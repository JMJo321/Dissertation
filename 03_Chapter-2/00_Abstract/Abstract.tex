\begin{abstract}
Does household electricity consumption respond only to the existence of peak-demand-hour price increases rather than to the magnitude of their marginal changes under Time-Of-Use (TOU) electricity pricing? The answer is yes if different types of electricity consumption are not taken into account separately. This paper re-explores the impact of TOU rates on residential electricity consumption with a different approach: by decomposing household electricity consumption into two different channels of consumption, which are classified according to their dependence on outdoor temperatures---i.e., consumption for non-temperature-control and temperature-control uses. I determine TOU-price-inducing change in each of the two distinct categories by applying Difference-In-Differences-style (DID-style) spline regression specifications to 30-minute interval metering data collected from an experiment in Ireland. My empirical results demonstrate that residential consumers sensitively respond to the marginal growth in peak-demand-hour price under the TOU tariff structure, while the high sensitivity is masked due to the opposite directional changes in the two types of electricity consumption. Moreover, my analysis also shows that the two channels of household electricity consumption evolve differently, and nonlinearly, according to daily heating degree days and the point electricity is consumed in time. Those multidimensional dynamics of residential electricity consumption under TOU tariffs imply that adopting autonomous heating control systems or augmenting additional across-day variations to the price scheme is required to maximize the benefits of TOU electricity pricing. 
\end{abstract}
