\begin{abstract}
\noindent
The existing literature has found that while Time-of-Use (TOU) electricity pricing causes reductions in aggregate household electricity consumption during peak-demand hours, the magnitude of these reductions is largely insensitive to incremental changes in the peak period price. In this paper, I re-examine the impact of TOU rates on residential electricity consumption with a different approach. Specifically, I decompose aggregate household electricity consumption into two different channels of consumption: consumption for non-temperature-control and temperature-control uses. I determine TOU-price-induced changes in both channels of consumption by applying Difference-In-Differences-style (DID-style) spline regression specifications using 30-minute interval metering data collected from an experiment in Ireland. My empirical results demonstrate that residential consumers are, in fact, quite sensitive to incremental changes in the peak-demand-hour price under the TOU tariff structures. However, this sensitivity is masked due to the opposite directional changes in the two channels of electricity consumption---i.e., during the peak hours, non-temperature-control-driven consumption falls as the peak price increases, while temperature-control-driven consumption actually increases as the peak price grows. Moreover, my analysis reveals that even for a given price, the response in each category of household electricity consumption, which is the combination of load-shedding and load-shifting, evolves differently according to not only daily heating degree days but also the point electricity is consumed in time. Those multidimensional dynamics of residential electricity consumption under TOU tariffs suggest that adopting autonomous heating control systems or augmenting additional across-day variations to the price scheme is required to maximize the benefits of TOU electricity pricing. 
\end{abstract}
