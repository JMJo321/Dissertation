Introduction
Many energy utilities are shifting customers onto Time-Of-Use (TOU) electricity rate structures, which have become feasible owing to the diffusion of renewable electricity generation capacity and smart metering technology.\footnote{According to \cite{A-Survey-of-Residential-TOU-Rates_FHS_2019}, a residential TOU rate is offered by about 15\% of all America's utilities in 2019.} Under a TOU tariff structure, the retail price of electricity varies across periods of the day---typically with a higher ``peak'' price during the late afternoon hours and lower ``off-peak'' prices during other hours. These TOU rates are intended to reduce electricity consumption during the peak demand hours of the day when the cost of supplying the electricity and the capacity constraints on transmission networks are at their greatest. In addition, shifting some of the consumption to lower demand hours potentially, when the cost of supplying electricity is far lower, is another intention of the dynamic rates. Ultimately, how effective the time-varying electricity prices are at reducing peak consumption, and shifting consumption across time, depends on how elastic consumers are to the magnitude of the price increase in peak-demand hours and the price decreases in the off-peak hours. In settings where households are unresponsive to within-day price changes, TOU programs may provide only small gains. In contrast, if consumers are very sensitive to the magnitude of the spread between the off-peak and peak electricity prices, that would suggest that additional gains could be achieved by adopting even more dynamic pricing, such as Real-Time Pricing (RTP), where the peak vs. off-peak price spread varies across days. While many evaluations of various dynamic electricity prices, including TOU programs, consistently document reductions in electricity consumption during peak hours \citep{Quantifying-Customer-Response-to-Dynamic-Pricing_Faruqui-and-George_2005, Dynamic-Pricing-of-Electricity-in-the-Mid-Atlantic-Region_Econometric-Results-from-the-Baltimore-Gas-and-Electric-Company-Experiment_Faruqui-et-al_2011, Dynamic-Pricing-of-Electricity-for-Residential-Customers_The-Evidence-from-Michigan_Faruqui-et-al_2013}, the literature often finds that households' consumption is quite inelastic to the magnitude of the within-day price changes \citep{Rethinking-Real-Time-Electricity-Pricing_Allcott_2011, Knowledge-is-Less-Power_Jessoe-and-Rapson_2014}. Notably, \cite{Peaking-Interest:How-Awareness-Drives-the-Effectiveness-of-Time-of-Use-Electricity-Pricing_Prest_2020} finds that, in a TOU pricing experiment in Ireland, households were highly insensitive to the incremental increases in the peak rate.\footnote{This paper, which also utilizes the CER experiment datasets, expresses the results as follows: ``Most of the overall response comes at the smallest price increase, with higher prices yielding strongly diminishing returns.''} That is, residential consumers seemed to respond only to the existence of the within-day price changes and not the magnitude of the within-day price changes. This paper aims to re-examine the TOU program evaluated by \cite{Peaking-Interest:How-Awareness-Drives-the-Effectiveness-of-Time-of-Use-Electricity-Pricing_Prest_2020} to understand why the households' aggregate consumption is so inelastic with respect to the magnitude of the within-day price changes. 

When re-measuring how sensitive residential consumers are to TOU tariffs, I decompose their electricity consumption into two distinct channels of consumption instead of merely investigating their consumption as a whole: 1) electricity consumption for non-temperature-control uses (e.g., lighting, operating appliances, and cooking), and 2) electricity consumption for temperature-control uses (e.g., cooling and heating). My empirical analysis focuses on those two broad categories of electricity consumption for two reasons. First, the two types of electricity consumption react differently to outdoor temperatures. Electricity consumed for temperature control will undoubtedly depend on outdoor temperatures. For example, more electricity will be utilized for heating on cold days compared to mild days. By contrast, electricity used for other non-temperature-control services will be largely independent of outdoor temperatures. These enable me to estimate how much electricity is consumed for each broad category by using temperature variations. Second, the two distinct electricity consumption categories may respond differently to TOU prices. For instance, TOU electricity pricing may cause households to relocate some non-temperature-control-driven services to non-peak hours without changing aggregate consumption across a day \citep{Residential-Response-to-Critical-Peak-Pricing-of-Electricity_California-Evidence_Herter-and-Wayland_2010, Empowering-Consumers-through-Data-and-Smart-Technology_HML_2016}. In contrast, if TOU rates induce them to lower their electricity use for heating, then there could be reductions in consumption across all hours. 

My study examines 30-minute interval residential electricity consumption data collected from a TOU pricing experiment conducted from July 2009 to December 2010 by the Commission for Energy Regulation (CER), the electricity and natural gas sector regulator in Ireland.\footnote{The CER changed its name to the Commission for Regulation of Utilities (CRU).} While the vast majority of homes in the sample utilize oil and gas as their primary energy source for heating, a sizable amount of electricity is used for heating in those homes. Notably, residential electricity consumption peaks during the winter months, typically reaching levels approximately 1.5 times higher than the consumption observed during the mild summer months. Using the observed household consumption throughout the day and measurements of the daily temperatures in Ireland, I estimate 1) the aggregate changes in temperature-control-driven and non-temperature-control-driven consumption caused by the TOU program, 2) how these consumption changes vary with the average daily outdoor temperature---more precisely, daily Heating Degree Days (HDDs)---, and 3) how these consumption changes vary with the magnitude of the peak-period price change.  

From my empirical analysis, I find that the households' non-temperature-control-driven electricity consumption was highly responsive to the magnitude of the peak price change. On the other hand, there is no evidence that the reduction in temperature-control-driven electricity consumption during the peak rate period increased as the magnitude of the peak price grew. Instead, there is weak evidence demonstrating that the reduction in temperature-control-driven electricity consumption during peaks went towards zero as the peak price increased. Interestingly, due to the opposite relationship between demand reductions and price changes in the two channels of electricity consumption, the high sensitivity of household electricity consumption in response to TOU pricing in the peak rate period was masked. In other words, when the estimated reductions in electricity consumption originating from the two channels are aggregated, the difference in the combined reduction between tariff groups is seemingly dampened because of the opposite correlations.\footnote{There were four tariff groups in the CER experiment. Refer to Figure \ref{Figure:Time-Of-Use-Pricing-Structures}.} Indeed, this is precisely the result discussed in \cite{Peaking-Interest:How-Awareness-Drives-the-Effectiveness-of-Time-of-Use-Electricity-Pricing_Prest_2020}. 

To explore why the two distinct categories of electricity consumption (i.e., temperature-control-driven and non-temperature-control-driven consumption) respond somewhat differently to the TOU prices during the peak price hours, I examine how both types of consumption change in the off-peak price hours---in particular, the hours leading up to and following the peak rate period (denoted the pre- and post-peak hours, respectively). In the TOU experiment, the households that experienced price increases during the peak hours also experienced decreases in the prices they paid for electricity in the hours surrounding the peak tariff period. Moreover, the higher the peak price the households had to pay, the lower the off-peak prices (i.e., the day and night rates) they had to pay. 

My regression analysis shows interesting non-temperature-control-driven consumption changes in the hours leading up to and following the peak rate period, even though not all relevant points estimates are statistically significant. Specifically, the TOU prices appeared to have a spillover effect on households' demand for electricity in the hours surrounding the peak rate period (i.e., in the pre- and post-peak periods): a reduction in their non-temperature-control-driven consumption even under a lower electricity price. In particular, the more considerable the peak price increase, the smaller the reduction in non-temperature-control-driven electricity consumption during non-peak hours. Furthermore, I find no concrete evidence suggesting that larger peak price increases, and corresponding more significant off-peak price decreases, caused households to shift some of their non-temperature-control consumption to the hours surrounding the peak hours. Therefore, along with the TOU-price-induced changes in non-temperature-control-driven consumption during peak hours, these findings imply that with respect to households' non-temperature-control-driven consumption, load-shedding was their dominant reaction to the peak rate increases in and near the peak price period.

My empirical analysis indicates that a different pattern emerged for the temperature-control-driven consumption changes in the pre- and post-peak hours, while the TOU tariffs also seemed to have a spillover effect on it. Although my results for the temperature-control-associated consumption are weak in terms of statistical significance, I find that during the pre-peak hours, households' temperature-control-driven electricity usage fell, and those reductions got larger as the magnitude of the peak price increased. That is, households exposed to a higher peak-demand-hour price appeared to reduce their pre-peak usage for heating by larger amounts. In contrast, my analysis demonstrates that households' temperature-control-driven electricity usage rose during post-peak hours. As opposed to the consumption changes in the pre-peak hours, these growths in electricity usage for heating during the post-peak hours got smaller as the size of the peak-hour price change increased. Furthermore, interestingly, those consumption changes near the peak rate period were observable only when outdoor temperatures were low enough. Altogether, those are not indicative of load-shifting (e.g., pre-heating their space and water prior to the peak rate period). Rather, those findings suggest that the TOU program caused a reduction in household demand for heating across the entire day. 

The findings described above could also contribute to the result that households' temperature-control-driven electricity consumption during the peak rate period was largely unresponsive to the magnitude of the peak price increase. For example, if households that experience a high peak price use less electricity for heating in the pre-peak hours, then they may not be as warm going into the peak hours. Consequently, more significant amounts of electricity may be consumed for heating during the peak price period than otherwise would have been absent the reduction in pre-peak heating. Effectively, households' temperature-control-driven electricity usage does appear to be sensitive to the size of the peak rate in that period. However, such responses are mostly seen prior to the peak rate period---and as a result, make the impacts during the peak hours look potentially more muted. In addition, this interpretation of the sequential behavioral changes related to temperature-control-driven consumption in time suggests an important policy implication: under TOU electricity pricing, impeding such pre-adjustment by exploiting an automation instrument, like Programmable Communicating Thermostats (PCTs), enables more reductions during peak hours.

In addition to their responsiveness to TOU prices, in my empirical analysis, the reduction in households' temperature-control-driven electricity consumption during the peak rate period showed a U-shaped profile over daily HDDs. The nonlinearity in TOU-tariff-induced temperature-control-associated reduction in household electricity consumption over households' daily heating needs discloses a veiled feature of TOU electricity pricing: its day-varying effects on the temperature-control-related part of residential electricity consumption. Suppose that the reductions obtained by adopting the TOU prices stem entirely from the non-temperature-control use of electricity. In that case, the degree of reductions does not vary across days because it is nearly irrelevant to across-day temperature variations. My empirical results, however, indicate that on days with moderate heating needs, a sizable share of reductions in household electricity consumption stemmed from electricity usage for temperature control during peak hours. Consequently, even though the TOU tariffs do not change across days, the tariffs already induce substantial reductions in electricity consumption for heating on typical winter days, in terms of daily HDDs, in Ireland. Therefore, on those days, the additional gains captured by switching TOU prices to Real-Time Pricing (RTP) are likely to be smaller than many economists have thought.\footnote{Under RTP, retail prices vary across not only hours of days but days according to contemporaneous generating costs.} 

The U-shaped evolving pattern of the reduction in temperature-control-driven electricity consumption over daily HDDs also implies that TOU pricing induces somewhat smaller decreases on days with relatively large heating needs, on which the grid is most burdened, in turn, the most significant diminution in electricity consumption is required. This undesirable quality of TOU electricity pricing, however, can be addressed by adopting a more flexible TOU-style pricing scheme in which household heating needs are integrated as an additional dimension of dynamics. According to my analysis, raising the size of a rate change in the peak-demand hours prevented the reduction driven by temperature-control-related consumption from disappearing. Furthermore, it produces more reduction in non-temperature-control-associated electricity consumption. In light of those findings, introducing an alternative pricing structure in which the magnitude of peak-hour price increases is proportionally coupled to daily HDDs could create additional gains on high-heating-needs days.  

To sum up, the results from my empirical analysis extend the previous work by isolating temperature-control-associated reduction in household electricity consumption from the entire TOU-price-induced demand declines. My results demonstrate that in and near the peak hours, the changes from each of the two channels of electricity consumption sensitively vary according to the magnitude of the price changes in the peak rate period. That is, in determining the electricity consumption level within a home under TOU tariff structures, not the mere existence of price changes, prices themselves still matter. Moreover, the day-varying performance under TOU prices suggests a vital policy implication: shifting from TOU towards RTP-like pricing can improve the reduction in residential electricity consumption on extremely cold days. In addition, examining the changes in electricity consumption from the two distinct categories of electricity consumption, not just in the peak rate period but in and near the period, enables unlocking the full benefits of TOU electricity pricing through the automation-technology-relevant policy implication.
