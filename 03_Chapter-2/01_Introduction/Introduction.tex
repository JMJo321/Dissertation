Many energy utilities are shifting customers onto Time-Of-Use (TOU) electricity rate structures, which have become feasible owing to the diffusion of smart metering technology.\footnote{According to \cite{A-Survey-of-Residential-TOU-Rates_FHS_2019}, residential TOU rates were offered by about 15\% of all America's utilities in 2019.} Under a TOU tariff structure, the retail price of electricity varies across periods of the day---typically with a higher ``peak'' price during the late afternoon hours and lower ``off-peak'' prices during other hours. These TOU rates are intended to reduce electricity consumption during the peak demand hours of the day when the cost of supplying the electricity and the capacity constraints on transmission networks are at their greatest. In addition to reducing peak demand for electricity, TOU pricing can also provide cost savings by shifting some of the consumption to lower demand hours or hours with excess renewable output. 

Ultimately, the cost savings that can be achieved by TOU tariff structures depend on two factors. First, the extent to which TOU rates can reduce electricity consumption during peaks, and shift consumption across time, relies on how elastic consumers are to the magnitude of the price increase in the peak demand hours and the price decreases in the off-peak hours. In settings where households are unresponsive to within-day price variations, TOU prices may provide only small gains. Second, the magnitude of the benefits achieved by TOU tariffs also depends on how the resulting reductions in electricity consumption differ across days. Intuitively, reducing electricity consumption will provide larger cost savings on days with very high electricity demand (e.g., days with extreme temperatures when demand for temperature control peaks). Suppose TOU tariffs merely cause a uniform reduction across days (e.g., households turn off their lights more often). In that case, the benefits from the time-varying rates will not vary across days. In contrast, if TOU rates induce households mainly to reduce their electricity consumption for heating or cooling, then the reductions in household electricity consumption are likely to be concentrated on peak demand days when the reductions will be the most valuable. 

In light of those factors, it is necessary not only to examine how responsive a household's aggregate consumption is to the magnitude of the peak vs. off-peak price changes, but also to decompose how much of the savings in household electricity consumption stem from reductions in the use of energy services that systematically vary across days (e.g., temperature control), in order to fully understand the full impacts of TOU electricity pricing on household electricity consumption. 

While many evaluations of various dynamic electricity prices, including TOU programs, consistently document reductions in electricity consumption during peak hours \citep{Quantifying-Customer-Response-to-Dynamic-Pricing_Faruqui-and-George_2005, Dynamic-Pricing-of-Electricity-in-the-Mid-Atlantic-Region_Econometric-Results-from-the-Baltimore-Gas-and-Electric-Company-Experiment_Faruqui-et-al_2011, Dynamic-Pricing-of-Electricity-for-Residential-Customers_The-Evidence-from-Michigan_Faruqui-et-al_2013}, the literature often finds that households' consumption is quite inelastic to the magnitude of the within-day price changes \citep{Rethinking-Real-Time-Electricity-Pricing_Allcott_2011, Knowledge-is-Less-Power_Jessoe-and-Rapson_2014}. Notably, \cite{Peaking-Interest:How-Awareness-Drives-the-Effectiveness-of-Time-of-Use-Electricity-Pricing_Prest_2020} finds that, in a TOU pricing experiment in Ireland, households were highly insensitive to the incremental increases in the peak rate.\footnote{This paper, which also utilizes the CER experiment datasets, expresses the results as follows: ``Most of the overall response comes at the smallest price increase, with higher prices yielding strongly diminishing returns.''} That is, residential consumers seemed to respond only to the existence of the within-day price changes and not the magnitude of the within-day price changes. This paper aims to re-examine the TOU program evaluated by \cite{Peaking-Interest:How-Awareness-Drives-the-Effectiveness-of-Time-of-Use-Electricity-Pricing_Prest_2020} to understand why household's aggregate consumption is so inelastic with respect to the magnitude of the within-day price changes. 

When re-measuring how sensitive residential consumers are to TOU tariffs, I decompose their electricity consumption into two distinct channels of consumption instead of merely investigating their consumption as a whole: 1) electricity consumption for non-temperature-control uses (e.g., lighting, operating appliances, and cooking), and 2) electricity consumption for temperature-control uses (e.g., cooling and heating). My empirical analysis focuses on those two broad categories of electricity consumption for two reasons. First, the two types of electricity consumption react differently to outdoor temperatures. Electricity consumed for temperature control will undoubtedly depend on outdoor temperatures. For example, more electricity will be used to heat on cold days compared to mild days. By contrast, electricity used for other non-temperature-control services will be largely independent of outdoor temperatures. These enable me to estimate how much electricity is consumed for each category by using temperature variation. Second, the two distinct electricity consumption categories may respond differently to TOU prices. For instance, TOU electricity pricing may cause households to relocate some non-temperature-control-driven services to non-peak hours without changing aggregate consumption across a day \citep{Residential-Response-to-Critical-Peak-Pricing-of-Electricity_California-Evidence_Herter-and-Wayland_2010, Empowering-Consumers-through-Data-and-Smart-Technology_HML_2016}. In contrast, if TOU rates induce them to lower their electricity use for heating, then there could be reductions in consumption across all hours. 

My study examines 30-minute interval household metering data collected from a TOU pricing experiment conducted from July 2009 to December 2010 by the Commission for Energy Regulation (CER), the electricity and natural gas sector regulator in Ireland.\footnote{The CER changed its name to the Commission for Regulation of Utilities (CRU).} While the vast majority of homes in the sample utilized oil and gas as their primary energy source for space and water heating, a sizable amount of electricity was still used for heating in those homes. Notably, residential electricity consumption peaked during the winter months, typically reaching levels approximately 1.5 times higher than the consumption observed during the mild summer months. Using the observed household consumption throughout the day and measurements of the daily temperatures in Ireland, I estimate 1) the changes in temperature-control-driven and non-temperature-control-driven consumption, respectively, caused by the TOU program, 2) how these consumption changes vary with the average daily outdoor temperature---more precisely, daily Heating Degree Days (HDDs)---, and 3) how these consumption changes vary with the magnitude of the peak-rate-period price change.  

From my empirical analysis, I document two key findings. First, the two broad categories of household electricity consumption were responsive to incremental changes in the peak-rate-period price, but in different ways. In the peak rate period, households' non-temperature-control-driven electricity consumption was highly sensitive to the magnitude of the price changes. On the other hand, there is no evidence that the reduction in temperature-control-driven electricity consumption during the peak rate period increased as the size of the incremental price changes grew. Instead, there is weak evidence demonstrating that in the peak rate period, the reduction in temperature-control-driven electricity consumption went towards zero as the price increased. Interestingly, due to the opposite relationship between demand reductions and price changes in the two channels of electricity consumption, the high sensitivity of household electricity consumption in response to TOU pricing in the peak rate period was masked. In other words, when the estimated reductions in electricity consumption originating from the two channels are aggregated, the difference in the combined reduction between tariff groups is seemingly dampened because of the opposite correlations.\footnote{There were four tariff groups in the CER experiment. See Figure \ref{Figure:Time-Of-Use-Pricing-Structures}.} Indeed, this finding precisely explains the price insensitivity discussed in \cite{Peaking-Interest:How-Awareness-Drives-the-Effectiveness-of-Time-of-Use-Electricity-Pricing_Prest_2020}.\footnote{See \textit{5.3 Price Insensitivity} in \cite{Peaking-Interest:How-Awareness-Drives-the-Effectiveness-of-Time-of-Use-Electricity-Pricing_Prest_2020}.}  

Even in the hours leading up to and following the peak rate period (denoted the pre- and post-peak hours/periods, respectively), the TOU tariffs also induced changes in households' demand for electricity, which cannot be explained simply by price drops in the hours surrounding the peak rate period. In the experiment, the households under the TOU tariff structures experienced price increases during the peak hours, whereas they faced decreases in the price they paid for electricity consumption in the hours surrounding the peak rate period. Moreover, the higher the price the households had to pay in the peak rate period, the lower the off-peak prices (i.e., the day and night rates) they had to pay. My regression analysis shows that households reduced their non-temperature-control-related electricity consumption in both off-peak periods. In other words, the load-shedding in the peak rate period spilled over into the pre- and post-peak hours, during which prices fell. On top of that, load-shifting from the peak to the off-peak hours, incentivized by across-rate-period price differences, seemed to occur, too. Furthermore, the revealed relationship between the size of the load-shifting and the magnitude of the peak-hour price change confirms the economic intuition about the price incentive for the load relocation. My analysis also suggests that the load-shifting only partly, or just barely, offset the spillovers. In the aggregate, in both off-peak periods, the more considerable the price increase in the peak rate period, the smaller the reduction.

For temperature-control-driven consumption changes, my empirical analysis indicates that a different pattern emerged in pre- and post-peak hours. I find that during the pre-peak hours, households' temperature-control-driven electricity usage fell, and those reductions got larger as the magnitude of the price jump in the peak rate period increased. That is, households exposed to a higher peak-demand-hour price appeared to reduce their pre-peak usage for heating by larger amounts. In contrast, my analysis demonstrates that households' temperature-control-driven electricity usage rose during post-peak hours. As opposed to the consumption changes in the pre-peak hours, these growths in electricity usage for heating during the post-peak hours got smaller as the size of the peak-hour price change increased. Altogether, in both non-peak periods, due to the opposite directional changes in the two categories of household electricity consumption, households' sensitive responsiveness to the TOU tariff structures was muted, as it was in the peak rate period. Interestingly, those temperature-control-relevant consumption changes near the peak rate period were observable only when outdoor temperatures were low enough. 

The second key finding from my empirical analysis is that the reduction in households' temperature-control-related electricity consumption during the peak hours showed a U-shaped profile over daily HDDs. The nonlinearity in TOU-tariff-induced temperature-control-associated reduction in household electricity consumption over households' daily heating needs discloses a veiled feature of TOU electricity pricing: its day-varying effects on the temperature-control-related part of residential electricity consumption. Suppose that the reductions obtained by adopting TOU prices stem entirely from the non-temperature-control use of electricity. In that case, the degree of reductions does not vary across days because it is nearly irrelevant to across-day temperature variation. My empirical results, however, indicate that on days with moderate heating needs, a sizable reduction in household electricity consumption stemmed from electricity usage for temperature control during the peak hours. For instance, in the case of the household subgroup that experienced a six-dollar price increase in the peak rate period, more than two-thirds of the reduction in their electricity consumption came from temperature-control-related consumption when the value of daily HDDs was ten. Consequently, even though the TOU electricity pricing only has intraday price variation, the pricing already induces a substantial reduction in electricity consumption for heating on typical winter days, in terms of daily HDDs, in Ireland. Therefore, on those days, the additional gains captured by switching TOU prices to Real-Time Pricing (RTP) will likely be smaller than many economists have thought.\footnote{Under RTP, retail prices vary across not only hours of days but days according to contemporaneous generating costs.} 

To sum up, the results from my empirical analysis extend the previous work by isolating temperature-control-associated reduction in household electricity consumption from the entire TOU-tariff-induced demand declines. My results demonstrate that in and near the peak hours, the changes from each of the two channels of electricity consumption are responsive to the magnitude of the price changes in the peak rate period. That is, in determining the electricity consumption level within a home under TOU tariff structures, not the mere existence of price changes, prices themselves---more clearly, the magnitude of the price increase in the peak hours---still matter. Moreover, the day-varying performance of TOU electricity pricing suggests a vital policy implication of an alternative electricity pricing that internalizes an additional layer of dynamics by nonlinearly synchronizing price increases in the peak hours with daily HDDs, causing a more significant reduction in household electricity consumption on extremely cold days. 
