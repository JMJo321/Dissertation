These days many energy utilities are moving towards Time-Of-Use (TOU) electricity pricing, which has become feasible owing to the diffusion of renewable electricity generation capacity and smart metering technology. Under a TOU tariff structure, the pre-determined growth in peak-hour rate, which is usually invariant across days, induces reductions in electricity consumption when the cost of supplying the electricity and the capacity constraints on transmission networks are at their greatest (i.e., during peaks). Many evaluations of experiments that assessed how households respond to TOU tariffs have consistently documented reductions in electricity consumption during peak hours. Furthermore, TOU prices can incentivize consumers to shift their electricity consumption from peak to off-peak hours. Ultimately, how effective the time-varying electricity prices are at reducing or relocating electricity consumption depends on how responsive electricity consumers are to the magnitude of the price increase in peak-demand hours. 
Suppose that consumers are very sensitive to the change in peak-hour electricity price. In that case, there might be additional gains from adopting even more granular forms of dynamic pricing, such as Real-Time Pricing (RTP). \cite{Peaking-Interest:How-Awareness-Drives-the-Effectiveness-of-Time-of-Use-Electricity-Pricing_Prest_2020}, however, finds that even during peaks, households were highly insensitive to the incremental risings in the peak rate.\footnote{This paper, which also utilizes the CER experiment datasets, expresses the results as follows: ``Most of the overall response comes at the smallest price increase, with higher prices yielding strongly diminishing returns.''} In other words, according to the previous work, residential consumers seemed to respond only to the existence of the price changes. In this paper, the main goal is to re-evaluate households' responsiveness to the magnitude of the peak-hour price increases in the setting exploited by the paper, but through a different approach. 

To measure how sensitive residential consumers are to the size of the price variations in peak-demand hours, I decompose their consumption changes in response to TOU tariffs into two distinct channels of electricity savings instead of simply investigating the changes as a whole: 1) savings from electricity consumption for non-temperature-control uses (e.g., lighting, operating appliances, and cooking), and 2) savings from electricity consumption for temperature-control uses (e.g., cooling and heating). The two categories of electricity use (i.e., the two sources of TOU-price-inducing electricity savings) are inherently different in timeliness---the lag between the moment electricity is consumed to create a specific service and the point the service is actually exploited in time. In the case of non-temperature-control-relevant electricity use, which is nearly independent of temperature variations, the timeliness is usually high. For example, lighting service has no lag because the service is available very the moment electricity is consumed. Electricity consumption for temperature-control uses, by contrast, can have a longer lag. Somebody might warm up his house before the time he gets home from work by using automation technology, like  Programmable Communicating Thermostats (PCTs). In that case, PCTs cause changes in electricity consumption across hours of the day. Due to the dissimilarity, examining the aggregate impact of TOU pricing on household electricity consumption is evidently insufficient to identify unique consumption changes relevant to each channel. In addition to the difference in timeliness, even for a given peak-hour price increase, induced consumption changes in for-temperature-control use of electricity on mild days could be considerably different from those on days with extreme temperatures. Moreover, different implications can be drawn depending on the share of electricity savings between the two sources. For instance, although TOU electricity pricing only has within-day price changes, the time-varying pricing can generate sizable variations in electricity savings across days if considerable savings come from temperature-control-related electricity use. For those reasons, in my empirical analysis, I isolate the temperature-control-use-associated savings from the whole by exploiting temperature variations across days. 

My study examines 30-minute interval residential electricity consumption data collected from a TOU pricing experiment conducted from July 2009 to December 2010 by the Commission for Energy Regulation (CER), the electricity and natural gas sector regulator in Ireland.\footnote{The CER changed its name to the Commission for Regulation of Utilities (CRU).} Due to Irish households' widespread use of non-electric fuels for space and water heating, the sample utilized in the empirical analysis only includes meter readings from non-electric heating households in order to draw more universal policy implications from my empirical results. Furthermore, instead of focusing on the peak rate period, my empirical analysis also covers intervals near the period (i.e., two-hour-length pre- and post-peak intervals), in which the level of household electricity consumption changed a lot. Using spline regressions inspired by the Difference-in-Differences (DID) strategy, I estimate not only how electricity savings from the TOU program vary with average daily Heating Degree Days (HDDs) but also how the savings alter with the magnitude of price changes in the peak rate period. By doing so, I identify three building blocks underlying the multi-faceted dynamics of the electricity savings arising from the two distinct drivers of household electricity consumption around peak hours (i.e., in and near peak hours): the magnitude of price spikes at peak hours, daily HDDs, and the point at which electricity is consumed in time. 

One of the most compelling findings from my empirical analysis is that in peak hours, the treated households were highly responsive to the level of price jumps. During the peak rate period, the savings from electricity consumption for non-temperature-control purposes were directly proportional to the price increases in that period. On the other hand, the saving related to the for-temperature-control use of electricity diminished as the degree of the peak-hour price changes became more prominent. Interestingly, due to the opposite relationship between demand reductions and price changes in the two channels of electricity savings, the high sensitivity of household electricity consumption in response to TOU pricing in the peak rate period was masked. In other words, when the estimated electricity savings originating from the two sources are aggregated, the difference in the combined savings between tariff groups is seemingly dampened because of the opposite correlations.\footnote{There were four tariff groups in the CER experiment. Refer to XYZ.} Indeed, this is precisely the result discussed in \cite{Peaking-Interest:How-Awareness-Drives-the-Effectiveness-of-Time-of-Use-Electricity-Pricing_Prest_2020}. In addition to such price sensitivities, as expected, their consumption changes depended on heating needs in a day (i.e., daily HDDs) for a given price spike in the peak rate period. To be specific, during peaks, the treatment effects on household electricity consumption for temperature-control uses showed a U-shaped profile over daily HDDs, which implies that the effectiveness of TOU pricing varies with daily heating needs. 

The nonlinearity in TOU-tariff-inducing electricity savings over households' daily heating needs discloses a veiled feature of TOU electricity pricing: its day-varying effects on residential electricity savings. Suppose that the savings obtained by adopting the TOU prices stem entirely from the non-temperature-control use of electricity. In that case, the degree of savings does not vary across days because it is nearly irrelevant to across-day temperature variations. My empirical results, however, illustrate that on days with moderate heating needs, a sizable share of savings does stem from electricity use for temperature-control purposes. Consequently, even though the TOU rates are not variable across days, the tariff structures already induce substantial reductions in electricity consumption on typical winter days, in terms of daily HDDs, in Ireland. Therefore, on those days, the additional gains captured by switching TOU prices to Real-Time Pricing (RTP) are likely to be smaller than many economists have thought.\footnote{Under RTP, retail prices vary across not only hours of days but days according to contemporaneous generating costs.} In contrast, the U-shaped evolving pattern of the temperature-control-relevant savings over daily HDDs implies that TOU pricing induces rather fewer savings on days with relatively large heating needs, on which the grid is most burdened, in turn, the most significant electricity savings are required. This undesirable quality of TOU electricity pricing, however, can be addressed by adopting a TOU-style pricing scheme in which household heating needs are integrated as an additional dimension of dynamics. According to my analysis, raising the size of a rate change in the peak-demand hours prevents the electricity savings driven by temperature-control-related consumption from disappearing. Furthermore, it produces more temperature-control-associated savings. In light of those findings, introducing an alternative pricing structure in which the magnitude of peak-hour price increases is proportionally coupled to daily HDDs might create additional savings on high-heating-needs days.  

Even in the pre- and post-peak intervals, the households assigned to the treatment group also sensitively adjusted their consumption behavior according to the magnitude of peak-hour price increases. In other words, the TOU prices facilitated spillover effects on households' consumption behavior in near-peak-hour intervals, during which they were not subject to the price raised to a pre-determined level. In both intervals, the households reduced their electricity consumption for non-temperature-control uses in inverse proportion to the size of the peak-rate-period price increases. In addition, with respect to their temperature-control-related consumption, they did not respond to the TOU tariffs until daily HDDs were sufficiently sizable. On days with relatively high HDDs, the TOU tariffs made the residential consumers reduce their for-heating consumption in the before-peak interval as the size of the peak-hour price changes increased. In the after-peak interval, the larger the magnitude of the peak-hour price changes, the smaller the heating-related additional consumption. As in the peak rate period, due to the flipped correlations between induced consumption changes and peak-demand-hour price variations in the two sources of electricity savings, the seemingly lessened responsiveness of the treated households occurred in the off-peak intervals as well. 

The estimated consumption changes allow me to infer how the treated households adapted their consumption behavior to the TOU program around the peak rate period. As discussed, the households' behavioral changes were not restricted to peak hours. Regarding the electricity consumption for non-temperature-control uses, in lieu of relocating their peak-hour consumption to off-peak hours, the households assigned to the treatment group in the experiment simply reduced their demand in and near the peak rate period. In other words, from the pre-peak to the post-peak intervals, the households reacted to the price jumps in peak-demand hours through not load-shifting but load-shedding. For temperature-control-associated electricity savings, on the other hand, the households' consumption changes in the pre-peak hours were likely to determine the degree of their behavioral changes in the following periods. Specifically, the electricity savings obtained from adjustment during the before-peak interval seemed to lead to fewer savings in the following period (i.e., the peak rate period), which in turn brought about limited additional consumption during the after-peak interval. Those sequential behavioral changes associated with temperature-control-related electricity use have an important policy implication: under TOU pricing, impeding such pre-adjustment by exploiting an automation instrument, like PCTs, enables more electricity savings during peaks.

To sum, the results from my empirical analysis extend the previous work by isolating temperature-control-associated electricity savings from the entire TOU-pricing-causing demand reductions. My results demonstrate that around peak hours, the savings from each of the two different channels sensitively vary according to the magnitude of the price changes in the peak rate period. That is, in determining household electricity consumption, not the mere existence of price changes, prices themselves still matter under TOU tariff structures. Moreover, the day-varying electricity savings under TOU prices suggest a vital policy implication: shifting from TOU towards RTP-like pricing can improve residential electricity savings on extremely cold days. In addition, examining the electricity savings from the two distinct sources, not in the peak rate period but around the period, enables unlocking the full benefits of TOU electricity pricing through the automation-technology-relevant policy implication.

