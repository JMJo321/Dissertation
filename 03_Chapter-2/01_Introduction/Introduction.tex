These days many utilities are moving towards Time-of-Use (TOU) electricity tariff structures that raise the rate for peak-demand hours to the predetermined level, which does not vary across days. Many evaluations of experiments that assessed how residential consumers respond to TOU rates consistently document reductions in electricity consumption during peak hours. Those studies, however, do not examine how such reductions in household electricity consumption are achieved. In this research, to understand the mechanism of the demand reductions in detail, I decompose the decline in residential electricity consumption into two different sources of electricity savings: 1) electricity savings derived from the reduction in electricity consumption for temperature-control uses (e.g., cooling and heating), and 2) electricity savings from non-temperature-control uses (e.g., lighting, operating appliances, and cooking). Furthermore, instead of focusing only on the peak hours, my empirical analysis also covers around peak hours (i.e., 2-hour-length pre- and post-peak intervals), in which the level of household electricity consumption changed a lot.

My study examines 30-minute interval residential electricity consumption data collected from a TOU pricing experiment conducted from July 2009 to December 2010 by the Commission for Energy Regulation (CER), the electricity and natural gas sector regulator in Ireland. Due to Irish households' widespread use of non-electric fuels for space and water heating, the sample utilized in the empirical analysis only includes meter readings from non-electric heating households. Using Difference-in-Differences (DID) strategy, I estimate the Average Treatment Effects (ATEs) of the TOU prices on household electricity consumption. To be specific, I measure how electricity savings from the TOU program vary with average daily temperatures. In addition, I also estimate how the savings alter with the magnitude of price changes during peak hours. By doing so, I find that around peak hours, the electricity savings stemming from the two distinct drivers of household electricity consumption (i.e., temperature-control and non-temperature-control uses) evolve differently depending on the point in time where the electricity is consumed, daily Heating Degree Days (HDDs), and the size of peak-hour price spikes.

The empirical analysis reveals that the TOU-price-causing variations in residential electricity demand are not restricted to the peak rate period. In addition to the consumption reductions during the peak rate period, my results show that participating households tend to cut their electricity consumption down before directly experiencing a jump to the predetermined peak rate. Moreover, such pre-adjustments are observed both from non-for- and for-heating electricity uses. 

From the detailed analysis, I also find that temperature-control-use-associated electricity savings are a nonlinear function of daily HDDs around peak hours. In other words, for a given peak-hour price jump, the impact of the TOU electricity tariffs on residential demand at an instant of time varies with daily HDDs. Specifically, the changes in household electricity consumption for heating show a U-shape profile over daily HDDs in the peak rate period, while those in the pre- and post-peak intervals do not emerge until daily HDDs are sufficiently sizable. 

The most absorbing finding from the empirical analysis is that households are highly responsive to the level of price changes in peak-demand hours. In the two-hour-length interval just before the peak-demand hours, the HDD-varying treatment effect on household electricity consumption for temperature-control uses is proportional to the size of the peak-demand-hours rate changes. On the other hand, the treatment effect on non-temperature-control-use-related electricity demand is inversely proportional to it. Interestingly, those relationships are flipped for the peak-rate-period electricity consumption. To be specific, the savings related to the for-heating use of electricity increase as the price jumps in the peak-demand interval become more prominent, while the savings stemming from the electricity consumption for non-for-heating purposes is proportional to the jumps. Due to the opposite order of the magnitude of the demand reductions, in each interval, the high sensitivity of household electricity consumption to the TOU tariffs around peak hours is masked. Indeed, this is precisely the result discussed in \cite{Peaking-Interest:How-Awareness-Drives-the-Effectiveness-of-Time-of-Use-Electricity-Pricing_Prest_2020}, which also utilizes the CER experiment datasets.\footnote{\cite{Peaking-Interest:How-Awareness-Drives-the-Effectiveness-of-Time-of-Use-Electricity-Pricing_Prest_2020} expresses the result as follows: ``Most of the overall response comes at the smallest price increase, with higher prices yielding strongly diminishing returns.''}

How households adapted their consumption behavior to the TOU tariff structures newly introduced can be deduced from those empirical findings above. Regarding the electricity consumption for non-temperature-control uses, the households assigned to the treatment group in the experiment simply reduced their demand around the peak rate period in lieu of reallocating it to off-peak hours. In other words, participating households reacted to the price jumps in peak demand hours, not through load-shifting but load-shedding. On the other hand, for-heating-associated electricity savings accompanied more complicated behavioral changes. Households adjusted their electricity consumption during the pre- and post-peak hours only when HDDs were large enough. And the reductions obtained from the pre-adjustment seem to lead to fewer savings in the following period (i.e., the peak rate period), especially on freezing days. 

Those findings disclose a veiled advantage of TOU electricity pricing: day-varying effects on residential electricity savings. Let us suppose the electricity savings obtained by adopting the TOU prices stem entirely from the reductions in non-temperature-control uses. In that case, intuitively, the magnitude of the savings does not vary across days. In other words, the amount of the savings does not depend on across-day temperature variations when residential demand for electricity peaks. However, my empirical results illustrate that on days with moderate heating needs, a sizable share of the electricity savings does stem from reductions in the use of electricity for temperature-control uses. Consequently, even though the TOU rates do not change across days, the tariff structures already induce substantial reductions in electricity consumption on a typical winter day, in terms of daily HDDs, in Ireland. 

The empirical results also provide new insights into the potential benefits of adopting even more dynamic price structures, e.g., Real-Time Pricing (RTP).\footnote{Under RTP, retail prices vary across not only hours of days but days according to contemporaneous generating costs.} As discussed earlier, the U-shaped evolving pattern of the temperature-control-use-associated electricity savings over daily HDDs in the peak rate period implies that TOU electricity pricing becomes less effective on days with extreme outside temperatures. So on those days that the grid is most burdened, in turn, the most significant electricity savings are required, the additional gains from switching from TOU prices to RTPs might be smaller than many economists have thought. Nevertheless, considering that a high price change in the peak-demand hours prevents the temperature-control-use-driven electricity savings from disappearing, allowing the peak-hour prices to rise by synchronizing it with daily HDDs would induce moderately more savings on high-demand days.

In addition, the identified gap, in terms of the most attainable for-heating-use-related electricity savings, between the lowest and the highest rate changes in the peak rate period suggests the potential gains from adopting automation technologies. As already discussed, the gap is likely a side effect of households' behavior change during the pre-peak interval. Hence, if it is possible to impede such changes in temperature-control-related electricity consumption from pre-peak to peak hours by exploiting an automation instrument, like an automated thermostat, more electricity savings can be achieved under TOU electricity pricing. 

To sum, the results from my empirical analysis extend the previous work by decoupling temperature-control-use-related electricity savings from the entire TOU-pricing-causing demand reductions. My results demonstrate that around peak hours, the savings from two different sources sensitively vary according to the magnitude of the price change in the peak rate period. That is, prices still matter under TOU tariff structures. Moreover, the day-varying electricity savings under TOU prices suggest a vital policy implication: shifting from TOU towards RTP-like pricing can improve residential electricity savings on extremely cold days. In addition, examining the electricity savings from two distinct sources, not in the peak rate period but around the peak rate period, enables unlocking the full benefits of TOU electricity pricing through the automation-technology-relevant policy implication. 
