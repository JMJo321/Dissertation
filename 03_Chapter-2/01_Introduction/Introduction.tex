% TODO: Add contents (Target length = 2.5 ~ 3 pages)
These days many utilities are moving towards Time-of-Use (TOU) electricity tariff structures that raise the rate for peak demand hours to the predetermined level, which does not vary across days. Many evaluations of experiments that assess how residential consumers respond to TOU rates consistently document reductions in electricity consumption during peak hours. However, those studies do not examine how the reductions in household electricity consumption are being achieved. In this research project, I decompose the demand reductions during the peak hours into two different sources of energy savings: 1) energy savings derived from the reduction in electricity consumption for temperature-control (e.g., cooling and heating), and 2) energy savings from non-temperature-control uses (e.g., lighting, operating appliances, and cooking). 

My study examines 30-minute interval electricity consumption data from 4,096 Irish households that participated in a TOU pricing experiment conducted from July 2009 to December 2010 by the Commission for Energy Regulation (CER), the regulator for electricity and natural gas sectors in Ireland. In the region studied, heating accounts for a large portion of residential electricity consumption. Using Difference-in-Differences (DID) strategy, I estimate how savings obtained from the TOU program vary with the magnitude of the increase in price during the peak hours. In addition, I measure how savings alter with the daily temperature. By doing so, I can identify the share of energy savings during the peak hours stemming from the two distinct drivers of household electricity consumption (i.e., temperature-control and non-temperature-control uses). Overall, I find that deploying TOU prices reduces residential peak-hour demand for electricity by nearly 10\%. Importantly, I also find that nearly half of the energy savings stem from reductions in electricity consumed for temperature-control use. 

The findings provide new insights into the potential benefits of adopting even more dynamic price structures, e.g., Real-Time Pricing (RTP), which would vary the retail price not only across hours of the day but across days according to contemporaneous generating costs. Intuitively, if the energy savings caused by adopting TOU prices stemmed entirely from reductions in non-temperature-control-driven uses, then we would not expect the savings to vary meaningfully across days. In other words, the energy savings would be similar on days with mild temperatures and days with extreme temperatures, when the demand for electricity (and the cost of supplying it) peaks. However, my results illustrate that a sizable share of the energy savings caused by moving towards TOU tariffs does stem from reductions in the use of electricity for temperature-control use. Consequently, even though the TOU rates do not vary across days, the tariff structure already induces substantial reductions in electricity consumption on days in which the temperatures are extreme, in turn, the grid is most burdened. And this suggests that, at least currently, the additional gains from switching from TOU prices to RTPs may be smaller than many economists have thought. 

To further examine the potential gains from switching from TOU to RTP regime, I utilize my estimate of the heterogeneity in electricity consumption reductions under the TOU experiment to predict the impact of adopting an RTP-like tariff structure. While my results imply that allowing the peak-hour prices to vary across days would induce moderately larger energy savings on days with more extreme temperatures, the magnitude of the increased savings on high-demand days would be fairly small. This prediction comes from the fact that the reductions in electricity consumption during the peak hours in the experiment are quite insensitive to the size of the peak-period price increase. Indeed, this is precisely the result discussed in \cite{Peaking-Interest:How-Awareness-Drives-the-Effectiveness-of-Time-of-Use-Electricity-Pricing}. Importantly, my findings extend the previous work by highlighting that it is the temperature-control-driven energy savings that are insensitive to the magnitude of the price changes. And according to my empirical results, the reduction in electricity consumption for non-temperature-control uses is proportional to the size of the tariff change. So, the low responsiveness to the peak price variations in the paper seems reasonable because the energy savings from non-temperature-control uses only account for half of the total savings.

To sum, my results demonstrate that, by reducing household electricity consumption for temperature-control use, TOU prices are already effectively lowering the demand for electricity during the peak hours of the peak demand days. Moreover, I ultimately predict that moving from TOU towards RTP-like pricing would not significantly alter residential electricity consumption, especially during the peak demand hours of the days with extreme temperatures. An important policy implication of the empirical findings is that designing an incentive to make residential electricity consumption for temperature-control use more responsive to price variations could be a key to unlocking the full benefits that time-varying price structures could provide.
