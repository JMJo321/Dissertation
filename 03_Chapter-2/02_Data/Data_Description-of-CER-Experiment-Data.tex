\afterpage{
    \begin{table}
        \centering
        \caption{Treatment and Control Group Assignments}
        \label{Table:Treatment-and-Control-Group-Assignments}
        \vspace{0.2cm}
        \begin{adjustbox}{scale = 0.85}
            \begin{tabular}{
                >{\raggedright}m{4.5cm} |
                >{\raggedleft}m{1.5cm} 
                >{\raggedleft}m{1.5cm} 
                >{\raggedleft}m{1.5cm} 
                >{\raggedleft}m{1.5cm} 
                >{\raggedleft}m{1.5cm} ||
                >{\raggedleft\arraybackslash}m{2.1cm}
            }
                \hline \hline
                \multicolumn{1}{c|}{Stimuli} & \multicolumn{5}{c||}{Tariffs} & \multicolumn{1}{c}{Total} \\
                \cline{2-6}
                \multicolumn{1}{c|}{}  & \multicolumn{1}{c}{Control} & \multicolumn{1}{c}{A} & \multicolumn{1}{c}{B} & \multicolumn{1}{c}{C} & \multicolumn{1}{c||}{D} & \multicolumn{1}{c}{} \\
                \hline
                Monthly Bill & 0 & 79 & 37 & 89 & 28 & 233 \\
                Bi-Monthly Bill & 0 & 81 & 34 & 76 & 34 & 225 \\
                Bi-Monthly Bill $+$ IHD & 0 & 79 & 22 & 86 & 30 & 217 \\
                Bi-Monthly Bill $+$ OLR & 0 & 90 & 27 & 84 & 34 & 235 \\
                Control & 260 & 0 & 0 & 0 & 0 & 260 \\
                \hline
                Total & 260 & 329 & 120 & 335 & 126 & 1,170 \\
                \hline \hline
            \end{tabular}
        \end{adjustbox}
%        \begin{tablenotes}[flushleft]
%            \small
%            \textit{Note}: (The purpose of this sentence is to determine the width of the table. Therefore, this note should be replaced or deleted after polishing the table size.)
%        \end{tablenotes}

    \end{table}
}
The CER experiment dataset disseminated by the Irish Social Science Data Archive (ISSDA) consists of participating households' electricity consumption and survey data.\footnote{Many papers have explored the CER dataset with different focuses. See \cite{Reducing-Household-Electricity-Demand-through-Smart-Metering_Carroll-et-al_2014}, \cite{Unintended-Outcomes-of-Electricity-Smart-Metering_McCoy-and-Lyons_2016}, \cite{Nudging-Electricity-Consumption-using-TOU-Pricing-and-Feedback_Cosmo-and-OHora_2017}, and \cite{Estimating-the-Impact-of-Time-Of-Use-Pricing-on-Irish-Electricity-Demand_Di-Cosmo-et-al_2014}.}  

Throughout the baseline and treatment periods, meter reads for residential participants were recorded in 30-minute intervals. The high granularity of the electricity consumption data generated from a well-designed experiment enables quantifying the energy savings when transferring to TOU electricity pricing for each of the three rate periods. 

The survey data contains participants' responses to more than 300 questions in pre- and post-trial surveys. The primary purpose of the two surveys was to gather trial-associated experiential and attitudinal feedback from the households. The surveys also included questions intended to collect residential participants' socio-demographic characteristics. In addition, questions about the physical attributes of their house were included in the surveys. 

My empirical analysis throughout this paper uses a longitudinal sample that consists of observations satisfying certain conditions only. First of all, the sample is constructed by including observations only for the second half of each experiment period.\footnote{I exclude the observations for the first half of the treatment period because there is no counterpart observation in the baseline period.} From this sample, I keep observations for non-holiday weekdays in the published electricity consumption data because the TOU rates were active just on those days. And then, only households that continuously exploited non-electric fuels for their space and water heating during the experiment periods (i.e., the baseline and the treatment periods) are preserved in the sample.\footnote{From the survey data, it is possible to find out what type of fuel each responding household used for each heating purpose during each period. \par
There are two reasons why only non-electric-heating households are exploited in the following empirical analysis. First, in Ireland, non-electric fuels, such as oil, gas, and solid fuels, fulfill most of the residential heating demand. Specifically, according to \cite{Heating-and-Cooling-in-Ireland-Today_SEAI_2022}, only 4\% of Irish households utilize electricity to heat their space and water. Therefore, with respect to fuels for heating in Ireland, the sample consisting of non-electric heating households only is representative. Second, as Figure \ref{Figure:Pre-and-Post-Treatment-Household-Average-Daily-Electricity-Consumption} demonstrates, even non-electric-heating households consumed more electricity as temperatures decreased. In other words, electricity is still essential for non-electric-heating households to warm their space or water. Hence, the sample, including non-electric-heating households only, is well aligned with one of the primary purposes of this research: to evaluate the impact of TOU pricing on temperature-control-driven residential electricity consumption separately.} Moreover, among the non-electric-heating households, those with unreliable meter reads are excluded from the sample.\footnote{To be specific, the residential participants who had no consumption for eight days or more are excluded from the sample. In addition, I drop the meter reads for the days when several participating households' consumption data were missed. \par
Although I utilize the sample satisfying the following criteria too for the empirical analysis, applying the criteria does not change the results: 1) Exclude the day immediately following the end of daylight-saving time due to noticeably different consumption levels in the same hours of the day; 2) Drop the observations for the last five days of the baseline and treatment periods because of extraordinarily high electricity demand on those days.} This process results in 1,170 households. Table \ref{Table:Treatment-and-Control-Group-Assignments} summarizes the assignment distribution of the 1,170 households. 

The control and treatment groups in the sample are largely balanced, as shown in Table \ref{Table:Summary-Statistics-and-Differences-in-Means}. Although several variables are statistically significant at the 5\% level, the Bonferroni familywise $p$-value is not significant at that level. The absence of differences between the two groups over many observables are consistent with previous studies examining the CER experiment dataset.\footnote{To check the balance between the control and treatment groups, \cite{Peaking-Interest:How-Awareness-Drives-the-Effectiveness-of-Time-of-Use-Electricity-Pricing_Prest_2020} employs a linear probability model, while a probit model is used in \cite{The-Effect-of-Information-on-TOU-Electricity-Use:An-Irish-Residential-Study_Pon_2017}.} 
\afterpage{
    \begin{table}
        \centering
        \caption{Summary Statistics and Differences in Means}
        \label{Table:Summary-Statistics-and-Differences-in-Means}
        \vspace{0.2cm}
        \begin{adjustbox}{scale = 0.8}
            \begin{tabular}{
                >{\raggedright}m{6.5cm}
                >{\raggedleft}m{1.25cm}
                >{\raggedleft}m{1.25cm}
                >{\raggedleft}m{0.1cm}
                >{\raggedleft}m{1.25cm}
                >{\raggedleft}m{1.25cm}
                >{\raggedleft}m{0.1cm}
                >{\raggedleft}m{1.25cm}
                >{\raggedleft}m{1.25cm}
                >{\raggedleft\arraybackslash}m{1.25cm}
            }
                \hline \hline
                \multicolumn{1}{c}{} & \multicolumn{2}{c}{Control} & \multicolumn{1}{c}{} & \multicolumn{2}{c}{Treatment} & \multicolumn{1}{c}{} & \multicolumn{3}{c}{Difference} \\
                \cline{2-3} \cline{5-6} \cline{8-10}
                \multicolumn{1}{c}{} & \multicolumn{1}{c}{Mean} & \multicolumn{1}{c}{(S.E.)} & \multicolumn{1}{c}{} & \multicolumn{1}{c}{Mean} & \multicolumn{1}{c}{(S.E.)} & \multicolumn{1}{c}{} & \multicolumn{1}{c}{Mean} & \multicolumn{1}{c}{(S.E.)} & \multicolumn{1}{c}{$p$-value} \\
                \hline
                \underline{Electricity Consumption during Baseline Period \ ($kWh$)} &  &  &  &  &  &  &  &  &  \\
                \hspace{0.2cm} Daily & 22.122 & (0.674) &  & 23.529 & (0.379) &  & 1.407 & (0.773) & 0.069 \\
                \hspace{0.2cm} Hourly & 0.939 & (0.028) &  & 0.996 & (0.016) &  & 0.057 & (0.032) & 0.074 \\
                \hspace{0.2cm} Hourly, Night Rate & 0.524 & (0.018) &  & 0.560 & (0.010) &  & 0.035 & (0.021) & 0.088 \\
                \hspace{0.2cm} Hourly, Day Rate & 1.128 & (0.034) &  & 1.193 & (0.019) &  & 0.065 & (0.039) & 0.095 \\
                \hspace{0.2cm} Hourly, Peak Rate & 1.537 & (0.053) &  & 1.642 & (0.029) &  & 0.105 & (0.060) & 0.080 \\
                \underline{Demographics} &  &  &  &  &  &  &  &  &  \\
                \hspace{0.2cm} Age Group: 65+? & 0.277 & (0.028) &  & 0.225 & (0.014) &  & -0.052 & (0.031) & 0.096 \\
                \hspace{0.2cm} Education: Primary or less? & 0.208 & (0.025) &  & 0.144 & (0.012) &  & -0.064 & (0.028) & 0.022 \\
                \hspace{0.2cm} Education: Secondary? & 0.462 & (0.031) &  & 0.457 & (0.017) &  & -0.005 & (0.035) & 0.889 \\
                \hspace{0.2cm} Unemployed? & 0.081 & (0.017) &  & 0.101 & (0.010) &  & 0.020 & (0.020) & 0.304 \\
                \hspace{0.2cm} Number of People over 15 in Home & 2.488 & (0.061) &  & 2.506 & (0.032) &  & 0.019 & (0.077) & 0.808 \\
                \hspace{0.2cm} Number of People under 15 in Home & 1.754 & (0.060) &  & 1.964 & (0.035) &  & 0.210 & (0.138) & 0.132 \\
                \underline{Housing Characteristics} &  &  &  &  &  &  &  &  &  \\
                \hspace{0.2cm} Owned House? & 0.904 & (0.018) &  & 0.932 & (0.008) &  & 0.028 & (0.020) & 0.165 \\
                \hspace{0.2cm} Number of Bedrooms & 3.335 & (0.054) &  & 3.465 & (0.028) &  & 0.130 & (0.061) & 0.035 \\
                \hspace{0.2cm} Timer for Space Heating & 0.792 & (0.025) &  & 0.802 & (0.013) &  & 0.010 & (0.028) & 0.728 \\
                \hline \hline
            \end{tabular}
        \end{adjustbox}
        \begin{minipage}{15.5cm}
            \small \singlespacing
            \textit{Note}: In the table, variable descriptions with question mark suggest that these variables are binary.
        \end{minipage}
    \end{table}
}

