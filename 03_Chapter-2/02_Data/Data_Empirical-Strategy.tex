Figure \ref{Figure:Pre-and-Post-Treatment-Household-Average-Daily-Electricity-Consumption}, showing not only household average daily electricity consumption over temperature (in Panel A) but also percentage changes in electricity consumption (in Panel B), clearly demonstrates the motivation of this research.\footnote{An important feature also stands out from the figure: the minimum household electricity consumption occurred at around 60$^{\circ}F$. This phenomenon supports the setting of the reference temperature for calculating daily HDDs at the very level.} As illustrated in Panel A of the figure, household demand for electricity grew gradually as the temperature decreased. That is, for Irish households, in addition to temperature-insensitive electricity demand (i.e., for non-temperature-control uses), there was a sizeable electricity demand for heating (i.e., for temperature-control uses), which seems to be highly responsive to temperature variations. In this research, I determine not only how much variations in household electricity consumption occurred, on average, in response to the deployment of the TOU tariffs but also how their impact varied according to daily HDDs. In other words, the dynamic-pricing-causing effects on for-heating and non-for-heating electricity uses are separately estimated to figure out the primary source of electricity savings. As shown in the figure, households in the control group consumed less electricity during the treatment period, especially on days with low temperatures, although their percentage reductions seem less than those of the treated households.\footnote{In Panel A, non-treated households consumed more electricity during the baseline period, especially on days with higher heating needs. The fact that for a given temperature bin, the total daily HDDs during the baseline period were generally greater than those during the treatment period is a plausible explanation for the phenomenon.} In light of this, it is necessary to employ an identification strategy that accounts for the before and after differences in household electricity consumption under the traditional tariff structure (i.e., a flat rate of 14.1 cents per kWh for all hours).

I employ a Difference-In-Differences (DID) approach to estimate the electricity savings caused by the TOU price program. The CER experiment dataset primarily utilized in the following empirical analysis was generated from a carefully developed Randomized Controlled Trial (RCT). So, in principle, the effect of the TOU tariffs on household electricity consumption can be measured simply through the difference in average usage between the two groups during the treatment period.\footnote{Because random assignment of participating households puts selection bias right, observed differences in electricity consumption between the control and treatment groups after introducing the TOU tariffs are only attributable to their differences in exposure to the time-varying electricity prices.} However, as discussed, there exist non-trivial differences in electricity demand between the control and treatment groups during the baseline period. Following the previous studies exploiting the same data, I utilize a DID estimator to address the possible source of bias. 

I include daily HDDs as an explanatory variable directly in my econometric models. In the previous papers using the identical dataset, Fixed-Effects (FEs) were utilized to control for time-varying factors influencing household electricity consumption. Since those studies focused on quantifying how households responded, on average, to the TOU price regimes newly introduced, adding such FEs to their models served their research purpose. In other words, they did not need to explicitly model the relationship between temperature and household electricity consumption to estimate the Average Treatment Effects (ATEs). However, a primary interest of this research is to understand how electricity savings vary with the temperature after shifting to TOU prices. Therefore, more direct controls rather than FEs, not sweeping out temperature variations across days, are required in my empirical analysis. For that reason, I extend a typical panel DID specification and allow the treatment effect to vary as a function of daily HDDs.\footnote{Under three identifying assumptions, applying a DID strategy to measure electricity savings obtained from adopting the TOU prices makes sense. First, the parallel trend assumption is required for the DID estimator. Considering that the 30-minute interval meter reads for participating households were collected during the trial, the assumption implies that the pre-treatment-period load profile for the treated households should be very similar to that for the non-treated households. Figure \ref{Figure:Average-Hourly-Electricity-Consumption-by-Time-of-Day}, showing average within-day load profiles for the two groups during the baseline period, supports the plausibility of the parallel trend assumption. In addition, the electricity consumption profile for the control group illustrated in Figure \ref{Figure:Average-Daily-Electricity-Consumption}, which smoothly evolved over the entire experiment period although heavily fluctuated daily, suggests its high reliability as a counterfactual under the assumption.
The assumption of common temporal shocks is the second identifying assumption necessary for the plausibility of the identification strategy employed. This assumption implies that a treatment-status-irrelevant unexpected event occurring at the same time as or following the deployment of the dynamic prices should have the same impact on both the control and treatment groups. Although the common shocks assumption cannot be tested directly, the similar trends in electricity demand profiles for the control and treatment groups shown in Figure \ref{Figure:Average-Daily-Electricity-Consumption} support the assumption required for the DID approach.
Third, the stable unit treatment value assumption (SUTVA) must hold too. The SUTVA requires that introducing the TOU prices did not affect the electricity consumption of the untreated households. That is, the SUTVA allows no spillovers. During the recruitment process, the locational distribution of the participating households was aligned with that of the total Irish population to construct a representative sample of the national population. Because only a few thousand households scattered geospatially participated in the nationwide experiment, it is unlikely that the treated households influenced the households allocated to the control group. This again supports the SUTVA required under the DID identification strategy.} That is, I estimate the ATEs of the dynamic prices on household electricity demand by exploiting the within-household electricity consumption changes across not only rate periods but temperatures.\footnote{The attrition rate during the RCT was about 20\%. The main reasons for participant attrition were changes in tenancy and supplier. Due to such imperfect compliance, the estimates must be interpreted as local average treatment effects (LATEs). However, according to \cite{Electricity-Smart-Metering-Customer-Behaviour-Trials-Findings-Report_CER_2011}, attritions were unlikely to be associated with the RCT. Furthermore, the level of attritions varied only marginally across treatment status.}

A caveat to my empirical analysis is that a tariff group in my sample's treatment group consists of four subgroups that were subject to one of the four different DSM stimuli. Because of this, part of the estimated ATEs should be attributable to the DSM stimuli. But as shown in Table \ref{Table:Treatment-and-Control-Group-Assignments}, the proportions of the four distinct DSM stimuli, constituting each tariff group, are similar in my sample. Therefore, within a tariff group, a specific DSM stimulus is unlikely to play a prominent role in causing changes in household electricity consumption. 
