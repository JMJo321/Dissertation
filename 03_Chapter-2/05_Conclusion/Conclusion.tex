The primary aim of various types of time-varying electricity pricing is to reshape load curves, especially around the peak-demand hours. Under the dynamic pricing of electricity, prices---more precisely, price variations---, which reflect instantaneous generation costs, are utilized to incentivize consumers to change their consumption behavior. Therefore, their responsiveness to the price changes in the tariff structures determines whether the time-varying electricity prices, including TOU pricing, will work as intended. In this paper, I quantify how sensitively households adjust their electricity consumption in response to TOU prices in and near the peak rate period. The results from my empirical analysis reveal two interesting points: household electricity consumption, consisting of two categories of electricity use---non-temperature-control-driven and temperature-control-driven consumption---, 1) sensitively responded to the magnitude of the price change in the peak rate period, and 2) also depended on daily heating degree days as well as the point electricity was consumed in time for a given rate change. In other words, my empirical analysis discloses the multidimensional dynamics of households' responses to the TOU tariffs. 

Those findings provide important policy implications for TOU electricity pricing. First, along with residential consumers' high price sensitivity, the nonlinearity in their responses to daily heating needs proposes an alternative pricing scheme: TOU pricing with additional flexibility induced by synchronizing the magnitude of the peak-demand-hour price jump with daily heating degree days. Second, taking a close look at the relationship between the size of the peak-hour price increase and the changes in electricity consumption for temperature-control uses in chronological order emphasizes the importance of adopting home automation technologies, like Programmable Communicating Thermostats (PCTs), to improve the performance of TOU pricing. 

My empirical findings and the policy implications derived from them ultimately indicate that an integrated understanding of the multidimensional dynamics of households' responses to TOU electricity pricing is required to make the price structure function with its full potential as a demand management tool. Furthermore, even for stakeholders in the electricity market, such as power generators, investors, regulators, and policymakers, comprehending how electricity consumption reacts to the time-varying pricing is critical because consumers' behavioral changes are an important piece of information in their decision makings.
