My empirical results obviously illustrate that the effectiveness of TOU tariffs, as measured by the amount of an induced reduction in household electricity consumption, nonlinearly varies with daily HDDs. As discussed, the alteration in electricity consumption caused by the deployment of TOU electricity pricing consists of two elements: the change in non-temperature-control-driven electricity consumption and that in temperature-control-driven electricity consumption. By definition, the change originating from non-temperature-control-related electricity consumption is independent of ever-changing weather conditions, including daily HDDs. Hence, the nonlinearity in the effectiveness of the TOU tariff structures, as illustrated in Figure \ref{Figure:Treatment-Effects-as-a-Linear-Function-of-Price-Changes-in-the-Peak-Rate-Period}, is utterly attributable to the other type of electricity consumption, that for heating. 

The nonlinear relationship between the amount of change in temperature-control-associated electricity consumption and daily HDDs indicates an interesting characteristic of TOU pricing: the day-varying effect of TOU pricing on residential electricity consumption. Daily HDDs, one of the critical determinants of temperature-control-relevant electricity consumption, fluctuate day by day. Therefore, it is intuitive that in response to daily changing household heating needs, the TOU-price-induced change in electricity consumption for heating also alters every day. 

The day-varying effectiveness of TOU electricity pricing suggests a significant implication in connection with Real-Time Pricing (RTP), a more granular time-varying electricity tariff structure.\footnote{\cite{Household-Responses-to-Time-Varying-Electricity-Prices_Harding-and-Sexton_2017} provides a detailed description of various kinds of time-varying electricity tariff structures.} Contrary to TOU pricing, rates typically change hourly under RTP. So compared to TOU pricing, RTP has an advantage in reflecting generation costs contemporaneously. In other words, RTP imposes a higher price in the situation that electricity demand is high, followed by high generation costs, to curb household electricity consumption. Economists, therefore, often advocate RTP over TOU pricing. 

Because of the reduction in temperature-control-driven electricity consumption that covaries with daily HDDs, TOU electricity pricing can somewhat emulate the favorable feature of RTP on relatively warm winter days in Ireland---roughly speaking, on days when the value of daily HDDs is below ten. As evidently illustrated in Figure \ref{Figure:Average-Daily-Electricity-Consumption}, households' heating needs drive the demand for electricity in Irish households. So, a more significant diminution in household electricity consumption is required on cold winter days to relieve the burden on the power grid. According to Figure \ref{Figure:Treatment-Effects-as-a-Linear-Function-of-Price-Changes-in-the-Peak-Rate-Period}, for example, for the households in Tariff Group A, the reduction in heating-associated electricity consumption in the peak rate period on warm winter days (i.e., on days when the value of daily HDDs fell between zero and ten), whose amount was more than half of the aggregated reduction in household electricity consumption under the TOU program at its maximum, expanded as households' heating needs became larger. This empirical finding means that comparing two warm winter days in Ireland, which have different values of daily HDDs, despite no across-day price variation under the price scheme, TOU electricity pricing induces a larger reduction in household electricity consumption during peak hours on the day with higher HDDs (i.e., on the day demonstrating higher generation costs due to more significant electricity demand). Consequently, in that case, the additional gains obtained by switching to RTP might not be as substantial as economists have expected. The excellent feature of TOU electricity pricing, however, gradually disappeared as daily HDDs grew above the value of ten, even though a more considerable reduction in household electricity consumption is required to ease the burden on the power grid. 
