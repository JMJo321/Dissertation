Figure \ref{Figure:Pre-and-Post-Treatment-Household-Average-Daily-Electricity-Consumption}, showing not only household average daily electricity consumption over temperature (in Panel A) but also percentage changes in electricity consumption (in Panel B), clearly demonstrates the motivation of this research.\footnote{An important feature also stands out from the figure: the minimum household electricity consumption occurred at around 60$^{\circ}F$. This phenomenon supports the setting of the reference temperature for calculating daily HDDs at the very level.} As illustrated in Panel A of the figure, household demand for electricity grew gradually as the temperature decreased. That is, for Irish households, in addition to temperature-insensitive electricity demand (i.e., for non-temperature-control uses), there was a sizeable electricity demand for heating (i.e., for temperature-control uses), which seems to be highly responsive to temperature variations. In this research, I determine not only how much variations in household electricity consumption occurred, on average, in response to the deployment of the TOU tariffs but also how their impact varied according to daily HDDs. In other words, the dynamic-pricing-causing effects on for-heating and non-for-heating electricity uses are separately estimated to figure out the primary source of electricity savings. As shown in the figure, households in the control group consumed less electricity during the treatment period, especially on days with low temperatures, although their percentage reductions seem less than those of the treated households.\footnote{In Panel A, non-treated households consumed more electricity during the baseline period, especially on days with higher heating needs. The fact that for a given temperature bin, the total daily HDDs during the baseline period were generally greater than those during the treatment period is a plausible explanation for the phenomenon.} In light of this, it is necessary to employ an identification strategy that accounts for the before and after differences in household electricity consumption under the traditional tariff structure (i.e., a flat rate of 14.1 cents per kWh for all hours).
\afterpage{
    \begin{figure}[t!]
        \centering
        \includegraphics[scale = 0.048]{03_Chapter-2/00A_Figures/Figure_For-Motivation_Daily-Consumption-with-Percentage-Changes_Base-only-the-Second-Half_Step-Size-2.png}
        \caption{Pre- and Post-Treatment Household Average Daily Electricity Consumption}
        \caption*{
            {\small
            \textit{Note}: Panel A in this figure illustrates, for each group, how within-household average daily electricity consumption evolved over average daily temperatures during each experiment period. In addition, Panel B of the figure demonstrates the percentage changes in residential electricity consumption after the deployment of TOU tariff structures at different mean daily temperatures. The treatment group showed larger percentage reductions on typical winter days (roughly speaking, when the average daily temperature was lower than the value of 45$^{\circ}F$), while the control group exhibited wider percentage reductions on exceptionally cold days in Ireland.
        }}
        \label{Figure:Pre-and-Post-Treatment-Household-Average-Daily-Electricity-Consumption}
    \end{figure}
}

I employ a Difference-In-Differences (DID) approach to estimate the electricity savings caused by the TOU price program. The CER experiment dataset primarily utilized in the following empirical analysis was generated from a carefully developed Randomized Controlled Trial (RCT). So, in principle, the effect of the TOU tariffs on household electricity consumption can be measured simply through the difference in average usage between the two groups during the treatment period.\footnote{Because random assignment of participating households puts selection bias right, observed differences in electricity consumption between the control and treatment groups after introducing the TOU tariffs are only attributable to their differences in exposure to the time-varying electricity prices.} However, as discussed, there exist non-trivial differences in electricity demand between the control and treatment groups during the baseline period. Following the previous studies exploiting the same data, I utilize a DID estimator to address the possible source of bias. 

I include daily HDDs as an explanatory variable directly in my econometric models. In the previous papers using the identical dataset, Fixed-Effects (FEs) were utilized to control for time-varying factors influencing household electricity consumption. Since those studies focused on quantifying how households responded, on average, to the TOU price regimes newly introduced, adding such FEs to their models served their research purpose. In other words, they did not need to explicitly model the relationship between temperature and household electricity consumption to estimate the Average Treatment Effects (ATEs). However, a primary interest of this research is to understand how electricity savings vary with the temperature after shifting to TOU prices. Therefore, more direct controls rather than FEs, not sweeping out temperature variations across days, are required in my empirical analysis. For that reason, I extend a typical panel DID specification and allow the treatment effect to vary as a function of daily HDDs.

A caveat to my empirical analysis is that a tariff group in my sample's treatment group consists of four subgroups that were subject to one of the four different DSM stimuli. Because of this, part of the estimated ATEs should be attributable to the DSM stimuli. But as shown in Table \ref{Table:Treatment-and-Control-Group-Assignments}, the proportions of the four distinct DSM stimuli, constituting each tariff group, are similar in my sample. Therefore, within a tariff group, a specific DSM stimulus is unlikely to play a prominent role in causing changes in household electricity consumption. 
