% Figures
\afterpage{
    \begin{figure}[t!]
        \centering
        \includegraphics[scale = 0.155]{03_Chapter-2/00A_Figures/Figure_Time-of-Use-Tariff-Structures}
        \caption{Time-Of-Use Pricing Structures}
        \subcaption*{\textit{Note}: This figure illustrates the CER experiment in terms of TOU tariff structures. The households in the control group were subjected to a flat rate (i.e., 14.1 cents per $kWh$) during the entire experiment period. On the contrary, the treated households are assigned to one of four TOU tariff groups. And for each tariff group, there were three rate periods: night, day, and peak. Only the unit rate in the peak rate period was higher than the flat rate.}
        \label{Figure:Time-Of-Use-Pricing-Structures}
    \end{figure}
}
\clearpage

\afterpage{
    \begin{figure}[t!]
        \centering
        \includegraphics[scale = 0.125]{03_Chapter-2/00A_Figures/Figure_Average-Electricity-Consumption_Within-Day-Average-Hourly-Consumption.png}
        \caption{Average Hourly Electricity Consumption by Time of Day}
        \subcaption*{\textit{Note}: The figure shows, during each experiment period, household average hourly electricity consumption for the control and treatment groups, respectively. In general, during the baseline period, households assigned to the treatment group consumed more electricity at a given hour of the day. Although both groups reduced their electricity consumption during the treatment period, the reduction in electricity consumption for the treatment group was much more remarkable for the treatment group than for the control group.}
        \label{Figure:Average-Hourly-Electricity-Consumption-by-Time-of-Day}
    \end{figure}
}
\clearpage

\afterpage{
    \begin{figure}[t!]
        \centering
        \includegraphics[scale = 0.19]{03_Chapter-2/00A_Figures/Figure_Distribution-of-HDDs.png}
        \caption{Distribution of Heating Degree Days during Experiment Periods}
        \subcaption*{\textit{Note}: This histogram shows the distribution of HDDs, with the mean and median values, in each experiment period. Only the second halves of 2009 and 2010 are utilized to generate the histogram. }
        \label{Figure:Distribution-of-Heating-Degree-Days-during-Experiment-Periods}
    \end{figure}
}
\clearpage

\afterpage{
    \begin{figure}[t!]
        \centering
        \includegraphics[scale = 0.125]{03_Chapter-2/00A_Figures/Figure_Average-Electricity-Consumption_Average-Daily-Consumption-by-Date.png}
        \caption{Average Daily Electricity Consumption}
        \subcaption*{\textit{Note}: The figure depicts, for households that exploit non-electric energy sources for their space and water heating, not only the average daily electricity consumption with 95\% confidence intervals for each group (red and yellow lines) but also the mean daily temperature (blue line). From this figure, it is apparent that household daily electricity consumption is negatively correlated with the average daily temperature. In other words, in Ireland, outdoor temperatures are a crucial driver of within-household electricity consumption.}
        \label{Figure:Average-Daily-Electricity-Consumption}
    \end{figure}
}
\clearpage

\afterpage{
    \begin{figure}[t!]
        \centering
        \includegraphics[scale = 0.052]{03_Chapter-2/00A_Figures/Figure_For-Motivation_Daily-Consumption-with-Percentage-Changes_Base-only-the-Second-Half_Step-Size-2.png}
        \caption{Pre- and Post-Treatment Household Average Daily Electricity Consumption}
        \subcaption*{\textit{Note}: Panel A in this figure illustrates, for each group, how within-household average daily electricity consumption evolved over average daily temperatures during each experiment period. In addition, Panel B of the figure demonstrates the percentage changes in residential electricity consumption after the deployment of TOU tariff structures at different mean daily temperatures. The treatment group showed larger percentage reductions on typical winter days (roughly speaking, when the average daily temperature was lower than the value of 45$^{\circ}F$), while the control group exhibited wider percentage reductions on exceptionally cold days in Ireland.}
        \label{Figure:Pre-and-Post-Treatment-Household-Average-Daily-Electricity-Consumption}
    \end{figure}
}
\clearpage

\afterpage{
    \begin{figure}
        \centering
        \includegraphics[scale = 0.105]{03_Chapter-2/00A_Figures/Figure_Time-Profile-of-Half-Hourly-ATEs.png}
        \caption{Half-hourly Average Treatment Effects}
        \subcaption*{\textit{Note}: This figure depicts the time profile of half-hourly average treatment effects with 95\% confidence intervals. Standard errors are clustered at the household and date levels to adjust for serial correlation. As clearly illustrated, the treated households significantly reduced their electricity consumption during peak hours. A more interesting phenomenon is that they reduced their electricity consumption in hours leading up to and following the peak rate period, during which the applicable unit rate was lower than the flat rate in the baseline period, even though most of the estimated treatment effects are statistically insignificant in those hours. }
        \label{Figure:Half-Hourly-Average-Treatment-Effects}
    \end{figure}
}
\clearpage

\afterpage{
    \begin{figure}[t!]
        \centering
        \includegraphics[scale = 0.113]{03_Chapter-2/00A_Figures/Figure_Breakdown-of-Hourly-ATEs_For-Different-Intervals_All_Knot-10.png}
        \caption{Breakdown of Hourly Average Treatment Effects}
        \subcaption*{\textit{Note}: This figure is a graphical summary of the regression results in the first three columns of Table \ref{Table:Breakdown-of-Hourly-ATEs}. The order of panes corresponds to that of columns. As clearly illustrated, each two-hour interval shows distinct evolving patterns of two broad categories of household electricity consumption. The changes in non-temperature-control-driven household electricity consumption are straight lines because they are independent of outdoor temperature variations. On the other hand, the changes in temperature-control-associated residential electricity consumption are a nonlinear function of daily HDDs.}
        \label{Figure:Breakdown-of-Hourly-ATEs-in-the-Peak-Rate-Period}
    \end{figure}
}
\clearpage

\afterpage{
    \begin{figure}[t!]
        \centering
        \includegraphics[scale = 0.105]{03_Chapter-2/00A_Figures/Figure_Predicted-Electricity-Savings_By-Intervals-and-HDDs_Spline_Knot-10.png}
        \caption{Treatment Effects as a Linear Function of Peak-hour Price Changes}
        \subcaption*{\textit{Note}: This figure depicts, for four different price changes in the peak rate period, estimated treatment effects as a linear function of price changes. The first row in the figure shows the treatment effects on non-temperature-control-driven household electricity consumption. The treatment effects on temperature-control-related residential electricity consumption are illustrated in the second row. The aggregate effects are presented in the last row. The first three columns correspond to the three two-hour periods (i.e., pre-peak, peak, and post-peak periods). The fourth column demonstrates the total changes in the three periods. }
        \label{Figure:Treatment-Effects-as-a-Linear-Function-of-Price-Changes-in-the-Peak-Rate-Period}
    \end{figure}
}
\clearpage

\afterpage{
    \begin{figure}[t!]
        \centering
        \includegraphics[scale = 0.135]{03_Chapter-2/00A_Figures/Figure_Time-Profile-of-ATEs_Using-Tariff-Groups-A-and-D_Knot-10.png}
        \caption{Relative Comparison of Tariff Group D to Tariff Group A}
        \subcaption*{\textit{Note}: This figure shows time profiles of hourly treatment effects, with 95\% confidence intervals, measured by setting Tariff Groups A and D as the control and treatment groups. Each row in the figure corresponds to one of the three indicator variables of interest. The time profiles suggest only limited evidence of load-shifting in and near the peak rate period. }
        \label{Figure:Relative-Comparison-of-Tariff-Group-D-to-Tariff-Group-A}
    \end{figure}
}
\clearpage

\afterpage{
    \begin{figure}[t!]
        \centering
        \includegraphics[scale = 0.135]{03_Chapter-2/00A_Figures/Figure_Additional-Electricity-Savings_Knot-10.png}
        \caption{Additional Gains from an Alternative Electricity Pricing Scheme}
        \subcaption*{\textit{Note}: This figure illustrates two different price schemes. Under a typical TOU electricity pricing scheme, the rate in the peak rate period is 6 cents per $kWh$ regardless of daily HDDs. On the contrary, under an alternative tariff structure that is TOU-style but has extra flexibility across daily HDDs, the peak-hour price escalates as household heating needs grow. The shaded areas depict additional gains obtained by adopting the redesigned pricing scheme, which are mainly attributable to more significant reductions in non-temperature-control-driven household electricity consumption. }
        \label{Figure:Additional-Savings-from-an-Alternative-Electricity-Pricing-Scheme}
    \end{figure}
}
\clearpage


% Tables
\afterpage{
    \begin{table}
        \centering
        \caption{Treatment and Control Group Assignments}
        \label{Table:Treatment-and-Control-Group-Assignments}
        \vspace{0.2cm}
        \begin{adjustbox}{scale = 1.0}
            \begin{tabular}{
                >{\raggedright}m{4.5cm} |
                >{\raggedleft}m{1.5cm} 
                >{\raggedleft}m{1.5cm} 
                >{\raggedleft}m{1.5cm} 
                >{\raggedleft}m{1.5cm} 
                >{\raggedleft}m{1.5cm} ||
                >{\raggedleft\arraybackslash}m{2.1cm}
            }
                \hline \hline
                \multicolumn{1}{c|}{Stimuli} & \multicolumn{5}{c||}{Tariffs} & \multicolumn{1}{c}{Total} \\
                \cline{2-6}
                \multicolumn{1}{c|}{}  & \multicolumn{1}{c}{Control} & \multicolumn{1}{c}{A} & \multicolumn{1}{c}{B} & \multicolumn{1}{c}{C} & \multicolumn{1}{c||}{D} & \multicolumn{1}{c}{} \\
                \hline
                Monthly Bill & 0 & 79 & 37 & 89 & 28 & 233 \\
                Bi-Monthly Bill & 0 & 81 & 34 & 76 & 34 & 225 \\
                Bi-Monthly Bill $+$ IHD & 0 & 79 & 22 & 86 & 30 & 217 \\
                Bi-Monthly Bill $+$ OLR & 0 & 90 & 27 & 84 & 34 & 235 \\
                Control & 260 & 0 & 0 & 0 & 0 & 260 \\
                \hline
                Total & 260 & 329 & 120 & 335 & 126 & 1,170 \\
                \hline \hline
            \end{tabular}
        \end{adjustbox}
%        \begin{tablenotes}[flushleft]
%            \small
%            \textit{Note}: (The purpose of this sentence is to determine the width of the table. Therefore, this note should be replaced or deleted after polishing the table size.)
%        \end{tablenotes}

    \end{table}
}
\clearpage

\afterpage{
    \begin{table}
        \centering
        \caption{Summary Statistics and Differences in Means}
        \label{Table:Summary-Statistics-and-Differences-in-Means}
        \vspace{0.2cm}
        \begin{adjustbox}{scale = 0.9}
            \begin{tabular}{
                >{\raggedright}m{6.0cm}
                >{\raggedleft}m{1.25cm}
                >{\raggedleft}m{1.25cm}
                >{\raggedleft}m{0.1cm}
                >{\raggedleft}m{1.25cm}
                >{\raggedleft}m{1.25cm}
                >{\raggedleft}m{0.1cm}
                >{\raggedleft}m{1.25cm}
                >{\raggedleft}m{1.25cm}
                >{\raggedleft\arraybackslash}m{1.25cm}
            }
                \hline \hline
                \multicolumn{1}{c}{} & \multicolumn{2}{c}{Control} & \multicolumn{1}{c}{} & \multicolumn{2}{c}{Treatment} & \multicolumn{1}{c}{} & \multicolumn{3}{c}{Difference} \\
                \cline{2-3} \cline{5-6} \cline{8-10}
                \multicolumn{1}{c}{} & \multicolumn{1}{c}{Mean} & \multicolumn{1}{c}{(S.E.)} & \multicolumn{1}{c}{} & \multicolumn{1}{c}{Mean} & \multicolumn{1}{c}{(S.E.)} & \multicolumn{1}{c}{} & \multicolumn{1}{c}{Mean} & \multicolumn{1}{c}{(S.E.)} & \multicolumn{1}{c}{$p$-value} \\
                \hline
                \underline{Electricity Consumption during Baseline Period \ ($kWh$)} &  &  &  &  &  &  &  &  &  \\
                \hspace{0.2cm} Daily & 22.122 & (0.674) &  & 23.529 & (0.379) &  & 1.407 & (0.773) & 0.069 \\
                \hspace{0.2cm} Hourly & 0.939 & (0.028) &  & 0.996 & (0.016) &  & 0.057 & (0.032) & 0.074 \\
                \hspace{0.2cm} Hourly, Night Rate & 0.524 & (0.018) &  & 0.560 & (0.010) &  & 0.035 & (0.021) & 0.088 \\
                \hspace{0.2cm} Hourly, Day Rate & 1.128 & (0.034) &  & 1.193 & (0.019) &  & 0.065 & (0.039) & 0.095 \\
                \hspace{0.2cm} Hourly, Peak Rate & 1.537 & (0.053) &  & 1.642 & (0.029) &  & 0.105 & (0.060) & 0.080 \\
                \underline{Demographics} &  &  &  &  &  &  &  &  &  \\
                \hspace{0.2cm} Age Group: 65+? & 0.277 & (0.028) &  & 0.225 & (0.014) &  & -0.052 & (0.031) & 0.096 \\
                \hspace{0.2cm} Education: Primary or less? & 0.208 & (0.025) &  & 0.144 & (0.012) &  & -0.064 & (0.028) & 0.022 \\
                \hspace{0.2cm} Education: Secondary? & 0.462 & (0.031) &  & 0.457 & (0.017) &  & -0.005 & (0.035) & 0.889 \\
                \hspace{0.2cm} Unemployed? & 0.081 & (0.017) &  & 0.101 & (0.010) &  & 0.020 & (0.020) & 0.304 \\
                \hspace{0.2cm} Number of People over 15 in Home & 2.488 & (0.061) &  & 2.506 & (0.032) &  & 0.019 & (0.077) & 0.808 \\
                \hspace{0.2cm} Number of People under 15 in Home & 1.754 & (0.060) &  & 1.964 & (0.035) &  & 0.210 & (0.138) & 0.132 \\
                \underline{Housing Characteristics} &  &  &  &  &  &  &  &  &  \\
                \hspace{0.2cm} Owned House? & 0.904 & (0.018) &  & 0.932 & (0.008) &  & 0.028 & (0.020) & 0.165 \\
                \hspace{0.2cm} Number of Bedrooms & 3.335 & (0.054) &  & 3.465 & (0.028) &  & 0.130 & (0.061) & 0.035 \\
                \hspace{0.2cm} Timer for Space Heating & 0.792 & (0.025) &  & 0.802 & (0.013) &  & 0.010 & (0.028) & 0.728 \\
                \hline \hline
            \end{tabular}
        \end{adjustbox}
        \begin{tablenotes}[flushleft]
            \small
            \textit{Note}: Variable descriptions with question marks suggest that these variables are binary. 
        \end{tablenotes}

    \end{table}
}

\clearpage

\afterpage{
    \begin{table}[t!]
        \centering
        \caption{Correlations in Average Daily Temperatures between Weather Stations}
        \label{Table:Correlations-in-Average-Daily-Temperatures-among-Weather-Stations}
        \vspace{0.2cm}
        \begin{adjustbox}{scale = 1.0}
            \begin{tabular}{
                >{\raggedright}m{4.8cm} |
                >{\centering}m{5.5cm} 
                >{\centering\arraybackslash}m{5.5cm}
            }
                \hline \hline
                \multicolumn{1}{c|}{Stations} & \multicolumn{2}{c}{Correlation Coefficients} \\
                \cline{2-3}
                \multicolumn{1}{c|}{}  & \multicolumn{1}{c}{For Sample Period} & \multicolumn{1}{c}{For Experiment Period} \\
                \hline
                Ballyhaise & 0.98291 & 0.98244 \\
                Belmullet & 0.96089 & 0.96361 \\
                Cork Airport & 0.97121 & 0.97130 \\
                Gurteen & 0.98389 & 0.98307 \\
                Johnstown & 0.98189 & 0.97958 \\
                Mace & 0.95870 & 0.95921 \\
                Malin & 0.95632 & 0.95705 \\
                Markree Castle & 0.97194 & 0.97179 \\
                Moore Park & 0.98057 & 0.97798 \\
                Mount Dillon & 0.97945 & 0.97782 \\
                Mullingar & 0.98876 & 0.98654 \\
                Newport Furnace & 0.97015 & 0.97211 \\
                Oak Park & 0.99074 & 0.98925 \\
                Shannon Airport & 0.97696 & 0.97582 \\
                Sherkin Island & 0.95342 & 0.95411 \\
                \hline \hline
            \end{tabular}
        \end{adjustbox}
        \begin{tablenotes}[flushleft]
            \small
            \textit{Note}: For each weather station, historical weather data from the weather station at Dublin airport is utilized to compute the two correlation coefficients. I do not provide the $p$-value of each correlation coefficient because it is arbitrarly small in magnitude. And the experiment period is the period between July 2009 to December 2010, while the sample period is the second half of 2009 and 2010. 
        \end{tablenotes}
    \end{table}
}

\clearpage

\afterpage{
    \begin{sidewaystable}[t!]
        \centering
        \caption{Hourly Average Treatment Effects in and near the Peak Rate Period}
        \label{Table:Hourly-Average-Treatment-Effects-in-and-near-the-Peak-Rate-Period}
        \small
        \begin{adjustbox}{scale = 0.65}
            \begin{threeparttable}
                \begin{tabular}{@{\extracolsep{1pt}}lccccccccccccccc}
                    \\[-5.5ex]
                    \hline \hline
                    \\[-3.0ex]
%                    & \multicolumn{15}{c}{Dependent Variable} \\
%                    \\[-3.0ex]
%                    \cline{2-16}
%                    \\[-3.0ex]
                    & \multicolumn{15}{c}{Hourly Electricity Consumption  (kWh/Hour)} \\
                    \\[-3.0ex]
                    & (1) & (2) & (3) & (4) & (5) & (6) & (7) & (8) & (9) & (10) & (11) & (12) & (13) & (14) & (15)\\
                    \\[-3.0ex]
                    \hline
                    \\[-2.0ex]
                    $\mathbb{1}$[Treatment \& Post] & $-$0.048$^{***}$ & $-$0.053$^{*}$ & $-$0.002 & $-$0.049 & $-$0.032$^{***}$ & $-$0.125$^{***}$ & $-$0.161$^{***}$ & $-$0.119$^{***}$ & $-$0.249$^{***}$ & $-$0.143$^{***}$ & $-$0.082$^{***}$ & $-$0.055$^{*}$ & $-$0.015 & $-$0.113$^{**}$ & $-$0.058$^{***}$ \\
                    & (0.016) & (0.027) & (0.017) & (0.031) & (0.011) & (0.020) & (0.036) & (0.022) & (0.044) & (0.015) & (0.020) & (0.030) & (0.021) & (0.048) & (0.015) \\
                    & & & & & & & & & & & & & & & \\
                    \hline
                    \\[-2.0ex]
                    Description of Interval & Pre-Peak & Pre-Peak & Pre-Peak & Pre-Peak & Pre-Peak & Peak & Peak & Peak & Peak & Peak & Post-Peak & Post-Peak & Post-Peak & Post-Peak & Post-Peak \\
                    Interval of Hours & 15 to 16 & 15 to 16 & 15 to 16 & 15 to 16 & 15 to 16 & 17 to 18 & 17 to 18 & 17 to 18 & 17 to 18 & 17 to 18 & 19 to 20 & 19 to 20 & 19 to 20 & 19 to 20 & 19 to 20 \\
                    Tariff Group & A & B & C & D & All & A & B & C & D & All & A & B & C & D & All \\
                    FEs: Household by Half-Hourly Time Window & Yes & Yes & Yes & Yes & Yes & Yes & Yes & Yes & Yes & Yes & Yes & Yes & Yes & Yes & Yes \\
                    FEs: Day of Week by Half-Hourly Time Window & Yes & Yes & Yes & Yes & Yes & Yes & Yes & Yes & Yes & Yes & Yes & Yes & Yes & Yes & Yes \\
                    FEs: Month of Year & Yes & Yes & Yes & Yes & Yes & Yes & Yes & Yes & Yes & Yes & Yes & Yes & Yes & Yes & Yes \\
                    Observations & 506,540 & 326,800 & 511,700 & 331,960 & 1,006,200 & 506,540 & 326,800 & 511,700 & 331,960 & 1,006,200 & 506,540 & 326,800 & 511,700 & 331,960 & 1,006,200 \\
                    Adjusted R$^{2}$ & 0.312 & 0.330 & 0.320 & 0.327 & 0.308 & 0.384 & 0.397 & 0.383 & 0.367 & 0.379 & 0.371 & 0.389 & 0.376 & 0.361 & 0.372 \\
                    \\[-2.0ex]
                    \hline \hline
                    \\[-4.5ex]
                \end{tabular}
                \begin{tablenotes}[flushleft]
                    \footnotesize
                    \item \textit{Note}: * $p < 0.1$, ** $p < 0.05$, and *** $p < 0.01$.
                \end{tablenotes}
            \end{threeparttable}
        \end{adjustbox}
    \end{sidewaystable}
}
   
%\clearpage

\afterpage{
    \begin{table}[t!]
        \centering
        \caption{Breakdown of Hourly Average Treatment Effects}
        \label{Table:Breakdown-of-Hourly-ATEs}
        \vspace{0.3cm}
        \small
        \begin{adjustbox}{scale = 0.75}
            \begin{threeparttable}
                \begin{tabular}{@{\extracolsep{10pt}}lccccccc}
                    \\[-5.5ex]
                    \hline \hline
                    \\[-3.0ex]
%                    & \multicolumn{15}{c}{Dependent Variable} \\
%                    \\[-3.0ex]
%                    \cline{2-100}
%                    \\[-3.0ex]
                    & \multicolumn{7}{c}{Hourly Electricity Consumption  (kWh/Hour)} \\
                    \\[-3.0ex]
                    & (1) & (2) & (3) & (4) & (5) & (6) & (7) \\
                    \\[-3.0ex]
                    \hline
                    \\[-2.0ex]
                    $\mathbb{1}$[Treatment \& Post] & $-$0.017 & $-$0.092$^{***}$ & $-$0.022 & $-$0.057$^{*}$ & $-$0.127$^{***}$ & $-$0.078$^{**}$ & $-$0.189$^{***}$ \\
                    & (0.020) & (0.025) & (0.024) & (0.030) & (0.039) & (0.031) & (0.041) \\
                    & & & & & & & \\
                    $\mathbb{1}$[Treatment \& Post] $\times$ HDDs & 0.00005 & $-$0.007$^{**}$ & $-$0.002 & $-$0.010$^{***}$ & $-$0.002 & $-$0.004 & $-$0.009 \\
                    & (0.003) & (0.003) & (0.003) & (0.004) & (0.005) & (0.003) & (0.006) \\
                    & & & & & & & \\
                    $\mathbb{1}$[Treatment \& Post] $\times$ HDDs$^{*}$ & $-$0.002 & 0.008$^{***}$ & 0.006$^{*}$ & 0.013$^{***}$ & 0.003 & 0.005 & 0.011$^{*}$ \\
                    & (0.003) & (0.003) & (0.003) & (0.005) & (0.006) & (0.003) & (0.006) \\
                    & & & & & & & \\
                    \hline
                    \\[-2.0ex]
                    Description of Period & Pre-Peak & Peak & Post-Peak & Peak & Peak & Peak & Peak \\
                    Period of Hours & 15 to 16 & 17 to 18 & 19 to 20 & 17 to 18 & 17 to 18 & 17 to 18 & 17 to 18 \\
                    Tariff Group & All & All & All & A & B & C & D \\
                    Price Change in the Peak Rate Period & [-] & [-] & [-] & +6 & +12 & +18 & +24 \\
                    Knot & 10 & 10 & 10 & 10 & 10 & 10 & 10 \\
                    FEs: Day of Week by Half-Hourly Time Window & Yes & Yes & Yes & Yes & Yes & Yes & Yes \\
                    Observations & 1,006,200 & 1,006,200 & 1,006,200 & 506,540 & 326,800 & 511,700 & 331,960 \\
                    Adjusted R$^{2}$ & 0.024 & 0.047 & 0.040 & 0.046 & 0.044 & 0.044 & 0.045 \\
                    \\[-2.0ex]
                    \hline \hline
                    \\[-4.5ex]
                \end{tabular}
                \begin{tablenotes}[flushleft]
                    \footnotesize
                    \item \textit{Note}: * $p < 0.1$, ** $p < 0.05$, and *** $p < 0.01$.
                \end{tablenotes}
            \end{threeparttable}
        \end{adjustbox}
    \end{table}
}

\clearpage

\afterpage{
    \begin{table}[t!]
        \centering
        \caption{Hourly Treatment Effects as a Linear Function of Peak-rate-period Price Changes}
        \label{Table:Hourly-ATEs-as-a-Linear-Function-of-Peak-Rate-Period-Price-Changes_Coefficients-of-Interest-only}
        \vspace{0.3cm}
        \small
        \begin{adjustbox}{scale = 1.0}
            \begin{threeparttable}
                \begin{tabular}{@{\extracolsep{40pt}}lccc}
                    \\[-5.5ex]
                    \hline \hline
                    \\[-3.0ex]
%                    & \multicolumn{15}{c}{Dependent Variable} \\
%                    \\[-3.0ex]
%                    \cline{2-100}
%                    \\[-3.0ex]
                    & \multicolumn{3}{c}{Hourly Electricity Consumption  (kWh/Hour)} \\
                    \\[-3.0ex]
                    & (1) & (2) & (3) \\
                    \\[-3.0ex]
                    \hline
                    \\[-2.0ex]
                    $\mathbb{1}$[Treatment \& Post] & $-$0.045 & $-$0.028 & $-$0.053 \\
                    & (0.029) & (0.035) & (0.035) \\
                    & & & \\
                    $\mathbb{1}$[Treatment \& Post] $\times$ $\Delta$PC & 0.002 & $-$0.005$^{**}$ & 0.002 \\
                    & (0.002) & (0.002) & (0.002) \\
                    & & & \\
                    $\mathbb{1}$[Treatment \& Post] $\times$ HDDs & $-$0.0001 & $-$0.010$^{**}$ & $-$0.001 \\
                    & (0.004) & (0.004) & (0.004) \\
                    & & & \\
                    $\mathbb{1}$[Treatment \& Post] $\times$ HDDs$^{*}$ & 0.001 & 0.012$^{**}$ & 0.005 \\
                    & (0.005) & (0.006) & (0.005) \\
                    & & & \\
                    $\mathbb{1}$[Treatment \& Post] $\times$ HDDs $\times$ $\Delta$PC & 0.00001 & 0.0002 & $-$0.0001 \\
                    & (0.0002) & (0.0002) & (0.0003) \\
                    & & & \\
                    $\mathbb{1}$[Treatment \& Post] $\times$ HDDs$^{*}$ $\times$ $\Delta$PC & $-$0.0002 & $-$0.0003 & 0.00004 \\
                    & (0.0003) & (0.0003) & (0.0003) \\
                    & & & \\
                    \hline
                    \\[-2.0ex]
                    Description of Period & Pre-Peak & Peak & Post-Peak \\
                    Period of Hours & 15 to 16 & 17 to 18 & 19 to 20 \\
                    Knot & 10 & 10 & 10 \\
                    FEs: Day of Week by Half-Hourly Time Window & Yes & Yes & Yes \\
                    Observations & 1,006,200 & 1,006,200 & 1,006,200 \\
                    Adjusted R$^{2}$ & 0.024 & 0.047 & 0.040 \\
                    \\[-2.0ex]
                    \hline \hline
                    \\[-4.5ex]
                \end{tabular}
                \begin{tablenotes}[flushleft]
                    \footnotesize
                    \item
                    \item \textit{Note}: This table shows the results of the regression in Equation (\ref{Eq:Model-Specification_Breakdown-of-Hourly-Average-Treatment-Effect-as-a-Linear-Function-of-Price-Changes}). Only coefficients of interest are presented in the table. See Table \ref{Table:Hourly-ATEs-as-a-Linear-Function-of-Peak-Rate-Period-Price-Changes_For-Appendix} to look at the full results. Standard errors in parentheses are clustered at the household and day of experiment levels to correct for serial correlation; * $p < 0.1$, ** $p < 0.05$, and *** $p < 0.01$.
                \end{tablenotes}
            \end{threeparttable}
        \end{adjustbox}
    \end{table}
}

