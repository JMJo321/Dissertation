The U-shaped curve of temperature-control-use-associated electricity savings in the peak rate period is not a desirable feature of TOU pricing. The fundamental intention of the time-varying tariff scheme is to reshape load profiles, especially in the peak-demand period, to avoid excessive investment in power generation capacity. So a higher amount of savings in electricity consumption for heating on freezing days (i.e., on days in which the grid is most burdened) serves the purpose of the price scheme. In light of that, the U-shaped evolving pattern of the savings over daily HDDs is unattractive because on days with high heating needs, the price structure induces even less savings in for-heating-relevant household electricity consumption. 

An alternative electricity pricing scheme, a TOU-like tariff structure with additional flexibility in price variations across daily HDDs, could address the disadvantage of typical TOU pricing revealed from my analysis (i.e., fewer electricity savings on days with very low temperatures). My empirical findings illustrate two important relationships between TOU-tariff-inducing electricity savings and the price variations in the peak-demand hours. First, the savings from electricity consumption for non-temperature-control uses are directly proportional to the size of price increases during peaks. Second, raising the magnitude of price changes in the peak rate period somewhat inhibits heating-related electricity savings from disappearing even at a high level of daily HDDs. Those two points collectively imply that scaling up the size of rate changes in the peak rate period as daily HDDs escalate allows for achieving more considerable TOU-price-inducing savings in residential electricity consumption.  
 
Figure XYZ depicts the predicted electricity savings under the alternative pricing scheme. (...) 
