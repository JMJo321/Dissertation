Exploring participating households' electricity consumption, following a time sequence around the peak rate period, facilitates comprehending how they adapted to the deployment of TOU electricity pricing more completely. Intuitively, residential consumers can respond to a peak TOU price by conserving their electricity consumption during peaks, leading to an overall reduction in their demand for electricity. Instead of reducing their electricity consumption, they can shift it to off-peak hours so as not to be subject to the peak rate as much as possible. In this case, the level of their net electricity consumption is maintained. Of course, those two ways of responding to the time-varying price structure can co-occur. Because those two ways of reacting to the time-varying tariff scheme reshape load curves around the peak rate period, it is natural to examine the TOU-tariff-inducing electricity savings as a whole from a time-moving perspective in order to grasp the dynamics of households' behavioral changes. In the following paragraphs, interpretations of the changes in households' consumption behavior relevant to each of the two channels of electricity savings are followed by a policy implication suggested through them. 

Regarding residential electricity demand for non-temperature-control uses, the leading reaction of the treated households to the TOU tariffs was to reduce their heating-irrelevant consumption around the peak rate period. As discussed, to the magnitude of the peak-hour price changes under the TOU program, the not-for-heating electricity savings were directly proportional in the peak rate period while inversely proportional in the pre- and post-peak intervals. In the case of Tariff Group A, although there was almost zero price variation relative to the flat rate in the before- and after-peak intervals, the amount of electricity savings for that group was nearly the same in all three intervals. Meanwhile, despite the price decreases, the remaining tariff groups (maintained or) conserved their consumption in both intervals. In sum, the price changes in the peak rate period caused a spillover effect in those pre- and post-peak intervals: reductions in electricity consumption for non-temperature-control uses. In other words, with respect to non-temperature-control-related electricity consumption, the households allocated to the treatment group responded to the TOU program, on the whole, not through load-shifting but load-shedding. 

With respect to temperature-control-use-related household electricity consumption, Figure XYZ depicts that the treated households' primary response to the TOU program was also load-shedding. The program caused savings in for-heating electricity use during the peak rate period, especially around moderate values of daily HDDs. In the pre-peak interval, heating-associated electricity savings only occurred on days with low temperatures. In the post-peak interval, although high daily HDDs incurred additional electricity consumption after introducing TOU tariffs, which might be a consequence of load-shifting or rate decline, its amount was not large enough to offset, for given heating needs in a day, the savings in the preceding intervals.

Measuring the electricity savings of the households in Tariff Group D relative to Tariff Group A validates the load-shedding interpretations. Suppose that for the treated residential consumers, load-shifting is a primary countermeasure against the TOU program. Then residential consumers in Tariff Group D, compared to those in Tariff Group A, had more incentive to reallocate a portion of their electricity consumption to off-peak hours because they faced a much larger price increase in the peak rate period. So in both near-peak intervals, the savings for Tariff Group D must be significantly smaller than those for Tariff Group A. However, Figure XYZ, which shows point estimates obtained by setting Tariff Groups A and D as the control and treatment groups, respectively, does not demonstrate a meaningful difference between them. That is, load-shifting did not play a role in reshaping households' load profiles in and near the peak rate period. 

Going through the curves of the predicted savings related to temperature-control electricity use for the three intervals simultaneously but by taking account of their time sequence suggests a significant implication of the effectiveness of the TOU prices in the peak rate period. According to Figure XYZ, as the magnitude of the peak-hour price escalations increases, the temperature-control-related savings in the pre-peak interval expanded proportionally, while those in the peak rate period decreased gradually. Collectively, it is likely that a larger pre-adjustment leads to smaller reductions in electricity demand for heating during peaks, which in turn results in limited additional consumption in the post-peak interval. Considering that the TOU tariffs are intended to conserve electricity consumption during the peak rate period, it is inferable that fewer savings caused by too large pre-adjustment deteriorate the performance of the TOU tariffs. 
