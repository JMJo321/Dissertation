My empirical results obviously show that the effectiveness of the TOU tariffs, as measured by the magnitude of the induced electricity savings, nonlinearly varies with daily HDDs. As discussed, the total electricity savings caused by the deployment of TOU pricing consists of two elements: the savings from electricity consumption for non-temperature-control uses and those from electricity consumption for temperature-control uses. By definition, the savings originating from non-for-heating electricity consumption are independent of daily HDDs. Hence, the nonlinearity in the effectiveness of the TOU structures is utterly attributable to the other source of electricity savings, electricity consumption for heating. 

The nonlinear relationship between the amount of TOU-price-causing electricity savings and daily HDDs suggests an interesting characteristic of the tariff structure: the day-varying effects of TOU pricing on residential electricity consumption. Daily HDDs, which are one of the critical determinants of for-heating-relevant saving, vary day by day. Therefore, it is natural that in proportion to daily changing household heating needs, the total amount of TOU-price-inducing electricity savings also alters every day. 

The day-varying effectiveness of TOU electricity pricing suggests an interesting implication in connection with Real-Time Pricing (RTP), a type of time-varying electricity tariff structure.\footnote{\cite{Household-Responses-to-Time-Varying-Electricity-Prices_Harding-and-Sexton_2017} provides a detailed description of various kinds of time-varying electricity tariff structures.} Contrary to TOU pricing, rates typically change hourly under RTP. So compared to TOU pricing, RTP has an advantage in reflecting generation costs contemporaneously. Economists, therefore, prefer RTP to TOU pricing. But because TOU-tariff-inducing electricity savings covariate with daily HDDs, TOU electricity pricing can somewhat emulate the favorable feature of RTP, especially on days with extreme temperatures. For example, on typical winter days in Ireland, Tariff Group A's heating-associated electricity savings in the peak rate period is almost half of the total savings under the TOU program. In other words, the time-varying rate structure already induces substantial reductions in electricity consumption according to real-time generation costs, even though there were only within-day price variations. Consequently, in that case, the additional gains obtained by switching to RTP might not be significant as economists have expected. 

