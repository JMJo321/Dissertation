As noted in Section \ref{Sub-subsection:Household-Consumption-Behavior-in-and-near-the-Peak-Rate-Period}, under the TOU program, households' adjustments to their behavior for temperature-control-driven electricity consumption during the pre-peak hours seem to determine the degree of a reduction in that use of electricity during the following period (i.e., during the peak rate period) in lieu of price signals. In Figure \ref{Figure:Treatment-Effects-as-a-Linear-Function-of-Price-Changes-in-the-Peak-Rate-Period}, the gap in the temperature-control-related treatment effect at given daily HDDs between the lowest and the highest peak-hour rate changes, therefore, might be understood as potentially attainable gains when the pre-adjustments are suppressed. This explanation motivates the necessity of adopting home automation technologies, like Programmable Communicating Thermostats (PCTs), to restrict such adjustments only to the peak rate period. Considering the fact that households generally set a target temperature instead of micromanaging their heating devices according to ever-changing outdoor temperatures, PCTs with recommended default settings for temperature-control-associated use of electricity are highly likely to contribute to minimizing their behavioral changes prior to the peak rate period.\footnote{\cite{Default-Effects-and-Follow-on-Behavior_Evidence-from-an-Electricity-Pricing-Program_Fowlie-et-al_2021} examines default effects in a randomized controlled trial, in which the participants assigned to the control group defaulted into a residential electricity pricing program. Default effects have been studied in a range of settings, such as organ donation \citep{Medicine_Do-Defaults-Save-Lives_Johnson-and-Goldstein_2003, The-Impact-of-Presumed-Consent-Legislation-on-Cadaveric-Organ-Donoation_Abadie-and-Gay_2006}, car insurance \citep{Framing-Probability-Distortions-and-Insurance-Decisions_Johnson-et-al_1993}, and participation in retirement savings plans \citep{Status-Quo-Bias-in-Decision-Making_Samuelson-and-Zeckhauser_1988, The-Power-of-Suggestion_Madrian-and-Shea_2001, For-Better-or-For-Worse_Choi-et-al_2019}.} Moreover, the additional gains realized by utilizing the automated instruments provide legitimacy for the ongoing SEAI-offering Home Energy Grants, in which heating controls are an essential part.\footnote{Sustainable Energy Authority of Ireland (SEAI) is Ireland's national sustainable energy authority whose goal is to promote and assist the development of sustainable energy in Ireland. Detailed information about Home Energy Grants is available at \url{https://www.seai.ie/grants/research-funding/}.} 
