As noted in Section XYZ, under the TOU program, households' adjustments to their consumption behavior for temperature-control electricity use during the pre-peak hours seem to result in fewer savings in the following period (i.e., the peak rate period). In Figure XYZ, the gap in the temperature-control-related savings at given daily HDDs between the lowest and the highest peak-hour rate changes, therefore, might be understood as potentially attainable savings when the pre-adjustments are suppressed. This explanation motivates the necessity of adopting home automation technologies, like Programmable Communicating Thermostats (PCTs), to restrict such adjustments only to the peak rate period. Considering the fact that households generally set a target temperature instead of micromanaging their heating devices according to ever-changing outside temperatures, PCTs with recommended default settings for temperature-control use of electricity are highly likely to contribute to minimizing the behavioral changes before the peak rate period. Moreover, the benefits obtained by utilizing the automated instruments provide legitimacy for the ongoing SEAI-offering Home Energy Grants, in which heating controls are an essential part.\footnote{Sustainable Energy Authority of Ireland (SEAI) is Ireland's national sustainable energy authority whose goal is to promote and assist the development of sustainable energy in Ireland. And detailed information about Home Energy Grants is available at \url{https://www.seai.ie/grants/research-funding/}.} 

Confining the impact of TOU prices on household electricity consumption for temperature-control uses to the peak rate period by exploiting an automation technology provides more than realizing the potential electricity savings in the period. As discussed in Section 4.1.2, TOU electricity pricing can induce substantially larger electricity savings on days when the temperatures are more extreme and the demand on the grid is higher, even though the rates under the tariff structure do not vary across days. Because an automated system for heating controls causes additional savings in electricity consumption for temperature-control uses during peaks, especially on typical winter days in Ireland, the savings are comparable to those from more granular types of dynamic price schemes. 

