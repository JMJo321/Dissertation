Examining participating households' electricity consumption, following a time sequence from the pre-peak to the post-peak period, facilitates a complete understanding of how they adapted to the TOU tariff structures in the CER experiment. Intuitively, residential consumers can respond to a peak TOU price by conserving their electricity consumption during peaks, leading to an overall reduction in their demand for electricity. Instead of reducing their electricity consumption, they can shift it to off-peak hours so as not to be subject to the peak rate as much as possible. In this case, the level of their net electricity consumption is maintained. Of course, those two ways of responding to time-varying price structures can co-occur. Because those two ways reshape load curves not only in the peak rate period but also in the hours surrounding that period, it will be natural to examine the impact of the TOU program on household electricity consumption from a time-moving perspective in order to grasp the whole dynamics of households' behavioral changes. In the following paragraphs, I will provide interpretations of the changes in households' consumption behavior, which are observed in my empirical analysis. 

Regarding residential electricity demand for non-temperature-control uses, the leading reaction of the treated households to the TOU tariffs was to reduce their consumption in and near the peak rate period. According to my regression results summarized in Figure \ref{Figure:Treatment-Effects-as-a-Linear-Function-of-Price-Changes-in-the-Peak-Rate-Period}, in the peak period, the reduction in non-temperature-control-related electricity consumption increased as the magnitude of a peak-rate-period price change under the TOU program grew. Non-temperature-control-driven electricity consumption for the pre- and post-peak periods showed an opposite variation---i.e., the reduction originating from households' non-for-heating consumption diminished as the degree of a price increase in the peak rate period became larger. In the case of Tariff Group A, although there was almost zero price variation relative to the flat rate (i.e., only 0.1 cents per kWh) in the pre- and post-peak periods, the amount of the diminution in non-temperature-control-related electricity consumption for that group was nearly the same in all three periods. Meanwhile, despite more sizable price decreases, the remaining tariff groups also conserved their consumption for non-temperature-control uses in both surrounding periods. In sum, the price increases in the peak rate period caused a spillover effect in those pre- and post-peak periods: a reduction in electricity consumption for non-temperature-control uses. In other words, with respect to non-temperature-control-driven electricity consumption, the households assigned to the treatment group responded to the TOU program, on the whole, via not load-shifting but load-shedding. Interestingly, the total non-temperature-control-relevant reduction in and near the peak rate period, which is depicted in the fourth column of the first row in the figure, did not vary with the level of a peak-hour price increase. 

With respect to temperature-control-related household electricity consumption, Figure \ref{Figure:Treatment-Effects-as-a-Linear-Function-of-Price-Changes-in-the-Peak-Rate-Period} depicts that the treated households' primary response to the TOU program was also load-shedding. The program caused a reduction in for-heating electricity use during the peak rate period, especially around typical values of daily HDDs during winter in Ireland\footnote{See Figure \ref{Figure:Distribution-of-Heating-Degree-Days-during-the-Experiment-Period}.}---interestingly, the smaller the magnitude of a peak-demand-hour price increase, the larger the induced reduction in temperature-control-related consumption in the peak period. That is, the reduction violated the law of demand. A possible explanation for this phenomenon will be discussed later. As described in Figure \ref{Figure:Time-Of-Use-Pricing-Structures}, there were price drops in the hours surrounding the peak rate period. Furthermore, for marginal electricity consumption, because the tariff group that paid the highest price in the peak rate period (i.e., Tariff Group D) paid the lowest price in the surrounding hours, the households in that group were more incentivized to relocate their peak-hour electricity consumption to off-peak hours. Therefore, the reduction in electricity consumption for heating in the pre-peak period, which occurred only on days with heavy heating needs, cannot be explained as a consequence of a price decrease or load-shifting. In other words, regarding temperature-control-driven household electricity consumption, in addition to the peak rate period, price signals did not function well in the pre-peak period. In the post-peak period, although high daily HDDs incurred additional electricity consumption for heating after introducing TOU tariffs, which also cannot be justified by price signals for the same reasons as in the pre-peak period, its amount was generally not large enough to fully offset, for given heating needs in a day, the reductions in the preceding periods.

Measuring the induced consumption reduction of households in Tariff Group D relative to Tariff Group A validates the load-shedding interpretation. Suppose that for the treated residential consumers, load-shifting is a primary countermeasure against the TOU program. Then the residential consumers in Tariff Group D, compared to those in Tariff Group A, had more incentive to reallocate a portion of their peak-hour electricity consumption to off-peak hours because they faced a much larger price increase in the peak rate period as well as a much larger price decrease in the pre- and post-peak periods. So, compared to those in Tariff Group A, the households in Tariff Group D should consume more electricity in both periods surrounding the peak rate period, while their electricity consumption should be less in the peak rate period. However, Figure \ref{Figure:Relative-Comparison-of-Tariff-Group-D-to-Tariff-Group-A}, which shows point estimates obtained by setting Tariff Groups A and D as the control and treatment groups, respectively, exhibits only a little hint of load-shifting only in the post-peak period, though the reduction in non-temperature-control-driven household electricity consumption was evident. That is, load-shifting did not play a role in reshaping households' load profiles in and near the peak rate period. 

From Figure \ref{Figure:Treatment-Effects-as-a-Linear-Function-of-Price-Changes-in-the-Peak-Rate-Period}, examining the curves of aggregate change in temperature-control-associated electricity consumption for three consecutive periods simultaneously, but taking account of their time sequence, suggests a significant implication of the effectiveness of the TOU prices in the peak rate period. According to the figure, as the degree of peak-hour price escalation increased, the temperature-control-related consumption reduction in the pre-peak period expanded, while those in the peak period decreased gradually. Altogether, it is likely that a larger pre-adjustment leads to a smaller reduction in electricity demand for heating during peak-demand hours, which in turn seems to result in limited additional consumption during the following post-peak period. Compared to the case that a household does not reduce for-heating electricity consumption during the pre-peak period, consuming more for-heating electricity during peak hours seems necessary to prevent indoor temperatures from falling too much or persisting at a low level when the household significantly reduces its temperature-control-driven consumption during the pre-peak period.
%\footnote{This interpretation is in line with the concept ``discomfort'' in \cite{Smart-Thermoststs-Automation-and-Time-Varying-Prices_Blonz-et-al_2021}. See Section 3.4 in the paper.} 
In addition, the household will have less incentive to increase its electricity consumption for heating during post-peak hours since its room temperatures will be higher than if it were to considerably reduce its electricity consumption for heating during peak hours. In light of the fact that TOU tariffs are intended to conserve electricity consumption during peak-demand hours, it is reasonable to conclude that a lower reduction in peak hours due to a too large pre-adjustment results in a deterioration in the performance of the TOU tariffs.
