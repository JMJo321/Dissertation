Examining participating households' electricity consumption, following a time sequence from the pre-peak to the post-peak period, facilitates a complete understanding of how they adapted to the TOU tariff structures in the CER experiment. Intuitively, residential consumers can respond to a peak TOU price by conserving their electricity consumption during peaks, leading to an overall reduction in their demand for electricity. Instead of reducing their electricity consumption, they can shift it to off-peak hours so as not to be subject to the peak rate as much as possible. In this case, the level of their net electricity consumption is maintained. Of course, those two ways of responding to time-varying price structures can co-occur. Because those two ways reshape load curves not only in the peak rate period but also in the hours surrounding that period, it will be natural to examine the impact of the TOU program on household electricity consumption from a time-moving perspective in order to grasp the whole dynamics of households' behavioral changes. In the following paragraphs, I will provide interpretations of the changes in households' consumption behavior, which are observed in my empirical analysis. 

Regarding residential electricity demand for non-temperature-control uses, the leading reaction of the treated households to the TOU tariffs was to reduce their consumption in and near the peak rate period. According to my regression results summarized in Figure \ref{Figure:Treatment-Effects-as-a-Linear-Function-of-Price-Changes-in-the-Peak-Rate-Period}, in the peak period, the reduction in non-temperature-control-related electricity consumption increased as the magnitude of a peak-rate-period price change under the TOU program grew. Non-temperature-control-driven electricity consumption for the pre- and post-peak periods showed a weak but opposite variation---i.e., the reduction originating from households' non-for-heating consumption moved towards zero as the degree of a price increase in the peak rate period became larger. In the case of Tariff Group A, although there was almost zero price variation relative to the flat rate (i.e., only 0.1 cents per kWh) in the pre- and post-peak periods, the amount of the diminution in non-temperature-control-related electricity consumption for that group was nearly the same in all three periods. Meanwhile, despite more sizable price decreases, the remaining tariff groups also conserved, or at least sustained, their consumption for non-temperature-control uses in both surrounding periods. In other words, my empirical results reveal that reductions in households' non-for-heating electricity consumption spilled over into non-peak periods (i.e., the pre- and post-peak periods) and that the smaller the price difference between the peak and non-peak periods, the larger the spillovers. 

A remarkable point with respect to the spillovers to non-peak hours, suggesting households' behavioral changes related to non-temperature-control-driven electricity consumption in the pre- and post-peak periods, is that they seemed to relocate a part of their not-for-heating electricity consumption during peak hours to those two periods. As described in Figure \ref{Figure:Time-Of-Use-Pricing-Structures}, there were price drops in the hours before and after the peak rate period. Furthermore, for marginal electricity consumption, because the tariff group that paid the highest price in the peak rate period (i.e., Tariff Group D) paid the lowest price in the surrounding hours, the households in that group were more incentivized to move their peak-hour electricity consumption to off-peak hours. Hence, the phenomenon that reductions in not-for-heating electricity consumption in the surrounding periods declined as the magnitude of a peak-rate-period price change increased is well in line with load-shifting motivated by a monetary incentive coming from price differences. As shown in Figure \ref{Figure:Treatment-Effects-as-a-Linear-Function-of-Price-Changes-in-the-Peak-Rate-Period}, the relocation-associated behavioral change, in general, did not fully outweigh the conservation-relevant behavioral change in both periods. 

In summary, with respect to non-temperature-control-driven electricity consumption, the households assigned to the treatment group responded to the TOU program, on the whole, via load-shedding as primary and load-shifting as secondary. Interestingly, the total non-temperature-control-relevant reduction in and near the peak rate period, which is depicted in the fourth column of the first row in Figure \ref{Figure:Treatment-Effects-as-a-Linear-Function-of-Price-Changes-in-the-Peak-Rate-Period}, did not vary with the level of a peak-hour price increase. 

With respect to temperature-control-related household electricity consumption, Figure \ref{Figure:Treatment-Effects-as-a-Linear-Function-of-Price-Changes-in-the-Peak-Rate-Period} depicts that the treated households' primary response to the TOU program was also load-shedding. The program caused a reduction in for-heating electricity use during the peak rate period, especially around typical values of daily HDDs during winter in Ireland\footnote{See Figure \ref{Figure:Distribution-of-Heating-Degree-Days-during-Experiment-Periods}.}---interestingly, and although statistically insignificant, the smaller the magnitude of a peak-demand-hour price increase, the larger the induced reduction in temperature-control-related consumption in the peak period. That is, the reduction violated the law of demand. As discussed above, the households assigned to Tariff Group D had the highest incentive to relocate their peak-hour electricity consumption to non-peak hours surrounding the peak-demand hours. Therefore, the reduction in electricity consumption for heating in the pre-peak period, which occurred only on days with heavy heating needs, cannot be explained as a consequence of a price decrease or load-shifting. In other words, regarding temperature-control-driven household electricity consumption, in addition to the peak rate period, the price signals did not function well in the pre-peak period. In the post-peak period, high daily HDDs incurred additional electricity consumption for heating after introducing TOU tariffs. The additional consumption, however, also cannot be justified by the price signals for the same reasons as in the pre-peak period.\footnote{The estimated changes in temperature-control-related electricity consumption with respect to peak-demand-hour price variations for the peak and post-peak periods, presented in Figure \ref{Figure:Treatment-Effects-as-a-Linear-Function-of-Price-Changes-in-the-Peak-Rate-Period}, seem rather to imply that the degree of load-shifting diminishes as the financial incentive, measured by the price difference between the two periods, increases.} And the amount of the additional consumption was generally not large enough to fully offset, for given heating needs in a day, the reductions in the preceding periods. In Section \ref{Sub-subsection:Household-Electricity-Consumption-for-Heating-in-a-Time-Line}, I will discuss a possible explanation for the consumption behavior not backed by the price signals.  

Measuring the induced consumption reduction of households in Tariff Group D relative to Tariff Group A re-confirms that load-shifting in household electricity consumption from peak to non-peak hours is likely to be insensitive to the magnitude of the pecuniary incentive, which is measured by the price difference across rate periods, particularly for temperature-control-driven electricity consumption. Suppose that for treated residential consumers, load-shifting strictly depends on the across-rate-period monetary incentive under the TOU program. Then the residential consumers in Tariff Group D, compared to those in Tariff Group A, are more willing to reallocate a portion of their peak-hour electricity consumption to non-peak hours because they face a much larger price increase in the peak rate period as well as a much larger price decrease in the pre- and post-peak periods. So, compared to those in Tariff Group A, the households in Tariff Group D should consume more electricity in both periods surrounding the peak rate period, while their electricity consumption should be less in the peak rate period. However, Figure \ref{Figure:Relative-Comparison-of-Tariff-Group-D-to-Tariff-Group-A}, which shows point estimates obtained by setting Tariff Groups A and D as the control and treatment groups, respectively, exhibits merely a slender hint of load-shifting. As illustrated in the figure, regarding non-temperature-control-driven household electricity consumption, there were apparent reductions in the peak rate period, while there was only a marginal increase in the hours immediately adjacent to the peak rate period. For temperature-control-related household electricity consumption, such a weak pattern was also identified only in the post-peak period on cold days. Consequently, load-shifting played a minimal role in reshaping households' load profiles in and near the peak rate period. 
