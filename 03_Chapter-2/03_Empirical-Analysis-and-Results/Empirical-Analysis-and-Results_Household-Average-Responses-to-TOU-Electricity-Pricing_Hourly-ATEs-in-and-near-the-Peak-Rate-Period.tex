Estimating by-tariff-group ATEs in and near the peak rate period allows understanding how the relationship between the degree of change in household electricity consumption and the magnitude of a peak-demand-hour price increase evolves in and near the peak rate period.\footnote{In this paper, the effects of four different information stimuli on household electricity consumption are not of interest. \cite{The-Effect-of-Information-on-TOU-Electricity-Use:An-Irish-Residential-Study_Pon_2017} studied the effects in detail using the same datasets.} To do so, I run the following regression for each of the four tariff groups:
\begin{equation}
\begin{split}
    \textit{kWh}_{ith} \ 
    & = \ \beta_{p} \mathbb{1}\big[ \text{Treatment \& Post} \big]_{it} \ + \ \alpha_{iw} \ + \ \gamma_{tw} \ + \ \delta_{m} \ + \ \epsilon_{ith}
\end{split}
\label{Eq:Model-Specification_Hourly-Average-Treatment-Effects}
\end{equation}

Excepting the dependent variable and the parameter of interest, the econometric model above is the same as (\ref{Eq:Model-Specification_Half-Hourly-Average-Treatment-Effects}). Specifically, the response variable $kWh_{ith}$, which means the electricity consumption by household $i$ on the day $t$ during the hour of the day $h$, is utilized due to its better accessibility in interpretation. The point estimates of $\beta_{p}$ indicate the ATE for each of the three intervals included in rate period $p$. Table \ref{Table:Hourly-Average-Treatment-Effects-in-and-near-the-Peak-Rate-Period} summarizes the regression results. 

The measured ATEs for the peak rate period re-confirm the finding provided in \cite{Peaking-Interest:How-Awareness-Drives-the-Effectiveness-of-Time-of-Use-Electricity-Pricing_Prest_2020}.\footnote{See Figure 6 in \cite{Peaking-Interest:How-Awareness-Drives-the-Effectiveness-of-Time-of-Use-Electricity-Pricing_Prest_2020}.} The table clearly shows that within-household aggregate demand for electricity during the peak rate period declined, with a significance level of 0.01, due to the deployment of TOU pricing. However, based on the point estimates for the four tariff groups, it is unclear whether an incremental change in peak-rate-period price increase induces a statistically meaningful additional change in household electricity consumption or not. 

To quantify how residential consumers responded to the TOU program in off-peak hours close to the peak rate period, I also estimate ATEs in periods of two hours before and after the peak rate period (i.e., in pre- and post-peak periods). Interestingly, the table also demonstrates that in the pre- and post-peak periods, the implementation of the TOU tariff structures resulted in reductions in household electricity consumption, which are statistically different from zero, even though TOU prices were lower than the flat rate of 14.1 cents per kWh.\footnote{Even insignificant point estimates (i.e., point estimates for Tariff Groups C and D in the pre-peak interval and Tariff Group C in the post-peak interval) have negative values.} The reductions in both periods surrounding the peak hours suggest that the impact of the price increases in the peak rate period overtook the impact of the price drops in each off-peak period. Therefore, in the following empirical analysis, I will focus on linking household electricity consumption in the pre- and post-peak periods with the price increases in the peak rate period, instead of the price decreases in those off-peak periods. 
