The preceding results from my empirical analysis highlight that households were quite responsive to incremental changes in the peak-demand-hour price. As the peak-hour price increased compared to the flat rate, non-temperature-control-related electricity consumption continued to decline in the peak rate period. In contrast, as the peak-hour price increased, temperature-control-driven consumption did indeed fall, but these reductions in residential electricity consumption occurred outside of the peak rate period (i.e., in the pre- and post-peak periods). In this section, I will further examine what can drive these different patterns in the responses to TOU prices. 

\subsubsection{Mechanism: Load-shedding vs. Load-shifting}
\label{Sub-subsection:Mechanism}
Examining participating households' electricity consumption, following a time sequence from the pre-peak to the post-peak period, facilitates a complete understanding of how they adapted to the TOU tariff structures in the CER experiment. Intuitively, residential consumers can respond to TOU tariffs by conserving their electricity consumption during the peak-demand hours, leading to an overall reduction in their demand for electricity. Instead of reducing their electricity consumption, they can shift it to off-peak hours so as not to be subject to the peak rate as much as possible. In this case, the level of their net electricity consumption in a day is maintained. Of course, those two ways of responding to time-varying price structures can co-occur. Because those two ways reshape load curves not only in the peak rate period but also in the hours surrounding that period, it will be natural to examine the impact of the TOU program on household electricity consumption from a time-moving perspective in order to grasp the whole dynamics of households' behavioral changes. In the following paragraphs, I will provide interpretations of the changes in households' consumption behavior, which are observed in my empirical analysis. 

Regarding residential electricity demand for non-temperature-control uses, the leading reaction of the treated households to the TOU tariffs was to reduce their consumption in and near the peak rate period. According to my regression results summarized in Figure \ref{Figure:Treatment-Effects-as-a-Linear-Function-of-Price-Changes-in-the-Peak-Rate-Period}, in the peak rate period, the reduction in non-temperature-control-related electricity consumption increased as the magnitude of the price change in that period under the TOU program grew. Non-temperature-control-driven electricity consumption in the pre- and post-peak periods showed a weak but opposite variation---i.e., the reduction originating from households' non-for-heating consumption moved towards zero as the degree of the price increase in the peak rate period became larger. In the case of Tariff Group A, although there was almost zero price variation relative to the flat rate (i.e., only 0.1 cents per kWh) in the pre- and post-peak periods, the amount of the diminution in non-temperature-control-related electricity consumption for that group was nearly the same in all three periods. Meanwhile, despite more sizable price decreases, the remaining tariff groups also conserved, or at least sustained, their consumption for non-temperature-control uses in both surrounding periods. In other words, my empirical results reveal that reductions in households' non-for-heating electricity consumption spilled over into non-peak periods (i.e., the pre- and post-peak periods). 

A remarkable point with respect to the spillovers to non-peak hours, suggesting households' behavioral changes related to non-temperature-control-driven electricity consumption in the pre- and post-peak periods, is that they seemed to relocate a part of their not-for-heating electricity consumption during peak hours to those two periods. As described in Figure \ref{Figure:Time-Of-Use-Pricing-Structures}, there were price drops in the hours before and after the peak rate period. Furthermore, for marginal electricity consumption, because the tariff group that paid the highest price in the peak rate period (i.e., Tariff Group D) paid the lowest price in the surrounding hours, the households in that group were more incentivized to move their peak-hour electricity consumption to off-peak hours. Hence, the phenomenon that the reduction in not-for-heating electricity consumption in the surrounding periods declined as the magnitude of the peak-rate-period price change increased is well explained by combining the load-shedding with the load-shifting, which was motivated by the monetary incentive from the price differences between the peak and off-peak periods. As shown in Figure \ref{Figure:Treatment-Effects-as-a-Linear-Function-of-Price-Changes-in-the-Peak-Rate-Period}, the relocation-associated consumption change, in general, did not fully outweigh the conservation-relevant one in both periods. 

Taken together, with respect to non-temperature-control-driven electricity consumption, the households assigned to the treatment group responded to the TOU program via load-shedding as primary and load-shifting as secondary reactions. Interestingly, the total non-temperature-control-relevant reduction in and near the peak rate period, which is depicted in the fourth column of the first row in Figure \ref{Figure:Treatment-Effects-as-a-Linear-Function-of-Price-Changes-in-the-Peak-Rate-Period}, did not vary with the level of a peak-hour price increase. This outcome might reflect households' limited capability not only to identify possible sources of reducing their electricity consumption but also to realize lower consumption from the sources. 

With respect to temperature-control-related household electricity consumption, Figure \ref{Figure:Treatment-Effects-as-a-Linear-Function-of-Price-Changes-in-the-Peak-Rate-Period} depicts that the program caused a reduction in for-heating electricity use during the peak rate period, especially around typical values of daily HDDs during winter in Ireland.\footnote{See Figure \ref{Figure:Distribution-of-Heating-Degree-Days-during-Experiment-Periods}.} Interestingly, although statistically insignificant, the smaller the magnitude of the peak-demand-hour price change increase, the larger the induced reduction in temperature-control-related consumption in the peak period. That is, the change in for-heating electricity consumption seems to violate the law of demand. As discussed above, the households assigned to Tariff Group D had the highest incentive to relocate their peak-hour electricity consumption to non-peak hours surrounding the peak-demand hours due to the largest across-period price difference. Therefore, the reduction in electricity consumption for heating in the pre-peak period, which occurred only on days with heavy heating needs, cannot be explained as a consequence of either the price decrease in that period or load-shifting. In other words, regarding temperature-control-driven household electricity consumption, as did in the peak rate period, the price signals did not function well in the pre-peak period. In the post-peak period, high daily HDDs incurred additional electricity consumption for heating after introducing TOU tariffs. The degree of the additional consumption, however, also cannot be justified by the price signals for the same reasons as in the pre-peak period.\footnote{The estimated changes in temperature-control-related electricity consumption with respect to peak-demand-hour price variation for the peak and post-peak periods, presented in Figure \ref{Figure:Treatment-Effects-as-a-Linear-Function-of-Price-Changes-in-the-Peak-Rate-Period}, seem rather to imply that the degree of load-shifting diminished as the financial incentive, measured by the price difference between the two periods, increased.} And the amount of the additional consumption was generally not large enough to fully offset, for given heating needs in a day, the reduction in the preceding periods. In Section \ref{Sub-subsection:Household-Electricity-Consumption-for-Heating-in-a-Time-Line}, I will discuss a possible explanation for the consumption behavior not backed by the price signals. 



\subsubsection{Household Consumption Behavior in and near the Peak Rate Period}
\label{Sub-subsection:Household-Consumption-Behavior-in-and-near-the-Peak-Rate-Period}
From Figure \ref{Figure:Treatment-Effects-as-a-Linear-Function-of-Price-Changes-in-the-Peak-Rate-Period}, examining the curves that illustrate the change in temperature-control-associated electricity consumption for three consecutive periods simultaneously, but taking account of their time sequence, suggests a significant implication of the effectiveness of the TOU prices in the peak rate period. According to the figure, as the degree of peak-hour price escalation increased, the temperature-control-related consumption reduction in the pre-peak period expanded, while those in the peak period decreased gradually. Altogether, it is likely that a larger pre-adjustment leads to a smaller reduction in electricity demand for heating during peak-demand hours, which in turn seems to result in limited additional consumption during the following post-peak period. Compared to the case that a household does not reduce for-heating electricity consumption during the pre-peak period, consuming more for-heating electricity during peak hours seems necessary to prevent indoor temperatures from falling too much or persisting at a low level when the household significantly reduces its temperature-control-driven consumption during the pre-peak period.\footnote{This interpretation is in line with the concept ``discomfort'' in \cite{Smart-Thermoststs-Automation-and-Time-Varying-Prices_Blonz-et-al_2021}. See Section 3.4 in the paper.} In addition, the household will have less incentive to increase its electricity consumption for heating during post-peak hours since its room temperatures will be higher than if it were to reduce its electricity consumption for heating during peak hours considerably. In light of the fact that TOU tariffs are intended to conserve electricity consumption during peak-demand hours, it is reasonable to conclude that a lower reduction in peak hours due to a too large pre-adjustment results in a deterioration in the performance of the TOU tariffs. 

As discussed in detail, under the TOU program, households' adjustments to their behavior for temperature-control-driven electricity consumption during the pre-peak hours seem to determine the degree of a reduction in that use of electricity during the following period (i.e., during the peak rate period) in lieu of price signals. In Figure \ref{Figure:Treatment-Effects-as-a-Linear-Function-of-Price-Changes-in-the-Peak-Rate-Period}, the gap in the temperature-control-related treatment effect at given daily HDDs between the lowest and the highest peak-hour rate changes, therefore, might be understood as potentially attainable gains when the pre-adjustments are suppressed. This explanation motivates the necessity of adopting home automation technologies, like Programmable Communicating Thermostats (PCTs), to restrict such adjustments only to the peak rate period. Considering the fact that households generally set a target temperature instead of micromanaging their heating devices according to ever-changing outdoor temperatures, PCTs with recommended default settings for temperature-control-associated use of electricity are highly likely to contribute to minimizing their behavioral changes prior to the peak rate period.\footnote{\cite{Default-Effects-and-Follow-on-Behavior_Evidence-from-an-Electricity-Pricing-Program_Fowlie-et-al_2021} examines default effects in a randomized controlled trial, in which the participants assigned to the control group defaulted into a residential electricity pricing program. Default effects have been studied in a range of settings, such as organ donation \citep{Medicine_Do-Defaults-Save-Lives_Johnson-and-Goldstein_2003, The-Impact-of-Presumed-Consent-Legislation-on-Cadaveric-Organ-Donoation_Abadie-and-Gay_2006}, car insurance \citep{Framing-Probability-Distortions-and-Insurance-Decisions_Johnson-et-al_1993}, and participation in retirement savings plans \citep{Status-Quo-Bias-in-Decision-Making_Samuelson-and-Zeckhauser_1988, The-Power-of-Suggestion_Madrian-and-Shea_2001, For-Better-or-For-Worse_Choi-et-al_2019}.} Moreover, the additional gains realized by utilizing the automated instruments provide legitimacy for the ongoing SEAI-offering Home Energy Grants, in which heating controls are an essential part.\footnote{Sustainable Energy Authority of Ireland (SEAI) is Ireland's national sustainable energy authority whose goal is to promote and assist the development of sustainable energy in Ireland. Detailed information about Home Energy Grants is available at \url{https://www.seai.ie/grants/research-funding/}.} 