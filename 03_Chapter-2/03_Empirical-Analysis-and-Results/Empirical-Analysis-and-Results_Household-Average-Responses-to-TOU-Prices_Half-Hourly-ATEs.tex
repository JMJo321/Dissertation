Utilizing a panel DID identification strategy, I first measure the impact of the TOU prices on 30-minute-interval household electricity consumption. To obtain the ATE for each half-hour interval, I estimate the following specification:
\begin{equation}
\begin{split}
    \textit{kWh}_{itw} \ 
    & = \ \beta_{w} \mathbb{1}\big[ \text{Treatment \& Post} \big]_{it} \ + \ \alpha_{iw} \ + \ \gamma_{dw} \ + \ \delta_{m} \ + \ \epsilon_{itw}
\end{split}
\label{Eq:Model-Specification_Half-Hourly-Average-Treatment-Effects}
\end{equation}
The term $kWh_{itw}$ is the electricity consumption by household $i$ on the day $t$ during the half-hourly time window $w$. The indicator variable $\mathbb{1}\big[ \text{Treatment \& Post} \big]_{it}$ is equal to 1 only if household $i$ is in the treatment group and the day $t$ is in the treatment period. The terms $\alpha_{iw}$, $\gamma_{dw}$, and $\delta_{m}$ are household-by-half-hourly-interval, day-of-week-by-half-hourly-time-window, and month-of-year fixed effects, respectively. In the specification, the point estimates of $\beta_{w}$ representing the ATE for each 30-minute interval $w$ are the parameters of interest. I cluster the standard errors at the household and the day of experiment levels to correct for serial correlation.

\begin{figure}
    \caption{Half-Hourly Average Treatment Effects}
    \label{Figure:Half-Hourly-Average-Treatment-Effects}
\end{figure}
Figure \ref{Figure:Half-Hourly-Average-Treatment-Effects} summarizes the estimated ATEs in the form of a time profile. As also demonstrated in \cite{Peaking-Interest:How-Awareness-Drives-the-Effectiveness-of-Time-of-Use-Electricity-Pricing}, peak hours (i.e., from 5:00 p.m. to 7:00 p.m.), during which the inefficiency of fixed flat-rate tariffs is greatly intensified, show dominant electricity savings. In the following empirical analysis, I continually focus on household electricity demand responses to the time-varying prices during the peak rate period.