To fully understand how residential consumers adjust their consumption behavior as a set of reactions to the price changes under the TOU program, it is necessary to explicitly examine, for each of the three periods (i.e., the pre-peak, peak, and post-peak periods), the relationship between the size of a price increase in the peak rate period and the changes in the two distinct categories of household electricity consumption. For that reason, I quantitatively determine the relationship by utilizing the following econometric model:
\begin{equation}
\small
\begin{split}
    kWh_{ith} \ 
%    & = \ \beta_{1} HDD_{t} \ + \ \beta_{2} HDD_{t}^{*} \\
%    & \hspace{0.7cm} + \ \beta_{3} \mathds{1}[\text{Treatment}]_{i} \ + \ \beta_{4} \mathds{1}[\text{Treatment}]_{i} \Delta RC_{i} \\
%    & \hspace{0.7cm} + \ \beta_{5} HDD_{t} \mathds{1}[\text{Treatment}]_{i} \ + \ \beta_{6} HDD_{t} \mathds{1}[\text{Treatment}]_{i} \Delta RC_{i} \\
%    & \hspace{0.7cm} + \ \beta_{7} HDD_{t}^{*} \mathds{1}[\text{Treatment}]_{i} \ + \ \beta_{8} HDD_{t}^{*} \mathds{1}[\text{Treatment}]_{i} \Delta RC_{i} \\
%    & \hspace{0.7cm} + \ \beta_{9} \mathds{1}[\text{Post}]_{t} \ + \ \beta_{10} HDD_{t} \mathds{1}[\text{Post}]_{t} \ + \ \beta_{11} HDD_{t}^{*} \mathds{1}[\text{Post}]_{t} \\
%    & \hspace{0.7cm} + \ \beta_{12} \mathds{1}[\text{Treatment \& Post}]_{it} \ + \ \beta_{13} \mathds{1}[\text{Treatment \& Post}]_{i} \Delta RC_{i} \\
%    & \hspace{0.7cm} + \ \beta_{14} HDD_{t} \mathds{1}[\text{Treatment \& Post}]_{it} \ + \ \beta_{15} HDD_{t} \mathds{1}[\text{Treatment \& Post}]_{i} \Delta RC_{i} \\
%    & \hspace{0.7cm} + \ \beta_{16} HDD_{t}^{*} \mathds{1}[\text{Treatment \& Post}]_{it} \ + \ \beta_{17} HDD_{t}^{*} \mathds{1}[\text{Treatment \& Post}]_{i} \Delta RC_{i} \ + \ \alpha_{dw} \ + \ \epsilon_{ith}
%    & = \ \beta_{1} HDD_{t} \ + \ \beta_{2} HDD_{t}^{*} \\
%    & \hspace{0.7cm} + \ \big( \beta_{3} \ + \ \beta_{4} HDD_{t} \ + \ \beta_{5} HDD_{t}^{*} \big) \mathds{1}[\text{Treatment}]_{i} \\
%    & \hspace{0.7cm} + \ \big( \beta_{6} \ + \ \beta_{7} HDD_{t} \ + \ \beta_{8} HDD_{t}^{*} \big) \mathds{1}[\text{Treatment}]_{i} \Delta RC_{i} \\
%    & \hspace{0.7cm} + \ \big( \beta_{9} \ + \ \beta_{10} HDD_{t} \ + \ \beta_{11} HDD_{t}^{*} \big) \mathds{1}[\text{Post}]_{t} \\
%    & \hspace{0.7cm} + \ \big( \beta_{12} \ + \ \beta_{13} HDD_{t} \ + \ \beta_{14} HDD_{t}^{*} \big) \mathds{1}[\text{Treatment \& Post}]_{it} \\
%    & \hspace{0.7cm} + \ \big( \beta_{15} \ + \ \beta_{16} HDD_{t} \ + \ \beta_{17} HDD_{t}^{*} \big) \mathds{1}[\text{Treatment \& Post}]_{i} \Delta RC_{i} \\
%    & \hspace{0.7cm} + \ \alpha_{dw} \ + \ \epsilon_{ith}
    & = \ \beta_{1} HDD_{t} \ + \ \beta_{2} HDD_{t}^{*} \ + \ \big( \beta_{3} \ + \ \beta_{4} HDD_{t} \ + \ \beta_{5} HDD_{t}^{*} \big) \mathds{1}[\text{Treatment}]_{i} \\
    & \hspace{0.7cm} + \ \big( \beta_{6} \ + \ \beta_{7} HDD_{t} \ + \ \beta_{8} HDD_{t}^{*} \big) \mathds{1}[\text{Treatment}]_{i} \Delta PC_{i} \ + \ \big( \beta_{9} \ + \ \beta_{10} HDD_{t} \ + \ \beta_{11} HDD_{t}^{*} \big) \mathds{1}[\text{Post}]_{t} \\
    & \hspace{0.7cm} + \ \big( \beta_{12} \ + \ \beta_{13} HDD_{t} \ + \ \beta_{14} HDD_{t}^{*} \big) \mathds{1}[\text{Treatment \& Post}]_{it} \\
    & \hspace{0.7cm} + \ \big( \beta_{15} \ + \ \beta_{16} HDD_{t} \ + \ \beta_{17} HDD_{t}^{*} \big) \mathds{1}[\text{Treatment \& Post}]_{i} \Delta PC_{i} \ + \ \alpha_{dw} \ + \ \epsilon_{ith}
\end{split}
\end{equation}
The model is the same with (\ref{Eq:Model-Specification_Breakdown-of-Hourly-Average-Treatment-Effect}) except for interaction terms between treatment-status-relevant indicator variables (i.e., $\mathds{1}[\text{Treatment}]_{i}$ and $\mathds{1}[\text{Treatment \& Post}]_{it}$) and $\Delta PC_{i}$, where $\Delta PC_{i}$ is the difference between the peak-hour prices in the treatment period and the flat rate in the baseline period. The coefficients of those interaction terms capture the impacts of deploying TOU tariffs on household electricity consumption as a linear function of the degree of a peak-demand-hour price change.

The estimates of the six coefficients of interest (i.e., from $\beta_{12}$ to $\beta_{17}$) presented in Table \ref{Table:Hourly-ATEs-as-a-Linear-Function-of-Peak-Rate-Period-Price-Changes_Coefficients-of-Interest-only} are summarized graphically in Figure \ref{Figure:Treatment-Effects-as-a-Linear-Function-of-Price-Changes-in-the-Peak-Rate-Period}. And this figure, showing the estimated treatment effects for the two consumption channels and the sum of the treatment effects in each of the three intervals, re-confirms the finding of peak-rate-period price increases' diminishing returns in \cite{Peaking-Interest:How-Awareness-Drives-the-Effectiveness-of-Time-of-Use-Electricity-Pricing_Prest_2020}. 

In the peak rate period, the reduction in non-temperature-control-associated electricity consumption increased as the magnitude of a peak-hour price increase grew. On the contrary, at given daily HDDs, the reduction in temperature-control-related electricity consumption weakly moved towards zero as the size of a peak-demand-hour tariff escalation increased. As well illustrated in the figure, for a given value of daily HDDs, the differences in treatment effect across the level of price growth are seemingly dampened when the estimated treatment effects from two distinct categories of electricity consumption are aggregated due to the opposite response to peak-hour price increases in the two consumption categories. Indeed, this empirical result is consistent with the finding discussed in the paper that a higher price results in a larger diminution in electricity demand, while additional gains diminish in the peak interval.  

The opposite order of by-rate-change treatment effects for given daily HDDs in two different types of electricity consumption also holds in the pre-peak interval, although in a contrary manner. The interval shows a more significant reduction in non-temperature-control-driven electricity consumption for a more minor change in peak-hour price. By contrast, the diminution in non-temperature-control-related electricity consumption exhibits an inverse relationship with the peak-rate-period price change. For the same reason as in the peak interval, the aggregate treatment effects of the TOU tariffs described in the last row of Figure \ref{Figure:Treatment-Effects-as-a-Linear-Function-of-Price-Changes-in-the-Peak-Rate-Period} are seemingly less sensitive to the peak-hour prices. Note that regarding electricity consumption for heating during the pre-peak interval, TOU pricing played a role only when household heating needs were sufficiently high. 

Irish residential consumers adjusted their electricity consumption behavior during the post-peak period as well. As in the pre-peak period, consumption changes stemming from non-temperature-control-related electricity use increased as the size of a peak-demand-hour rate change diminished. The TOU-price-induced change in temperature-control-driven electricity consumption evolved over daily HDDs somewhat complicatedly. Though depending on the magnitude of a peak-hour price increase, TOU tariffs reduced household electricity consumption for heating on Ireland's typical winter days in that period. Interestingly, the CER TOU program provoked additional heating-related consumption during the post-peak period on extremely cold days in Ireland. In addition, as the level of peak-demand-hour price alteration grew, the profile of measured treatment effect for temperature-control-associated consumption moved downward. Consequently, a higher price increase in the peak rate period resulted in a more significant reduction in electricity consumption for heating when heating demands were lower, while a smaller addition to electricity consumption for heating on cold winter days. 

In summary, under TOU electricity pricing, the degree of a price change in peak-demand hours, not just its existence, still matters to residential consumers' electricity consumption. The empirical results above suggest that the opposite directional changes in the two channels of electricity consumption make Irish households appear insensitive to the time-varying price structure. In other words, their high sensitivity to TOU prices is revealed only when their electricity consumption is disaggregated. Together with the empirical findings in previous sections, the results imply that three simultaneously interacting factors govern the dynamics of residential electricity consumption under TOU pricing: the timing when electricity is consumed, daily HDDs, and the magnitude of price increase in the peak rate period.  
