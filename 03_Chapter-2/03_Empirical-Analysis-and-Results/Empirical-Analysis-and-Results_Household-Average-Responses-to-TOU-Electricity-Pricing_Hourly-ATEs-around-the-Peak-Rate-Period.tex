Estimating by-tariff-group ATEs around the peak rate period allows us to justify whether or not the law of demand is satisfied between the responsiveness of Irish households and the magnitudes of price changes in TOU electricity pricing.\footnote{In this paper, the effects of four different information stimuli on household electricity consumption are not of interest. \cite{The-Effect-of-Information-on-TOU-Electricity-Use:An-Irish-Residential-Study_Pon_2017} studied the effects in detail using the same datasets.} To do so, I run the following regression for each of the four tariff groups:
\begin{equation}
\begin{split}
    \textit{kWh}_{ith} \ 
    & = \ \beta_{p} \mathbb{1}\big[ \text{Treatment \& Post} \big]_{it} \ + \ \alpha_{iw} \ + \ \gamma_{tw} \ + \ \delta_{m} \ + \ \epsilon_{ith}
\end{split}
\label{Eq:Model-Specification_Hourly-Average-Treatment-Effects}
\end{equation}
Excepting the dependent variable and the parameter of interest, the econometric model above is the same as (\ref{Eq:Model-Specification_Half-Hourly-Average-Treatment-Effects}). Specifically, as the response variable, $kWh_{ith}$ that means the electricity consumption by household $i$ on the day $t$ during the hour of the day $h$ is utilized due to its better accessibility in interpretation. The point estimates of $\beta_{p}$ indicate the ATE for each of the two-hour-length intervals included in rate period $p$. Table \ref{Table:Average-Treatment-Effects-in-the-Peak-Rate-Period} summarizes the regression results. 

The measured peak-rate-period ATEs re-confirm the finding suggested in \cite{Peaking-Interest:How-Awareness-Drives-the-Effectiveness-of-Time-of-Use-Electricity-Pricing_Prest_2020}: a critical determinant of the effectiveness of TOU electricity pricing in the peak rate period is nothing more than its existence. As demonstrated in Table \ref{Table:Average-Treatment-Effects-in-the-Peak-Rate-Period}, the estimated ATEs for the peak-demand hours generally follow the law of demand. In other words, the reductions in household demand for electricity in the peak rate period grow with the degree of price changes in that period. But the marginal gain of the time-varying price structure is diminishing. 

Interestingly, the law of demand does not hold in both the pre- and post-peak intervals. In spite of the price drops in those intervals, compared to the flat rate of 14.1 cents per kWh, the treated households reduced their electricity consumption. Although the mechanism that caused the changes in residential electricity consumption is not explicit, such changes evidently suggest that the households assigned to the treatment group adjusted their electricity consumption not only prior to but also following the price spikes in the peak rate period. That is, the TOU tariffs have some spillover effects on household demand for electricity in the off-peak intervals. 

The results discussed above collectively imply that in and near peak-demand hours, at least one of the two distinct sources of electricity savings from TOU pricing, temperature-control- and non-temperature-control-related electricity consumption, is driven by the magnitude of tariff changes in the peak rate period. Motivated by this implication, the relative responsiveness of the two drivers of electricity savings to the TOU tariff structures is quantified in the following section. 

%\begin{table}[!th]
\caption{Average Treatment Effects in the Peak Rate Period}
\centering
\small
\begin{tabular}{@{\extracolsep{20pt}}lcccc} 
\\[-5.5ex]
\hline \hline
\\[-3.0ex] 
 & \multicolumn{4}{c}{Hourly Electricity Consumption  (kWh/Hour)} \\
\\[-3.0ex]  
\cline{2-5} 
\\[-3.0ex]
% & Tariff A & Tariff B & Tariff C & Tariff D \\ 
%\\[-4.0ex]
 & (1) & (2) & (3) & (4) \\
\\[-3.0ex] 
\hline
\\[-2.0ex] 
$\mathbb{1}$[Treatment \& Post] & $-$0.136$^{***}$ & $-$0.168$^{***}$ & $-$0.161$^{***}$ & $-$0.210$^{***}$ \\ 
 & (0.015) & (0.023) & (0.015) & (0.023) \\ 
 & & & & \\ 
\hline
\\[-2.0ex] 
Tariff Group & A & B & C & D \\
FEs: Household by Half-Hourly Time Window & Yes & Yes & Yes & Yes \\ 
FEs: Day of Week by Half-Hourly Time Window & Yes & Yes & Yes & Yes \\ 
FEs: Month of Year & Yes & Yes & Yes & Yes \\ 
Observations & 1,771,600 & 1,147,240 & 1,795,680 & 1,155,840 \\ 
Adjusted R$^{2}$ & 0.360 & 0.376 & 0.362 & 0.360 \\ 
\\[-2.0ex]
\hline \hline
\\[-4.5ex] 
\end{tabular}
\begin{tablenotes}
    \small
    \textit{Note}: (...) 
\end{tablenotes}
\label{Table:Average-Treatment-Effects-in-the-Peak-Rate-Period}
\end{table}

