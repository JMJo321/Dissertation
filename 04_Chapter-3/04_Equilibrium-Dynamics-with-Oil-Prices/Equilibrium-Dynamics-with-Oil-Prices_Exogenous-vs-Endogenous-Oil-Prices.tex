This section examines the equilibrium path of drilling for each of the exogenous and endogenous oil prices. Endogenizing the time path of oil prices makes the probability of drilling at time $t = 0$ decrease due to the initial production (i.e., $Q_{0} > 0$). Since increasing drilling, and thus production, causes oil prices endogenously determined to fall, there would be no incentive to rapidly raise the rate of drilling. For these reasons, drilling under endogenous oil prices (denoted $D_{t}^{en}$) would be small relative to that under exogenous oil prices (denoted $D_{t}^{ex}$) for some period after $t = 0$. But at some time point, $D_{t}^{en}$ would be larger than $D_{t}^{ex}$ because of the lower level of undrilled reserves in the exogenous-price case. Figure \ref{Figure:Time-Paths-for-Drilling-and-Reserves-under-Endogenous-and-Exogenous-Oil-Prices}, which shows the time paths for drilling and the remaining reserves for each type of oil price, supports the predictions.
