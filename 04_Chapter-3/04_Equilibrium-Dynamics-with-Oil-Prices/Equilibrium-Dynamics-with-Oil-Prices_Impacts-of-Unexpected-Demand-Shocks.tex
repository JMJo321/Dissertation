The Implicit Function Theorem (IFT) allows us to predict the impact of sudden demand shocks on the equilibrium paths for drilling and production. Applying the IFT to equation (\ref{Equation:Firms-Problem_Euler-Equation}) yields
\begin{equation}
\begin{split}
    % \frac{ \ f_{ik} \ + \ \lambda_{a} \sigma \big( \gamma \ - \ \ln(1 - Pr_{k}) \big) \ }{\rho} \
    % & = \ \left\{ \psi_{i1k} \ + \ \sigma \big( \gamma \ - \ \ln(Pr_{k}) \big) \right\} \ - \ \sigma \big( \gamma \ - \ \ln(1 - Pr_{k}) \big) \\
    % \frac{1}{\rho} \frac{\partial f_{ik}}{\partial p_{k}} \ + \ \frac{\lambda_{a} \sigma}{\rho (1 - Pr_{k})} \frac{\partial Pr_{k}}{\partial p_{k}} \
    % & = \ \frac{\partial \psi_{i1k}}{\partial p_{k}} \ - \ \frac{\sigma}{Pr_{k}} \frac{\partial Pr_{k}}{\partial p_{k}} \ - \ \frac{\sigma}{1 - Pr_{k}} \frac{\partial Pr_{k}}{\partial p_{k}} \\
    % \left( \frac{\lambda_{a} \sigma}{\rho (1 - Pr_{k})} \ + \ \frac{\sigma}{Pr_{k}} \ + \ \frac{\sigma}{1 - Pr_{k}} \right) \frac{\partial Pr_{k}}{\partial p_{k}} \
    % & = \ -\frac{1}{\rho} \frac{\partial f_{ik}}{\partial p_{k}} \ + \ \frac{\partial \psi_{i1k}}{\partial p_{k}} \\
    \frac{\partial Pr_{k}}{\partial p_{k}} \
    & = \ \left\{ \frac{\rho Pr_{k} (1 - Pr_{k})}{\sigma (\lambda_{a} Pr_{k} + \rho)} \right\} \left( \frac{\partial \psi_{i1k}}{\partial p_{k}} \ - \ \frac{1}{\rho} \frac{\partial f_{ik}}{\partial p_{k}} \right).
\end{split}
\label{Equation:Equilibrium-Paths_Applying-IFT}
\end{equation}
This equation suggests that an unexpected positive price shock will lead to a higher drilling rate if the cost-shock-induced incremental gains from drilling are better than those from waiting. 

Figure \ref{Figure:Equilibrium-Paths-under-Unexpected-Price-Shocks} depicts how the paths of drilling probability, drilling, reserves, and oil price respond to two unanticipated demand shocks.\footnote{Given parameter values utilized for this simulation, the condition for the positive relationship between drilling probability and unexpected price shocks in (\ref{Equation:Equilibrium-Paths_Applying-IFT}) (i.e., $\partial \psi_{i1k} / \partial p_{k} - (1/\rho)(\partial f_{ik} / \partial p_{k}) > 0$) holds.} The first negative demand shock causes drilling probability discontinuously decreases. Due to the reduction in drilling probability, drilling demonstrates a discontinuous decrease too. Moreover, the negative demand shock also reduces the depletion rate of the reserves. As shown in the last panel in the figure, the oil price jumps down on impact after the negative demand shock, then gradually rises.\footnote{Drilling, and thus production, rapidly diminishes for a while after $t = 0$. Then, its rate of change gradually decreases, and drilling eventually converges to a lower bound. In other words, the time path of drilling has a convex profile. Because equation (\ref{Equation:DCDP-Model_Oil-Prices}) determines the oil price at time $t$, the time path for the endogenous oil price is a concave curve.} The later positive demand shock induces the opposite reactions in the equilibrium paths. 
