To examine how the firm's drilling activity at well sites of heterogeneous quality responds to unexpected demand shocks, we now assume that potential drilling locations can be of low quality, in which case a well's production is $\alpha^{L}$ or of high quality, in which case production is $\alpha^{H} \ (> \alpha^{L})$. The firm's choice-specific instantaneous payoff is then $\psi_{i1k}(p_{k}) = \alpha^{g} p_{k} - c$, $g \in \mathcal{Q} = \{ L, H \}$.\footnote{In the formulation, we implicitly normalize the oil production from a low-quality well location to 1.} 

Figure \ref{Figure:Heterogeneous-Impacts-of-Unexpected-Price-Shocks-on-Equilibrium-Paths} illustrates the results from a simulation. As discussed in Section \ref{C3-SubSubSection:The-Role-of-Geological-Quality-in-Horizontal-Drilling}, our empirical analysis reveals that fracking firms in North Dakota more significantly reduced drilling at low-quality well sites, more than at the high-quality ones, when experiencing sharp oil price declines. The time paths for drilling and production presented in the figure clearly show that the model-predicted elasticity of drilling s greater on low-quality well sites, just as illustrated in Figure \ref{Figure:High-Sensitivity-of-Firm-Level-Low-Quality-Well-Drilling}.\footnote{The term $Pr_{k}(1 - Pr_{k})$ in equation (\ref{Equation:Equilibrium-Paths_Applying-IFT}) is a parabolic curve that goes to zero as $Pr_{k}$ approaches to zero or one and that has its maximum value at $Pr_{k} = 1/2$. These properties of the term suggest the possibility that the drilling of high-quality well sites is less responsive to a demand shock.} As demonstrated in the third panel of the figure, per-drilling production, i.e., productivity, increases discontinuously due to the negative demand shock. Consequently, as described in the fourth panel, per-drilling oil production is also improved. 
