Figure \ref{Figure:Heterogeneous-Impacts-of-Unexpected-Price-Shocks-on-Equilibrium-Paths} illustrates how the firm's drilling activity at well sites of heterogeneous quality responds to unexpected demand shocks. As discussed in Section \ref{C3-SubSubSection:The-Role-of-Geological-Quality-in-Horizontal-Drilling}, our analysis reveals that fracking firms in North Dakota more significantly reduced drilling at low-quality well sites, more than at the high-quality ones, when experiencing sharp oil price declines. The time paths for drilling and production presented in the figure clearly indicate that our model accurately emulates the firm-level drilling activity empirically observed in North Dakota, which are provided in Figure \ref{Figure:High-Sensitivity-of-Firm-Level-Low-Quality-Well-Drilling}.\footnote{The term $Pr_{k}(1 - Pr_{k})$ in equation (\ref{Equation:Equilibrium-Paths_Applying-IFT}) is a parabolic curve that goes to zero as $Pr_{k}$ approaches to zero or one and that has its maximum value at $Pr_{k} = 1/2$. These properties of the term suggest the possibility that the drilling of high-quality well sites is less responsive to a demand shock.} As demonstrated in the third panel of the figure, the share of drilling from high-quality well sites increases discontinuously due to the negative demand shock. Consequently, as described in the fourth panel, per-drilling oil production is also improved. 