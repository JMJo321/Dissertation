To examine how the firm's drilling activity at well sites of heterogeneous quality responds to unexpected demand shocks, we suppose that a particular well location $i$ has a quality type that falls in one from the quality set $\mathcal{Q} = \{ L(ow), H(igh) \}$. Furthermore, we also assume that when the well location is not drilled yet, the choice-specific instantaneous payoff $\psi_{i \alpha k}$ takes the following functional form:
\begin{align}
    \psi_{iak}(p_{k}, a; \boldsymbol{\theta}_{\psi}) \
    & = \ \big[ \big\{ g_{i}^{L} \ + \ (1 - g_{i}^{L}) \alpha^{H} \big\} p_{k} \ - \ c \big] a,
\label{Equation:Firms-Problem_Instantaneous-Payoff-for-Heterogeneous-Quality}
\end{align}
where $g_{i}^{L}$ is a binary indicator with the value of one when well location $i$ is a low-quality well site and $\alpha^{H}$, which is greater than 1, is the normalized oil production from the site $i$ whose quality type is $H$.\footnote{In the formulation, we implicitly normalize the oil production from a well location to 1.} 

Figure \ref{Figure:Heterogeneous-Impacts-of-Unexpected-Price-Shocks-on-Equilibrium-Paths} illustrates the results from a simulation. As discussed in Section \ref{C3-SubSubSection:The-Role-of-Geological-Quality-in-Horizontal-Drilling}, our empirical analysis reveals that fracking firms in North Dakota more significantly reduced drilling at low-quality well sites, more than at the high-quality ones, when experiencing sharp oil price declines. The time paths for drilling and production presented in the figure clearly show that the model-predicted elasticity of drilling s greater on low-quality well sites, just as illustrated in Figure \ref{Figure:High-Sensitivity-of-Firm-Level-Low-Quality-Well-Drilling}.\footnote{The term $Pr_{k}(1 - Pr_{k})$ in equation (\ref{Equation:Equilibrium-Paths_Applying-IFT}) is a parabolic curve that goes to zero as $Pr_{k}$ approaches to zero or one and that has its maximum value at $Pr_{k} = 1/2$. These properties of the term suggest the possibility that the drilling of high-quality well sites is less responsive to a demand shock.} As demonstrated in the third panel of the figure, per-drilling production, i.e., productivity, increases discontinuously due to the negative demand shock. Consequently, as described in the fourth panel, per-drilling oil production is also improved. 
