In this section, we study the impact of unanticipated price shocks on the equilibrium paths for drilling and production under exogenously given oil prices. Applying the Implicit Function Theorem to equation (\ref{Equation:Firms-Problem_Euler-Equation}) yields
\begin{equation}
\begin{split}
    % \frac{ \ f_{ik} \ + \ \lambda_{a} \sigma \big( \gamma \ - \ \ln(1 - Pr_{k}) \big) \ }{\rho} \
    % & = \ \left\{ \psi_{i1k} \ + \ \sigma \big( \gamma \ - \ \ln(Pr_{k}) \big) \right\} \ - \ \sigma \big( \gamma \ - \ \ln(1 - Pr_{k}) \big) \\
    % \frac{1}{\rho} \frac{\partial f_{ik}}{\partial p_{k}} \ + \ \frac{\lambda_{a} \sigma}{\rho (1 - Pr_{k})} \frac{\partial Pr_{k}}{\partial p_{k}} \
    % & = \ \frac{\partial \psi_{i1k}}{\partial p_{k}} \ - \ \frac{\sigma}{Pr_{k}} \frac{\partial Pr_{k}}{\partial p_{k}} \ - \ \frac{\sigma}{1 - Pr_{k}} \frac{\partial Pr_{k}}{\partial p_{k}} \\
    % \left( \frac{\lambda_{a} \sigma}{\rho (1 - Pr_{k})} \ + \ \frac{\sigma}{Pr_{k}} \ + \ \frac{\sigma}{1 - Pr_{k}} \right) \frac{\partial Pr_{k}}{\partial p_{k}} \
    % & = \ -\frac{1}{\rho} \frac{\partial f_{ik}}{\partial p_{k}} \ + \ \frac{\partial \psi_{i1k}}{\partial p_{k}} \\
    \frac{\partial Pr_{k}}{\partial p_{k}} \
    & = \ \left\{ \frac{\rho Pr_{k} (1 - Pr_{k})}{\sigma (\lambda_{a} Pr_{k} + \rho)} \right\} \left( \frac{\partial \psi_{i1k}}{\partial p_{k}} \ - \ \frac{1}{\rho} \frac{\partial f_{ik}}{\partial p_{k}} \right).
\end{split}
\label{Equation:Equilibrium-Paths_Applying-IFT}
\end{equation}
This equation suggests that a positive price shock increases the probability of drilling when the marginal change in the choice-specific payoff is larger than in the current-value flow payoff. Therefore, an unexpected positive price shock leads to a higher drilling rate if drilling is better than waiting at the limit. 

Figure \ref{Figure:Equilibrium-Paths-under-Unexpected-Price-Shocks} depicts a time path of exogenous oil prices, including two unanticipated price shocks. The first negative price shock causes a reduction in the probability of drilling, being predicted above.\footnote{Given parameter values utilized for this simulation, the condition for the positive relationship between drilling probability and unexpected price shocks in (\ref{Equation:Equilibrium-Paths_Applying-IFT}) (i.e., $\partial \psi_{i1k} / \partial p_{k} - (1/\rho)(\partial f_{ik} / \partial p_{k}) > 0$) holds.} Due to the reduction in drilling probability, drilling demonstrates a discontinuous decrease too. Moreover, the negative price shock also reduces the depletion rate of the reserves. The later positive price shock induces the opposite reactions in the equilibrium paths. 

Figure \ref{Figure:Heterogeneous-Impacts-of-Unexpected-Price-Shocks-on-Equilibrium-Paths} illustrates how the firm's drilling activity on well sites with heterogeneous quality responds to unexpected price shocks. As discussed in Section \ref{C3-SubSubSection:Correlation-between-Oil-Prices-and-Horizontal-Drilling-in-ND}, our analysis reveals that fracking firms in North Dakota more significantly reduced drilling of low-quality well sites, compared to one of high-quality well locations, when experiencing sharp oil price declines. The time paths for drilling and production presented in the figure clearly indicate that our model accurately emulates firm-level drilling activity empirically observed in North Dakota, which are provided in Figure \ref{Figure:High-Sensitivity-of-Firm-Level-Low-Quality-Well-Drilling}. As demonstrated in the third panel of the figure, the share of drilling from high-quality well sites discontinuously increases due to the negative price shock. Consequently, as described in the fourth panel, per-drilling oil production is also improved. 
