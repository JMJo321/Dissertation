We can also obtain the equilibrium path for drilling under endogenous oil prices. Under the assumption that $p_{0} = \bar{p}$, endogenizing the time path of oil prices makes the probability of drilling at time $t = 0$ decrease due to the initial production (i.e., $Q_{0} > 0$). In addition, since increasing drilling, and therefore production, causes oil prices endogenously determined to decrease, there would be no incentive to raise the rate of drilling rapidly. For these reasons, drilling under endogenous oil prices (denoted $D_{t}^{en}$) would be small relative to that under exogenous oil prices (denoted $D_{t}^{ex}$) for some time after $t = 0$. But at some time point, $D_{t}^{en}$ would be larger than $D_{t}^{ex}$ because of the lower level of reserves undrilled in the exogenous-price case. The time paths for drilling and the remaining reserves in Figure \ref{Figure:Time-Paths-for-Drilling-and-Undrilled-Reserves-under-Endogenous-and-Exogenous-Oil-Prices} support the predictions. 
