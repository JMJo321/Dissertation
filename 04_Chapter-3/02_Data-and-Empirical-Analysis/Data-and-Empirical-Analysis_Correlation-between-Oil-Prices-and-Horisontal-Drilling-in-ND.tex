Figure \ref{Figure:Time-Series-of-the-Number-of-Drilling-in-North-Dakota} shows how well drilling in North Dakota evolved between 2009 and 2020. As clearly illustrated, well completions dramatically increased from the beginning of 2010. In addition, the explosive growth of well drilling in that region was driven mainly by horizontal drilling. 

According to the figure, it is evident that drilling horizontal wells in North Dakota is closely correlated with oil prices, especially after 2009. On the whole, oil prices significantly increased between 2009 and 2010 and remained high until mid-2014. And there was a sharp plunge in oil prices from mid-2014 to the end of 2015. The number of horizontal wells drilled in that time range significantly declined too. And when oil prices gradually climbed between 2016 and 2020, North Dakota's drilling activities also recovered. To summarize, oil prices and the number of horizontal drilling in North Dakota seem to be positively correlated. Importantly, such a positive correlation between oil prices and horizontal drilling in the state suggests that fracking firms' drilling decisions strongly depend on oil prices. In Section \ref{C3-SubSubSection:The-Role-of-Geological-Quality-in-Horizontal-Drilling}, we show that their drilling decisions are linked with oil prices through the geological features of well sites.
