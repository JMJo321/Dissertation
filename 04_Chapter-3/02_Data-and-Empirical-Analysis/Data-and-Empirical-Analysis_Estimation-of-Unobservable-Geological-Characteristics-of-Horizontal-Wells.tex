\textit{\textbf{Estimation of Unobservable Geological Characteristics of Horizontal Wells}} --- Not all well-specific information on geological features is available to econometricians, including us. The NDGS geological survey data only include estimates of four different measurements of geological properties at a given location. Because the geospatial data was published to the public in 2008, it is likely that fracking firms, whose objective is to maximize their profits, have already exploited the contents of the maps. And as discussed in \cite{Learning-where-to-drill_Agerton_2020}, learning about the spatial distribution of deposits by drilling wells is one of three economic factors that govern firms' where-to-drill decisions. So, it is reasonable to suppose that hydraulic frackers have private information about the Bakken area's spatial distribution of geological characteristics, which is not accessible to researchers. 

The geological characteristics observed only by fracking firms play two different roles in their well-drilling-related decisions. The unobservables are choice variables themselves. For a given output (i.e., oil or gas) price, frackers need to locate their wells in a spot with good enough unobservable geological features (i.e., a well site with good quality) to achieve sufficient oil or gas production in order to make their cash flow positive.\footnote{Refer to a blog post of Bloomberg Finance L.P. (\href{https://about.bnef.com/blog/economics-u-s-shale-oil-production}{Economics of U.S. Shale Oil Production}) that discusses basin-level break-even costs of horizontal wells.} Moreover, the unknown features determine optimal input choices. For given input prices, the cost required for an incremental well would be a function of the well's characteristics unobserved by researchers. Therefore, estimating the unobservable quality of rocks into which wells are drilled is necessary to uncover their impact on drilling decisions made by hydraulic fracturing firms in North Dakota. 

Following \cite{The-Economics-of-Time-Limited-Development-Options_2020_Herrnstadt-Kellogg-and-Lewis}, we employ Robinson's partially linear model to determine the unobservable quality of the horizontal wells completed between 2009 and 2018. We first specify the oil production from a horizontal well as
\begin{equation}
\begin{split}
    \log \left( y_{i} \right) \ 
    & = \ \log \left( \boldsymbol{X}_{i} \right)' \boldsymbol{\beta} \ - \ \lambda \left( longitude_{i}, latitude_{i} \right) \ + \ \epsilon_{i}.
\end{split}
\label{Equation:Production-Function}
\end{equation}
In this specification, $y_{i}$ are horizontal well $i$'s cumulative oil production at its cumulative production month 24.\footnote{That is, unobservable geological features are estimated cross-sectionally in our estimation.} And the covariate vector $\boldsymbol{X}_{i}$ for well $i$ includes simulation inputs (i.e., fluid volume, proppant weight, and length of horizontal drilling), cumulative producing days, and observable geological characteristics (i.e., thickness, total organic contents, and thermal maturity). In addition, $\lambda(longitude_{i}, latitude_{i})$, a nonparametric function of each well's coordinates, means well $i$'s unobservable resource quality that partly determines its productivity. Lastly, $\epsilon_{i}$ are error terms. As implied by the specification, we implicitly attribute a given horizontal well $i$'s productivity unexplained by simulation inputs, cumulative producing days, and observable geological features to unobservable resource quality. 

We run the following econometric model with horizontal wells whose cumulative production month equals or exceeds 24 to identify the deterministic part of the production function (i.e., $\boldsymbol{\beta}$):
\begin{equation}
\begin{split}
    \log \left( y_{i} \right) \ - \ \widehat{m}_{y_{i}} \ 
    & = \ \left( \log \left( \boldsymbol{X}_{i} \right) \ - \ \widehat{\boldsymbol{m}}_{\boldsymbol{X}_{i}} \right)' \boldsymbol{\beta} \ + \ \epsilon_{i}.
\end{split}
\label{Equation:Partially-Linear-Model}
\end{equation}
In this model, $\widehat{m}_{y_{i}}$ are predictions from a non-parametric regression of $\log \left( y_{i} \right)$ on well $i$'s coordinates $(longitude_{i}, latitude_{i})$. Predictions $\widehat{\boldsymbol{m}}_{\boldsymbol{X}_{i}}$ are obtained from different nonparametric regressions whose dependent and independent variables are $\log \left( \boldsymbol{X}_{i} \right)$ and $(longitude_{i}, latitude_{i})$, respectively. The values of primary interest $\widehat{\lambda}_{i}$ (i.e., the unobservable geological quality of horizontal well $i$) are estimated as follows\footnote{For details of Robinson's difference estimator, refer to \textit{9.7.3 Partially Linear Model} in \cite{MicroEconometrics-Methods-and-Applications_Cameron-and-Trivedi_2005}}\footnote{The fact that no observation for horizontal wells not drilled yet exists in our sample implies that the estimated unobservable resource quality $\widehat{\lambda}_{i}$ is biased upward.}:
\begin{equation}
\begin{split}
    \widehat{\lambda}_{i} \
    & = \ \widehat{m}_{y_{i}} \ - \ \widehat{\boldsymbol{m}}_{\boldsymbol{X}_{i}}' \widehat{\boldsymbol{\beta}}
\end{split}
\label{Equation:Estimates}
\end{equation}
Figure \ref{Figure:Spatial-Distribution-of-the-Estimated-Geological-Characteristic-by-Year} shows the spatial distribution of the estimated quality.
