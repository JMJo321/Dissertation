\par
\vspace{0.3cm}
\noindent
\textit{\textbf{Simultaneous Drilling of Horizontal Wells with Heterogeneous Geological Qualities}} ---
Figure \ref{Figure:Simultaneous-Drilling-of-Horizontal-Wells-with-Heterogeneous-Geological-Quality}, summarizing the estimated geological qualities of horizontal wells in scatter plots, clearly demonstrates that horizontal wells with a range of geological qualities were drilled simultaneously in the Bakken region of North Dakota. And as shown in Figure \ref{Figure:High-Sensitivity-of-Firm-Level-Low-Quality-Well-Drilling}, simultaneous drilling is observed even at the firm level. 

The simultaneous drilling empirically discovered contradicts the well-known least-cost-first extraction rule in nonrenewable resource extraction.\footnote{The cost that matters here is the marginal cost. And the marginal cost consists of two distinct costs: the marginal cost of drilling a new well and the marginal cost of extracting oil from existing wells.} According to the rule derived from the canonical Hotelling model, deposits of an exhaustible resource should be exploited in strict order, beginning with the lowest cost deposit. Because a larger estimate of the unobservable geological quality implies higher ultimate oil production, if the rule holds, wells with larger estimates, thus having lower per-barrel marginal cost, should be first extracted. Those figures, however, do not show the theoretical prediction at all. 

The interesting empirical finding of the simultaneous drilling of horizontal wells with heterogeneous quality could be due to limited extraction capacity constraints. According to \cite{Extraction-Capacity-and-the-Optimal-Order-of-Extraction_Holland_2003}, the extraction of a high-cost deposit of a natural resource can be prior to that of a lower-cost one if the capacity of resource extraction is limited. Therefore, the empirical result likely indicates the need to add extraction-capacity-related constraints to our theoretical framework. 
