\par
\vspace{0.3cm}
\noindent
\textit{\textbf{The High Sensitivity of Drilling Low-quality Horizontal Wells to Negative Price Shocks}} --- In addition to the simultaneous drilling of horizontal wells with heterogeneous qualities, Figure \ref{Figure:Simultaneous-Drilling-of-Horizontal-Wells-with-Heterogeneous-Geological-Quality} demonstrates an interesting point: the remarkable responsiveness of low-quality well drilling to sharp oil price declines from mid-2014 to the end of 2015. The high sensitivity of drilling activities for low-quality horizontal wells to negative price shocks during the period is also pronounced even at the firm level, as illustrated in Figure \ref{Figure:High-Sensitivity-of-Firm-Level-Low-Quality-Well-Drilling}. 

Figure \ref{Figure:Held-by-Production-vs-Non-Held-by-Production-Horizontal-Well-Drilling} shows that drilling associated with held-by-production did not drive the relationship between oil prices and low-quality well drilling, especially between mid-2014 and the end of 2015. The upper panel in the figure illustrates by-quality time series of the number of drilled horizontal wells that were the first ones for a given section.\footnote{In the Public Land Survey System, a \textit{section}, which is one of 36 sections in a township, is a one-mile-square area.} In other words, the panel demonstrates time trends of horizontal well drilling suspected to be well drilling related to held-by-production. The relatively high drilling rate of low-quality wells, especially between mid-2011 and mid-2013, seems to be consistent with the empirical result of \cite{The-Economics-of-Time-Limited-Development-Options_2020_Herrnstadt-Kellogg-and-Lewis}: firms bound to a lease contract including use-it-or-lose-it requirements tend to drill low-productivity well locations just before the first lease expires.

The evolving pattern of drilling for each of the three quality levels presented in the lower panel of Figure \ref{Figure:Held-by-Production-vs-Non-Held-by-Production-Horizontal-Well-Drilling}, regarded as post-held-by-production drilling, shows completely different movements from those in the upper panel. Until 2014, horizontal wells of heterogeneous quality were drilled equally and at the same growth rate. But drilling of low-productivity horizontal wells more sensitively reacted to negative price shocks between mid-2014 and 2015, compared to drilling medium- and high-quality wells. The new theoretical approach, required to rationalize the simultaneous drilling of wells with heterogeneous qualities, needs to explain the high sensitivity of low-quality well drilling.