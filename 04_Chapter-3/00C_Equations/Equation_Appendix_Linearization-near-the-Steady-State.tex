Using a Taylor series expansion, $\dot{R}_{t}$ can be approximated near the steady state ($R_{ss}$, $\pi_{ss}$):
\begin{equation*}
\begin{split}
    \dot{R}_{t} \ 
    & \approx \ (-R_{ss} Pr_{ss} \ + \ E) \\
    & \hspace{1.0cm} + \ \frac{\partial}{\partial R_{t}} (-R_{t} Pr_{t} \ + \ E) (R_{t} \ - \ R_{ss}) \ + \ \frac{\partial}{\partial \pi_{t}} (-R_{t} Pr_{t} \ + \ E) (\pi_{t} \ - \ \pi_{ss}) \\
    & = \ 0 \ + \ \left( -Pr_{t} \ - \ \frac{\partial Pr_{t}}{\partial R_{t}} R_{t} \right) (R_{t} \ - \ R_{ss}) \ + \ \left( -\frac{\partial Pr_{t}}{\partial \pi_{t}} R_{t} \right) (\pi_{t} \ - \ \pi_{ss}) \\
    & = \ \left( -Pr_{t} \ - \ \frac{\partial Pr_{t}}{\partial R_{t}} R_{t} \right) (R_{t} \ - \ R_{ss}) \ + \ \left( -\frac{\partial Pr_{t}}{\partial \pi_{t}} R_{t} \right) (\pi_{t} \ - \ \pi_{ss}).
\end{split}
\end{equation*}
In the same way, the linear approximation of $\dot{\pi}_{t}$ near the steady state ($R_{ss}$, $\pi_{ss}$) is given by
\begin{equation*}
\begin{split}
    \dot{\pi}_{t} \ 
    & \approx \ \left(r \pi_{ss} \ - \ \sigma \big( \gamma \ - \ \ln(1 - Pr_{ss}) \big) \right) \\
    & \hspace{1.0cm} + \ \frac{\partial}{\partial R_{t}} \left(r \pi_{t} \ - \ \sigma \big( \gamma \ - \ \ln(1 - Pr_{t}) \big) \right) (R_{t} \ - \ R_{ss}) \\
    & \hspace{1.0cm} + \ \frac{\partial}{\partial \pi_{t}} \left(r \pi_{t} \ - \ \sigma \big( \gamma \ - \ \ln(1 - Pr_{t}) \big) \right) (\pi_{t} \ - \ \pi_{ss}) \\
    & = \ 0 \ + \ \left( -\frac{\sigma}{1 - Pr_{t}} \frac{\partial Pr_{t}}{\partial R_{t}} \right) (R_{t} \ - \ R_{ss}) \ + \ \left( r \ - \ \frac{\sigma}{1 - Pr_{t}} \frac{\partial Pr_{t}}{\partial \pi_{t}} \right) (\pi_{t} \ - \ \pi_{ss}) \\
    & = \ \left( -\frac{\sigma}{1 - Pr_{t}} \frac{\partial Pr_{t}}{\partial R_{t}} \right) (R_{t} \ - \ R_{ss}) \ + \ \left( r \ - \ \frac{\sigma}{1 - Pr_{t}} \frac{\partial Pr_{t}}{\partial \pi_{t}} \right) (\pi_{t} \ - \ \pi_{ss}).
\end{split}
\end{equation*}
From those two approximations, the linearized system near the steady state ($R_{ss}$, $\pi_{ss}$) is
\begin{equation*}
\begin{split}
    \begin{pmatrix}
        \dot{R}_{t} \\
        \dot{\pi}_{t}
    \end{pmatrix} \ 
    & = \ 
    \begin{pmatrix}
        -Pr_{t} \ - \ \frac{\partial Pr_{t}}{\partial R_{t}} R_{t} & -\frac{\partial Pr_{t}}{\partial \pi_{t}} R_{t} \\
        -\frac{\sigma}{1 - Pr_{t}} \frac{\partial Pr_{t}}{\partial R_{t}} & r \ - \ \frac{\sigma}{1 - Pr_{t}} \frac{\partial Pr_{t}}{\partial \pi_{t}}
    \end{pmatrix}
    \begin{pmatrix}
        R_{t} \ - \ R_{ss} \\
        \pi_{t} \ - \ \pi_{ss}
    \end{pmatrix} \\
    & = \
    \begin{pmatrix}
        [1] & [2] \\
        [3] & [4]
    \end{pmatrix}
    \begin{pmatrix}
        R_{t} \ - \ R_{ss} \\
        \pi_{t} \ - \ \pi_{ss}
    \end{pmatrix}.
\end{split}
\end{equation*}

Applying the Implicit Function Theorem to necessary condition (\ref{Equation:Social-Planners-Problem_Meaning-of-Costate-Variable}), we can obtain the followings:
\begin{equation*}
\begin{cases}
    \begin{split}
        \frac{\partial Pr_{t}}{\partial R_{t}} \ 
        & = \ - \frac{ \big( \alpha^{2} u''(\alpha R_{t} Pr_{t}) - c''(R_{t} Pr_{t}) \big) Pr_{t}}{ \ \big( \alpha^{2} u''(\alpha R_{t} Pr_{t}) - c''(R_{t} Pr_{t}) \big) R_{t} \ - \ \frac{\sigma}{Pr_{t}(1 - Pr_{t})} \ } \\
        \frac{\partial Pr_{t}}{\partial \pi_{t}} \
        & = \ \frac{1}{ \ \big( \alpha^{2} u''(\alpha R_{t} Pr_{t}) - c''(R_{t} Pr_{t}) \big) R_{t} \ - \ \frac{\sigma}{Pr_{t}(1 - Pr_{t})} \ } 
    \end{split}
\end{cases}
\end{equation*}
Because $\alpha^{2} u''(\alpha R_{t} Pr_{t}) - c''(R_{t} Pr_{t}) < 0$ in our setting, $\partial Pr_{t} / \partial R_{t} < 0$ and $\partial Pr_{t} / \partial \pi_{t} < 0$.

In the coefficient matrix,
\begin{equation*}
\begin{split}
    [1] \ 
    & : \ \ -Pr_{t} \ - \ \frac{\partial Pr_{t}}{\partial R_{t}} R_{t} \ = \frac{\frac{\sigma}{Pr_{t}(1 - Pr_{t})}}{ \ \big( \alpha^{2} u''(\alpha R_{t} Pr_{t}) - c''(R_{t} Pr_{t}) \big) R_{t} \ - \ \frac{\sigma}{Pr_{t}(1 - Pr_{t})} \ } \ < \ 0; \\
    [2] \
    & : \ \ -\frac{\partial Pr_{t}}{\partial \pi_{t}} R_{t} \ > \ 0; \\
    [3] \
    & : \ \ -\frac{\sigma}{1 - Pr_{t}} \frac{\partial Pr_{t}}{\partial R_{t}} \ > \ 0; \ \text{and} \\
    [4] \ 
    & : \ \ r \ - \ \frac{\sigma}{1 - Pr_{t}} \frac{\partial Pr_{t}}{\partial \pi_{t}} \ > \ 0.
\end{split}
\end{equation*}
Therefore, the determinant of the coefficient matrix clearly has a negative value (i.e., $[1] \times [4] - [2] \times [3] < 0$). 