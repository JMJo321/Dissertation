In our framework, the instantaneous Bellman equation can be approximated as follows:
\begin{equation*}
\begin{split}
    (1 + \tau \rho) V_{ik} (t) \
    & \approxeq \ \tau (f_{ik} \ + \ \lambda_{a} \tilde{f}_{ik}) \\
    & \hspace{0.7cm} + \ \tau \lambda_{a} E\Big[ \underset{a \in \mathcal{A}}{\max} \left\{ V_{i,\ell(i, a, k)} (t + \tau) \ + \ \psi_{iak} \ + \ \tilde{\psi}_{iak} \ + \ \epsilon_{iak} \right\} \Big] \\
    & \hspace{0.7cm} + \ \sum_{\ell \neq k} \tau \lambda_{k\ell} V_{i\ell}(t + \tau) \\
    & \hspace{0.7cm} + \ \left\{ 1 \ - \ \tau \left( \lambda_{a} \ + \ \sum_{\ell \neq k} \lambda_{k\ell} \right) \right\} V_{ik} (t + \tau),
\end{split}
\end{equation*}
where $1 + \tau \rho$ is the discount factor for the time increment $\tau$, $\tau \lambda_{a}$ is the probability that the firm in state $k$ choose an action $a$ in an incremental time period $\tau$, and $\sum_{\ell \neq k} \tau \lambda_{k\ell}$ is the probability of moving from state $k$ to state $\ell$. The curly bracket of the fourth line in the expression, therefore, means the probability that the firm remains at state $k$. 

Rearranging terms, dividing by $\tau$, and letting $\tau \rightarrow 0$ yield a simpler expression:
\begin{equation*}
\begin{split}
    & -\big\{ V_{ik} (t + \tau) \ + \ V_{ik} (t) \big\} \ + \ \tau \rho V_{ik} (t) \ + \ \tau \left( \lambda_{a} \ + \ \sum_{\ell \neq k} \lambda_{k\ell} \right) V_{ik}(t + \tau) \\
    & \hspace{1.0cm} \approxeq \ \tau (f_{ik} \ + \ \lambda_{a} \tilde{f}_{ik}) \ + \ \tau \lambda_{a} E\Big[ \underset{a \in \mathcal{A}}{\max} \left\{ V_{i,\ell(i, a, k)} (t + \tau) \ + \ \psi_{iak} \ + \ \tilde{\psi}_{iak} \ + \ \epsilon_{iak} \right\} \Big] \\
    & \hspace{1.7cm} + \ \sum_{\ell \neq k} \tau \lambda_{k\ell} V_{i\ell}(t + \tau) \\
    & -\frac{1}{\tau} \big\{ V_{ik} (t + \tau) \ + \ V_{ik} (t) \big\} \ + \ \rho V_{ik}(t) \ + \ \left( \lambda_{a} \ + \ \sum_{\ell \neq k} \lambda_{k\ell} \right) V_{ik}(t + \tau) \\
    & \hspace{1.0cm} \approxeq \ (f_{ik} \ + \ \lambda_{a} \tilde{f}_{ik}) \ + \ \lambda_{a} E\Big[ \underset{a \in \mathcal{A}}{\max} \left\{ V_{i,\ell(i, a, k)} (t + \tau) \ + \ \psi_{iak} \ + \ \tilde{\psi}_{iak} \ + \ \epsilon_{iak} \right\} \Big] \\
    & \hspace{1.7cm} + \ \sum_{\ell \neq k} \lambda_{k\ell} V_{i\ell}(t + \tau) \\
    & -\dot{V}_{ik} (t) \ + \ \left( \rho \ + \ \lambda_{a} \ + \ \sum_{\ell \neq k} \lambda_{k\ell} \right) V_{ik}(t) \\
    & \hspace{1.0cm} = \ (f_{ik} \ + \ \lambda_{a} \tilde{f}_{ik}) \ + \ \lambda_{a} E\Big[ \underset{a \in \mathcal{A}}{\max} \left\{ V_{i,\ell(i, a, k)} (t) \ + \ \psi_{iak} \ + \ \tilde{\psi}_{iak} \ + \ \epsilon_{iak} \right\} \Big] \ + \ \sum_{\ell \neq k} \lambda_{k\ell} V_{i\ell}(t)
\end{split}
\end{equation*}