The assumption of constant oil price implies no change in state. In other words, the assumption suggests the following conditions:
\begin{itemize}
    \item
    $V_{ik}(t) \ = \ 0$;
    
    \item
    for all $\ell$, $\lambda_{k\ell} = 0$;
    
    \item
    $\tilde{f}_{ik} = 0$; and
    
    \item
    $\tilde{\psi}_{iak} = 0$.
    
\end{itemize}

Using those conditions, the firm's value function, which is presented in equation (\ref{Equation:Firms-Problem_Value-Function}), is simplified as follows:
\begin{equation*}
\begin{split}
    (\rho \ + \ \lambda_{a}) V_{ik}(t) \
    & = \ f_{ik} \ + \ \lambda_{a} E\Big[ \underset{a \in \mathcal{A}}{\max} \left\{ V_{i,\ell(i, a, k)} (t) \ + \ \psi_{iak} \ + \ \epsilon_{iak} \right\} \Big]
\end{split}
\end{equation*}
In addition, equation (\ref{Equation:Firms-Problem_Emax}) allows yielding the even more simple functional form of the firm's value function. When $a = 0$,
\begin{equation*}
\begin{split}
    V_{ik} (t) \
    & = \ \frac{ \ f_{ik} \ + \ \lambda_{a} \sigma \big( \gamma \ - \ \ln(1 - Pr_{k}) \big) \ }{\rho}.
\end{split}
\end{equation*}
In the case of $a = 1$,
\begin{equation*}
\begin{split}
    V_{ik} (t) \
    & = \ \frac{ \ f_{ik} \ + \ \lambda_{a} \left\{ \psi_{i1k} \ + \ \sigma \big( \gamma \ - \ \ln(Pr_{k}) \big) \right\} \ }{\rho \ + \ \lambda_{a}}.
\end{split}
\end{equation*}
Equating the two equations, with some algebra, allows us having the Euler equation:
\begin{equation*}
\begin{split}
    \frac{ \ f_{ik} \ + \ \lambda_{a} \sigma \big( \gamma \ - \ \ln(1 - Pr_{k}) \big) \ }{\rho} \
    & = \ \left\{ \psi_{i1k} \ + \ \sigma \big( \gamma \ - \ \ln(Pr_{k}) \big) \right\} \ - \ \sigma \big( \gamma \ - \ \ln(1 - Pr_{k}) \big).
\end{split}
\end{equation*}