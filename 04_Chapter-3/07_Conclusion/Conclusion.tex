In this paper, we first present two interesting empirical findings about the drilling decisions of hydraulic fracturing companies in North Dakota. First, the oil producers drilled horizontal wells into sites of heterogeneous quality, not in a strict order, but simultaneously. Second, the drilling of low-quality well sites was more responsive to the significant plunge in oil prices in the second half of 2014 than that of high-quality ones. 

We develop a continuous-time Discrete Choice Dynamic Programming (DCDP) to explain the empirical facts that are not captured in the economic models developed in other papers. Our theoretical framework is analytically tractable. Moreover, our economic model, in which choice-specific cost shocks at each decision opportunity allow us to avoid modeling specific constraints (e.g., capacity and transmission constraints), provides us with insightful implications for how the drilling activity of fracking firms operating in North Dakota evolves in response to oil price changes. Furthermore, in our model of optimal drilling, well-level drilling decisions naturally lead to the optimal firm-level path of drilling. 

In addition to the analytical advantages, our model is also empirically manageable. We can easily take our model to detailed well-level drilling and production data. Therefore, utilizing the empirically estimated values of structural parameters, we can perform counterfactual analysis on real-world issues in the oil and gas industry, such as a change in the severance tax rate in North Dakota. In other words, our theoretical framework can bridge Hotelling-style theoretical approaches with empirical research that relies heavily on microeconomic data. 
