Necessary conditions (\ref{Equation:Social-Planners-Problem_Necessary-Conditions_Costate-Variable}) and (\ref{Equation:Social-Planners-Problem_Meaning-of-Costate-Variable}) have important economic implications of the optimal path of well sites' depletion. Firstly,  necessary condition (\ref{Equation:Social-Planners-Problem_Necessary-Conditions_Costate-Variable}) demonstrates significant implications of $\pi_{t}$'s growth over time. We can re-express this condition as follows\footnote{Because $\dot{\pi}_{t}$ must be zero at the steady state, this equation implies that $\pi_{t} = \left\{ f_{t} \ + \ \sigma \big( \gamma \ - \ \ln(1 - Pr_{t}) \big) \right\} / r$ at the steady state. It is clear that both $f_{t}$ and $Pr_{t}$ must be a constant at the steady state. Based on this observation, we can draw the same expression for $\pi_{t}$ at the steady state from equation (\ref{Equation:Social-Planners-Problem_Necessary-Conditions_Costate-Variable-with-Integration}): $\pi_{t} = e^{rt} \left\{ f_{t} \ + \ \sigma \big( \gamma \ - \ \ln(1 - Pr_{t}) \big) \right\} \int_{t}^{\infty} e^{-r\tau} \ d\tau = \left\{ f_{t} \ + \ \sigma \big( \gamma \ - \ \ln(1 - Pr_{t}) \big) \right\} / r$.}:
\begin{equation}
\begin{split}
%    \dot{\pi}_{t} \ 
%    & = \ r \pi_{t} \ - \ \sigma \big( \gamma \ + \ \ln(1 - Pr_{t}) \big) \\
    \dot{\pi}_{t} \
    & = \ r \left\{ \pi_{t} \ - \ \frac{ \ f_{t} \ + \ \sigma \big( \gamma \ - \ \ln(1 - Pr_{t}) \big) \ }{r} \right\}.
%    \frac{\dot{\pi_{t}}}{\pi_{t}} \ + \  \frac{\sigma \big( \gamma \ - \ \ln(1 - Pr_{t}) \big)}{\pi_{t}} \ 
%    & = \ r
\end{split}
\label{Equation:Social-Planners-Problem_Growth-Rate-of-Costate-Variable}
\end{equation}
This expression clearly indicates that $\pi_{t}$ grows slower than the rate of interest $r$. The necessary condition also suggests that $\pi_{t}$ increases concavely with $Pr_{t}$ and converges in the limit, unlike the exponential growth of the shadow price in Hotelling's framework.\footnote{From the beginning of drilling (i.e., $t = 0$), $R_{t}$ decreases. If $Pr_{t}$ is maintained at the initial level, the rate of drilling will quickly converge to zero. So, $Pr_{t}$ must continuously grow to keep drilling. In equation (\ref{Equation:Social-Planners-Problem_Growth-Rate-of-Costate-Variable}), the increase in $Pr_{t}$ leads to the reduction in $\dot{\pi}_{t}$. Moreover, as discussed earlier, $Pr_{t}$ approaches to $Pr_{ss}$.} 

Secondly, necessary condition (\ref{Equation:Social-Planners-Problem_Meaning-of-Costate-Variable}) directly provides us with what $\pi_{t}$ means. In this condition, the three terms in the curly bracket collectively mean the net benefit from the output (i.e., oils) produced from the marginally drilled well site at time $t$. The last term among them is the expected value of the cost shock for the marginal well location at time $t$ when the social planner decides to drill it (i.e., $e_{1t}$). The remaining term in this condition represents the opportunity cost of drilling the marginal site at time $t$.\footnote{If the social planner decides not to drill a horizontal well into the marginal well site at time $t$, then the expected value of $\epsilon_{0t}$ conditional on $a_{t} = 0$ (i.e., $e_{0t}$) is the only gain the social planner gets from the decision.} Hence, the necessary condition indicates that the costate variable $\pi_{t}$ implies the net shadow value of the marginally drilled well site in the current-value term at time $t$. 

Necessary condition (\ref{Equation:Social-Planners-Problem_Necessary-Conditions_Costate-Variable}) also enables us to understand what $\pi_{t}$ means from a different perspective. We can re-write this necessary condition as follows:
\begin{equation}
\begin{split}
%    \dot{\pi}_{t} \ 
%    & = \ r \pi_{t} \ - \ \Big\{ f_{t} \ + \ \lambda_{a} \sigma \big( \gamma \ - \ \ln(1 - Pr_{t}) \big) \Big\} \\
%    (\dot{\pi}_{t} \ - \ r\pi_{t}) e^{-rt} \
%    & = \ - \Big\{ f_{t} \ + \ \lambda_{a} \sigma \big( \gamma \ - \ \ln(1 - Pr_{t}) \big) \Big\} e^{-rt} \\
%%    \pi_{t} e^{-rt} \
%%    & = \ \int_{t}^{\infty} e^{-r\tau} \Big\{ f_{t} \ + \ \lambda_{a} \sigma \big( \gamma \ - \ \ln(1 - Pr_{t}) \big) \Big\} \ d\tau \ + \ \mathcal{C} \\  % C = 0 from the equation for \pi_{t} at the steady state.
%%    \pi_{t} \
%%    & = \ e^{rt} \left[ \int_{t}^{\infty} e^{-r\tau} \Big\{ f_{t} \ + \ \lambda_{a} \sigma \big( \gamma \ - \ \ln(1 - Pr_{\tau}) \big) \Big\} \ d\tau \ + \ \mathcal{C} \right],
%    \pi_{t} e^{-rt} \
%    & = \ \lim_{T \to \infty} \int_{t}^{T} e^{-r\tau} \Big\{ f_{t} \ + \ \lambda_{a} \sigma \big( \gamma \ - \ \ln(1 - Pr_{t}) \big) \Big\} \ d\tau \\
    \pi_{t} \
    & = \ e^{rt} \left[ \lim_{T \to \infty} \int_{t}^{T} e^{-r\tau} \Big\{ f_{t} \ + \ \lambda_{a} \sigma \big( \gamma \ - \ \ln(1 - Pr_{\tau}) \big) \Big\} \ d\tau \right].
\end{split}
\label{Equation:Social-Planners-Problem_Necessary-Conditions_Costate-Variable-with-Integration}
\end{equation}
This equation implies that $\pi_{t}$ is the marginally undrilled well site's aggregate future payoff (i.e., the sum of the flow payoff and the expected value of $\epsilon_{0t}$'s from time $t$ and beyond), as the current value at time $t$, if the well site will remain undeveloped. Therefore, it is clear that drilling well locations is economic depletion in our framework, as extracting an exhaustible resource is in Hotelling's model. 

From the above discussions, necessary conditions (\ref{Equation:Social-Planners-Problem_Meaning-of-Costate-Variable}) and (\ref{Equation:Social-Planners-Problem_Necessary-Conditions_Costate-Variable-with-Integration}) collectively suggest that on the optimal path of drilling, the marginal undeveloped well site will be drilled at time $t$ if the net gains from drilling it at time $t$ equal the undrilled site' aggregate future gains from time $t$ and beyond. In other words, at the margin, drilling a horizontal well today is an optimal choice if its value is indifferent to the value of simply holding it forever. Indeed, this implication holds under the Hotelling framework. 

In addition, the costate variable $\pi_{t}$ can be expressed in terms of $W_{t}^{sp}$. Specifically, $\pi_{t}$, which indicates the net shadow value of the marginally drilled well site as the current value at time $t$, is the marginal welfare with respect to $D_{t}$ as shown below:
\begin{equation}
\begin{split}
	\frac{\partial W_{t}^{sp}}{\partial D_{t}} \
	& = \ \alpha u'(\alpha D_{t}) \ - \ c'(D_{t}) \ + \ \sigma \big( \gamma \ - \ \ln(Pr_{t}) \big) \ - \ \sigma \big( \gamma \ - \ \ln( 1 - Pr_{t}) \big).
\end{split}
\label{Equation:Social-Planners-Problem_Marginal-Welfare}
\end{equation} 
That is, the marginally drilled site's shadow value is the same as the marginal welfare of drilling. 

The value of $\sigma$, which is the dispersion parameter of the I.I.D. T1EV cost shocks, provides two interesting implications. First, the social planner's problem reverts to the Hotelling model of the optimal extraction of a nonrenewable resource when $\sigma$ goes to zero. Taking limits to zero for necessary conditions (\ref{Equation:Social-Planners-Problem_Necessary-Conditions_Costate-Variable}) and (\ref{Equation:Social-Planners-Problem_Meaning-of-Costate-Variable}) with the assumption of $f_{t} = 0$ yields the followings, which are identical to the two necessary conditions in Hotelling's classic model of depletion\footnote{We take the assumption of zero flow payoff because the payoff element is not introduced in Hotelling's theoretical model.}:
\begin{equation}
\begin{cases}
        \begin{split}
        \ \lim_{\sigma \to 0} \dot{\pi}_{t} \
        & = \ r\pi_{t} \\
        \ \lim_{\sigma \to 0} \pi_{t} \
        & = \ \alpha u'(\alpha \lambda_{a} R_{t} Pr_{t}) \ - \ c'(\lambda_{a} R_{t} Pr_{t}).
        \end{split}
    \end{cases}
\label{Equation:Social-Planners-Problem_Reverting-to-the-Hotelling-Model}
\end{equation}
Intuitively, the limiting case that $\sigma$ takes the value of zero means that the drilling decision for the marginal well location depends only on drilling costs and the interest rate $r$, which are not stochastic, unlike cost shocks $\epsilon_{at}$'s.

Second, the magnitude of $\sigma$ determines the rate of drilling, and also production. To be specific, an increase in the magnitude of $\sigma$ reduces the drilling rate. When the value of $\sigma$ grows, the importance of the cost shocks increases relative to the observable components in the payoff function (i.e., $\psi(\boldsymbol{X}_{t})$). In other words, as more payoff comes from the cost shocks, the option value of each well location increases. Therefore, a larger value of $\sigma$ makes the social planner wait for a better cost shock, which in turn, delays well drilling. 

Transversality condition (\ref{Equation:Social-Planners-Problem_Transversality-Condition}) rules out too aggressive depletion of well sites. Note that the transversality condition holds even when $R_{t} \neq 0$.
