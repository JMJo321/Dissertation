Necessary conditions (\ref{Equation:Social-Planners-Problem_Meaning-of-Costate-Variable}) and (\ref{Equation:Social-Planners-Problem_Simplified-Costate-Variable}) have important economic implications of the optimal path of well sites' depletion. Firstly, necessary condition (\ref{Equation:Social-Planners-Problem_Meaning-of-Costate-Variable}) directly provides us with what $\pi_{t}$ means. In this condition, the three terms in the curly bracket collectively mean the net benefit from the output (i.e., oil or gas) produced from the marginally drilled well site at time $t$. Of note, the last term among them in the curly bracket is the expected value of the cost shock at time $t$ when the firm decides to drill it (i.e., $e(1)$). The remaining term in this condition represents the opportunity cost of drilling the marginal site at time $t$.\footnote{If the firm decides not to drill a horizontal well into the marginal well site, then the expected value of $\epsilon_{0,t}$ conditional on $a_{t} = 0$ (i.e., $e(0)$) is the only gain the firm gets from the decision.} Hence, the necessary condition indicates that the costate variable $\pi_{t}$ implies the net shadow value of the marginally drilled well site in the current-value term at time $t$. 

Necessary condition (\ref{Equation:Social-Planners-Problem_Simplified-Costate-Variable}) enables us to understand what $\pi_{t}$ means from a different perspective. We can re-write this necessary condition as follows\footnote{We ignore an arbitrary integration constant $\mathcal{C}$ in the derivation because $\mathcal{C} = 0$ from the fact that $Pr_{t}$ is a constant at the steady state as well as equation (\ref{Equation:Social-Planners-Problem_System-of-Equations-for-Steady-State}).}:
\begin{equation}
\begin{split}
    & \frac{ \ \dot{\pi}_{t} \ + \ \left\{ f_{t} \ + \ \lambda_{a} \sigma \big( \gamma \ - \ \ln(1 - Pr_{t}) \big) \right\} \ }{r} \\
    & = \ \alpha u'(\alpha \lambda_{a} R_{t} Pr_{t}) \ - \ c'(\lambda_{a} R_{t} Pr_{t}) \ + \ \sigma \big( \gamma \ - \ \ln(Pr_{t}) \big) \ - \ \sigma \big( \gamma \ - \ \ln(1 - Pr_{t}) \big).
\end{split}
\label{Equation:Social-Planners-Problem_Euler-Equation}
\end{equation}
This equation implies that $\pi_{t}$ is the marginally undrilled well site's aggregate expected utility (i.e., the sum of the expected value of $\epsilon_{0,t}$'s over time), as the current value at time $t$, if the well site will remain undeveloped.

Collectively, necessary conditions (\ref{Equation:Social-Planners-Problem_Meaning-of-Costate-Variable}) and (\ref{Equation:Social-Planners-Problem_Simplified-Costate-Variable}) suggest that on the optimal path of drilling, the marginal undeveloped well site will be drilled at time $t$ if the net gains from drilling it at time $t$ equal the undrilled site' aggregate future gains from time $t$. In other words, drilling a horizontal well today is an optimal choice for the firm if its value is, at the margin, indifferent to the value of simply holding it forever. 

Necessary condition (\ref{Equation:Social-Planners-Problem_Simplified-Costate-Variable}) demonstrates significant implications of $\pi_{t}$'s growth over time. We can re-express this condition as follows:
\begin{equation}
\begin{split}
%    \dot{\pi}_{t} \ 
%    & = \ r \pi_{t} \ - \ \sigma \big( \gamma \ + \ \ln(1 - Pr_{t}) \big) \\
    \dot{\pi}_{t} \
    & = \ r \left\{ \pi_{t} \ - \ \frac{ \ f_{t} \ + \ \sigma \big( \gamma \ - \ \ln(1 - Pr_{t}) \big) \ }{r} \right\}.
%    \frac{\dot{\pi_{t}}}{\pi_{t}} \ + \  \frac{\sigma \big( \gamma \ - \ \ln(1 - Pr_{t}) \big)}{\pi_{t}} \ 
%    & = \ r
\end{split}
\label{Equation:Social-Planners-Problem_Growth-Rate-of-Costate-Variable}
\end{equation}
Since the probability of drilling is a control variable, it does not change during the infinitesimal period of time between two consecutive drilling decisions. From this observation, it is natural that the second term in the curly bracket is the net gains from drilling the marginal well site at time $t$.\footnote{From equation (\ref{Equation:Social-Planners-Problem_Euler-Equation}), $\pi_{t} = e^{rt} \sigma \big( \gamma \ - \ \ln(1 - Pr_{t}) \big) \int_{t}^{\infty} e^{-r\tau} d\tau = \sigma \big( \gamma \ - \ \ln(1 - Pr_{t}) \big) / r$.} Therefore, the necessary condition indicates that equalizing the growth in $\pi_{t}$ over time and $r$\% of the change, from the net gains, in $\pi_{t}$ makes the firm indifferent between drilling a horizontal well at time $t$ and drilling it at some time in the future.\footnote{Of note, the firm discounts the future at the interest rate $r$.} It is also clear that $\pi_{t}$ grows slower than the rate of interest $r$. In addition, the necessary condition suggests that $\pi_{t}$ increases concavely with $Pr_{t}$ and converges in the limit, unlike the exponential growth of the shadow price on the law of motion in Hotelling's framework.\footnote{From the beginning of drilling (i.e., $t = 0$), $R_{t}$ decreases. The decreasing path of $R_{t}$ implies that $Pr_{t}$ must continuously grow to keep drilling. In equation (\ref{Equation:Social-Planners-Problem_Growth-Rate-of-Costate-Variable}), $Pr_{t}$'s increase leads to the reduction in $\dot{\pi}_{t}$.}

The value of $\sigma$, which is the dispersion parameter of the I.I.D. T1EV cost shocks, provides two interesting implications. First, the social planner's problem reverts to the Hotelling model of the optimal extraction of a nonrenewable resource when $\sigma$ goes to zero. Taking limits to zero for necessary conditions (\ref{Equation:Social-Planners-Problem_Meaning-of-Costate-Variable}) and (\ref{Equation:Social-Planners-Problem_Simplified-Costate-Variable}) yields the followings, which are identical to the two necessary conditions in Hotelling's classic model of depletion:
\begin{equation}
\begin{cases}
        \begin{split}
        \ \lim_{\sigma \to 0} \dot{\pi}_{t} \
        & = \ r\pi_{t} \\
        \ \lim_{\sigma \to 0} \pi_{t} \
        & = \ \alpha u'(\alpha \lambda_{a} R_{t} Pr_{t}) \ - \ c'(\lambda_{a} R_{t} Pr_{t}).
        \end{split}
    \end{cases}
\label{Equation:Social-Planners-Problem_Reverting-to-the-Hotelling-Model}
\end{equation}
Intuitively, the limiting case that $\sigma$ takes a value of zero means that the drilling decision for the marginal well location depends only on oil prices, drilling costs, and the interest rate $r$, which are not stochastic, unlike cost shocks $\epsilon(a_{t})$'s. 

Second, the magnitude of $\sigma$ determines the rate of drilling, and also production. As suggested in (\ref{Equation:Social-Planners-Problem_Growth-Rate-of-Costate-Variable}), a larger value of $\sigma$, allowing more extreme utility shocks, makes the size of $\dot{\pi}_{t}$ smaller for a given value of $Pr_{t}$. Because a lower growth rate of $\pi_{t}$ means less incentive for drilling due to a smaller marginal value of drilling, an increase in the magnitude of $\sigma$ reduces drilling. In other words, the increase in the size of the dispersion parameter makes waiting for the next drilling opportunity more valuable due to the greater expected value of the cost shock conditional $a_{t} = 0$. Considering the proportional relationship between drilling and oil production, a larger value of $\sigma$ leads to a decrease in oil supply. 

Transversality condition (\ref{Equation:Social-Planners-Problem_Transversality-Condition}) also has an economically meaningful implication. The endpoint condition demonstrates that the present discounted value of the limiting well site should not grow faster than the interest rate $r$. In other words, the condition rules out too aggressive depletion of reserves. Note that the transversality condition holds even when $R_{t} \neq 0$.
