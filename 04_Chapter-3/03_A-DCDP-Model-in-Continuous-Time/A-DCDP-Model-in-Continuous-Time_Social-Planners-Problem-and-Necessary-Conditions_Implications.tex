Necessary conditions (\ref{Equation:Social-Planners-Problem_Necessary-Conditions_Costate-Variable}) and (\ref{Equation:Social-Planners-Problem_Meaning-of-Costate-Variable}) have important economic implications of the optimal path of well sites' depletion. Firstly, necessary condition (\ref{Equation:Social-Planners-Problem_Meaning-of-Costate-Variable}) directly provides us with what $\pi_{t}$ means. In this condition, the three terms in the curly bracket collectively mean the net benefit from the output (i.e., oil or gas) produced from the marginally drilled well site at time $t$. Of note, the last term among them is the expected value of the marginal well location's cost shock at time $t$ when the firm decides to drill it (i.e., $e(1)$). The remaining term in this condition represents the opportunity cost of drilling the marginal site at time $t$.\footnote{If the firm decides not to drill a horizontal well into the marginal well site, then the expected value of $\epsilon_{0,t}$ conditional on $a_{t} = 0$ (i.e., $e(0)$) is the only gain the firm gets from the decision.} Hence, the necessary condition indicates that the costate variable $\pi_{t}$ implies the net shadow value of the marginally drilled well site in the current-value term at time $t$. 

Necessary condition (\ref{Equation:Social-Planners-Problem_Necessary-Conditions_Costate-Variable}) enables us to understand what $\pi_{t}$ means from a different perspective. We can re-write this necessary condition as follows\footnote{We ignore an arbitrary integration constant $\mathcal{C}$ in the derivation because $\mathcal{C} = 0$ from the fact that $Pr_{t}$ is a constant at the steady state as well as equation (\ref{Equation:Social-Planners-Problem_System-of-Equations-for-Steady-State}).}:
\begin{equation}
\begin{split}
    % \dot{\pi}_{t} \ 
    % & = \ r \pi_{t} \ - \ \sigma \big( \gamma \ + \  \ln(1 - Pr_{t}) \big) \\
    % (\dot{\pi}_{t} \ - \ r\pi_{t}) e^{-rt} \
    % & = \ - \sigma \big( \gamma \ - \ \ln(1 - Pr_{t}) \big) e^{-rt} \\
    % \pi_{t} e^{-rt} \
    % & = \ \int_{t}^{\infty} e^{-r\tau} \sigma \big( \gamma \ - \ \ln(1 - Pr_{\tau}) \big) d\tau \ + \ \mathcal{C} \\  % C = 0 from the equation for \pi_{t} at the steady state.
    \pi_{t} \
    & = \ e^{rt} \left[ \int_{t}^{\infty} e^{-r\tau} \Big\{ f_{t} \ + \ \sigma \big( \gamma \ - \ \ln(1 - Pr_{\tau}) \big) \Big\} d\tau \right],
\end{split}
\label{Equation:Social-Planners-Problem_Euler-Equation}
\end{equation}
This equation implies that $\pi_{t}$ is the marginally undrilled well site's aggregate expected utility (i.e., the sum of the expected value of $\epsilon_{0,t}$'s over time), as the current value at time $t$, if the well site will remain undeveloped. Therefore, it is clear that drilling well locations is economic depletion in our framework, as extracting an exhaustible resource is in Hotelling's model. 

Collectively, necessary conditions (\ref{Equation:Social-Planners-Problem_Necessary-Conditions_Costate-Variable}) and (\ref{Equation:Social-Planners-Problem_Meaning-of-Costate-Variable}) suggest that on the optimal path of drilling, the marginal undeveloped well site will be drilled at time $t$ if the net gains from drilling it at time $t$ equal the undrilled site' aggregate future gains from time $t$. In other words, at the margin, drilling a horizontal well today is an optimal choice for the firm if its value is indifferent to the value of simply holding it forever. Indeed, this implication holds under the Hotelling framework. 

Necessary condition (\ref{Equation:Social-Planners-Problem_Necessary-Conditions_Costate-Variable}) demonstrates significant implications of $\pi_{t}$'s growth over time. We can re-express this condition as follows\footnote{From equation (\ref{Equation:Social-Planners-Problem_Euler-Equation}), $\pi_{t} = e^{rt} \sigma \big( \gamma \ - \ \ln(1 - Pr_{t}) \big) \int_{t}^{\infty} e^{-r\tau} d\tau = \sigma \big( \gamma \ - \ \ln(1 - Pr_{t}) \big) / r$ at the steady state.}:
\begin{equation}
%    \dot{\pi}_{t} \ 
%    & = \ r \pi_{t} \ - \ \sigma \big( \gamma \ + \ \ln(1 - Pr_{t}) \big) \\
    \dot{\pi}_{t} \ = \ r \left\{ \pi_{t} \ - \ \frac{ \ f_{t} \ + \ \lambda_{a} \sigma \big( \gamma \ - \ \ln(1 - Pr_{t}) \big) \ }{r} \right\}.
%    \frac{\dot{\pi_{t}}}{\pi_{t}} \ + \  \frac{\sigma \big( \gamma \ - \ \ln(1 - Pr_{t}) \big)}{\pi_{t}} \ 
%    & = \ r
\label{Equation:Social-Planners-Problem_Growth-Rate-of-Costate-Variable}
\end{equation}
This expression clearly indicates that $\pi_{t}$ grows slower than the rate of interest $r$. The necessary condition also suggests that $\pi_{t}$ increases concavely with $Pr_{t}$ and converges in the limit, unlike the exponential growth of the shadow price on the law of motion in Hotelling's framework.\footnote{From the beginning of drilling (i.e., $t = 0$), $R_{t}$ decreases. If $Pr_{t}$ is maintained at a lower level, the rate of drilling will quickly converge to zero. So, $Pr_{t}$ must continuously grow to keep drilling. In equation (\ref{Equation:Social-Planners-Problem_Growth-Rate-of-Costate-Variable}), $Pr_{t}$'s increase leads to the reduction in $\dot{\pi}_{t}$.} 

The value of $\sigma$, which is the dispersion parameter of the I.I.D. T1EV cost shocks, provides two interesting implications. First, the social planner's problem reverts to the Hotelling model of the optimal extraction of a nonrenewable resource when $\sigma$ goes to zero. Taking limits to zero for necessary conditions (\ref{Equation:Social-Planners-Problem_Necessary-Conditions_Costate-Variable}) and (\ref{Equation:Social-Planners-Problem_Meaning-of-Costate-Variable}) yields the followings, which are identical to the two necessary conditions in Hotelling's classic model of depletion\footnote{In the limiting case, we ignore the flow utility $f_{t}$, which is not introduced in Hotelling's theoretical model.}:
\begin{equation}
\begin{cases}
        \begin{split}
        \ \lim_{\sigma \to 0} \dot{\pi}_{t} \
        & = \ r\pi_{t} \\
        \ \lim_{\sigma \to 0} \pi_{t} \
        & = \ \alpha u'(\alpha R_{t} Pr_{t}) \ - \ c'(R_{t} Pr_{t}).
        \end{split}
    \end{cases}
\label{Equation:Social-Planners-Problem_Reverting-to-the-Hotelling-Model}
\end{equation}
Intuitively, the limiting case that $\sigma$ takes a value of zero means that the drilling decision for the marginal well location depends only on drilling costs and the interest rate $r$, which are not stochastic, unlike cost shocks $\epsilon(a_{t})$'s.

Second, the magnitude of $\sigma$ determines the rate of drilling, and also production. To be specific, an increase in the magnitude of $\sigma$ reduces drilling. When the value of $\sigma$ grows, the importance of the cost shocks increases relative to the observable components in the utility function (i.e., $\tilde{U}(\cdot)$). In other words, as more utility comes from the cost shocks, the option value of each well location increases. Therefore, a larger value of $\sigma$ makes the social planner wait for a better shock, which in turn, delays well drilling. 

Transversality condition (\ref{Equation:Social-Planners-Problem_Transversality-Condition}) rules out too aggressive depletion of well sites. Note that the transversality condition holds even when $R_{t} \neq 0$.
