The necessary condition (\ref{Equation:Social-Planners-Problem_Necessary-Conditions_Costate-Variable}) and the equation (\ref{Equation:Social-Planners-Problem_Meaning-of-Costate-Variable}) have significant economic interpretations of the optimal path. When $R_{t} > 0$, we can get a new expression for $\pi_{t}$ from the condition:
\begin{equation}
\begin{split}
    & \frac{ \ \dot{\pi}_{t} \ + \ \left\{ f_{t} \ + \ \lambda_{a} \sigma \big( \gamma \ - \ \ln(1 - Pr_{t}) \big) \right\} \ }{r} \\
    & = \ \alpha u'(\alpha \lambda_{a} R_{t} Pr_{t}) \ - \ c'(\lambda_{a} R_{t} Pr_{t}) \ + \ \sigma \big( \gamma \ - \ \ln(Pr_{t}) \big) \ - \ \sigma \big( \gamma \ - \ \ln(1 - Pr_{t}) \big).
\end{split}
\label{Equation:Social-Planners-Problem_Euler-Equation}
\end{equation}
where $\mathcal{C}$ is an arbitrary integration constant. The expression implies that $\pi_{t}$ is an undrilled well site's total expected utilities (i.e., the sum of the expected value of future $\epsilon_{0, t}$'s), as the current value at time $t$, if the well site will remain undeveloped. 

The equation (\ref{Equation:Social-Planners-Problem_Meaning-of-Costate-Variable}) enables us to understand what $\pi_{t}$ means from a different perspective. 
In this equation, the three terms in the curly bracket collectively mean the net benefit of consuming oils produced from the marginally drilled well site at time $t$. The remaining term in the equation represents the opportunity cost of drilling at time $t$. Hence, the equation indicates that the costate variable $\pi_{t}$ implies the net shadow value of the marginal site at time $t$.

Collectively, the necessary condition (\ref{Equation:Social-Planners-Problem_Necessary-Conditions_Costate-Variable}) and the equation (\ref{Equation:Social-Planners-Problem_Meaning-of-Costate-Variable}) suggest that on the path of the optimal drilling, the marginal undeveloped well site is drilled if the net gains from drilling it equal the total future gains from the undrilled site. 

The necessary condition (\ref{Equation:Social-Planners-Problem_Necessary-Conditions_Costate-Variable}) demonstrates an implication of $\pi_{t}$'s change over time. This condition can be re-written when $R_{t} > 0$ (i.e., $\lambda_{1,t} = 0$):
\begin{equation}
\begin{split}
%    \dot{\pi}_{t} \ 
%    & = \ r \pi_{t} \ - \ \sigma \big( \gamma \ + \ \ln(1 - Pr_{t}) \big) \\
    \dot{\pi}_{t} \
    & = \ r \left\{ \pi_{t} \ - \ \frac{ \ f_{t} \ + \ \sigma \big( \gamma \ - \ \ln(1 - Pr_{t}) \big) \ }{r} \right\}.
%    \frac{\dot{\pi_{t}}}{\pi_{t}} \ + \  \frac{\sigma \big( \gamma \ - \ \ln(1 - Pr_{t}) \big)}{\pi_{t}} \ 
%    & = \ r
\end{split}
\label{Equation:Social-Planners-Problem_Growth-Rate-of-Costate-Variable}
\end{equation}
Clearly, this equation implies that the sum of the shadow value of the marginal well site and the return to holding the marginal site rises at the interest rate $r$. In other words, both the shadow value and the return grow slower than $r$. 

The value of $\sigma$, which is the dispersion parameter of the I.I.D. T1EV utility shocks, provides two interesting implications. First, the social planner's problem reverts to the Hotelling model of the optimal extraction of a nonrenewable resource when $\sigma$ goes to zero. Taking limits to zero for the necessary condition (\ref{Equation:Social-Planners-Problem_Necessary-Conditions_Costate-Variable}) and the equation (\ref{Equation:Social-Planners-Problem_Meaning-of-Costate-Variable}) yields the followings, which are identical to the two necessary conditions in the classic model of depletion:
\begin{equation}
\begin{cases}
        \begin{split}
        \ \lim_{\sigma \to 0} \dot{\pi}_{t} \
        & = \ r\pi_{t} \\
        \ \lim_{\sigma \to 0} \pi_{t} \
        & = \ \alpha u'(\alpha \lambda_{a} R_{t} Pr_{t}) \ - \ c'(\lambda_{a} R_{t} Pr_{t}).
        \end{split}
    \end{cases}
\label{Equation:Social-Planners-Problem_Reverting-to-the-Hotelling-Model}
\end{equation}
Intuitively, the limiting case that $\sigma$ takes a value of zero means that the drilling decision for the marginal well location depends only on oil prices and drilling costs. 

Second, $\sigma$ is a governing factor for drilling activities. As suggested in (\ref{Equation:Social-Planners-Problem_Growth-Rate-of-Costate-Variable}), a larger value of $\sigma$, allowing more extreme utility shocks, makes the size of $\dot{\pi}_{t}$ smaller for a given $Pr_{t}$. Because a lower growth rate of $\pi_{t}$ means less incentive for drilling due to a smaller marginal value of drilling, an increase in the magnitude of $\sigma$ reduces drilling. In other words, the increase in the size of the dispersion parameter makes waiting for the next drilling opportunity more valuable. Considering the proportional relationship between drilling and oil production, a larger value of $\sigma$ leads to a decrease in oil supply. 

The transversality condition (\ref{Equation:Social-Planners-Problem_Transversality-Condition}) also has an economically meaningful implication. The endpoint condition demonstrates that the present discounted value of the limiting well site should not grow faster than the interest rate $r$. In other words, the condition rules out too aggressive depletion of reserves. Note that the transversality condition holds even when $R_{t} \neq 0$.
