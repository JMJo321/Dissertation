In our formulation, a particular well site $i$ can be in state $k$ at some time $t$. This $k$ is an integer scalar index $k = 1, 2, \cdots, K$, by which every available state in a finite state space $\mathcal{X}$ is enumerated.\footnote{As implied, $\mathcal{X}$ is a discrete state space.} For simplicity, it is assumed that the firm can drill only one horizontal well in that well location $i$. 

Exploiting the same index system, we discretize oil prices that are a state variable in the firm's problem. Specifically, for a given oil production $Q$, the oil price in state $k$ (denoted $p_{k}$) is determined by the following:
\begin{equation}
\begin{split}
	p_{k} \
	& = \ p_{0,k} \ - \ \widebar{p}_{1} Q,
\end{split}
\label{Equation:Firms-Problem_Oil-Prices}
\end{equation}
where $p_{0,k}$ and $\widebar{p}_{1}$ are non-negative. In our setting, oil prices can vary due to two distinct finite-state Markov jump processes.\footnote{A Markov jump process with finite states is a stochastic process that has discrete movements governed by a Poisson arrival process. For details, see \cite{Time-Discretization-of-Markov-Chains_Doytchinov-and-Irby_2010}.} One is a jump process on $\mathcal{X}$. Parameters $\lambda_{k\ell}$ that indicate the rates at which particular state transitions from $k$ to $\ell \neq k$ occur govern this process. The firm's actions, following a Poisson arrival process with rate parameter $\lambda_{a}$, drive the other process. Specifically, when the firm chooses the action $a = 1$, which means drilling a horizontal well in site $i$, from the discrete choice set $\mathcal{A} = \{ 0, 1 \}$, $p_{\ell(i, a, k)}$ is the price in the resulting state $\ell(i, a, k)$.\footnote{This implicitly assumes that oil prices are determined endogenously and that the firm takes oil prices as given in a competitive equilibrium.}\footnote{$a = 0$ is a costless continuation choice.}

The firm, which is forward-looking and discounts future payoffs at rate $\rho \in (0, \infty)$, receives two different types of payoffs. First, the firm receives flow payoff with respect to a particular undrilled well site $i$ being in state $k$. We formulate the flow payoff $f_{ik}$ as a function of oil prices: $f_{ik} (p_{k}; \boldsymbol{\theta_{f}})$.\footnote{In this formulation, $\boldsymbol{\theta}_{f}$ is a vector of parameters, which depends on a specific functional form.} Here, we assume that the firm exits the market after drilling a horizontal well into the site $i$. In other words, the firm gets no flow payoff from well locations already drilled. 

Regarding a given well site $i$ in state $k$, the firm also receives an instantaneous payoff when taking action $a \in \mathcal{A}$. This choice-dependent payoff consists of a choice-specific payoff $\psi_{iak}$ and a choice-specific payoff shock $\epsilon_{iak}$, which is observable only to the firm. We formulate the choice-specific payoff as follows:
\begin{equation}
\begin{split}
    \psi_{iak}(\boldsymbol{x}_{ik}, a; \boldsymbol{\theta}_{\psi}) \
    & = \ 
    \begin{cases}
        \ \big\{ g_{i}^{L} \ + \ (1 - g_{i}^{L}) \alpha^{H} \big\} p_{k} \ - \ c \hspace{0.5cm} \text{if $s_{ik} = 0$ and $a = 1$} \\
        \ 0 \hspace{4.65cm} \text{others},
    \end{cases}
\end{split}
\label{Equation:Firms-Problem_Instantaneous-Payoff}
\end{equation}
where $\boldsymbol{\theta}_{\psi}$, which is a vector of parameters, consists of two elements: $\alpha$ is the normalized oil production from the site $i$, whereas $c$ stands for drilling costs.\footnote{That is, $\boldsymbol{\theta}_{\psi} = (\alpha, c)$.} Clearly, for a given undrilled well site $i$, the firm gets an instantaneous payoff $\psi_{iak} + \epsilon_{iak}$, whose value varies with the firm's choice $a \in \mathcal{A}$. And the firm gets no instantaneous payoff from already drilled sites. Here, we assume that the $\epsilon_{iak}$'s are I.I.D. and follow $T1EV(0, \sigma)$. 

In our continuous-time framework, the value function for a particular well site $i$ in state $k$ is given as follows\footnote{Detailed derivation is presented in \ref{C3-Appendix_Derivations_Value-Function-in-Continuous-Time}.}:
\begin{equation}
\begin{split}
    % \left( \rho \ + \ \sum_{\ell \neq k} \lambda_{k\ell} \ + \ \lambda_{d} \right) V_{ik} \ 
    % & = \ f_{ik} \ + \ \sum_{\ell \neq k} \lambda_{k\ell} V_{i\ell} \ + \ \lambda_{d} E\bigg[ \underset{a}{\max} \left\{ V_{i,\ell(i, a, k)} \ + \ \psi_{iak} \ + \ \epsilon_{iak} \right\} \bigg].
%    V_{ik} \ 
%    & = 
%    \ \frac{
%        \ f_{ik} \ + \ \sum_{\ell \neq k} \lambda_{k\ell} V_{i\ell} \ + \ \lambda_{a} E\Big[ \underset{a \in \mathcal{A}}{\max} \left\{ V_{i,\ell(i, a, k)} \ + \ \psi_{iak} \ + \ \epsilon_{iak} \right\} \Big] \ 
%    }{
%        \rho \ + \ \sum_{\ell \neq k} \lambda_{k\ell} \ + \ \lambda_{a}
%    }.
    & -\dot{V}_{ik} (t) \ + \ \left( \rho \ + \ \lambda_{a} \ + \ \sum_{\ell \neq k} \lambda_{k\ell} \right) V_{ik}(t) \\
    & \hspace{1.0cm} = \ (f_{ik} \ + \ \lambda_{a} \tilde{f}_{ik}) \ + \ \lambda_{a} E\Big[ \underset{a \in \mathcal{A}}{\max} \left\{ V_{i,\ell(i, a, k)} (t) \ + \ \psi_{iak} \ + \ \tilde{\psi}_{iak} \ + \ \epsilon_{iak} \right\} \Big] \ + \ \sum_{\ell \neq k} \lambda_{k\ell} V_{i\ell}(t)
\end{split}
\label{Equation:Firms-Problem_Value-Function}
\end{equation}
Here, the value function $V_{ik}$ represents the present discounted value of all payoffs obtained from starting at state $k$ and behaving optimally in all subsequent periods. $\dot{V}_{ik}$ is the time derivative of $V_{ik}$. The three terms in the round bracket on the left-hand side are the sum of the discount factor and the rates of all possible (i.e., exogenous as well as endogenous) state changes. The right-hand side consists of the flow payoffs, the expected values relying on the firm's decisions, and the rate-weighted values related to exogenous state transitions. The expectation is for the joint distribution of $\epsilon_{i0k}$ and $\epsilon_{i1k}$. $\tilde{f}_{ik}$ and $\tilde{\psi}_{iak}$ indicate payoff changes due to state transitions as a result of the firm's decisions. 

For a given opportunity to choose an action $a \in \mathcal{A}$, the probability of drilling a horizontal well conditional on state $k$, denoted $Pr_{k}$, can be defined as follows\footnote{For given values of parameters, we can compute the value of each $Pr_{k}$, $k = 1, 2, \cdots, K$, by using value function iterations.}:
\begin{equation}
\begin{split}
	Pr_{k} \
	& \equiv \ \Pr \big[ \ \psi_{i1k} + \epsilon_{i1k} \ \geq \ V_{i,\ell(i,0,k)} + \psi_{i0k} + \epsilon_{i0k} \ | \ k \ \big].
\end{split}
\label{Equation:Firms-Problem_CCP}
\end{equation}
From this definition, the hazard rate of action $a$ is $h_{ak} = \lambda_{a} Pr_{k}$. 
As shown in \cite{Estimation-of-Dynamic-Discrete-Choice-Models-in-Continuous-Time_ABBE_2016}, for each action $a \in \mathcal{A}$, the third term in the numerator of equation (\ref{Equation:Firms-Problem_Value-Function}) is the following:
\begin{equation}
%\small
\begin{split}
	\lambda_{a} E\Big[ \underset{a \in \mathcal{A}}{\max} \left\{ V_{ik} + \psi_{iak} + \epsilon_{iak} \right\} \Big] 
	& =  
	\underbrace{\lambda_{a} \left\{ V_{ik} + \sigma \big( \gamma - \ln(1 - Pr_{k}) \big) \right\} }_{\text{if \ $a = 0$}} = \underbrace{\lambda_{a} \left\{ \psi_{i1k} + \sigma \big( \gamma - \ln(Pr_{k}) \big) \right\} }_{\text{if \ $a = 1$}},
\end{split}
\label{Equation:Firms-Problem_Emax}
\end{equation}

In the case in which the oil price remains constant at $\widebar{p}$, some algebraic manipulation on the value function (\ref{Equation:Firms-Problem_Value-Function}), with the expressions in (\ref{Equation:Firms-Problem_Emax}), yields the Euler equation that drives the dynamics of the firm's optimal drilling decisions\footnote{In our setting, constant oil price implies no state transition.}:
\begin{equation}
\begin{split}
    & \frac{ \ \dot{V}_{ik} \ + \ f_{ik} \ + \ \lambda_{a} \sigma \big( \gamma \ - \ \ln(1 - Pr_{k}) \big) \ }{\rho} \\
    & = \ \alpha p_{k} \ - \ c \ + \ \sigma \big( \gamma \ - \ \ln(Pr_{k}) \big) \ - \ \sigma \big( \gamma \ - \ \ln(1 - Pr_{k}) \big).
\end{split}
\label{Equation:Firms-Problem_Euler-Equation}
\end{equation}
In this equation, the left-hand side represents the firm's payoff when deciding not to drill a horizontal well in state $k$. In other words, the left-hand side is the payoff for the case that the firm chooses $a = 0$ at a decision point and remains in state $k$. The right-hand side is the firm's payoff if it chooses to drill a horizontal well into well site $i$. Importantly, it is evident that equation (\ref{Equation:Firms-Problem_Euler-Equation}) drawn from the firm's well-level decisions coincides with necessary condition (\ref{Equation:Social-Planners-Problem_Meaning-of-Costate-Variable}) of the social planner's problem.\footnote{The only difference is $\lambda_{a}$ on the left-hand side in equation (\ref{Equation:Firms-Problem_Euler-Equation}), which governs how often the firm makes decisions of whether or not to drill a given well site.}
