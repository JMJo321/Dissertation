The theoretical framework for optimal oil drilling and extraction delineated in \cite{Hotelling-under-Pressure_AKS_2018} (AKS) can be augmented by integrating heterogeneity in the quality of well locations. Suppose that the fracking firm owns well sites of different qualities, indexed by $g \in \{L(ow), H(igh)\}$, and that a homogeneous good (i.e., oil) is yielded from the sites in which new horizontal wells are drilled. Furthermore, suppose that the unit price of the output, $p$, is determined exogenously due to the firm's total production being negligible in comparison to the global market for the output. The maximization problem of the firm owning a continuum of infinitesimal well locations with disparate qualities can be articulated as follows:
\begin{equation}
\begin{split}
    \underset{d^{g}(t), \ g \in \{L, H\}}{\max} \hspace{0.2cm} \int_{0}^{\infty} e^{-rt} \left\{ \widebar{p} \sum_{g} \alpha^{g} d^{g}(t) \ - \ C\left( \sum_{g} d^{g}(t) \right) \right\} dt
\end{split}
\label{Equation:AKS-Style-Model_Objective-Function}
\end{equation}
subject to
%\begin{equation}
%\begin{split}
%    \dot{K}(t) \ = \ - \delta \sum_{g} q^{g}(t) \ + \ \sum_{g} I^{g} d^{g}(t), \hspace{0.3cm} K_{0} \ = \ K(0) \ \text{given,}
%\end{split}
%\end{equation}
\begin{equation}
\begin{split}
    \dot{R}^{g}(t) \ = \ - d^{g}(t), \hspace{0.3cm} R_{0}^{g} \ = \ R^{g}(0) \ \text{given,} 
\end{split}
\end{equation}
%\begin{equation}
%\begin{split}
%    q^{g}(t) \ \geq \ 0, \hspace{0.3cm} 0 \ \leq \ \sum_{g} q^{g}(t) \ \leq \ K(t),
%\end{split}
%\end{equation}
\begin{equation}
\begin{split}
    d^{g}(t) \ \geq \ 0, \hspace{0.3cm} R^{g}(t) \ \geq \ 0.
\end{split}
\end{equation}

In this formulation, state variables $R^{g}(t)$ denote the measure of undrilled well sites at a given time $t$. Contrary to the original AKS model, we exclude the accessible capacity for oil flow $K(t)$ in the formulation due to the discussion in the paper that oil extractors usually operate at their production capacity constraint.\footnote{In other words, we simply assume that the capacity constraint is always binding.} Control variables $d^{g}(t)$ represent the rate at which new horizontal wells are drilled at time $t$. $\alpha^{g}$ are the quantity of oil production from the marginally drilled well. Here, we assume that $\alpha^{H} > \alpha^{L}$. $C(\cdot)$, indicating the total instantaneous cost of drilling, is solely a function of the drilling rates.\footnote{Regarding the total cost of oil production, we follow the assumption made in \cite{Hotelling-under-Pressure_AKS_2018}: per-barrel extraction costs from existing wells are negligible.} Of note, in this formulation for $C(\cdot)$, we assume that locations are perfect substitutes on the cost side. And the profit obtained at time $t$ is discounted at the interest rate $r$. 

The current-value Hamiltonian-Lagrangian of the firm's problem is
\begin{equation}
\begin{split}
    \mathcal{H} \ 
    & = \ p(t) \sum_{g} q^{g}(t) \ - \ C\left( \sum_{g} d^{g}(t) \right) \\
    & \hspace{0.5cm} + \ \pi_{K}(t) \left\{ - \delta \sum_{g} q^{g}(t) \ + \sum_{g} I^{g} d^{g}(t) \right\} \\
    & \hspace{0.5cm} + \ \sum_{g} \pi_{R}^{g}(t) \left( -d^{g}(t) \right) \\ 
    & \hspace{0.5cm} + \ \sum_{g} \lambda_{1}^{g}(t) q^{g}(t) \ + \ \lambda_{2}(t) \sum_{g} q^{g}(t) \ + \ \lambda_{3}(t) \left\{ K(t) \ - \ \sum_{g} q^{g}(t) \right\} \\
    & \hspace{0.5cm} + \ \sum_{g} \lambda_{4}^{g}(t) d^{g}(t) \ + \ \sum_{g} \lambda_{5}^{g}(t) R^{g}(t),
\end{split}
\label{Equation:AKS-Style-Model_Current-Value-Hamiltonian}
\end{equation}

where $\pi^{g}_{R}$ are costate variables for the state variables $R^{g}$. $\lambda_{j}, \ j \in \{1, 2\}$ are the shadow cost of each constraint.

For a given quality level $g$, two necessary conditions characterize the firm's optimal rate of drilling:
\begin{equation}
\begin{split}
    d^{g}(t) \ \geq \ 0, \hspace{0.2cm} \alpha^{g} \widebar{p} \ - \ C'\left( \sum_{g} d^{g}(t) \right) \ - \ \pi^{g}(t) \ + \ \lambda_{1}^{g}(t) \ \leq \ 0, \hspace{0.2cm} C.S.,
\end{split}
\label{Equation:AKS-Style-Model_Necessary-Conditions_pi-K}
\end{equation}
\begin{equation}
\begin{split}
    \dot{\pi}^{g}(t) \ = \ r \pi^{g}(t) \ - \ \lambda_{2}^{g}(t).
\end{split}
\label{Equation:AKS-Style-Model_Necessary-Conditions_pi-R}
\end{equation}

When horizontal wells with heterogeneous quality are drilled simultaneously (i.e., for each $g$, $d^{g}(t) >0$, which leads to $\lambda_{1}^{g}(t) = 0$), necessary condition (\ref{Equation:AKS-Style-Model_Necessary-Conditions_pi-K}) implies that the shadow price on the resource constraint at time $t$ equals the profit on the marginal well:
\begin{equation}
\begin{split}
    \pi^{g} (t) \
    & = \ \alpha^{g} \widebar{p} \ - \ C'\left( \sum_{g} d^{g}(t) \right).
\end{split}
\label{Equation:AKS-Style-Model_Necessary-Conditions_Shadow-Price-pi}
\end{equation}

In addition, when both types of horizontal well sites are not fully exhausted (i.e., for each $g$, $R^{g}(t) > 0$, which in turn $\lambda_{2}^{g}(t) = 0$), necessary condition (\ref{Equation:AKS-Style-Model_Necessary-Conditions_pi-R}) means that the shadow value of the marginal undrilled well at time $t$ grows at the rate of $r$:
\begin{equation}
\begin{split}
    \dot{\pi}^{g} (t) \
    & = \ r \pi^{g} (t).
\end{split}
\label{Equation:AKS-Style-Model_Necessary-Conditions_Simplified-pi}
\end{equation}


The necessary conditions collectively suggest that the simultaneous drilling of horizontal wells with heterogeneous quality cannot be justified in the AKS framework when $d^{g}(t) > 0$ and $R^{g}(t) > 0$, which hold before all available well sites are developed. The followings stem from equation (\ref{Equation:AKS-Style-Model_Necessary-Conditions_Shadow-Price-pi}):
\begin{equation}
\begin{split}
    \dot{\pi}^{L}, \ \dot{\pi}^{H} \
    & = \ \left( \sum_{g} \dot{d}^{g}(t) \right) C'\left( \sum_{g} d^{g}(t) \right),
\end{split}
\label{Equation:AKS-Style-Model_pi-dots}
\end{equation}
\begin{equation}
\begin{split}
    \pi^{H} - \pi^{L} \
    & = \ (\alpha^{H} - \alpha^{L}) \widebar{p}.
\end{split}
\label{Equation:AKS-Style-Model_difference-between-pis}
\end{equation}

Here, based on equation (\ref{Equation:AKS-Style-Model_Necessary-Conditions_pi-R}), the equation (\ref{Equation:AKS-Style-Model_pi-dots}) implies $\pi^{H} = \pi^{L}$. However, this equality contradicts equation (\ref{Equation:AKS-Style-Model_difference-between-pis}) because $\alpha^{H} > \alpha^{L}$, which implies $\pi^{H} > \pi^{L}$. 

The contradiction could be attributable to not introducing any constraint for the rate of drilling (i.e., $d^{g}(t)$, \ $g \in \{ L, H \}$). As discussed in Section \ref{C3-SubSubSection:Correlation-between-Oil-Prices-and-Horizontal-Drilling-in-ND}, limited extraction capacity can change the standard order of extraction. However, in the current framework, setting the upper bound of an extraction-related constraint seems too arbitrary and complicates drawing implications from necessary conditions. From the empirical perspective, it is also intractable to quantify the margin for market-wide, or firm-wide, extraction capacity for each drilling decision. Furthermore, empirical estimation of the model from microeconomic data on drilling and production is, in general, too demanding. Those difficulties call for a new theoretical approach to build a framework that thoroughly explains our empirical findings for drilling decisions made by fracking firms in North Dakota. 
