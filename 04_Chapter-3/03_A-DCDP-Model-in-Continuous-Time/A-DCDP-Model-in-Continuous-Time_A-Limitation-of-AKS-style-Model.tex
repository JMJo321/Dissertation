The theoretical framework for optimal oil drilling and extraction delineated in \cite{Hotelling-under-Pressure_AKS_2018} can be augmented by integrating heterogeneity in the quality of well locations. Suppose that the fracking firm owns well sites of different qualities, indexed by $g \in \{L(ow), H(igh)\}$, and that a homogeneous good (i.e., oil) is yielded from the sites in which new horizontal wells are drilled. Furthermore, suppose that the unit price of the output, $p$, is determined exogenously due to the firm's total production being negligible in comparison to the global market for the output. The maximization problem of the firm owning a continuum of infinitesimal well locations with disparate qualities can be articulated as follows:
\begin{equation}
\begin{split}
    \underset{d^{g}(t), \ g \in \{L, H\}}{\max} \hspace{0.2cm} \int_{0}^{\infty} e^{-rt} \left\{ \widebar{p} \sum_{g} \alpha^{g} d^{g}(t) \ - \ C\left( \sum_{g} d^{g}(t) \right) \right\} dt
\end{split}
\label{Equation:AKS-Style-Model_Objective-Function}
\end{equation}
subject to
%\begin{equation}
%\begin{split}
%    \dot{K}(t) \ = \ - \delta \sum_{g} q^{g}(t) \ + \ \sum_{g} I^{g} d^{g}(t), \hspace{0.3cm} K_{0} \ = \ K(0) \ \text{given,}
%\end{split}
%\end{equation}
\begin{equation}
\begin{split}
    \dot{R}^{g}(t) \ = \ - d^{g}(t), \hspace{0.3cm} R_{0}^{g} \ = \ R^{g}(0) \ \text{given,} 
\end{split}
\end{equation}
%\begin{equation}
%\begin{split}
%    q^{g}(t) \ \geq \ 0, \hspace{0.3cm} 0 \ \leq \ \sum_{g} q^{g}(t) \ \leq \ K(t),
%\end{split}
%\end{equation}
\begin{equation}
\begin{split}
    d^{g}(t) \ \geq \ 0, \hspace{0.3cm} R^{g}(t) \ \geq \ 0.
\end{split}
\end{equation}

In this formulation, two state variables $R^{g}(t)$ and $K(t)$ denote, at a given time $t$, the measure of undrilled well sites and the accessible capacity for oil flow, respectively. Moreover, two control variables $q^{g}(t)$ and $d^{g}(t)$ represent the oil production and the rate at which new horizontal wells are drilled at time t, respectively. $C(\cdot)$, indicating the total instantaneous cost of drilling, is solely a function of the drilling rate.\footnote{Regarding the total cost of oil production, we follow the assumption made in \cite{Hotelling-under-Pressure_AKS_2018}: per-barrel extraction costs from existing wells are negligible.} $I^{g}$ symbolizes the maximum oil flow from a newly drilled well. And the profit obtained at time $t$ is discounted at the interest rate $r$. 

The current-value Hamiltonian-Lagrangian of the firm's problem is
\begin{equation}
\begin{split}
    \mathcal{H} \ 
    & = \ p(t) \sum_{g} q^{g}(t) \ - \ C\left( \sum_{g} d^{g}(t) \right) \\
    & \hspace{0.5cm} + \ \pi_{K}(t) \left\{ - \delta \sum_{g} q^{g}(t) \ + \sum_{g} I^{g} d^{g}(t) \right\} \\
    & \hspace{0.5cm} + \ \sum_{g} \pi_{R}^{g}(t) \left( -d^{g}(t) \right) \\ 
    & \hspace{0.5cm} + \ \sum_{g} \lambda_{1}^{g}(t) q^{g}(t) \ + \ \lambda_{2}(t) \sum_{g} q^{g}(t) \ + \ \lambda_{3}(t) \left\{ K(t) \ - \ \sum_{g} q^{g}(t) \right\} \\
    & \hspace{0.5cm} + \ \sum_{g} \lambda_{4}^{g}(t) d^{g}(t) \ + \ \sum_{g} \lambda_{5}^{g}(t) R^{g}(t),
\end{split}
\label{Equation:AKS-Style-Model_Current-Value-Hamiltonian}
\end{equation}

where $\pi_{K}$ and $\pi_{R}$ are costate variables for the state variables $K$ and $R$, respectively. $\lambda_{j}, \ j \in \{1, 2, 3, 4, 5\}$ are the shadow cost of each constraint.

For a given quality level $g$, two necessary conditions abstract the dynamic optimization problem:
\begin{equation}
\begin{split}
    d^{g}(t) \ \geq \ 0, \hspace{0.2cm} \alpha^{g} \widebar{p} \ - \ C'\left( \sum_{g} d^{g}(t) \right) \ - \ \pi^{g}(t) \ + \ \lambda_{1}^{g}(t) \ \leq \ 0, \hspace{0.2cm} C.S.,
\end{split}
\label{Equation:AKS-Style-Model_Necessary-Conditions_pi-K}
\end{equation}
\begin{equation}
\begin{split}
    \dot{\pi}^{g}(t) \ = \ r \pi^{g}(t) \ - \ \lambda_{2}^{g}(t).
\end{split}
\label{Equation:AKS-Style-Model_Necessary-Conditions_pi-R}
\end{equation}

When horizontal wells with heterogeneous quality are drilled simultaneously (i.e., for each $g$, $d^{g}(t) >0$, which leads to $\lambda_{4}^{g}(t) = 0$), necessary condition (\ref{Equation:AKS-Style-Model_Necessary-Conditions_pi-K}) implies that the marginal value of an additional production capacity at time $t$ equals the marginal cost of drilling (i.e., the sum of the per-barrel shadow value of the marginal undrilled well and the per-barrel instantaneous drilling cost):
\begin{equation}
\begin{split}
    \pi_{K} \ 
    & = \ \frac{\pi_{R}^{g}(t) \ + \ C'\left( \sum_{g} d^{g}(t) \right)}{I^{g}}.
\end{split}
\label{Equation:AKS-Style-Model_Necessary-Conditions_Simplified-pi-K}
\end{equation}

In addition, when both types of horizontal well sites are not fully exhausted (i.e., for each $g$, $R^{g}(t) > 0$, which in turn $\lambda_{5}^{g}(t) = 0$), necessary condition (\ref{Equation:AKS-Style-Model_Necessary-Conditions_pi-R}) means that the shadow value of the marginal undrilled well at time $t$ grows at the rate of $r$:
\begin{equation}
\begin{split}
    \pi_{R}^{g}(t) \
    = \ \pi_{R,0}^{g} e^{rt}.
\end{split}
\label{Equation:AKS-Style-Model_Necessary-Conditions_Simplified-pi-R}
\end{equation}


Two conditions (\ref{Equation:AKS-Style-Model_Necessary-Conditions_Simplified-pi-K}) and (\ref{Equation:AKS-Style-Model_Necessary-Conditions_Simplified-pi-R}) collectively suggest that the simultaneous drilling of horizontal wells with heterogeneous quality cannot be justified in the AKS framework when $d^{g}(t) > 0$ and $R^{g}(t) > 0$, which hold before all available well sites are developed. From the two conditions, at time $t$, the marginal cost of drilling can be expressed as follows:
\begin{equation}
\begin{split}
    C'\left( \sum_{g} d^{g}(t) \right) \ 
    % & = \ \frac{I^{L} \pi_{R}^{H}(t) \ - \ I^{H} \pi_{R}^{L}(t)}{I^{H} \ - \ I^{L}} \\
%    & = \ \left( \frac{I^{L} \pi_{R,0}^{H} \ - \ I^{H} \pi_{R,0}^{L}}{I^{H} \ - \ I^{L}} \right) e^{rt} \\
    & = \ \left( \frac{\pi_{R,0}^{H} \ - \ (I^{H}/I^{L}) \pi_{R,0}^{L}}{(I^{H}/I^{L}) \ - \ 1} \right) e^{rt} \\
    & = \ C_{0} e^{rt},
\end{split}
\label{Equation:AKS-Style-Model_The-Total-Cost-of-Drilling}
\end{equation}

where $C_{0}$ is a strictly positive constant.\footnote{$C_{0} = 0$ implies that the marginal cost of drilling is always zero, which in turn suggests $d^{g}(t) = 0$.} This relationship contradicts the transversality condition at the terminal time $T$, $\lim_{t \to \infty}d^{g}(t) = 0$.
