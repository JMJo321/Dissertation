\textit{\textbf{The Firm's (Expected) Payoff Maximization Problem}} ---
The firm aims to maximize the total payoffs from drilling its well sites. Because the instantaneous payoff includes choice-dependent shocks $\epsilon_{iak}$, which follow the distribution of $TIEV(0, \sigma)$, the payoff the firm receives when drilling an individual well site $i$ is an expected value. So, the firm's problem is given by
\begin{footnotesize}
\begin{equation}
\begin{split}
    \underset{\{Pr_{t}\}_{t = 0}^{\infty}}{\max} \hspace{0.1cm} \int_{0}^{\infty} e^{-\rho t} \sum_{g \in \mathcal{Q}} R_{t}^{g} \left[ f_{t}^{g} \ + \ \lambda_{a} \Big\{ Pr_{t}^{g} \cdot \Big( \psi_{t}^{g} \ + \ \sigma \big( \gamma - \ln(Pr_{t}^{g}) \big) \Big) \ + \ (1 - Pr_{t}^{g}) \cdot \sigma \big( \gamma - \ln(1 - Pr_{t}^{g}) \big) \Big\} \right] dt
\end{split}
\label{Equation:Firms-Problem_Expected-Payoff-Maximization-Problem}
\end{equation}
\end{footnotesize}
subject to
\begin{equation}
\begin{split}
    \dot{R}_{t}^{g} \ = \ -R_{t}^{g} \big( \lambda_{a} Pr_{t}^{g} \big) \ + \ E^{g}, \hspace{0.3cm} R_{0}^{g} \ = \ R^{g}(0) \ = \ 1 \hspace{0.2cm} \text{given,}
\end{split}
\end{equation}
\begin{equation}
\begin{split}
    R_{t}^{g} \ \geq \ 0, \hspace{0.3cm} 0 \leq \ Pr_{t}^{g} \ \leq \ 1.
\end{split}
\end{equation}
As shown, we normalize the low- and high-quality reserves of well sites to 1. In this formulation, $\lambda_{a} Pr_{t}^{g} \hspace{0.15cm} (\equiv h_{t}^{g})$ means the hazard rate of drilling at time $t$. 

When the firm is drilling (i.e., $R_{t}^{g} > 0$ and $Pr_{t}^{g} \in (0, 1)$), the necessary conditions of the Hamiltonian-Lagrangian for the firm's problem are as follows\footnote{The Hamiltonian-Lagrangian of the firm's problem is presented in \ref{C3-Appendix_Derivations_Firms-Problem_Necessary-Conditions}.}:
\begin{equation}
\begin{split}
    & R_{t} \lambda_{a} \big\{ \psi_{t}^{g} - \ \sigma \ln(Pr_{t}^{g}) \ + \ \sigma \ln(1 - Pr_{t}^{g}) \ - \ \pi_{t}^{g} \big\} \ - \ \lambda_{2,t} \ + \ \lambda_{3,t} \ \leq \ 0, \hspace{0.2cm} Pr_{t}^{g} \ \geq \ 0,  \hspace{0.2cm} \text{C.S.},
\end{split}
\label{Equation:Firms-Problem_Necessary-Conditions_Drilling-Probability}
\end{equation}
\begin{equation}
\begin{split}
    \dot{\pi}_{t}^{g} \ 
    & = \ r \pi_{t}^{g} \ - \ \big\{ f_{t}^{g} \ + \ \lambda_{a} \sigma \big( \gamma \ - \ \ln(1 - Pr_{t}^{g}) \big) \big\} \ - \ \lambda_{1,t},
\end{split}
\label{Equation:Firms-Problem_Necessary-Conditions_Costate-Variable}
\end{equation}
\begin{equation}
\begin{split}
    \lim_{t \rightarrow \infty} e^{-rt} (R_{t}^{g} \pi_{t}^{g}) \ = \ 0.
\end{split}
\label{Equation:Firms-Problem__Transversality-Condition}
\end{equation}
With some algebra and the assumption of no exogenous price change, the necessary conditions (\ref{Equation:Firms-Problem_Necessary-Conditions_Drilling-Probability}) and (\ref{Equation:Firms-Problem_Necessary-Conditions_Costate-Variable}) yields the equation (\ref{Equation:Firms-Problem_Euler-Equation}).\footnote{The derivation details are described in \ref{C3-Appendix_Derivations_Euler-Equation-for-the-Firms-Problem}.}  Importantly, the identical equation in both optimization levels suggests that the firm's optimal drilling decision for a particular well site leads to the optimal path at the firm level. 


\par
\vspace{0.3cm}
\noindent
\textit{\textbf{Aggregate Drilling and Production}} ---
For the firm's reserves of well sites with a specific quality $g \in \mathcal{Q}$, the drilling and production are as follows:
\begin{equation}
\begin{cases}
    \begin{split}
        D_{t}^{g} \
        & = \ R_{t}^{g} h_{t}^{g} \hspace{0.2cm} (= -\dot{R}_{t}^{g} \ + \ E^{g}) \\
        Q_{t}^{g} \
        & = \ \alpha^{g} D_{t}^{g}.
    \end{split}
\end{cases}
\label{Equation:Firms-Problem_Aggregate-Drilling-and-Production}
\end{equation}
Therefore, the firm's aggregate drilling and production are simply the sum of drilling and production over different qualities. 

The equations for aggregate drilling and production imply two interesting points. First, $D_{t}^{g}$ converges to $E^{g}$ as time goes by, which implies no change in the amount of the remaining reserves, because $\lim_{t \to \infty} R_{t}^{g} = E^{g} / h_{t}^{g}$.\footnote{With some algebra, we find that $R_{t}^{g} = E^{g}/h_{t}^{t} \ + \ (R_{0}^{g} - E^{g}/h_{t}^{g}) e^{-h^{g} t}$.} Second, the firm first develops high-quality well locations. In other words, per-drilling oil production decreases as time $t$ goes to infinity. 
