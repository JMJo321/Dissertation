In our formulation, a particular well site $i$ can be in state $k$ at some time $t$. This $k$ is an integer scalar index $k = 1, 2, \cdots, K$, by which every available state in a finite state space $\mathcal{X}$ is enumerated.\footnote{As implied, $\mathcal{X}$ is a discrete state space.} For simplicity, it is assumed that the firm can drill only one horizontal well in that well location $i$. 

We utilize discretized oil prices. Because oil prices are one of the state variables in our formulation, we exploit the same index system. For example, $p_{k}$ denotes the oil price in state $k$. In our setting, oil prices can vary due to two distinct processes. One is a Markov jump process on $\mathcal{X}$. Parameters $\lambda_{k\ell}$ that indicate the rates at which particular state transitions from $k$ to $\ell \neq k$ occur govern this process. The firm's actions, following a Poisson arrival process with rate parameter $\lambda_{a}$, drive the other process. Specifically, when the firm chooses the action $a = 1$, which means drilling a horizontal well in site $i$, from the discrete choice set $\mathcal{A} = \{ 0, 1 \}$, $p_{\ell(i, a, k)}$ is the price in the resulting state $\ell(i, a, k)$.\footnote{This implicitly assumes that oil prices are determined endogenously. The firm's action does not impact oil prices when oil prices are exogenous.}\footnote{$a = 0$ is a costless continuation choice.}

A vector $\boldsymbol{x}_{ik}$, representing a particular well site $i$ in state $k$ at each instant $t \in [0, \infty)$, has three elements:
\begin{center}
    $\boldsymbol{x}_{ik} \ = \ (g_{ik}^{L}, s_{ik}, p_{k}).$
\end{center}
The first element $g_{ik}^{L}$ characterizes a given well site $i$'s quality type, which falls in one from the quality set $\mathcal{Q} = \{ L(ow), H(igh) \}$. This binary indicator has the value of one when well location $i$ is a low-quality well site. The second element $s_{ik} \in \{ 0, 1 \}$ indicates whether a well has been drilled in well site $i$. For a given well site $i$, $s_{ik} = 1$ means that the firm already drilled a horizontal well in this site. The last term shows the oil price in state $k$.

The firm, which is forward-looking and discounts future payoffs at rate $\rho \in (0, \infty)$, receives two different types of payoffs. First, the firm receives flow payoff with respect to a particular well site $i$ being in state $k$. We formulate the flow payoff $f_{ik}$ as a function of the state variables:
\begin{equation}
\begin{split}
    f_{ik} (s_{ik}, p_{k}, g_{ik}^{L}; \theta_{1}, \theta_{2}, \theta_{3}) \
    & = \ (1 - s_{ik}) \big\{ \theta_{1} \ + \ \theta_{2}p_{k} \ + \ \theta_{3}(1 - g_{ik}^{L}) \big\}.
\end{split}
\label{Equation:Firms-Problem_Flow-Payoff}
\end{equation}
where $\boldsymbol{\theta}_{f}$ is a vector of parameters and $\widetilde{f} (\cdot)$ is a linear function of $g_{ik}$ as well as $p_{k}$. As shown, well sites already drilled get no flow payoff. 

Regarding a given well site $i$ in state $k$, the firm also receives an instantaneous payoff when taking action $a \in \mathcal{A}$. This choice-dependent payoff consists of a choice-specific payoff $\psi_{iak}$ and a choice-specific payoff shock $\epsilon_{iak}$, which is observable only to the firm. We formulate the choice-specific payoff as follows:
\begin{equation}
\begin{split}
    \psi_{iak}(\boldsymbol{x}_{ik}, a; \boldsymbol{\theta}_{\psi}) \
    & = \ 
    \begin{cases}
        \ \big\{ g_{i}^{L} \ + \ (1 - g_{i}^{L}) \alpha^{H} \big\} p_{k} \ - \ c \hspace{0.5cm} \text{if $s_{ik} = 0$ and $a = 1$} \\
        \ 0 \hspace{4.65cm} \text{others},
    \end{cases}
\end{split}
\label{Equation:Firms-Problem_Instantaneous-Payoff}
\end{equation}
where $\boldsymbol{\theta}_{\psi}$, which is a vector of parameters, consists of two elements: $\alpha^{H}$ is the normalized oil production from the site $i$ whose quality type is $H$, whereas $c$ stands for drilling costs.\footnote{That is, $\boldsymbol{\theta}_{\psi} = (\alpha^{H}, c)$. Here, we normalize the oil production from a low-quality well location to 1. Therefore, $\alpha^{H} > 1$ implicitly.} Clearly, for a given undrilled well site $i$, the firm gets an instantaneous payoff $\psi_{iak} + \epsilon_{iak}$, whose value varies with the firm's choice $a \in \mathcal{A}$. And the firm gets no instantaneous payoff from already drilled sites (i.e., $s_{ik} = 0$). In this formulation, the payoff shock $\epsilon_{iak}$ makes the simultaneous drilling of well sites with heterogeneous quality feasible. Here, we assume that the $\epsilon_{iak}$'s are I.I.D. and follow $T1EV(0, \sigma)$. 

In our continuous-time framework, the value function for a particular well site $i$ in state $k$ is given by
\begin{equation}
\begin{split}
    % \left( \rho \ + \ \sum_{\ell \neq k} \lambda_{k\ell} \ + \ \lambda_{d} \right) V_{ik} \ 
    % & = \ f_{ik} \ + \ \sum_{\ell \neq k} \lambda_{k\ell} V_{i\ell} \ + \ \lambda_{d} E\bigg[ \underset{a}{\max} \left\{ V_{i,\ell(i, a, k)} \ + \ \psi_{iak} \ + \ \epsilon_{iak} \right\} \bigg].
%    V_{ik} \ 
%    & = 
%    \ \frac{
%        \ f_{ik} \ + \ \sum_{\ell \neq k} \lambda_{k\ell} V_{i\ell} \ + \ \lambda_{a} E\Big[ \underset{a \in \mathcal{A}}{\max} \left\{ V_{i,\ell(i, a, k)} \ + \ \psi_{iak} \ + \ \epsilon_{iak} \right\} \Big] \ 
%    }{
%        \rho \ + \ \sum_{\ell \neq k} \lambda_{k\ell} \ + \ \lambda_{a}
%    }.
    & -\dot{V}_{ik} (t) \ + \ \left( \rho \ + \ \lambda_{a} \ + \ \sum_{\ell \neq k} \lambda_{k\ell} \right) V_{ik}(t) \\
    & \hspace{1.0cm} = \ (f_{ik} \ + \ \lambda_{a} \tilde{f}_{ik}) \ + \ \lambda_{a} E\Big[ \underset{a \in \mathcal{A}}{\max} \left\{ V_{i,\ell(i, a, k)} (t) \ + \ \psi_{iak} \ + \ \tilde{\psi}_{iak} \ + \ \epsilon_{iak} \right\} \Big] \ + \ \sum_{\ell \neq k} \lambda_{k\ell} V_{i\ell}(t)
\end{split}
\label{Equation:Firms-Problem_Value-Function}
\end{equation}
For a given well site $i$, the value function $V_{ik}$ represents the present discounted value of all payoffs obtained from starting at state $k$ and behaving optimally in all subsequent periods. 

If no exogenous change in oil prices exists, the value function can be simplified as follows\footnote{In this simplification, we treat $Pr_{k}$ as a constant. For given values of parameters, we compute the value of each $Pr_{k}$, $k = 1, 2, \cdots, K$, by using value function iterations.}:
\begin{equation}
    V_{ik} \ = \
    \begin{cases}
        \begin{split}
            & 
            \ \frac{
                \ f_{ik} \ + \ \lambda_{a} \left\{ V_{ik} \ + \ \sigma \big( \gamma \ - \ \ln(1 - Pr_{k}) \big) \right\} \
            }{
                \rho \ + \ \lambda_{a}
            } \hspace{0.7cm} \text{if \ \ $a = 0$} \\
            &
            \ \frac{
                \ f_{ik} \ + \ \lambda_{a} \left\{ \psi_{i1k} \ + \ \sigma \big( \gamma \ - \ \ln(Pr_{k}) \big) \right\} \
            }{
                \rho \ + \ \lambda_{a}
            } \hspace{1.25cm} \text{if \ \ $a = 1$}.
        \end{split}
    \end{cases}
\label{Equation:Firms-Problem_Euler-Equation}
\end{equation}
Some algebraic manipulation for this simplified value function yields the following equation that drives the dynamics of the firm's optimal drilling decisions:
\begin{equation}
\begin{split}
    & \frac{ \ \dot{V}_{ik} \ + \ f_{ik} \ + \ \lambda_{a} \sigma \big( \gamma \ - \ \ln(1 - Pr_{k}) \big) \ }{\rho} \\
    & = \ \alpha p_{k} \ - \ c \ + \ \sigma \big( \gamma \ - \ \ln(Pr_{k}) \big) \ - \ \sigma \big( \gamma \ - \ \ln(1 - Pr_{k}) \big).
\end{split}
\label{Equation:Firms-Problem_Euler-Equation}
\end{equation}
In this equation, the left-hand side represents the firm's payoff when deciding to stay in state $k$. In other words, the left-hand side is the payoff for the case that the firm chooses $a = 0$ at every decision point. The right-hand side is the firm's payoff if it chooses to drill a horizontal well into well site $i$. Importantly, it is evident that the equation (\ref{Equation:Firms-Problem_Euler-Equation}) drawn from the firm's well-level decisions is equivalent to the necessary condition (\ref{Equation:Social-Planners-Problem_Meaning-of-Costate-Variable}) of the social planner's problem. In addition, the equivalence between them suggests a significant economic implication that the oil market clears. 
