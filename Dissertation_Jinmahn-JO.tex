% ----------------------------------------
% Preambles
% ----------------------------------------
% ------- Set Document Class -------
\documentclass[11pt, final]{ucdavisthesis}


% ------- Load Files Regarding Preambles -------
% 1. To load packages required
\input{00_Preambles/Dissertation_Preamble_Packages.tex}



% ----------------------------------------
% Doc. Setup & Info. for Preliminary Pages
% ----------------------------------------
% ------- Info. for Preliminary Pages -------
\title {
    Essays on Energy and Natural Resource Economics
}
% Exact title of your thesis. Indicate italics where necessary by underlining or using italics. Please capitalize the first letter of each word that would normally be capitalized in a title.

\author {
    \textsc{Jinmahn Jo}
}
% Your full name as it appears on University records. Do not use initials.

\authordegrees {
    \
}
% Indicate your previous degrees conferred.

\officialmajor {
    \textsc{Agricultural and Resource Economics}
}
% This is your official major as it appears on your University records.

\graduateprogram {
    Graduate Program Name
}
% This is your official graduate program name. Used for UMI abstract.

\degreeyear {
    2023
}
% Indicate the year in which your degree will be officially conferred.

\degreemonth {
    September
}
% Indicate the month in which your degree will be officially conferred. Used for UMI abstract.

\committee{Kevin Novan}{Katrina K. Jessoe}{Mark J. Agerton}{}{}
% These are your committee members. The command accepts up to five committee members so be sure to have five sets of braces, even if there are empties.


% ------- Copyright -------
%\copyrightyear{2020}
\nocopyright


% ------- Dedication -------
\dedication{
    \textsl{
        I dedicate my dissertation to my lovely wife Yeana. \\
        \vspace{0.5cm}
        Thank you for all of your unconditional support along the way. 
    }
}


% ------- Abstract -------
\abstract{
This dissertation consists of two essays on how residential consumers respond to a range of different electricity price structures and one on the supply side of the energy sector. 

The first essay examines what information households respond to in their monthly energy bills and the implications of their behavioral responses to electricity pricing. Previous work has uncovered evidence that households largely respond to the average price they pay for energy in the previous billing period. In this essay, I re-examine whether households indeed pay attention solely to the average price. Using detailed hourly consumption data from over 100,000 households in Sacramento, California, I measure the impact of surpassing the first threshold of nonlinear tariff structures in a billing month on households' average daily electricity consumption in subsequent months. My empirical results illustrate that households that consumed enough to be subjected to a higher marginal price in the previous billing period reduced their electricity consumption in the succeeding billing month. This finding illustrates one of the many inefficiencies that arise from tiered rate structures: households' response to prices that are not reflecting current supply conditions but rather the household's past consumption levels.

The second essay studies how households respond to Time-Of-Use (TOU) electricity prices that vary throughout the course of the day. The primary purpose of the time-varying pricing scheme is to reshape households' electricity consumption in and near peak-demand hours---more specifically, to reduce their consumption during peak hours and shift some of their consumption to off-peak hours. The existing literature presents evidence that under TOU tariff structures, residential consumers reduced their electricity consumption during peak price periods, but these reductions were insensitive to the marginal changes in peak-hour prices. In this essay, I re-examine the impact of TOU rates but with a different strategy. Rather than estimating how aggregate consumption responds to TOU rates, I decompose household electricity consumption into two distinct categories: consumption for non-temperature-control and temperature-control uses. My empirical analysis shows that households indeed responded to the magnitude of the price increase in the peak rate period; however, the response was not the same for the two consumption categories. In particular, while non-temperature-control-driven consumption during the peak hours markedly fell as the peak price increased, temperature-control-driven consumption fell prior to the peak hours, and actually increased during the peak hours, relative to the reduction in pre-peak hours, as the peak price increased. Ultimately, the differences in the responses across these two channels masked households' high price sensitivity. This also illustrates that the two types of electricity consumption evolved differently, and nonlinearly, according to daily heating degree days and the point electricity was consumed in time. These findings suggest that adopting autonomous heating control systems or augmenting additional across-day flexibility to the typical structure of TOU electricity pricing is required to maximize the benefits of the pricing scheme.

The third essay develops a discrete choice dynamic programming (DCDP) framework in continuous time by formulating fracking firms' drilling decisions as an optimal stopping problem. In a recent paper, Hotelling's classic model of nonrenewable resource extraction is recast as a drilling problem to explain observations in Texas that drilling activity responds to oil prices sensitively, while oil production from existing wells does not respond. The model in this prior paper, however, cannot rationalize the empirical phenomenon that firms in North Dakota drilled wells in both low- and high-quality locations. The DCDP model uses cost shocks to rationalize the simultaneous drilling of resources with heterogeneous quality. Further, the model can be estimated empirically using microeconomic data and also solved analytically for an aggregate solution. In the limit, the model converges to the classic Hotelling model.
}


% ------- Acknowledgments -------
\acknowledgments{
I am deeply grateful for the teaching, guidance, and encouragement I received from my committee members: Kevin Nova, Katrina Jessoe, and Mark Agerton. \\

My graduate school colleagues have contributed to my academic and daily lives at UC Davis. Thanks especially to Seunghyun Lee, Yujing (Megan) Song, Emily McGlynn, and Sangwon Lee. \\

Funding from the Jastro \& Shields Research Fellowship is gratefully acknowledged. 
}



% ----------------------------------------
% Main Document 
% ----------------------------------------
\begin{document}

\makeintropages
% Processes/produces the preliminary pages


% ------- Chapter 1 -------
\chapter{Dynamic Consumption Behavior with the Lagged Price Signals}
\label{Chapter:Chapter-1}

\section{Introduction}
\label{C1-Section:Introduction}
Increasing-Block Pricing (IBP) is one of the most common electricity rate plans. Under the pricing scheme, the marginal price increases, in a nonlinear way, with the customer's usage. Specifically, the marginal price customers pay for their marginal electricity consumption is a step function of their aggregate consumption in a billing cycle. Naturally, the relationship between price and quantity consumed under IBP would imply that for customers, knowing how much electricity they have used from the first day of the current billing month is inevitable to properly adjust their consumption behavior according to the price variations determined by their past consumption decisions in that billing month. 

Electricity bills, usually issued every month, have been a primary, perhaps for some residential customers the only, source of information about their electricity consumption in a billing cycle, including the total amount of consumption and the marginal price. Unlike a decade ago, many households can access (near-)real-time data about their electricity consumption these days because of the diffusion of smart metering. But as illustrated in \cite{Dynamic-Salience-with-Intermittent-Billing_Gilbert-and-Zivin_2014}, households change their consumption behavior after receiving a monthly bill, even though the pieces of information, which are available directly or after some computations from it, are not up-to-date. In other words, the paper shows that electricity bills, which are intermittent tough, determine household electricity consumption at least partly. 

Economists have studied how households respond to monthly bill-providing information, especially to different prices. \cite{Do-Consumers-Respond-to-Marginal-or-Average-Price?-Evidence-from-Nonlinear-Electricity-Pricing_2014_(Ito)}, demonstrating that when faced with IBP, residential consumers respond to the average price rather than the marginal price, presents convincing evidence that households respond more to lagged average prices than contemporaneous ones. More recently, \cite{Misunderstanding-Nonlinear-Prices_2020_(Shaffer)} shows that not all households uniformly respond to the average price. To be specific, this paper exhibits that most residential consumers respond to the average price, but their response can be veiled by the response of a small number of households mistakenly applying the marginal price to all consumption units, not the last unit. And in both papers, it is stressed that the two distinct types of misunderstanding nonlinear prices will lead to households' inefficient electricity consumption, like over- or under-consumption. 

This paper examines another piece of information available from monthly electricity bills that could alter a household's consumption behavior: whether the total electricity consumption in a billing cycle surpasses a base usage quantity at which the marginal price sharply jumps. Specifically, I investigate how households adjust their electricity consumption in a billing month when they know that their total electricity consumption in the previous billing month just exceeded a threshold at which the marginal price discontinuously increases. In other words, focusing on two consecutive billing cycles, I study how households respond to a delayed price signal, indicating that they were subjected to a higher margin price in the previous billing month. 

To measure the impact of surpassing a base usage quantity in a billing cycle (denoted Period 0) on household electricity consumption in the following billing cycle (denoted Period 1), I utilize the sharp discontinuity in the marginal price at the threshold. Under IBP, a household's aggregate electricity consumption at some point completely determines the price the household pays for the marginal unit of consumption at that point. In other words, in my setting, the known-to-consumer price determination mechanism is what determines a household's treatment status. On top of that, households had no practical way to know how much electricity they had consumed from the first day of the very billing cycle at the point of consumption, especially before installing smart meters. They were able to find out whether their total consumption in a billing cycle surpassed the threshold only after receiving their monthly bills at the beginning of the subsequent billing cycle. Therefore, it is compelling that residential consumers were not able to control their electricity consumption precisely to avoid crossing the cutoff point. Resultingly, the two points demonstrate that my RD design clearly satisfies the fundamental assumption required for identification: around the threshold, the treatment assignment is random. 

Implementing the RD design to monthly billing records from hundreds of thousands of residential consumers in Sacramento, CA, I find households' interesting behavioral changes with respect to their electricity consumption: the treated households (i.e., households whose aggregate electricity consumption in a billing month exceeded the first base usage quantity) reduced their electricity consumption by 0.1\% in the following billing month. Importantly, this result suggests another inefficiency stemming from households' responses to nonlinear electricity pricing---households also respond to the delayed marginal price, reflecting not their contemporaneous consumption level but their past one. In addition, my empirical analysis demonstrates that among the treated households, the larger the electricity consumption during Period 0, the greater the magnitude of the reduction in electricity consumption during Period 1. I also examine heterogeneity in responsiveness to receiving the delayed price signal across households based on different rate codes and seasons. Furthermore, my investigation into the impacts of the discontinuous increase in the marginal price shows that reductions in household electricity consumption persist over multiple billing cycles but tend to dissipate gradually. 

The rest of this paper proceeds as follows. In Section 2, I discuss my empirical approach and data. Section 3 presents the results from my empirical analysis. Section 4 describes the policy implications of my key findings, and Section 5 concludes. 


\section{Data and Research Design}
\label{C1-Section:Data-and-Research-Design}
This section provides a detailed description of the data utilized for my empirical analysis. Furthermore, this section demonstrates a key feature of my research design.


% Institutional Background
\subsection{Background of Residential Rates of Sacramento Municipal Utility District}
\label{C1-Sub-section:Background-of-Residential-Rates-of-SMUD}
Sacramento Municipal Utility District (SMUD), which is the nation's sixth-largest community-owned electric utility, provides electricity to most of Sacramento County and small portions of adjoining Placer and Yolo Counties.\footnote{According to the company information presented on \href{https://www.smud.org/en/Corporate/About-us/Company-Information}{SMUD's website}, the size of this utility's service area is about 900 square miles.}

\afterpage{
    \begin{figure}[t!]
        \centering
        \includegraphics[scale = 0.097]{02_Chapter-1/00A_Figures/Figure_SMUD-Residential-Rates_Variable-Charges-and-Base-Usage-Quantities.png}
        \caption{Tier Rates and Base Usage Quantities of SMUD Residential Rates}
        \caption*{
            {\small
            \textit{Note}: 
            The figure illustrates how SMUD changed tier rates and base usage quantities of the three major residential rate plans (i.e., RSCH, RSEH, and RSGH) over time. Both tier rates and base usage quantities show significant seasonality. The two rate plans for electric-heating households (i.e., RSCH and RSEH) had the same base usage quantities. There have been only two tiers since September 2009.
        }}
        \label{Figure:SMUD-Residential-Rates_Variable-Charge-and-Base-Usage}
    \end{figure}
}
Before the default residential rate switched to the Time-Of-Day (TOD) rate, most SMUD residential customers chose residential rates having an increasing nonlinear block-tier structure.\footnote{In my sample, only 5\% of residential customers adopted the TOD rate, although SMUD already offered it.} The three most popular rates for SMUD residential customers were Standard General Service (RSGH), Standard Closed Electric-Heated Service (RSCH), and Standard Open Electric-Heated Service (RSEH).\footnote{Specifically, more than 75\% of SMUD residential customers in my dataset chose the RSGH rate, whereas 2\% and 20\% of households in my dataset adopted the RSCH and RSEH rates, respectively.} For those residential rates, the marginal price of the energy charge was a step function of monthly consumption relative to a base usage quantity per month, which varies seasonally. Figure \ref{Figure:SMUD-Residential-Rates_Variable-Charge-and-Base-Usage} illustrates variations in price and base usage quantity over time. Two points are noteworthy from this figure: first, both tier rates and base usage quantities of the energy charges showed substantial seasonality; second, the structure of the residential rates changed from three-tier to two-tier since September 2009.

In addition to the variable charge (i.e., the energy charge), households choosing one of the three rates should pay a per-month fixed charge, called the System Infrastructure Fixed Charge. As shown in Figure \ref{Figure:SMUD-Residential-Rates_Fixed-Charge}, the unit price of the fixed charge significantly increased between 2009 and 2014. 




% Data and Summary Statistics
\subsection{Data and Summary Statistics}
\label{C1-Sub-section:Data-and-Summary-Statistics}
From SMUD, I obtain household-level monthly billing history of residential consumers in the Sacramento area from 2004 to 2013. For each monthly record, account ID, premise ID, rate code, billing start and end dates, monthly consumption with its breakdown into each tier, monthly fixed charge, monthly variable charge only for kWh usage, total monthly bill, and an indicator related to solar adoption are included in the data. Of note, in my empirical analysis, I assume that a pair of account and premise IDs corresponds to an individual household. And because the monthly billing data contain no price and base usage quantity information, I append historical price schedules and base usage quantities presented by SMUD. Unfortunately, my monthly billing data also lack any socioeconomic and demographic information. 

First, I focus on households that consistently used one of the three major residential rates (i.e., RSGH, RSCH, and RSEH) in their billing history. Second, I only utilize billing records before 2010, from which SMUD got down to install smart meters. Third, I focus on households whose number of billing periods is greater than or equal to 24. Fourth, I focus on SMUD residential customers with reliable billing records only.\footnote{To be specific, I exclude, from the sample used for my empirical analysis, households that have billing records being applied to any of the following conditions: 1) observations whose length of the billing period is either less than 27 or greater than 34; 2) observations with negative values for either quantities or charges; 3) observations having overlapping billing periods within a pair of account and premise IDs; and 4) observations whose number of days from the previous billing period is greater than 14.} Fifth, my sample only includes households that crossed the lower threshold at least once in their billing history.\footnote{In other words, I drop always-light- and always-heavy-users from my sample. } The procedure results in 16,322,353 billing records for 365,975 households. Table \ref{Table:Summary-Statistics} provides summary statistics for my sample. Furthermore, Figure \ref{Figure:Household-Average-Daily-Electricity-Consumption-by-Month-of-Year} shows, for each rate code, households' average daily electricity consumption by month of the year. 

To take account of the impact of weather conditions, especially outdoor temperatures, on household electricity consumption, I utilize the Local Climatological Data (LCD) for the Sacramento International Airport, published by the National Oceanic and Atmospheric Administration (NOAA). Using daily heating degree days (HDDs) and daily cooling degree days (CDDs) in the LCD between 2004 and 2009, I calculate each monthly billing period's accumulated HDDs and CDDs, which are used to compute the average daily HDDs and CDDs. 




% Research Design
\subsection{Research Design}
\label{C1-Sub-Section:Research-Design}

\subsubsection{Monthly Bill as the Only Source of Electric Usage Information for Households}
\label{C1-Sub-Sub-section:Monthly-Bill-as-the-Only-Source-of-Electric-Usage-Information-for-Households}
Before 2009, there was no feasible way for SMUD residential customers to access real-time information related to their electricity use. SMUD initiated installations of smart meters, allowing its residential and business customers to view their electricity usage online when they want, in late 2009. The electric service completed it in the first quarter of 2012. Also, the three types of bill alert SMUD are offering were introduced in 2017.\footnote{SMUD provides its customers with three types of bill alerts, via text or e-mail, as a billing service: 1) Mid-Bill Alerts send an alert on the 16th day of a customer's billing period and advise what his usage has been and what the cost is as of that day, 2) High Bill Alters compare a customer's current billing cycle to the same billing cycle in the previous year and alerts the customer if their current usage is running higher than before, and 3) Bill Threshold allows a customer to know when his bill has reached a certain amount set in advance by himself.} Therefore, for households using SMUD-delivering electricity, the only practical source of information about their electricity consumption had been their monthly bill statements, which send out (either e-mail or U.S. mail) after 3 or 4 business days from the last day of each billing cycle. 

The issue of households' welfare losses due to their response to discontinuous changes in the lagged marginal price suggests the importance of providing seemly price information in an appropriate manner. Many studies about various time-varying electricity pricing show that households changed their consumption behavior in response to the information about consumption and prices \citep{Dynamic-Pricing-of-Electricity-in-the-Mid-Atlantic-Region_Econometric-Results-from-the-Baltimore-Gas-and-Electric-Company-Experiment_Faruqui-et-al_2011, Knowledge-is-Less-Power_Jessoe-and-Rapson_2014, The-Effect-of-Information-on-TOU-Electricity-Use:An-Irish-Residential-Study_Pon_2017, Information-vs-Automation-and-Implications-for-Dynamic-Pricing_Bollinger-and-Hartmann_2020}. My empirical finding demonstrates that even under IBP, such information, though lagged, still plays a role in household electricity consumption. In this respect, providing household-specific as well as current price information for residential consumers, via text messages or app notifications regularly, could encourage them to respond to \textit{true} price signals rather than lagged ones, which in turn avoid the negative impact on household welfare. Based on the dissipating effect of intermittently salient information discussed in \cite{Dynamic-Salience-with-Intermittent-Billing_Gilbert-and-Zivin_2014}, a high frequency of informing the latest tailored price information might maximize households' behavior change in electricity consumption. In addition, because sending such information-bearing notifications is available at a very low cost these days, this type of information provision would be a practical policy instrument for utilities, especially in developing countries where the transition toward dynamic electricity pricing is difficult due to substantial investments in installing smart metering systems. 


\subsubsection{Regression Discontinuity Design}
\label{C1-Sub-Sub-Section:Regression-Discontinuity-Design}
In this paper, I employ a Regression Discontinuity (RD) design to examine how households' electricity consumption responds to the marginal price informed via monthly energy statements under Increasing-Block Pricing (IBP). In previous studies, a common challenge in measuring consumption responses to price changes has been discussed repeatedly: constructing a well-defined control group is difficult due to that consumers typically experience the same price variation. However, the setting I exploit in this paper enables me to address the identification challenge.  

The RD design I implement in this paper relies on three points. First, the marginal price is a step function of consumption level in the increasing block-tier rate plans chosen by SMUD residential customers. That is, under IBP, the price a household pays for the marginal electricity consumption increases discontinuously at some pre-determined aggregate consumption in a billing cycle. Second, as discussed in Section \ref{C1-Sub-Sub-section:Monthly-Bill-as-the-Only-Source-of-Electric-Usage-Information-for-Households}, before 2009, SMUD residential customers had practically no way to know the marginal price in a billing cycle within the very cycle. They were informed of the price they paid for the marginal electricity consumption in a billing cycle only through their electricity bills delivered in the following cycle. Third, it is not generally feasible for households to consume only a pre-targeted amount of electricity within a billing cycle. In general, households have limited capability to control their electricity consumption due to the minimal essential demand (e.g., usage for refrigerators and lighting). In addition, because household electricity consumption heavily depends on outdoor temperature variation, managing one's own electricity usage not to exceed the target amount of electricity consumption could incur too high information cost, which might result in rational inattention \citep{Rational-Inattention-and-Energy-Efficiency_Sallee_(2014)}, even if households are available to adjust their consumption behavior with complete flexibility. 

\afterpage{
    \begin{figure}[t!]
        \centering
        \includegraphics[scale = 0.115]{02_Chapter-1/00A_Figures/Figure_Average-Daily-Electricity-Consumption-in-Period-1-over-NC0.png}
        \caption{Mean of Average Daily Electricity Consumption in Period 1 over $\overline{NC}_{0}$}
        \caption*{
            {\small
            \textit{Note}: 
            This figure's scatter points correspond to the average daily electricity consumption in Period 1, computed by bins with a bandwidth of 1\% of $\overline{NC}_{0}$. The solid line on each side of the vertical dot-dash line is a parametric fit obtained from the regression of the average daily electricity consumption on $\overline{NC}_{0}$. The dashed red line is an extension of the solid red line. The gap between the dashed red and solid blue lines seems to indicate a non-negligible treatment effect. 
        }}
        \label{Figure:Average-Daily-Electricity-Consumption-in-Period-1-over-NC0}
    \end{figure}
}
Regarding the first point, the discontinuities under the nonlinear electricity schedules allow utilizing a RD design. In my RD design, the running variable is the level of electricity consumption in a household during a billing period (denoted as Period 0), whereas the outcome variable corresponds to the household's average daily electricity consumption during the subsequent billing period (denoted as Period 1). So, in this quasi-experimental setting, I compare SMUD residential customers just above and below the thresholds of the tier rates, called base usage quantities. Under IBP, surpassing a threshold leads to an increase in the marginal price households pay for electricity consumption mechanically. Here, the discontinuous increase in the marginal price, which accompanies \textit{no discontinuous change in the average price}, applies only to Period 0, not to Period 1.\footnote{The average price smoothly grows around the cutoff point.} Moreover, information about whether households were subject to a higher marginal price in a billing period is delivered early in the subsequent billing period through their monthly electricity bills. Therefore, any changes with respect to the electricity consumption of households just above the threshold (i.e., households in the treatment group) in Period 1, compared to households just below the threshold (i.e., households in the control group), can be understood as their short-term behavioral responses stemming from the sharp jump in the marginal price in Period 0. Figure \ref{Figure:Average-Daily-Electricity-Consumption-in-Period-1-over-NC0}, showing how the mean of households' average daily electricity consumption in Period 1 evolves around the lower base usage quantity, seems to indicate the existence of such behavioral responses. 

The last two points demonstrate that the fundamental identifying assumption of the RD design is reasonable. The fundamental identifying assumption is that SMUD residential customers just below a base usage quantity are expected to be very similar to those just above it, along with observed and unobserved characteristics. In other words, a group of households in the small neighborhood of the threshold is not different from one obtained from a randomized experiment. In my setting for empirical analysis, SMUD residential customers were unable to be aware of how far away they were from a given cutoff point in real time. Furthermore, as discussed above, it is not convincing that they can perfectly control their electricity consumption during a billing cycle to use exactly a target amount of electricity by the end of the last day of the billing cycle. Hence, it is highly unlikely that the customers precisely adjusted their consumption behavior so as to avoid surpassing the cutoff point, which in turn prevented them from leading to a higher marginal price. That is, it seems plausible that households were not able to sort themselves around the threshold strategically. Therefore, any discontinuity gap in the outcome variable can be attributed to the discontinuous increase in the marginal price at the threshold in Period 0. 


\subsubsection{The Validity of the Regression Discontinuity Design}
\label{C1-Sub-Sub-Section:The-Validity-of-the-Regression-Discontinuity-Design}
\afterpage{
    \begin{figure}[t!]
        \centering
        \includegraphics[scale = 0.101]{02_Chapter-1/00A_Figures/Figure_Distribution-of-Electricity-Consumption.png}
        \caption{Distribution of Electricity Consumption by SMUD Residential Customers}
        \caption*{
            {\small
            \textit{Note}: 
            This figure presents histograms, with kernel density estimates, for electricity consumption by SMUD residential customers. Each of the six panels in the figure is for a pair of three major residential rates (i.e., RSCH, RSEH, and RSGH) and two seasons (i.e., summer and winter). Dot-dashed vertical lines in each panel are base usage quantities for the corresponding rate code and season.
        }}
        \label{Figure:SMUD-Billing-Data_Histogram_By-Season-and-Rate-Code}
    \end{figure}
}
Two pieces of evidence support the assumption that base usage quantities do not correspond to jumps in household characteristics. First, as illustrated in Figure \ref{Figure:SMUD-Billing-Data_Histogram_By-Season-and-Rate-Code}, each density plot of the running variable is very smooth, without any bump (i.e., excess mass), around base usage quantities at which marginal prices jump. The set of density plots that show apparent continuity at the thresholds suggests households' inability to precisely adjust their electricity consumption in order not to be subject to a higher marginal price. 

\afterpage{
    \begin{figure}[t!]
        \centering
        \includegraphics[scale = 0.115]{02_Chapter-1/00A_Figures/Figure_Average-Daily-Electricity-Consumption-in-Period-0-over-NC0.png}
        \caption{Mean of Average Daily Electricity Consumption in Period 0 over $\overline{NC}_{0}$}
        \caption*{
            {\small
            \textit{Note}: 
            In this figure, the scatter points correspond to the average daily electricity consumption in Period 0, calculated by binds with a bandwidth of 1\% of $\overline{NC}_{0}$. As can be seen, the average daily electricity consumption evolves smoothly around the cutoff point (i.e., $\overline{NC}_{0} = 0$). 
        }}
        \label{Figure:Average-Daily-Electricity-Consumption-in-Period-0-over-NC0}
    \end{figure}
}
Second, Figure \ref{Figure:Average-Daily-Electricity-Consumption-in-Period-0-over-NC0} demonstrates that households' average daily electricity consumption during Period 0 evolved smoothly around the lower cutoff point. This figure allows me, at a minimum, not to reject the assumption of local randomization around the base usage quantity, even though examining an observed covariate around the thresholds is not also a direct test for the validity of the assumption.




\section{Empirical Analysis and Results}
\label{C1-Section:Empirical-Analysis-and-Results}
\subsection{Household Responses to Lagged Marginal Prices in the RD Design}
\label{Sub-Section:Household-Responses-to-Lagged-Marginal-Prices-in-the-RD-Design}

\subsubsection{Econometric Model}
\label{Sub-Sub-Section:Econometric-Model}
Exploiting the sharp RD design described earlier, I estimate the following econometric specification to measure how SMUD residential customers respond, in terms of their electricity consumption in a billing month (i.e., Period 1), to the discontinuous change in the marginal price due to exceeding the lower base usage quantity in the previous billing month (i.e., Period 0):
\begin{equation}
\begin{split}
    ADC_{i, 1} \ 
    & = \ \beta \hspace{0.07cm} \mathbb{1}[Treatment]_{i, 0} \ + \ f\left( \overline{NC}_{i, 0} \right) \ + \ \boldsymbol{X}'\boldsymbol{\alpha} \ + \ \delta_{ym} \ + \ \epsilon_{i, 1}
\end{split}
\label{Eq:RD-Econometric-Model}
\end{equation}

The dependent variable $ADC_{i, 1}$ is the average daily consumption by household $i$ in Period 1. $\widebar{NC}_{i, 0}$ corresponds to the running variable, household $i$'s normalized consumption in Period 0:
\begin{equation}
\begin{split}
    \overline{NC}_{i, 0} \ 
    & = \ kWh_{i, 0} \ - \ BUQ_{i, 0}
\end{split}
\label{Eq:Normalized-Consumption}
\end{equation}

where $kWh_{i, 0}$ and $BUQ_{i, 0}$ are, in Period 0, household $i$'s aggregate electricity consumption and the lower base usage quantity, respectively. The binary indicator variable $\mathbb{1}[Treatment]_{i, 0}$ is equal to 1 only if household $i$'s aggregate electricity consumption in Period 0 exceeded the lower base usage quantity:
\begin{equation}
\begin{split}
    \mathbb{1}[Treatment]_{i, 0} \ 
    & = \ 
    \begin{cases}
        \hspace{0.2cm} 0 \hspace{0.7cm} \text{if} \hspace{0.3cm} \overline{NC}_{i, 0} \ \leq \ 0 \\
        \hspace{0.2cm} 1 \hspace{0.7cm} \text{if} \hspace{0.3cm} \overline{NC}_{i, 0} \ > \ 0
    \end{cases}
\end{split}
\label{Eq:Treatment-Status-Determination-Rule}
\end{equation}

The function $f\left( \cdot \right)$ is a continuous function of $\widebar{NC}_{i, 0}$ at $BUQ_{i, 0}$. $\boldsymbol{X}$ are covariates, such as average daily Cooling Degree Days (CDDs) and average daily Heating Degree Days (HDDs). $\gamma_{i}$ and $\delta{ym}$ are household and billing-year-by-billing-month fixed effects (FEs), respectively. The last term $\epsilon_{i, 1}$ is a stochastic error term. In this model, the coefficient of interest $\beta_{1}$ captures the treatment effect. I cluster the standard errors at account and premise IDs to correct for serial correlation. 

In the specification, each household's average daily electricity consumption, instead of the aggregate consumption, in a billing cycle is utilized as the dependent variable. My sample contains household-level monthly billing records. But because of the fact that each billing month consists of a different number of days, I use each billing month's average daily consumption for my empirical analysis. For the same reason, average daily CDDs and HDDs are exploited in later analysis. 



\subsubsection{Regression Discontinuity Results}
\label{Sub-Sub-Section:Regression-Discontinuity-Results}
Table \ref{Table:RD-Results} summarizes the regression results of several alternate specifications for the bandwidth of 20\%. Column (1) reports estimates from the simplest RD specification without any control and FEs. Column (2) adds controls for households' cooling and heating needs, significantly driving household electricity consumption. In addition to those two controls, columns (3) and (4) use household and billing-year-by-billing-month FEs, respectively. Here, the estimated treatment effects are substantially smaller, suggesting that controlling for both household-specific and time-varying factors is important. Column (5) shows results from the specification with the controls and both FEs. The estimated treatment effect is a discontinuous reduction in households' average daily electricity consumption by 0.024 kWh (i.e., about 0.1\% of the average daily consumption). This estimate is statistically different from 0 at the 5\% level. Columns from (6) to (10) additionally include the interaction term between the binary indicator and running variables. Adding the interaction terms to the specifications has only minimal impact on the estimates.  

The identified reduction in household electricity consumption clearly demonstrates that households respond to lagged marginal prices. As discussed in Section \ref{Sub-Sub-Section:Regression-Discontinuity-Design}, the discontinuous increase in the marginal price at the lower base usage quantity was not followed by any change in the average price. Moreover, the households in my sample were able to notice the price jump only through their monthly bill statements, which were delivered a few days after the first day of the new billing month. Collectively, my estimates reveal an inefficiency stemming from households' responses to nonlinear electricity pricing because the lagged marginal price reflects their consumption history, not their contemporaneous consumption. In other words, under IBP, the untimely price signal drives households' electricity consumption. 

Importantly, the estimated treatment effect also indicates that SMUD residential customers overreacted to the lagged marginal price under nonlinear electricity pricing. The discontinuous change in the marginal price at the lower base usage quantity occurred in a billing cycle (i.e., in Period 0). And my estimates show that in the following billing cycle (i.e., in Period 1), the customers reduced their electricity consumption as a response to the price variation. Consequently, the sharp increase in the marginal price at the cutoff point in Period 0 affected all consumption, not the marginal one, in Period 1. That is, households excessively applied the lagged marginal price to every unit of electricity consumption. 

Inspired by \cite{Misunderstanding-Nonlinear-Prices_2020_(Shaffer)}, the estimates could be interpreted differently. The paper finds that a subgroup of less than 10\% of households was driving the seemingly overall response. If this is also true in my setting, then the measured decrease in household electricity consumption would be attributed to a subset of my sample. Let us suppose that there are two distinct types of SMUD residential customers: households over-responding to the lagged marginal price and those not responding to it.\footnote{Here, I do not consider the type of households that respond to the average price because the change in the marginal price at the threshold does not accompany any change in the average price.} My back-of-the-envelope calculation suggests that about 4\% of over-responders produce the estimated treatment effect.\footnote{In my rough computation, I exploit the price elasticity for the lagged marginal price provided in \cite{Do-Consumers-Respond-to-Marginal-or-Average-Price?-Evidence-from-Nonlinear-Electricity-Pricing_2014_(Ito)}.} Interestingly, this calculation parallels the finding in the paper. 



\subsubsection{Robustness Checks}
\label{Sub-Sub-Section:Robustness-Checks}
\noindent
\textit{\textbf{Regression Discontinuity Results for Different Bandwidths and Functional Forms}} --- 
Table \ref{Table:Robustness-Checks_BWs} summarizes the regression results for a set of different bandwidths. The estimated treatment effect for the households in a very narrow range from the lower base usage quantity (i.e., the households within the bandwidth of 5\%) is not statistically significant even at the 10\% level. Except for the bandwidth of 5\%, the treatment estimates range from $-$0.038 to $-$0.069 and statistically differ from zero at least at the 5\% level. The estimated treatment effect is almost identical for the bandwidths of 10\% and 15\%. For wider bandwidths falling between 20\% and 40\%, the magnitude of the estimated treatment effect increases and remains stable.\footnote{The number of observations increases with the size of the bandwidth exploited, except the two widest ones. The exceptions are because I drop observations crossing the higher base usage quantity to avoid picking up the effect of surpassing the higher cutoff point.} Interestingly, this table clearly shows that the wider the bandwidth employed, the larger the estimated treatment effect. In other words, the treatment estimates approach zero as I move even closer to the lower base usage quantity.

There are several possible explanations for this monotonic trend in the treatment effect. First, it may be more difficult or demanding for SMUD residential customers near the threshold to notice, from their monthly bill statements, that their electricity consumption in the previous billing month barely exceeded the lower base usage quantity, which in turn made them experience a discontinuous increase in the marginal price. Second, households whose electricity consumption just surpassed the cutoff point in a billing cycle could intentionally ignore the lagged price signal in the subsequent billing cycle. Some of them likely understood that their immediate electricity consumption was utterly irrelevant to the signal. And it is also possible that adjusting their electricity consumption pattern against the lagged marginal price during a whole billing month led to too much cost for some treated households very near the threshold compared to its benefit. Third, households near the lower base usage quantity may respond differently to the lagged marginal price compared to those farther from the threshold. Specifically, conditional on a given magnitude of the increase in the lagged marginal price, heavy electricity consumers could be more responsive to the price signal. 

Tables \ref{Table:Robustness-Checks_Functional-Forms_1st-and-2nd-Order-Polynomial-Models} and \ref{Table:Robustness-Checks_Functional-Forms_3rd-and-4th-Order-Polynomial-Models} present the regression results from other specifications having different functional forms. As illustrated in Figure \ref{Figure:The-Impact-of-the-Change-in-the-MP-due-to-Surpassing-the-Lower-BUQ}, a linear regression function seems highly reasonable on both sides of the threshold, even for broader bandwidth. The robustness of the results from the first four columns in Table \ref{Table:Robustness-Checks_Functional-Forms_1st-and-2nd-Order-Polynomial-Models} confirms that the linear approximation of the regression line does not induce considerable biases in my RD estimates. In addition, the RD estimates in the two tables suggest that for wider bandwidths, adding higher-order polynomials of the running variables is still reasonable for the estimates to be precise.


\par \vspace{0.5cm}
\noindent
\textit{\textbf{Falsification Test}} ---
Figure \ref{Figure:Robustness-Checks_Falsification-Tests} summarizes the results from falsification tests that examine treatment effects at two placebo cutoff points (i.e., at $-$30\% and 40\% of the normalized electricity consumption in Period 0 from the \textit{true} lower base usage quantity).\footnote{That is, the two false thresholds are at 70\% and 140\% of the normalized consumption in Period 0. Following the suggestion in \cite{Regression-Discontinuity-Designs_A-Guide-to-Practice_Imbens-and-Lemieux_2008}, I select those false cutoff points that are close to the median of the running variable on each side of the \textit{true} cutoff point.} In the falsification tests, I only use bandwidths less than the distance between a false threshold and the (actual) lower base usage quantity to avoid capturing some of the treatment effect. As clearly demonstrated, no estimate is different from zero at the 5\% level, suggesting that my RD design is valid. 




\subsubsection{Heterogeneity in Households Responses to Lagged Marginal Prices in the RD Design}
\label{Sub-Sub-Section:Heterogeneity-in-Household-Responses-to-Lagged-Marginal-Prices-in-the-RD-Design}
(...)



\subsection{Multi-Period Household Responses to Lagged Marginal Prices in the RD Design}
\label{Sub-Section:Multi-Period-Household-Responses-to-Lagged-Marginal-Prices-in-the-RD-Design}
(...)


\section{Policy Implications}
\label{C1-Section:Policy-Implications}
Utilizing the Regression Discontinuity (RD) design described in Section \ref{C1-Sub-Sub-Section:Regression-Discontinuity-Design}, I show that under Increasing-Block Pricing (IBP), household electricity consumption responded to the lagged marginal price informed through their monthly bill statements. Between 2004 and 2010, SMUD residential customers had no feasible way to know, in real time, how much electricity they had consumed since the beginning of a billing cycle, how much they paid for the marginal unit, and so on. In such a situation, their behavior to exploit available information---, for example, the information contained in their monthly bills delivered after several days from the last day of each billing cycle---as much as possible in their decisions for electricity consumption seems reasonable. Nevertheless, it is undeniable that their electricity consumption responding to the lagged marginal price is suboptimal. As discussed, the marginal price household responded to is not for the marginal unit in the current billing period but for the last unit in the previous one. Responding to \textit{wrong} price signals, SMUD residential customers reduced their electricity consumption. In other words, the \textit{informed} consumption decisions made by households, based on the lagged marginal price, caused welfare losses to them. 

The issue of households' welfare losses due to their response to discontinuous changes in the lagged marginal price suggests the importance of providing seemly price information in an appropriate manner. Many studies about various time-varying electricity pricing show that households changed their consumption behavior in response to the information about consumption and prices \citep{Dynamic-Pricing-of-Electricity-in-the-Mid-Atlantic-Region_Econometric-Results-from-the-Baltimore-Gas-and-Electric-Company-Experiment_Faruqui-et-al_2011, Knowledge-is-Less-Power_Jessoe-and-Rapson_2014, The-Effect-of-Information-on-TOU-Electricity-Use:An-Irish-Residential-Study_Pon_2017, Information-vs-Automation-and-Implications-for-Dynamic-Pricing_Bollinger-and-Hartmann_2020}. My empirical finding demonstrates that even under IBP, such information, though lagged, still plays a role in household electricity consumption. In this respect, providing household-specific as well as current price information for residential consumers, via text messages or app notifications, could encourage them to respond to \textit{true} price signals rather than lagged ones, which in turn avoid the negative impact on household welfare. Based on the dissipating effect of intermittently salient information discussed in \cite{Dynamic-Salience-with-Intermittent-Billing_Gilbert-and-Zivin_2014}, a high frequency of informing the latest tailored price information might maximize households' behavior change in electricity consumption. In addition, because sending such information-bearing notifications is available at a very low cost these days, this type of information provision would be a practical policy instrument for utilities, especially in developing countries where the transition toward dynamic electricity pricing is difficult due to substantial investments in installing smart metering systems. 

No response to the lagged marginal price at the higher base usage quantity presented in Section \ref{C1-SubSubSection:Heterogeneity-in-Household-Response-to-the-Lagged-Marginal-Prices} suggests an implication for setting a price for each block in IBP. As discussed, the result might be attributable to the relatively small price increase at the cutoff point. This explanation is likely to parallel with the discussion, in \cite{Does-Marginal-Price-Matter?-A-Regression-Discontinuity-Approach-To-Estimating-Water-Demand_Nataraj-and-Hanemann_2011}, that regarding water demand management, imposing a higher marginal price seems required to deter consumption by heavy users. In light of this paper and my RD estimates at the higher threshold, setting a high enough price for the highest block in IBP seems necessary to curb demand from heavy electricity consumers, which appears to contribute to a higher spot price in the wholesale electricity market. 



\section{Conclusion}
\label{C1-Section:Conclusion}
In this paper, I examine how the electricity consumption of SMUD residential customers responded to the marginal price informed through monthly bill statements under Increasing-Block Pricing (IBP). In a setting with a valid regression discontinuity design, my empirical analysis shows that households, on average, reduced their electricity consumption in response to the discontinuous price change in the marginal price in the immediately preceding billing cycle. In other words, the empirical results of my analysis reveal an inefficiency of IBP. But at the same time, the interesting response demonstrates the potential to induce desired behavioral changes in household electricity consumption by providing appropriate, even lagged, price information. On top of that, the identified response, which may be driven by a subset of households in my sample, suggests that it is also important for households to correctly understand price-related information provided in order to make optimal decisions about electricity consumption. 



% ------- Chapter 2 -------
\chapter{Prices Still Matter: How Households Adjust Their Consumption Behavior under Time-Of-Use Electricity Pricing}
\label{Chapter:Chapter-2}

\section{Introduction}
\label{C2-Section:Introduction}
These days many energy utilities are moving towards Time-Of-Use (TOU) electricity pricing, which has become feasible owing to the diffusion of renewable electricity generation capacity and smart metering technology. Under a TOU tariff structure, the pre-determined growth in peak-hour rate, which is usually invariant across days, induces reductions in electricity consumption when the cost of supplying the electricity and the capacity constraints on transmission networks are at their greatest (i.e., during peaks). Many evaluations of experiments that assessed how households respond to TOU tariffs have consistently documented reductions in electricity consumption during peak hours. Furthermore, TOU prices can incentivize consumers to shift their electricity consumption from peak to off-peak hours. Ultimately, how effective the time-varying electricity prices are at reducing or relocating electricity consumption depends on how responsive electricity consumers are to the magnitude of the price increase in peak-demand hours. 
Suppose that consumers are very sensitive to the change in peak-hour electricity price. In that case, there might be additional gains from adopting even more granular forms of dynamic pricing, such as Real-Time Pricing (RTP). \cite{Peaking-Interest:How-Awareness-Drives-the-Effectiveness-of-Time-of-Use-Electricity-Pricing_Prest_2020}, however, finds that even during peaks, households were highly insensitive to the incremental risings in the peak rate.\footnote{This paper, which also utilizes the CER experiment datasets, expresses the results as follows: ``Most of the overall response comes at the smallest price increase, with higher prices yielding strongly diminishing returns.''} In other words, according to the previous work, residential consumers seemed to respond only to the existence of the price changes. In this paper, the main goal is to re-evaluate households' responsiveness to the magnitude of the peak-hour price increases in the setting exploited by the paper, but through a different approach. 

To measure how sensitive residential consumers are to the size of the price variations in peak-demand hours, I decompose their consumption changes in response to TOU tariffs into two distinct channels of electricity savings instead of simply investigating the changes as a whole: 1) savings from electricity consumption for non-temperature-control uses (e.g., lighting, operating appliances, and cooking), and 2) savings from electricity consumption for temperature-control uses (e.g., cooling and heating). The two categories of electricity use (i.e., the two sources of TOU-price-inducing electricity savings) are inherently different in timeliness---the lag between the moment electricity is consumed to create a specific service and the point the service is actually exploited in time. In the case of non-temperature-control-relevant electricity use, which is nearly independent of temperature variations, the timeliness is usually high. For example, lighting service has no lag because the service is available very the moment electricity is consumed. Electricity consumption for temperature-control uses, by contrast, can have a longer lag. Somebody might warm up his house before the time he gets home from work by using automation technology, like  Programmable Communicating Thermostats (PCTs). In that case, PCTs cause changes in electricity consumption across hours of the day. Due to the dissimilarity, examining the aggregate impact of TOU pricing on household electricity consumption is evidently insufficient to identify unique consumption changes relevant to each channel. In addition to the difference in timeliness, even for a given peak-hour price increase, induced consumption changes in for-temperature-control use of electricity on mild days could be considerably different from those on days with extreme temperatures. Moreover, different implications can be drawn depending on the share of electricity savings between the two sources. For instance, although TOU electricity pricing only has within-day price changes, the time-varying pricing can generate sizable variations in electricity savings across days if considerable savings come from temperature-control-related electricity use. For those reasons, in my empirical analysis, I isolate the temperature-control-use-associated savings from the whole by exploiting temperature variations across days. 

My study examines 30-minute interval residential electricity consumption data collected from a TOU pricing experiment conducted from July 2009 to December 2010 by the Commission for Energy Regulation (CER), the electricity and natural gas sector regulator in Ireland.\footnote{The CER changed its name to the Commission for Regulation of Utilities (CRU).} Due to Irish households' widespread use of non-electric fuels for space and water heating, the sample utilized in the empirical analysis only includes meter readings from non-electric heating households in order to draw more universal policy implications from my empirical results. Furthermore, instead of focusing on the peak rate period, my empirical analysis also covers intervals near the period (i.e., two-hour-length pre- and post-peak intervals), in which the level of household electricity consumption changed a lot. Using spline regressions inspired by the Difference-in-Differences (DID) strategy, I estimate not only how electricity savings from the TOU program vary with average daily Heating Degree Days (HDDs) but also how the savings alter with the magnitude of price changes in the peak rate period. By doing so, I identify three building blocks underlying the multi-faceted dynamics of the electricity savings arising from the two distinct drivers of household electricity consumption around peak hours (i.e., in and near peak hours): the magnitude of price spikes at peak hours, daily HDDs, and the point at which electricity is consumed in time. 

One of the most compelling findings from my empirical analysis is that in peak hours, the treated households were highly responsive to the level of price jumps. During the peak rate period, the savings from electricity consumption for non-temperature-control purposes were directly proportional to the price increases in that period. On the other hand, the saving related to the for-temperature-control use of electricity diminished as the degree of the peak-hour price changes became more prominent. Interestingly, due to the opposite relationship between demand reductions and price changes in the two channels of electricity savings, the high sensitivity of household electricity consumption in response to TOU pricing in the peak rate period was masked. In other words, when the estimated electricity savings originating from the two sources are aggregated, the difference in the combined savings between tariff groups is seemingly dampened because of the opposite correlations.\footnote{There were four tariff groups in the CER experiment. Refer to XYZ.} Indeed, this is precisely the result discussed in \cite{Peaking-Interest:How-Awareness-Drives-the-Effectiveness-of-Time-of-Use-Electricity-Pricing_Prest_2020}. In addition to such price sensitivities, as expected, their consumption changes depended on heating needs in a day (i.e., daily HDDs) for a given price spike in the peak rate period. To be specific, during peaks, the treatment effects on household electricity consumption for temperature-control uses showed a U-shaped profile over daily HDDs, which implies that the effectiveness of TOU pricing varies with daily heating needs. 

The nonlinearity in TOU-tariff-inducing electricity savings over households' daily heating needs discloses a veiled feature of TOU electricity pricing: its day-varying effects on residential electricity savings. Suppose that the savings obtained by adopting the TOU prices stem entirely from the non-temperature-control use of electricity. In that case, the degree of savings does not vary across days because it is nearly irrelevant to across-day temperature variations. My empirical results, however, illustrate that on days with moderate heating needs, a sizable share of savings does stem from electricity use for temperature-control purposes. Consequently, even though the TOU rates are not variable across days, the tariff structures already induce substantial reductions in electricity consumption on typical winter days, in terms of daily HDDs, in Ireland. Therefore, on those days, the additional gains captured by switching TOU prices to Real-Time Pricing (RTP) are likely to be smaller than many economists have thought.\footnote{Under RTP, retail prices vary across not only hours of days but days according to contemporaneous generating costs.} In contrast, the U-shaped evolving pattern of the temperature-control-relevant savings over daily HDDs implies that TOU pricing induces rather fewer savings on days with relatively large heating needs, on which the grid is most burdened, in turn, the most significant electricity savings are required. This undesirable quality of TOU electricity pricing, however, can be addressed by adopting a TOU-style pricing scheme in which household heating needs are integrated as an additional dimension of dynamics. According to my analysis, raising the size of a rate change in the peak-demand hours prevents the electricity savings driven by temperature-control-related consumption from disappearing. Furthermore, it produces more temperature-control-associated savings. In light of those findings, introducing an alternative pricing structure in which the magnitude of peak-hour price increases is proportionally coupled to daily HDDs might create additional savings on high-heating-needs days.  

Even in the pre- and post-peak intervals, the households assigned to the treatment group also sensitively adjusted their consumption behavior according to the magnitude of peak-hour price increases. In other words, the TOU prices facilitated spillover effects on households' consumption behavior in near-peak-hour intervals, during which they were not subject to the price raised to a pre-determined level. In both intervals, the households reduced their electricity consumption for non-temperature-control uses in inverse proportion to the size of the peak-rate-period price increases. In addition, with respect to their temperature-control-related consumption, they did not respond to the TOU tariffs until daily HDDs were sufficiently sizable. On days with relatively high HDDs, the TOU tariffs made the residential consumers reduce their for-heating consumption in the before-peak interval as the size of the peak-hour price changes increased. In the after-peak interval, the larger the magnitude of the peak-hour price changes, the smaller the heating-related additional consumption. As in the peak rate period, due to the flipped correlations between induced consumption changes and peak-demand-hour price variations in the two sources of electricity savings, the seemingly lessened responsiveness of the treated households occurred in the off-peak intervals as well. 

The estimated consumption changes allow me to infer how the treated households adapted their consumption behavior to the TOU program around the peak rate period. As discussed, the households' behavioral changes were not restricted to peak hours. Regarding the electricity consumption for non-temperature-control uses, in lieu of relocating their peak-hour consumption to off-peak hours, the households assigned to the treatment group in the experiment simply reduced their demand in and near the peak rate period. In other words, from the pre-peak to the post-peak intervals, the households reacted to the price jumps in peak-demand hours through not load-shifting but load-shedding. For temperature-control-associated electricity savings, on the other hand, the households' consumption changes in the pre-peak hours were likely to determine the degree of their behavioral changes in the following periods. Specifically, the electricity savings obtained from adjustment during the before-peak interval seemed to lead to fewer savings in the following period (i.e., the peak rate period), which in turn brought about limited additional consumption during the after-peak interval. Those sequential behavioral changes associated with temperature-control-related electricity use have an important policy implication: under TOU pricing, impeding such pre-adjustment by exploiting an automation instrument, like PCTs, enables more electricity savings during peaks.

To sum, the results from my empirical analysis extend the previous work by isolating temperature-control-associated electricity savings from the entire TOU-pricing-causing demand reductions. My results demonstrate that around peak hours, the savings from each of the two different channels sensitively vary according to the magnitude of the price changes in the peak rate period. That is, in determining household electricity consumption, not the mere existence of price changes, prices themselves still matter under TOU tariff structures. Moreover, the day-varying electricity savings under TOU prices suggest a vital policy implication: shifting from TOU towards RTP-like pricing can improve residential electricity savings on extremely cold days. In addition, examining the electricity savings from the two distinct sources, not in the peak rate period but around the period, enables unlocking the full benefits of TOU electricity pricing through the automation-technology-relevant policy implication.




\section{Data}
\label{C2-Section:Data}
% TODO: Write a section - Data
\subsection[Description of CER Experiment]{Description of CER Experiment\footnote{The detail about the CER experiment presented hereinbelow is a summary of \cite{Electricity-Smart-Metering-Customer-Behaviour-Trials-Findings-Report_CER_2011}.}}
\label{SubSection:Description-of-CER-Experiment}
The Commission for Energy Regulation (CER), the regulator for Ireland's electricity and natural gas sectors, conducted the Smart Metering Electricity Consumer Behavior Trial (hereafter, the ``trial'') between July 2009 and December 2010. As part of the Smart Metering Project initiated in 2007, the trial's purpose was to assess the impact of various TOU tariff structures, along with different Demand-Side Management (DSM) stimuli, on residential electricity consumption. The CER carefully recruited households to construct a representative sample of the national population. Opt-in to the trial was voluntary. Participants received balancing credits not to incur any extra costs than if they were on the regular electric tariff (i.e., the flat rate of 14.1 cents per kWh). Also, they received a thank-you payment of 25 cents after pre- and post-trial surveys. All credits were distributed outside the treatment period to avoid unintended effects on participants' electricity consumption.\footnote{While the first balancing credit was paid at the end of the base period (i.e., in December 2009), the participants received the second one at the immediate month after the treatment period (i.e., in January 2011). And the after-survey payments were credited to their bill with the balancing credits.}

The households who voluntarily opt-in to the experiment were randomly assigned to control and treatment groups.\footnote{The optimal sample size for the trial was determined to be 4,300 participants in the design phase. In the allocation phase, 5,028 households were assigned to the control and treatment groups to consider participant attrition. The published CER experiment data include electricity consumption data only for 4,225 households.} Baseline electricity consumption data were collected during the second half of 2009 (i.e., July to December 2009), while the treatment period was from January through December 2010. All treated households received two kinds of treatments simultaneously: 1) one of four TOU tariff structures and 2) one of four DSM stimuli, described in detail later. In other words, there were 16 distinct treatment subgroups. The CER provided the treated with a fridge magnet and stickers to facilitate accustoming them to the TOU pricing schemes.\footnote{The fridge magnet and stickers outlined the timebands during which different prices were applied. Moreover, they were tailored for each tariff group.} On the contrary, the households allocated to the control group remained on the normal flat tariff.

The four TOU tariff structures had different prices during each of the three rate periods in a day. The day in the treatment period was divided into three periods: 1) peak rate period from 5:00 p.m to 7:00 p.m., 2) day rate period from 8:00 a.m. to 5:00 p.m. and from 7:00 p.m. to 11:00 p.m., and 3) night rate period from 11:00 p.m. to 8:00 a.m. As illustrated in Figure \ref{Figure:Time-Of-Use-Pricing-Structures}, the order of magnitude in rate changes during the peak rate period is the opposite of that for the rest of the rate periods. The reason for designing the tariff structures in such a way is to enable participating households to face similar energy bills on average when maintaining their electricity consumption pattern, regardless of the rate structures to which they were assigned. 

The four DSM stimuli differed in the degree or the frequency of feedback on each household's electricity usage information. The control group received their bills in the same cycle (i.e., bi-monthly). On the contrary, all households assigned to the treatment group received a detailed energy usage statement combined with their bill, including their detailed weekly usage, average weekly costs, tips on reducing electricity use, and comparisons to peer households. The first stimulus subgroup received a bill with a detailed energy statement bi-monthly, while the second subgroup received the documents every month. An electricity monitor, which displays their usage against their pre-set daily budget, was also provided for the households allocated to the third DSM stimulus subgroup. The last stimulus subgroup received an Overall Load Reduction (OLR) incentive. Under the OLR incentive, the households that reached their 10\% reduction target over the eight-month period beginning May 2010 were rewarded with 20 Euros.\footnote{A household's reduction target in electricity consumption was set based on the participant's actual usage during the first four months of the treatment period. And the households in the last DSM stimulus subgroup were updated on their progress with each bi-monthly bill.}

\begin{figure}[!th]
%\includegraphics[scale = 0.095]{03_Chapter-2/00A_Figures/Figure_Time-of-Use-Tariff-Structures}
\caption{Time-Of-Use Pricing Structures}
\label{Figure:Time-Of-Use-Pricing-Structures}
\end{figure}



\subsection{Description of CER Experiment Data}
\label{SubSection:Description-of-CER-Experiment-Data}
The CER experiment dataset disseminated by the Irish Social Science Data Archive (ISSDA) consists of participating households' electricity consumption and survey data. 

Throughout the baseline and treatment periods, meter reads for residential participants were recorded in 30-minute intervals. The high granularity of the electricity consumption data generated from a well-designed experiment enables quantifying where the energy savings stem from when transferring to TOU electricity pricing for each of the three rate periods. 

The survey data contains participants' responses to more than 300 questions in both pre- and post-trial surveys. The primary purpose of the two surveys was to gather trial-associated experiential and attitudinal feedback from the households. The surveys also included questions intended to collect residential participants' socio-demographic characteristics. In addition, questions about the physical attributes of their house were included in the surveys. 

My empirical analysis throughout this paper uses the sample constructed by including observations only for non-holiday weekdays in the published electricity consumption data because the TOU rates were active just on those days.\footnote{The sample is a panel data of households with reliable meter reads only. Specifically, the residential participants who had no consumption for eight days or more are excluded from the sample. In addition, I drop the meter reads for the days when several participating households' consumption data were missed. \par
Although I utilize the sample satisfying the following criteria too for the empirical analysis, applying the criteria does not change results: 1) Exclude the day immediately following the end of daylight-saving time due to noticeably different consumption levels in the same hours of the day; 2) Drop the observations for the last five days of the baseline and treatment periods because of extraordinarily high electricity demand on those days.} This process results in 4,096 households.

The control and treatment groups in the sample are largely balanced, as shown in Table \ref{Table:Summary-Statistics-and-Differences-in-Means-for-Treatment-and-Control-Groups}. Such indifferences between the two groups over many observables are consistent with previous studies that also examined the CER experiment dataset.\footnote{To check the balance between the control and treatment groups, \cite{Peaking-Interest:How-Awareness-Drives-the-Effectiveness-of-Time-of-Use-Electricity-Pricing} employs a linear probability model, while a probit model is used in \cite{The-Effect-of-Information-on-TOU-Electricity-Use:An-Irish-Residential-Study}.} 


\begin{table}
\caption{Treatment and Control Group Assignments}
\label{Table:Treatment-and-Control-Group-Assignments}
\end{table}

\begin{table}
\caption{Summary Statistics and Differences in Means for Treatment and Control Groups}
\label{Table:Summary-Statistics-and-Differences-in-Means-for-Treatment-and-Control-Groups}
\end{table}

\begin{figure}
\caption{Average Consumption by Hour of Day}
\label{Figure:Average-Consumption-by-Hour-of-Day}
\end{figure}



\subsection{Description of Weather Data}
\label{SubSection:Description-of-Weather-Data}
In this research, weather data are an essential element. The main interest of most TOU papers has been to measure how consumers respond to TOU prices or the heterogeneity in their responsiveness across different information stimuli. Hence, those studies usually do not control temperature variations directly. For example, \cite{The-Effect-of-Information-on-TOU-Electricity-Use:An-Irish-Residential-Study_Pon_2017} and \cite{Peaking-Interest:How-Awareness-Drives-the-Effectiveness-of-Time-of-Use-Electricity-Pricing_Prest_2020}, which also exploited the CER experiment dataset, added weak-of-sample and month-by-year fixed effects (FEs) to their specifications, respectively, in order to control for variations in electricity usage due to seasonal changes. On the other hand, the primary objective of this paper is to decompose the TOU-price-inducing demand reductions into two parts--reductions in temperature-control and non-temperature-control uses. Since the electricity consumption for temperature-control uses is driven by weather, especially temperatures, it is necessary to link the 30-minute interval consumption data and reliable weather data with an appropriate level of resolution. 

The electricity savings associated with for-heating electricity consumption are disaggregated using average daily temperatures from the total savings resulting from the introduction of TOU prices. More granular temperatures, like hourly temperatures, are not a dominant determinant of electricity demand for temperature-control uses at a point in time. It is not easy to believe that households adjust their electricity consumption according to ever-changing outside temperatures elaborately and instantly. Furthermore, as shown in Figure \ref{Figure:Average-Hourly-Electricity-Consumption-by-Time-of-Day}, their electricity demand is the lowest in the early morning, the coldest time of the day. Considering those two points, I measure the TOU-tariff-inducing reductions in electricity consumption conditional on the average heating need in a given day. 

I exploit hourly temperature data for the Dublin airport weather station, provided by Met \'{E}ireann, Ireland's National Meteorological Service, to compute average daily temperatures. There is no available location information in the published CER experiment dataset for privacy and security reasons. Therefore, it is impossible to match a participant's consumption data with the weather data of the closest weather monitoring station to him. But fortunately, in Ireland, temperatures do not vary much across areas for a given day. As demonstrated in Table \ref{Table:Correlations-in-Average-Daily-Temperatures-among-Weather-Stations}, the temperature correlations between the Dublin station and stations near densely populated cities are evident. Because of the positive correlations, I use the mean daily temperatures obtained by averaging the Dublin airport station's hourly temperatures as the representative temperatures in the following analysis.

Using the average daily temperatures, I calculate daily Heating Degree Days (HDDs). Instead of 65 degrees of Fahrenheit ($^{\circ}F$), a normal base temperature in the United States, 60$^{\circ}F$ is utilized to compute daily HDDs, according to \cite{The-Impacts-of-Climate-Change-on-Domestic-Natural-Gas-Consumption-in-the-Greater-Dublin-Region_Liu-and-Sweeney_2012}. Figure \ref{Figure:Distribution-of-Heating-Degree-Days-during-the-Experiment-Period} shows that many days in the treatment period had lower average daily temperatures than the lowest one during the baseline period. The evolving pattern of heating-purpose demand for electricity on days with extreme temperatures could be significantly different under distinct rate structures--flat and TOU rates. If this is true, the lack of counterfactual consumption observations will cause bias in the measured impact of introducing TOU pricing on household electricity consumption. So, I drop observations for those days in the treatment period when constructing the sample to address the potential threat to the identification. 



\subsection{Empirical Strategy}
\label{SubSection:Empirical-Strategy}
Figure \ref{Figure:Pre-and-Post-Treatment-Household-Average-Daily-Electricity-Consumption}, showing not only household average daily electricity consumption over temperature (in Panel A) but also percentage changes in electricity consumption (in Panel B), clearly demonstrates the motivation of this research.\footnote{An important feature also stands out from the figure: the minimum household electricity consumption occurred at around 60$^{\circ}F$. This phenomenon supports the setting of the reference temperature for calculating daily HDDs at the very level.} As illustrated in Panel A of the figure, household demand for electricity grew gradually as the temperature decreased. That is, for Irish households, in addition to temperature-insensitive electricity demand (i.e., for non-temperature-control uses), there was a sizeable electricity demand for heating (i.e., for temperature-control uses), which seems to be highly responsive to temperature variations. In this research, I determine not only how much variations in household electricity consumption occurred, on average, in response to the deployment of the TOU tariffs but also how their impact varied according to daily HDDs. In other words, the dynamic-pricing-causing effects on for-heating and non-for-heating electricity uses are separately estimated to figure out the primary source of electricity savings. As shown in the figure, households in the control group consumed less electricity during the treatment period, especially on days with low temperatures, although their percentage reductions seem less than those of the treated households.\footnote{In Panel A, non-treated households consumed more electricity during the baseline period, especially on days with higher heating needs. The fact that for a given temperature bin, the total daily HDDs during the baseline period were generally greater than those during the treatment period is a plausible explanation for the phenomenon.} In light of this, it is necessary to employ an identification strategy that accounts for the before and after differences in household electricity consumption under the traditional tariff structure (i.e., a flat rate of 14.1 cents per kWh for all hours).

I employ a Difference-In-Differences (DID) approach to estimate the electricity savings caused by the TOU price program. The CER experiment dataset primarily utilized in the following empirical analysis was generated from a carefully developed Randomized Controlled Trial (RCT). So, in principle, the effect of the TOU tariffs on household electricity consumption can be measured simply through the difference in average usage between the two groups during the treatment period.\footnote{Because random assignment of participating households puts selection bias right, observed differences in electricity consumption between the control and treatment groups after introducing the TOU tariffs are only attributable to their differences in exposure to the time-varying electricity prices.} However, as discussed, there exist non-trivial differences in electricity demand between the control and treatment groups during the baseline period. Following the previous studies exploiting the same data, I utilize a DID estimator to address the possible source of bias. 

I include daily HDDs as an explanatory variable directly in my econometric models. In the previous papers using the identical dataset, Fixed-Effects (FEs) were utilized to control for time-varying factors influencing household electricity consumption. Since those studies focused on quantifying how households responded, on average, to the TOU price regimes newly introduced, adding such FEs to their models served their research purpose. In other words, they did not need to explicitly model the relationship between temperature and household electricity consumption to estimate the Average Treatment Effects (ATEs). However, a primary interest of this research is to understand how electricity savings vary with the temperature after shifting to TOU prices. Therefore, more direct controls rather than FEs, not sweeping out temperature variations across days, are required in my empirical analysis. For that reason, I extend a typical panel DID specification and allow the treatment effect to vary as a function of daily HDDs.\footnote{Under three identifying assumptions, applying a DID strategy to measure electricity savings obtained from adopting the TOU prices makes sense. First, the parallel trend assumption is required for the DID estimator. Considering that the 30-minute interval meter reads for participating households were collected during the trial, the assumption implies that the pre-treatment-period load profile for the treated households should be very similar to that for the non-treated households. Figure \ref{Figure:Average-Hourly-Electricity-Consumption-by-Time-of-Day}, showing average within-day load profiles for the two groups during the baseline period, supports the plausibility of the parallel trend assumption. In addition, the electricity consumption profile for the control group illustrated in Figure \ref{Figure:Average-Daily-Electricity-Consumption}, which smoothly evolved over the entire experiment period although heavily fluctuated daily, suggests its high reliability as a counterfactual under the assumption.
The assumption of common temporal shocks is the second identifying assumption necessary for the plausibility of the identification strategy employed. This assumption implies that a treatment-status-irrelevant unexpected event occurring at the same time as or following the deployment of the dynamic prices should have the same impact on both the control and treatment groups. Although the common shocks assumption cannot be tested directly, the similar trends in electricity demand profiles for the control and treatment groups shown in Figure \ref{Figure:Average-Daily-Electricity-Consumption} support the assumption required for the DID approach.
Third, the stable unit treatment value assumption (SUTVA) must hold too. The SUTVA requires that introducing the TOU prices did not affect the electricity consumption of the untreated households. That is, the SUTVA allows no spillovers. During the recruitment process, the locational distribution of the participating households was aligned with that of the total Irish population to construct a representative sample of the national population. Because only a few thousand households scattered geospatially participated in the nationwide experiment, it is unlikely that the treated households influenced the households allocated to the control group. This again supports the SUTVA required under the DID identification strategy.} That is, I estimate the ATEs of the dynamic prices on household electricity demand by exploiting the within-household electricity consumption changes across not only rate periods but temperatures.\footnote{The attrition rate during the RCT was about 20\%. The main reasons for participant attrition were changes in tenancy and supplier. Due to such imperfect compliance, the estimates must be interpreted as local average treatment effects (LATEs). However, according to \cite{Electricity-Smart-Metering-Customer-Behaviour-Trials-Findings-Report_CER_2011}, attritions were unlikely to be associated with the RCT. Furthermore, the level of attritions varied only marginally across treatment status.}

A caveat to my empirical analysis is that a tariff group in my sample's treatment group consists of four subgroups that were subject to one of the four different DSM stimuli. Because of this, part of the estimated ATEs should be attributable to the DSM stimuli. But as shown in Table \ref{Table:Treatment-and-Control-Group-Assignments}, the proportions of the four distinct DSM stimuli, constituting each tariff group, are similar in my sample. Therefore, within a tariff group, a specific DSM stimulus is unlikely to play a prominent role in causing changes in household electricity consumption. 




\section{Empirical Analysis and Results}
\label{C2-Section:Empirical-Analysis-and-Results}
% TODO: Write a section - Empirical Analysis and Results
\subsection{Household Average Responses to Time-Of-Use Prices}
\label{Subsection:Household-Average-Responses-to-Time-Of-Use-Prices}
\subsubsection{Half-Hourly Average Treatment Effects}
\label{Sub-subsection:Half-Hourly-Average-Treatment-Effects}
Utilizing a panel DID identification strategy, I first measure the impact of the TOU prices on 30-minute-interval household electricity consumption. To obtain the ATE for each half-hour interval, I estimate the following specification:
\begin{equation}
\begin{split}
    \textit{kWh}_{itw} \ 
    & = \ \beta_{w} \mathbb{1}\big[ \text{Treatment \& Post} \big]_{it} \ + \ \alpha_{iw} \ + \ \gamma_{dw} \ + \ \delta_{m} \ + \ \epsilon_{itw}
\end{split}
\label{Eq:Model-Specification_Half-Hourly-Average-Treatment-Effects}
\end{equation}
The term $kWh_{itw}$ is the electricity consumption by household $i$ on the day $t$ during the half-hourly time window $w$. The indicator variable $\mathbb{1}\big[ \text{Treatment \& Post} \big]_{it}$ is equal to 1 only if household $i$ is in the treatment group and the day $t$ is in the treatment period. The terms $\alpha_{iw}$, $\gamma_{dw}$, and $\delta_{m}$ are household-by-half-hourly-interval, day-of-week-by-half-hourly-time-window, and month-of-year fixed effects, respectively. In the specification, the point estimates of $\beta_{w}$ representing the ATE for each 30-minute interval $w$ are the parameters of interest. I cluster the standard errors at the household and the day of experiment levels to correct for serial correlation.

\begin{figure}
    \caption{Half-Hourly Average Treatment Effects}
    \label{Figure:Half-Hourly-Average-Treatment-Effects}
\end{figure}
Figure \ref{Figure:Half-Hourly-Average-Treatment-Effects} summarizes the estimated ATEs in the form of a time profile. As also demonstrated in \cite{Peaking-Interest:How-Awareness-Drives-the-Effectiveness-of-Time-of-Use-Electricity-Pricing}, peak hours (i.e., from 5:00 p.m. to 7:00 p.m.), during which the inefficiency of fixed flat-rate tariffs is greatly intensified, show dominant electricity savings. In the following empirical analysis, I continually focus on household electricity demand responses to the time-varying prices during the peak rate period.


\subsubsection{Hourly Average Treatment Effects in the Peak Rate Period}
\label{Sub-subsection:Hourly-Average-Treatment-Effects-in-the-Peak-Rate-Period}
Estimating peak-rate-period ATEs relative to the control group allows us to know whether or not the law of demand is satisfied between the responsiveness of Irish households and the magnitudes of price changes in TOU electricity pricing.\footnote{In this paper, the effects of four different information stimuli on household electricity consumption are not of interest. \cite{The-Effect-of-Information-on-TOU-Electricity-Use:An-Irish-Residential-Study} studied the effects in detail using the same datasets.} To do so, I run the following regression for each of the four tariff groups:
\begin{equation}
\begin{split}
    \textit{kWh}_{ith} \ 
    & = \ \beta_{p} \mathbb{1}\big[ \text{Treatment \& Post} \big]_{it} \ + \ \alpha_{iw} \ + \ \gamma_{dw} \ + \ \delta_{m} \ + \ \epsilon_{ith}
\end{split}
\label{Eq:Model-Specification_Hourly-Average-Treatment-Effects}
\end{equation}
Excepting the dependent variable and the parameter of interest, the econometric model above is the same as (\ref{Eq:Model-Specification_Half-Hourly-Average-Treatment-Effects}). Specifically, the response variable $kWh_{ith}$ means the electricity consumption by household $i$ on the day $t$ during the hour of the day $h$, and the point estimates of $\beta_{p}$ indicate the ATE for each of three rate periods $p$. Table \ref{Table:Average-Treatment-Effects-in-the-Peak-Rate-Period} summarizes the regression results. 

The results demonstrated in Table \ref{Table:Average-Treatment-Effects-in-the-Peak-Rate-Period} indicate that the measured ATEs generally follow the law of demand: in general, the reduction in household demand for electricity during the peak rate period grows with the size of the price jump. Importantly, the results imply that household electricity savings from temperature-control use or ones from non-temperature-control uses depend on the amount of the tariff change in the peak rate period. Motivated by this implication, the relative responsiveness of the two distinct drivers of energy savings to the time-varying prices introduced is quantified below.

\input{03_Chapter-2/00B_Tables/Average-Treatment-Effects-in-the-Peak-Rate-Period}



\subsection{Breakdown of Peak-Rate-Period Household Responses to Time-Of-Use Prices}
\label{Subsection:Breakdown-of-Responses-to-Time-Of-Use-Prices}
\subsubsection{Breakdown of Household Responses in the Peak Rate Period}
\label{Sub-subsection:Breakdown-of-Household-Responses-in-the-Peak-Rate-Period}
I decompose the TOU-tariff-causing reductions in household electricity consumption during the peak rate period into two parts to determine the share of energy savings stemming from two different sources: savings from non-temperature-control and temperature-control uses. Here, the non-temperature-control-related electricity savings mean the stable savings that occur every day regardless of each day's heating degrees. That is, the savings associated with non-temperature-control electricity use do not vary across days. On the contrary, the latter savings strictly depend on HDDs, which fluctuate daily. Therefore, the temperature-control-related electricity savings are additional savings that appear on days with positive HDDs due to reductions in electricity consumption for heating. Isolating the impact of TOU prices on household electricity demand for temperature-control use from the total reductions in electricity demand enables us to know how differently the TOU tariff structures function from day to day, whose implications will be discussed in the next section.

To break down peak-hours household responses to TOU prices, I exploit the following econometric model inspired by the DID framework:
\begin{equation}
\begin{split}
    \textit{kWh}_{ith} \ 
    & = \ \beta_{1} \textit{HDD}_{t} \ + \ \beta_{2} \textit{HDD}_{t} \cdot \mathbb{1}\left[ \text{Treatment} \right]_{i} \\ 
    & \hspace{0.7cm} + \ \beta_{3} \mathbb{1}\left[ \text{Post} \right]_{t} \ + \ \beta_{4} \textit{HDD}_{t} \cdot \mathbb{1}\left[ \text{Post} \right]_{t} \\ 
    & \hspace{0.7cm} + \ \beta_{5} \mathbb{1}\left[ \text{Treatment \& Post} \right]_{it} \ + \ \beta_{6} \textit{HDD}_{t} \cdot \mathbb{1}\left[ \text{Treatment \& Post} \right]_{it} \\ 
    & \hspace{0.7cm} + \ \alpha_{iw} \ + \ \gamma_{dw} \ + \ \delta_{m} \ + \ \epsilon_{ith}
\end{split}
\label{Eq:Model-Specification_Breakdown-of-Hourly-Average-Treatment-Effect}
\end{equation}
Like (\ref{Eq:Model-Specification_Hourly-Average-Treatment-Effects}), the dependent variable $kWh_{ith}$ is the electricity consumption by household $i$ on the day $t$ during the hour of the day $h$. There are three indicator variables in the model: the first indicator variable $\mathbb{1}[\text{Treatment}]_{i}$ has the value of 1 if household $i$ is assigned to the treatment group; the second indicator variable $\mathbb{1}[\text{Post}]_{t}$ equals 1 when the day $t$ is in the treatment period; the last indicator variable $\mathbb{1}[\text{Treatment \& Post}]_{it}$ is equal to 1 only for treatment households during the treatment period. The model also includes interaction terms between daily HDDs and those indicator variables. The terms $\alpha_{iw}$, $\gamma_{dw}$ and $\delta_{mw}$ are household-by-half-hourly-time-window, day-of-week-by-half-hourly-time-window and month-of-year-by-half-hourly-time-window fixed effects, respectively. 

The primary coefficients of interest in (\ref{Eq:Model-Specification_Breakdown-of-Hourly-Average-Treatment-Effect}) are $\beta_{5}$ and $\beta_{6}$. Both coefficients show how much electricity consumption households have reduced since the deployment of the TOU tariffs. To be specific, $\beta_{5}$ is the decrease in household electricity consumption for non-temperature-control uses, while $\beta_{6}$ is associated with the reductions in electricity consumed to satisfy household heating needs for given HDDs. 

Using the points estimates of the two coefficients of interest presented in Table \ref{Table:Breakdown-of-Average-Treatment-Effects-in-the-Peak-Rate-Period}, I show how the electricity savings caused by the TOU prices vary with daily HDDs in Figure \ref{Figure:Breakdown-of-Hourly-ATEs-in-the-Peak-Rate-Period}.\footnote{In Table \ref{Table:Breakdown-of-Average-Treatment-Effects-in-the-Peak-Rate-Period}, the second column demonstrates the estimates $\hat{\beta}_{5}$ and $\hat{\beta}_{6}$ obtained from the econometric model (\ref{Eq:Model-Specification_Breakdown-of-Hourly-Average-Treatment-Effect}). The first and the third columns are for robustness checks. As shown in the first column, adding household-level FEs instead of the indicator variable for assignment to the treatment group leads to the almost same regression result. The third column indicates that excluding covariates associated with the indicator variable for the treatment period results in very minimal changes in point estimates.} The figure clearly demonstrates that the households assigned to the treatment group significantly reduced their electricity consumption when they were subject to the TOU prices. Specifically, they reduced their consumption by about 10\% on a day with zero HDD. In addition, it is evident from the figure that the share of temperature-control-use-related demand reductions grows as household electricity needs for heating become serious. For example, the energy savings originating from electricity consumption for temperature-control use were close to half of the total TOU-pricing-inducing reductions in household electricity demand when Irish household needs for heating were at their peak (i.e., around daily HDDs of 35).  

\begin{figure}[!th]
\centering
\includegraphics[scale = 0.16]{03_Chapter-2/00A_Figures/Figure_Breakdown-of-Hourly-ATEs-in-the-Peak-Rate-Period.png}
\caption{Breakdown of Hourly Average Treatment Effects}
\label{Figure:Breakdown-of-Hourly-ATEs-in-the-Peak-Rate-Period}
\end{figure}

The specification (\ref{Eq:Model-Specification_Breakdown-of-Hourly-Average-Treatment-Effect}) is also utilized to examine the relationship between the degree of price increases and the electricity savings during the peak rate period. The point estimates of coefficients of interest, demonstrated in the last four columns of Table \ref{Table:Breakdown-of-Average-Treatment-Effects-in-the-Peak-Rate-Period}, are interesting in two points. First, the reduction in non-temperature-control electricity demand caused by introducing the TOU tariffs is positively proportional to the size of the change in price during peak hours. In other words, the electricity savings occurring on any day regardless of the average daily temperatures obviously follow the law of demand. Second, the savings associated with temperature-control electricity use are insensitive to the price jumps in the peak rate period. 

\input{03_Chapter-2/00B_Tables/Breakdown-of-ATEs_Hourly-in-the-Peak-Rate-Period}



\subsubsection{Peak-Rate-Period Household Responses as a Linear Function of Price Changes}
\label{Sub-subsection:Peak-Rate-Period-Household-Responses-as-a-Linear-Function-of-Price-Changes}
% TODO: Write a sub-subsection - Average Responses to TOU Prices: Empirical Strategy
(1. Description of average responses in the peak period: linear function of changes in unit rate) \\ 
(1.1. Econometric model) \\ 
\begin{equation}
    (...)
\end{equation}
(1.2. Results, with implications) \\ 
\begin{table}
    \caption{Treatment Effects as a Linear Function of Unit Rate Changes}
\end{table}






\section{An Alternative Electricity Pricing}
\label{C2-Section:An-Alternative-Electricity-Pricing}
\subsection{Household Consumption Behavior over Daily Heating Degree Days}
\label{Subsection:Household-Consumption-Behavior-over-Daily-Heating-Degree-Days}
\input{03_Chapter-2/04_An-Alternative-Electricity-Pricing/An-Alternative-Electricity-Pricing_Household-Consumption-Behavior-over-Daily-HDDs.tex}


\subsection{Time-Of-Use Pricing with Additional Dynamics over Daily Heating Degree Days}
\label{Subsection:Time-Of-Use-Pricing-with-Additional-Dynamics-over-Daily-Heating-Degree-Days}
The U-shaped curve of peak-demand-hour reduction in temperature-control-related electricity consumption is not a desirable feature of TOU electricity pricing. The fundamental intention of the time-varying tariff scheme is to reshape load profiles, especially in the peak rate period, in order to avoid excessive investment in power generation capacity. So a higher amount of reduction in electricity consumption for heating on freezing days (i.e., on days when the power grid is most burdened) serves the purpose of the price scheme. In light of that, the U-shaped evolving pattern over daily HDDs is unattractive because on days with high heating needs, TOU electricity pricing induces even less reduction in for-heating-relevant household electricity consumption. 

An alternative electricity pricing scheme, a TOU-like tariff structure with additional flexibility in price variations across daily HDDs, could address the disadvantage of typical TOU pricing revealed from my analysis (i.e., less effectiveness on days with very low temperatures). My empirical findings illustrate two important points with respect to the relationship between TOU-tariff-induced changes in household electricity consumption and price increases during the peak rate period. First, the reduction stemming from non-temperature-control-associated electricity consumption becomes larger as the magnitude of a price escalation in the peak period increases. Second, the gains obtained by marginally raising the peak-hour electricity price (i.e., an additional reduction in non-temperature-control-relevant electricity consumption) exceed the losses from such a marginal increase (i.e., a fewer reduction in temperature-control-driven electricity consumption). Those two points collectively imply that scaling up the size of a rate change in the peak rate period as daily HDDs rise enables achieving a more considerable TOU-price-induced aggregate reduction in residential electricity consumption. 
 
Figure \ref{Figure:Additional-Savings-from-an-Alternative-Electricity-Pricing-Scheme} depicts an alternative price scheme and additional gains from it. Under the price scheme proposed in the figure, the peak-demand-hour price jumps as household heating needs become serious. To be specific, prior to the value of daily HDDs that typical TOU pricing becomes ineffective, the magnitude of peak-rate-period price change is evenly six cents per $kWh$. After that point, every time daily HDDs rise by five, the degree of peak-demand-hour price change increases by six cents per $kWh$. As illustrated in the figure, compared to the case in which the size of peak-hour price growth is fixed at six cents for all values of daily HDDs, the alternative price scheme can induce a more significant reduction in household electricity consumption according to increasing household heating needs by synchronizing price increases in the peak rate period with daily HDDs. In other words, the weakness of typical TOU pricing is alleviated under the new price structure. 

The alternative price scheme is well in line with the key finding in \cite{Electricity-Retail-Rate-Design-in-a-Decarbonized-Economy_Schittekatte-et-al_2022}. According to this recent paper, TOU rates complemented with Critical Peak Pricing (CPP) work well for reflecting spot-price-providing within-day load-shifting incentives. Considering that CPP introduces dramatic but short-lived price escalations when generating costs exceed a certain threshold infrequently, a very high peak price linked with exceptionally large daily HDDs in Ireland under the proposed alternative price scheme is consonant with CPP events with which TOU prices are complemented as suggested in the paper. 

In addition, this proposed price structure is better than the typical TOU tariff structure with a higher fixed peak-demand-hour price. For example, Tariff Group D reduces household electricity consumption as much as the alternative price scheme on extremely cold days. However, compared to Tariff Group D, households under the proposed price structure can consume more electricity on warm days on which the power grid still has enough spare capacity to meet higher electricity demand. 




\section{Conclusion}
\label{C2-Section:Conclusion}
The primary aim of various types of time-varying electricity pricing is to reshape load curves, especially around the peak-demand hours. Under the dynamic pricing of electricity, prices---more precisely, price variations---, which reflect instantaneous generation costs, are utilized to incentivize consumers to change their consumption behavior. Therefore, their responsiveness to the price changes in the tariff structures determines whether the time-varying electricity prices, including TOU pricing, will work as intended. In this paper, I quantify how sensitively households adjust their electricity consumption in response to TOU prices in and near the peak rate period. The results from my empirical analysis reveal two interesting points: household electricity consumption, consisting of two categories of electricity use---non-temperature-control-driven and temperature-control-driven consumption---, 1) sensitively responded to the magnitude of the price change in the peak rate period, and 2) also depended on daily heating degree days as well as the point electricity was consumed in time for a given rate change. In other words, my empirical analysis discloses the multidimensional dynamics of households' responses to the TOU tariffs. 

Those findings provide important policy implications for TOU electricity pricing. First, along with residential consumers' high price sensitivity, the nonlinearity in their responses to daily heating needs proposes an alternative pricing scheme: TOU pricing with additional flexibility induced by synchronizing the magnitude of the peak-demand-hour price jump with daily heating degree days. Second, taking a close look at the relationship between the size of the peak-hour price increase and the changes in electricity consumption for temperature-control uses in chronological order emphasizes the importance of adopting home automation technologies, like Programmable Communicating Thermostats (PCTs), to improve the performance of TOU pricing. 

My empirical findings and the policy implications derived from them ultimately indicate that an integrated understanding of the multidimensional dynamics of households' responses to TOU electricity pricing is required to make the price structure function with its full potential as a demand management tool. Furthermore, even for stakeholders in the electricity market, such as power generators, investors, regulators, and policymakers, comprehending how electricity consumption reacts to the time-varying pricing is critical because consumers' behavioral changes are an important piece of information in their decision makings.



% ------- Chapter 3 -------
\chapter{From Hotelling to DCDP Model: New Approach for Microeconomic Empirical Work in Oil and Gas Extraction}
\label{Chpater:Chapter-2}
\textbf{{\large
    (Coauthored with Mark J. Agerton\footnote{Department of Agricultural and Resource Economics, University of California, Davis.})
}} \vspace{0.7cm}

\section{Introduction}
\label{C3-Section:Introduction}
Hotelling's model of exhaustible resource extraction provides simple but useful economic intuitions about the trade-off between extraction today and extraction in the future in the context of the forward-looking resource owners. The framework is flexible in applying to real-world resource extraction problems, such as exploration \citep{The-Optimal-Exploration-and-Production-of-Nonrenewable-Resources_Pindyck_1978, Optimal-Pricing-Use-and-Exploration-of-Uncertain-Natural-Resource-Stocks_Arrow-and-Chang_1982, Exploration-and-Exhaustible-Resources_The-Microfoundations-of-Aggregate-Models_Swierzbinski-and-Mendelsohn_1989, Exhaustible-Resources_A-Theory-of-Exploration_Quyen_1991}, uncertainty over reserves/demand/price \citep{Optimal-Depletion-of-an-Uncertain-Stock_Gilber_1979, Uncertainty-and-Exhaustible-Resource-Markets_Pindyck_1980, The-Optimal-Production-of-an-Exhaustible-Resource-When-Price-is-Exogenous-and-Stochastic_Pindyck_1981, Extraction-at-the-Intensive-Margin_Farrow-and-Krautkraemer_1989}, taxation effects \citep{Economics-of-Depletatble-Resources_Market-Forces-and-Intertemporal-Bias_Sweeney_1977, The-Taxation-of-Nonreplenishable-Natural-Resources-Revisited_Heaps_1985}, and technological improvement \citep{Growth-with-Exhaustible-Natural-Resources_Efficient-and-Optimal-Growth-Paths_Stiglitz_1974, Trends-in-Natural-Resource-Commodity-Prices_An-Analysis-of-the-Time-Domain_Slade_1982}. Accordingly, economists have utilized this canonical theory of the optimal depletion of nonrenewable resources for many decades to understand how exhaustible resource markets function. Hotelling's model, however, shows a different story in terms of empirical work. The main focus of the empirical literature on the Hotelling framework has been to test the well-known $r$-percent rule that a resource's shadow price has to rise at the rate of interest $r$. Unfortunately, such attempts have not been very fruitful due to various econometric issues and the fact that resource rents are generally unobservable.\footnote{See \cite{Natural-Resource-Economics-under-the-Rule-of-Hotelling_Gaudet_2007} and \cite{Whither-Hotelling-Tests-of-the-Theory-of-Exhaustible-Resources_Slade-and-Thille_2009}.} Furthermore, recent empirical work on oil and gas extraction tends not to use Hotelling's theoretical model.\footnote{
\cite{The-Effect-of-Uncertainty-on-Investment-Evidence-from-Texas-Oil-Drilling_Kellogg_2014} examines the relationship between drilling investments in Texas and oil price volatility. \cite{Experiential-Gains-with-a-New-Technology_2015_(Fitzgerald)} studies experiential gains in hydraulic fracturing. \cite{The-Housing-Market-Impacts-of-Shale-Gas-Development_Muehlenbach-Lucija-and-Timmins_2015}  investigates the impacts of shale gas development on the housing market. \cite{Drilling-Like-Theres-No-Tomorrow_Boomhower_2019} examines the effects of bankruptcy protection on industry structure and environmental outcomes. \cite{Patchwork-Policies-Spillovers-and-the-Search-for-Oil-and-Gas_Lewis_2019} studies the effects of a complex patchwork of mineral ownership on the oil and gas extraction outcomes. Using a government oil lease lottery, \cite{Information-Asymmetry-Trade-and-Drilling_Evidence-from-an-Oil-Lease-Lottery_Brehm-and-Lewis_2021} shows that initial assignment results in different trade, drilling, and production outcomes.} 

\cite{Hotelling-under-Pressure_AKS_2018} (AKS) extends Hotelling's model by adding a new layer to oil producers' decision-making. AKS allows extractors to manipulate the rate of extraction from each well (the intensive margin) as well as the rate of drilling new wells (the extensive margin). The authors document several stylized facts about oil production: 1) oil production from existing wells is unresponsive to oil prices, which is inconsistent with Hotelling's classic model; however, 2) the rate of drilling is responsive to oil prices. The authors can reconcile these stylized facts with their reformulation of the Hotelling model. 

Adding the heterogeneous geological features of different well sites is a natural augmentation to the AKS theoretical model. As discussed in \cite{Learning-where-to-drill_Agerton_2020}, variation in geological characteristics, which we denote \textit{resource quality}, is a key driver of well-level productivity and firms' extraction decisions. We extend the AKS framework to incorporate heterogeneity in resource quality and well-level cost shocks. This extension has several benefits. First, we can accommodate what we see empirically in U.S. production---that firms develop high- and low-quality resources at the same time. Second, our specification is both analytically tractable, allowing for analysis with standard optimal control methods, as well as empirically tractable, allowing for econometric estimation. 

To examine the validity of our extended model, we empirically analyze the resource quality of horizontal wells in North Dakota. As is well known, the geological quality of a given well site is usually observable only by extraction firms, though there are publicly available data that we can exploit to gauge it, such as the geological survey data published by the North Dakota Geological Survey (NDGS). Inspired by \cite{The-Economics-of-Time-Limited-Development-Options_2020_Herrnstadt-Kellogg-and-Lewis}, we use Robinson's partial linear model with detailed well-level data on horizontal wells drilled in North Dakota to estimate resource quality in each location. This estimation process provides us with two interesting stylized facts. One interesting fact is that fracking firms in North Dakota drilled well sites with different levels of resource quality simultaneously. The other empirical fact we discovered is that drilling of low-quality well locations decreased more than that of high-quality ones in response to the oil price drops during the second half of 2014. 

The stylized facts from our analysis of the quality of drilled well sites in North Dakota raise two issues in modeling fracking firms' drilling activity based on the extended AKS framework. First, we find that the AKS-style model incorporating well sites' heterogeneous resource quality cannot rationalize the simultaneous drilling of well sites with different quality levels. Simply put, the model is not able to explain the empirical finding. Second, the simultaneous drilling of well locations with heterogeneous resource quality is inconsistent with the well-known least-cost-first extraction rule in exhaustible resource extraction. \cite{Extraction-Capacity-and-the-Optimal-Order-of-Extraction_Holland_2003}, which shows that limited extraction capacity causes the rule not to hold, seems to indicate that we need to include additional extraction-capacity-associated constraints that are highly sophisticated to model to make our extended model describe the empirical facts well. In this context, we suggest adopting a different approach to developing an economic model that enables us to explain fracking firms' drilling activity observed in North Dakota convincingly. 

In this paper, following \cite{Estimation-of-Dynamic-Discrete-Choice-Models-in-Continuous-Time_ABBE_2016}, we develop a Discrete Choice Dynamic Programming (DCDP) framework in continuous time to model firms' drilling decisions. In this theoretical model, we formulate the decision to drill as an optimal stopping problem that trades off drilling a given well site today against drilling it at some time in the future. This trade-off is also the central idea of Hotelling's classic model. In our formulation, we introduce choice-specific cost shocks $\epsilon$'s that allow us to address the two problems we faced with respect to the extended AKS-style model. Because the cost shocks reflect a range of constraints that affect oil production companies' drilling decisions but are difficult to quantify by econometricians, they allow us to avoid adding various constraints to our model. Our model incorporating the cost shocks can also rationalize the empirical finding that fracking firms in North Dakota drilled horizontal wells with heterogenous quality simultaneously. Furthermore, under the assumption of a continuum of infinitesimally small well sites, our framework enables us to compute market-level drilling and production by aggregating the drilling decision for the marginal well location. In addition to its analytical tractability, one of the main advantages of the DCDP framework is that the model is estimable empirically using microeconomic data, which are available from both commercial and government databases. 

We examine the equilibrium dynamics implied by our DCDP model in continuous time, especially focusing on how hydraulic fracking firms adjust their drilling, and thus oil production, in response to changes in oil prices under different conditions. First, we investigate the impact of unexpected demand shocks on the evolution of optimal drilling paths. Our simulation shows that a negative demand shock results in an immediate decrease in drilling, oil production, and the equilibrium oil price and that the equilibrium oil price gradually increases after the discontinuous drop. Second, we simulate how firms' drilling activity on well locations with heterogenous resource quality responds to unexpected price shocks. The time paths from this simulation demonstrate an interesting result that is consistent with our empirical observation: they reduced the drilling of low-quality well sites more than that of high-quality ones. Third, we compute the time paths of optimal drilling of horizontal wells and oil production from them under two distinct types of oil prices---exogenous and endogenous oil prices. The obtained equilibrium paths show that exogenous oil prices cause a higher drilling rate over the early period in our simulation. 

The rest of this paper proceeds as follows. Section \ref{C3-Section:Data-and-Empirical-Analysis} discusses a set of data utilized for this research and the results from our empirical analysis. In Section \ref{C3-Section:A-DCDP-Model-in-Continuous-Time}, we develop a continuous-time DCDP model for drilling decisions in oil and gas extraction. Section \ref{C3-Section:Equilibrium-Dynamics-with-Oil-Prices} presents, under distinct conditions, the time paths of optimal drilling and oil production implied by our model, and Section \ref{C3-Section:Conclusion} concludes. 



\section{Data and Empirical Analysis}
\label{C3-Section:Data-and-Empirical-Analysis}
% Data and Empirical Analysis: Data
\subsection{Data}
\label{C3-SubSection:Data}
This section summarizes data on wells, geology, and oil prices, which are utilized to conduct empirical analysis and estimate a model of drilling behaviors observed in North Dakota. 

\subsubsection{Well Data}
\label{C3-SubSubSection:Well-Data}
Data for wells in North Dakota's Bakken region come from the Oil and Gas Division of the North Dakota Industrial Commission (NDIC), the regulator for the drilling and production of oil and gas in North Dakota.\footnote{NDIC's well data are available at \href{https://www.dmr.nd.gov/oilgas}{Official Portal for North Dakota State Government}.} NDIC-providing well data include a complete index of all wells permitted in North Dakota. The data contains basic information for each well, such as the type of well, completion and spud dates, location, the first and the current operator names, and targeting pool. Individual well's projection and injection histories, including producing days during each month, are also contained in the data. 

Detailed well completion data are also obtained from NDIC.\footnote{Using Form 6, Well Completion or Recompletion Report, filed by operators, NDIC has developed the detailed data.} The data contain how much water and proppant were consumed during well simulations.

The regulatory body also provides well-level survey data. The survey data include directions and lengths of legs in a well. Using the leg-related information in the data, we compute the total length of horizontal drillings in an individual well. 

The sample used throughout this paper consists of 24,520 horizontal wells that targeted the Bakken pool.\footnote{The Bakken pool includes Bakken, Three Forks, and Sanish formations.} Summary statistics for those wells are presented in Table \ref{Table:Summary-Statistics-for-Horizontal-Wells}.


\subsubsection{Geological Survey Data}
\label{C3-SubSubSection:Geological-Survey-Data}
We obtain geological survey data from the North Dakota Geological Survey (NDGS).\footnote{To be specific, we exploit NDGS maps GI-59 and GI-63.}  Regarding the upper and the lower Bakken shales, their estimates of thickness, thermal maturity, and total organic contents are illustrated in the geospatial data. We follow \cite{Experiential-and-Social-Learning-in-Firms_Covert_2015} to match the three geological characteristics with each horizontal well in our sample.\footnote{Refer to Section 2.4.1 of \cite{Experiential-and-Social-Learning-in-Firms_Covert_2015}.} Figure \ref{Figure:Spatial-Distributions-of-Geological-Characteristics} demonstrates the spatial distribution of well-level geological features. 


\subsubsection{Oil Price Data}
\label{C3_SubSubSection:Oil-Price-Data}
\afterpage{
     \begin{figure}[t!]
         \centering
         \includegraphics[scale = 0.105]{04_Chapter-3/00A_Figures/Figure_Completion-over-Time.png}
         \caption{Time Series of the Number of Well Completions in North Dakota}
         \caption*{
            {\small
             \textit{Note}: 
             This figure shows the time series of the number of well completions in North Dakota. Horizontal wells have been strictly dominant in that area. The solid line in the figure is the monthly per-barrel spot prices for West Texas Intermediate at Cushing, Oklahoma. The figure suggests that the spot prices positively correlate with horizontal well completions in North Dakota. 
         }}
         \label{Figure:Time-Series-of-the-Number-of-Well-Completions-in-North-Dakota}
     \end{figure}
}
We collect the monthly per-barrel spot prices for West Texas Intermediate at the Cushing, Oklahoma from the Energy Information Administration.\footnote{Time series data for Cushing, OK WTI Spot Price FOB is available \href{https://www.eia.gov/dnav/pet/PET\_PRI\_SPT\_S1\_M.htm}{here}.} As shown in Figure \ref{Figure:Time-Series-of-the-Number-of-Well-Completions-in-North-Dakota}, there was a striking movement in oil prices between 2014 and 2016. Specifically, oil prices, maintained at around \$100 per bbl during the first half of 2014, had continued to plunge, reaching less than \$50 per bbl in January 2015. After recovering to \$60 per bbl during the first half of 2015, oil prices had fallen to \$30 per bbl by the end of the year. Since then, oil prices have gradually risen until they declined again during the final quarter of 2018.



% Data and Empirical Analysis: Empirical Analysis
\subsection{Empirical Analysis}
\label{C3-SubSection:Empirical-Analysis}

\subsubsection{Correlation between Oil Prices and Horizontal Drilling in North Dakota}
\label{C3-SubSubSection:Correlation-between-Oil-Prices-and-Horizontal-Drilling-in-ND}
Figure \ref{Figure:Time-Series-of-the-Number-of-Well-Completions-in-North-Dakota} shows how well completions in North Dakota evolved between 2009 and 2020. As clearly illustrated, well completions, which were driven mainly by horizontal wells, dramatically increased from the beginning of 2010. 

According to the figure, it is evident that drilling horizontal wells in North Dakota is closely correlated with oil prices, especially after 2009. On the whole, oil prices significantly increased between 2009 and 2010 and remained high until mid-2014. Then, there was a sharp plunge in oil prices from mid-2014 to the end of 2015, and horizontal well drilling declined too. When oil prices gradually climbed between 2016 and 2020, North Dakota's drilling activities also recovered. To summarize, oil prices and the number of horizontal drilling in North Dakota seem to be positively correlated. Importantly, such a positive correlation between oil prices and horizontal drilling in the state suggests that fracking firms' drilling decisions strongly depend on oil prices. In Section \ref{C3-SubSubSection:The-Role-of-Geological-Quality-in-Horizontal-Drilling}, we show that their drilling decisions are linked with oil prices through the geological features of well sites.


\subsubsection{The Role of Geological Quality in Horizontal Drilling}
\label{C3-SubSubSection:The-Role-of-Geological-Quality-in-Horizontal-Drilling}
\textit{\textbf{Estimation of Unobservable Geological Characteristics of Horizontal Wells}} --- Not all well-specific information on geological features is available to econometricians. The NDGS geological survey data only include estimates of four different measurements of geological properties at a given location. Because the geospatial data was published to the public in 2008, it is likely that fracking firms, whose objective is to maximize their profits, have already exploited the contents of the maps. As discussed in \cite{Learning-where-to-drill_Agerton_2020}, learning about the spatial distribution of deposits by drilling wells is one of three economic factors that govern firms' where-to-drill decisions. So, it is reasonable to suppose that firms have private information about the Bakken area's spatial distribution of geological characteristics, which is not accessible to researchers. 

The geological characteristics observed only by firms play two different roles in their drilling decisions. First, firms choose whether to drill a location based on its resource quality. Thus, the sample of wells we observe is not random: it has been selected based on unobservable (to us) resource quality. Second, firms' choice of inputs during hydraulic fracturing of each well may be correlated with the unobservable resource quality. For these reasons, accounting for resource quality is critical in modeling firms' decisions and production functions. 

Following \cite{The-Economics-of-Time-Limited-Development-Options_2020_Herrnstadt-Kellogg-and-Lewis}, we employ Robinson's partially linear model to determine the unobservable quality of the horizontal wells completed between 2009 and 2018. We first specify the oil production from a horizontal well as
\begin{equation}
\begin{split}
    \log \left( y_{i} \right) \ 
    & = \ \log \left( \boldsymbol{X}_{i} \right)' \boldsymbol{\beta} \ - \ \lambda \left( longitude_{i}, latitude_{i} \right) \ + \ \epsilon_{i}.
\end{split}
\label{Equation:Production-Function}
\end{equation}
In this specification, $y_{i}$ are horizontal well $i$'s cumulative oil production at its cumulative production month 24.\footnote{That is, unobservable geological features are estimated cross-sectionally in our estimation.} The covariate vector $\boldsymbol{X}_{i}$ for well $i$ includes hydraulic fracturing inputs (i.e., fluid volume, proppant weight, and length of horizontal drilling), cumulative producing days, and observable geological characteristics (i.e., thickness, total organic contents, and thermal maturity). The term $\lambda(longitude_{i}, latitude_{i})$ is a nonparametric function of each well's coordinates and captures well $i$'s unobservable resource quality. Lastly, $\epsilon_{i}$ is a productivity shock not correlated with resource quality. 

To operationalize model (\ref{Equation:Production-Function}), we estimate the following partially linear model:
\begin{equation}
\begin{split}
    \log \left( y_{i} \right) \ - \ \widehat{m}_{y_{i}} \ 
    & = \ \left( \log \left( \boldsymbol{X}_{i} \right) \ - \ \widehat{\boldsymbol{m}}_{\boldsymbol{X}_{i}} \right)' \boldsymbol{\beta} \ + \ \epsilon_{i}
\end{split}
\label{Equation:Partially-Linear-Model}
\end{equation}
Here, $\widehat{m}_{y_{i}}$ are predictions from a non-parametric regression of $\log \left( y_{i} \right)$ on well $i$'s coordinates $(longitude_{i}, latitude_{i})$. The $\widehat{m}_{y_{i}}$ terms are smoothed means. Differencing these means out serves the same role as the within-transformation in a fixed effects model. In fact, if one used a uniform kernel function, $\widehat{m}_{y_{i}}$, within discrete cells, the estimator would be mathematically identical to fixed effect estimation with spatial fixed effects for each well. Predictions $\widehat{\boldsymbol{m}}_{\boldsymbol{X}_{i}}$ are obtained from different nonparametric regressions whose dependent and independent variables are $\log \left( \boldsymbol{X}_{i} \right)$ and $(longitude_{i}, latitude_{i})$, respectively. The values of primary interest $\widehat{\lambda}_{i}$ (i.e., the unobservable geological quality of horizontal well $i$) are estimated as follows\footnote{For details of Robinson's difference estimator, refer to \textit{9.7.3 Partially Linear Model} in \cite{MicroEconometrics-Methods-and-Applications_Cameron-and-Trivedi_2005}.}:
\begin{equation}
\begin{split}
    \widehat{\lambda}_{i} \
    & = \ \widehat{m}_{y_{i}} \ - \ \widehat{\boldsymbol{m}}_{\boldsymbol{X}_{i}}' \widehat{\boldsymbol{\beta}}
\end{split}
\label{Equation:Estimates}
\end{equation}
\afterpage{
    \begin{figure}[t!]
        \centering
        \includegraphics[scale = 0.13]{04_Chapter-3/00A_Figures/Figure_Cross-Sectional-Approach_Estimates-from-Robinson-Estimator_Time-Trend-of-Unobservable-Geological-Quality.png}
        \caption{Simultaneous Drilling of Horizontal Wells with Heterogeneous Geological Quality}
        \caption*{
            {\small
            \textit{Note}: 
            This figure indicates the estimated geological feature for each horizontal well, depicted as a dot. Those dots definitely suggest the simultaneous drilling of horizontal wells with heterogeneous geological quality. In the figure, percentiles of the estimates, with the 95\% confidence interval of each, are also presented. The solid red line is the time series of the monthly per-barrel spot prices for West Texas Intermediate at Cushing, Oklahoma. Oil prices plunged significantly between 2014 and 2015 and rose gradually. The percentile lines skewed upward, especially lower ones, as of the second half of 2014.  
        }}
        \label{Figure:Simultaneous-Drilling-of-Horizontal-Wells-with-Heterogeneous-Geological-Quality}
    \end{figure}
}
In our analysis, we define low- and high-quality locations by dividing our sample of well sites into those with $\widehat{m}_{y_{i}}$ below and above the median. Even though $\widehat{m}_{y_{i}}$ is estimated from only higher-quality locations with observed drilling, and therefore biased upward, the ordinal ranking of well sites should be affected less. Figure \ref{Figure:Spatial-Distribution-of-the-Estimated-Geological-Characteristic-by-Year} shows the spatial distribution of the estimated quality.

\par
\vspace{0.3cm}
\noindent
\textit{\textbf{Simultaneous Drilling of Horizontal Wells with Heterogeneous Geological Qualities}} --- Figure \ref{Figure:Simultaneous-Drilling-of-Horizontal-Wells-with-Heterogeneous-Geological-Quality}, summarizing the estimated geological qualities of horizontal wells in scatter plots, clearly demonstrates that horizontal wells with a range of geological qualities were drilled simultaneously in the Bakken region of North Dakota. And as shown in Figure \ref{Figure:High-Sensitivity-of-Firm-Level-Low-Quality-Well-Drilling}, simultaneous drilling is observed even at the firm level. 

The simultaneous drilling empirically obtained contradicts the well-known least-cost-first extraction rule in nonrenewable resource extraction.\footnote{The cost that matters here is the marginal cost. And the marginal cost consists of two distinct costs: the marginal cost of drilling a new well and the marginal cost of extracting oil from existing wells.} According to the rule derived from the canonical Hotelling model, deposits of an exhaustible resource should be exploited in strict order, beginning with the lowest cost deposit. Because a larger estimate of the unobservable geological quality implies higher ultimate oil production, if the rule holds, wells with larger estimates, thus having lower per-barrel marginal cost, should be first extracted. Those figures, however, do not show the theoretical prediction at all. This situation calls for a new theoretical approach to rationalize the simultaneous drilling of horizontal wells with heterogeneous qualities. 

\par
\vspace{0.3cm}
\noindent
\textit{\textbf{The High Sensitivity of Drilling Low-quality Horizontal Wells to Negative Price Shocks}} ---
In addition to the simultaneous drilling of horizontal wells with heterogeneous qualities, Figure \ref{Figure:Simultaneous-Drilling-of-Horizontal-Wells-with-Heterogeneous-Geological-Quality} demonstrates an interesting point: the responsiveness of low-quality well drilling to sharp oil price declines from mid-2014 to the end of 2015. The high sensitivity of drilling activities for low-quality horizontal wells to negative price shocks during the period is also pronounced even at the firm level, as illustrated in Figure \ref{Figure:High-Sensitivity-of-Firm-Level-Low-Quality-Well-Drilling}. 

\afterpage{
    \begin{figure}[t!]
        \centering
        \includegraphics[scale = 0.12]{04_Chapter-3/00A_Figures/Figure_Drilling-over-Time_By-Well-Quality-and-Whether-First-Drilled.png}
        \caption{Held-by-Production vs. Non-Held-by-Production Horizontal Well Drilling}
        \caption*{
            {\small
            \textit{Note}: 
            This figure depicts the drilling of horizontal wells classified into two quality levels. The upper panel shows how the first drilling in each section, regarded as held-by-production drilling, has evolved. There was only a limited number of held-by-production drilling between 2015 and 2019. The lower panel indicates all subsequent drilling in sections. The collapse in oil prices between mid-2014 and 2015 made post-held-by-production drilling decrease. Drilling of low-quality well locations showed a more significant reduction, especially in 2015. High-quality sites were drilled more than low-quality ones during the period of oil price recovery from 2016 to 2019. In each panel, the dot-dashed line is the time series of the monthly per-barrel spot prices for West Texas Intermediate at Cushing, Oklahoma.
        }}
        \label{Figure:Held-by-Production-vs-Non-Held-by-Production-Horizontal-Well-Drilling}
    \end{figure}
}
Figure \ref{Figure:Held-by-Production-vs-Non-Held-by-Production-Horizontal-Well-Drilling} shows that drilling associated with held-by-production did not drive the relationship between oil prices and low-quality well drilling, especially between mid-2014 and the end of 2015. The upper panel in the figure illustrates the by-quality time series of the number of horizontal wells drilled that are supposed to be drilling related to Held-By-Production (HBP). In our empirical analysis, we assume that for each of the sections into which horizontal wells in our sample were drilled, the purpose of the first drilling in that section was just HBP.\footnote{In the Public Land Survey System, a \textit{section}, which is one of 36 sections in a township, is a one-mile-square area.} The relatively high drilling rate of low-quality wells, especially between mid-2011 and mid-2013, seems to be consistent with the empirical result of \cite{The-Economics-of-Time-Limited-Development-Options_2020_Herrnstadt-Kellogg-and-Lewis}: firms bound to a lease contract including use-it-or-lose-it requirements tend to drill low-productivity well locations just before the first lease expires.

The evolving pattern of drilling for each of the three quality levels presented in the lower panel of Figure \ref{Figure:Held-by-Production-vs-Non-Held-by-Production-Horizontal-Well-Drilling}, regarded as post-held-by-production drilling, shows completely different movements from those in the upper panel. Until 2014, horizontal wells of heterogeneous quality were drilled equally and at the same growth rate. But drilling of low-productivity horizontal wells more sensitively reacted to negative price shocks between mid-2014 and 2015, compared to drilling medium- and high-quality wells. The new theoretical approach, required to rationalize the simultaneous drilling of wells with heterogeneous qualities, needs to explain the high sensitivity of low-quality well drilling.



\section{A DCDP Model in Continuous Time for Drilling Decisions in Oil and Gas Extraction}
\label{C3-Section:A-DCDP-Model-in-Continuous-Time}
This section delves into the analysis of two distinct economic frameworks. In Section \ref{C3-SubSection:A-Limitation-of-AKS-style-Model}, the optimal drilling and extraction model developed in \cite{Hotelling-under-Pressure_AKS_2018} is reformulated by introducing heterogeneity in resource quality. And it is shown that the recast model cannot justify the empirically observed simultaneous drilling of well locations with varying quality levels. In the subsequent portions of this section, a continuous-time Discrete Choice Dynamic Programming (DCDP) model for oil and gas extraction is developed, successfully articulating the simultaneous drilling of horizontal wells with heterogeneous quality.

% A Limitation of AKS-style Model
\subsection{A Limitation of AKS-style Model}
\label{C3-SubSection:A-Limitation-of-AKS-style-Model}
The theoretical framework for optimal oil drilling and extraction delineated in \cite{Hotelling-under-Pressure_AKS_2018} (AKS) can be augmented by integrating heterogeneity in the quality of well locations. Suppose that the fracking firm owns well sites of different qualities, indexed by $g \in \{L(ow), H(igh)\}$, and that a homogeneous good (i.e., oil) is yielded from the sites in which new horizontal wells are drilled. Furthermore, suppose that the unit price of the output, $p$, is determined exogenously due to the firm's total production being negligible in comparison to the global market for the output. The maximization problem of the firm owning a continuum of infinitesimal well locations with disparate qualities can be articulated as follows:
\begin{equation}
\begin{split}
    \underset{q^{g}(t), d^{g}(t), \ i \in \{L, H\}}{\max} \hspace{0.2cm} \int_{0}^{\infty} e^{-rt} \left\{ p(t) \sum_{g} q^{g}(t) \ - \ C\left( \sum_{g} d^{g}(t) \right) \right\} dt
\end{split}
\label{Equation:AKS-Style-Model_Objective-Function}
\end{equation}
subject to
\begin{equation}
\begin{split}
    \dot{K}(t) \ = \ - \delta \sum_{g} q^{g}(t) \ + \ \sum_{g} I^{g} d^{g}(t), \hspace{0.3cm} K_{0} \ = \ K(0) \ \text{given,}
\end{split}
\end{equation}
\begin{equation}
\begin{split}
    \dot{R}^{g}(t) \ = \ - d^{g}(t), \hspace{0.3cm} R_{0}^{g} \ = \ R^{g}(0) \ \text{given,} 
\end{split}
\end{equation}
\begin{equation}
\begin{split}
    q^{g}(t) \ \geq \ 0, \hspace{0.3cm} 0 \ \leq \ \sum_{g} q^{g}(t) \ \leq \ K(t),
\end{split}
\end{equation}
\begin{equation}
\begin{split}
    d^{g}(t) \ \geq \ 0, \hspace{0.3cm} R^{g}(t) \ \geq \ 0.
\end{split}
\end{equation}

In this formulation, state variables $R^{g}(t)$ denote the measure of undrilled well sites at a given time $t$. Contrary to the original AKS model, we exclude the accessible capacity for oil flow $K(t)$ in the formulation due to the discussion in the paper that oil extractors usually operate at their production capacity constraint.\footnote{In other words, we simply assume that the capacity constraint is always binding.} Control variables $d^{g}(t)$ represent the rate at which new horizontal wells are drilled at time $t$. $\alpha^{g}$ are the quantity of oil production from the marginally drilled well. Here, we assume that $\alpha^{H} > \alpha^{L}$. $C(\cdot)$, indicating the total instantaneous cost of drilling, is solely a function of the drilling rates.\footnote{Regarding the total cost of oil production, we follow the assumption made in \cite{Hotelling-under-Pressure_AKS_2018}: per-barrel extraction costs from existing wells are negligible.} Of note, in this formulation for $C(\cdot)$, we assume that locations are perfect substitutes on the cost side. And the profit obtained at time $t$ is discounted at the interest rate $r$. 

The current-value Hamiltonian-Lagrangian of the firm's problem is
\begin{equation}
\begin{split}
    \mathcal{H} \ 
    & = \ \widebar{p} \sum_{g} \alpha^{g} d^{g}(t) \ - \ C\left( \sum_{g} d^{g}(t) \right) \\
    & \hspace{0.5cm} + \ \sum_{g} \pi^{g}(t) \left( -d^{g}(t) \right) \\
    & \hspace{0.5cm} + \ \sum_{g} \lambda_{1}^{g}(t) d^{g}(t) \ + \ \sum_{g} \lambda_{2}^{g}(t) R^{g}(t),
\end{split}
\label{Equation:AKS-Style-Model_Current-Value-Hamiltonian}
\end{equation}

where $\pi^{g}_{R}$ are costate variables for the state variables $R^{g}$. $\lambda_{j}, \ j \in \{1, 2\}$ are the shadow cost of each constraint.

For a given quality level $g$, two necessary conditions characterize the firm's optimal rate of drilling:
\begin{equation}
\begin{split}
    d^{g}(t) \ \geq \ 0, \hspace{0.2cm} -C'\left( \sum_{g} d^{g}(t) \right) \ + \ I^{g} \pi_{K}(t) \ - \ \pi_{R}^{g}(t) \ + \ \lambda_{4}^{g}(t) \ \leq \ 0, \hspace{0.2cm} C.S.,
\end{split}
\label{Equation:AKS-Style-Model_Necessary-Conditions_pi-K}
\end{equation}
\begin{equation}
\begin{split}
    \dot{\pi}_{R}^{g}(t) \ = \ r \pi_{R}^{g}(t) \ - \ \lambda_{5}^{g}(t).
\end{split}
\label{Equation:AKS-Style-Model_Necessary-Conditions_pi-R}
\end{equation}

When horizontal wells with heterogeneous quality are drilled simultaneously (i.e., for each $g$, $d^{g}(t) >0$, which leads to $\lambda_{1}^{g}(t) = 0$), necessary condition (\ref{Equation:AKS-Style-Model_Necessary-Conditions_pi-K}) implies that the shadow price on the resource constraint at time $t$ equals the profit on the marginal well:
\begin{equation}
\begin{split}
    \pi^{g} (t) \
    & = \ \alpha^{g} \widebar{p} \ - \ C'\left( \sum_{g} d^{g}(t) \right).
\end{split}
\label{Equation:AKS-Style-Model_Necessary-Conditions_Shadow-Price-pi}
\end{equation}

In addition, when both types of horizontal well sites are not fully exhausted (i.e., for each $g$, $R^{g}(t) > 0$, which in turn $\lambda_{2}^{g}(t) = 0$), necessary condition (\ref{Equation:AKS-Style-Model_Necessary-Conditions_pi-R}) means that the shadow value of the marginal undrilled well at time $t$ grows at the rate of $r$:
\begin{equation}
\begin{split}
    \dot{\pi}^{g} (t) \
    & = \ r \pi^{g} (t).
\end{split}
\label{Equation:AKS-Style-Model_Necessary-Conditions_Simplified-pi}
\end{equation}


The necessary conditions collectively suggest that the simultaneous drilling of horizontal wells with heterogeneous quality cannot be justified in the AKS framework when $d^{g}(t) > 0$ and $R^{g}(t) > 0$, which hold before all available well sites are developed. The followings stem from equation (\ref{Equation:AKS-Style-Model_Necessary-Conditions_Shadow-Price-pi}):
\begin{equation}
\begin{split}
    \dot{\pi}^{L}, \ \dot{\pi}^{H} \
    & = \ \left( \sum_{g} \dot{d}^{g}(t) \right) C'\left( \sum_{g} d^{g}(t) \right),
\end{split}
\label{Equation:AKS-Style-Model_pi-dots}
\end{equation}
\begin{equation}
\begin{split}
    \pi^{H} - \pi^{L} \
    & = \ (\alpha^{H} - \alpha^{L}) \widebar{p}.
\end{split}
\label{Equation:AKS-Style-Model_difference-between-pis}
\end{equation}

Here, based on equation (\ref{Equation:AKS-Style-Model_Necessary-Conditions_pi-R}), the equation (\ref{Equation:AKS-Style-Model_pi-dots}) implies $\pi^{H} = \pi^{L}$. However, this equality contradicts equation (\ref{Equation:AKS-Style-Model_difference-between-pis}) because $\alpha^{H} > \alpha^{L}$, which implies $\pi^{H} > \pi^{L}$. 

The contradiction could be attributable to not introducing any constraint for the rate of drilling (i.e., $d^{g}(t)$, \ $g \in \{ L, H \}$). As discussed in Section \ref{C3-SubSubSection:Correlation-between-Oil-Prices-and-Horizontal-Drilling-in-ND}, limited extraction capacity can change the standard order of extraction. However, in the current framework, setting the upper bound of an extraction-related constraint seems too arbitrary and complicates drawing implications from necessary conditions. From the empirical perspective, it is also intractable to quantify the margin for market-wide, or firm-wide, extraction capacity for each drilling decision. Furthermore, empirical estimation of the model from microeconomic data on drilling and production is, in general, too demanding. Those difficulties call for a new theoretical approach to build a framework that thoroughly explains our empirical findings for drilling decisions made by fracking firms in North Dakota. 


% Setup
\subsection{Setup}
\label{C3-SubSection:Setup}
We develop a continuous-time Discrete Choice Dynamic Programming (DCDP) model that formulates a fracking firm's drilling decision on a particular well site as an optimal stopping problem. Under this new theoretical framework, without specifying any capacity constraint, we can rationalize the simultaneous drilling of horizontal wells with heterogeneous resource quality.   
Moreover, our DCDP model yields predictions, being testable by utilizing available well-level data on drilling and extraction in North Dakota, about how fracking firms' drilling and production activities vary with oil prices. 

In this section, we begin by presenting the basic elements and assumptions of our DCDP framework. Following \cite{Hotelling-under-Pressure_AKS_2018}, we assume a continuum of infinitesimally small well sites in which horizontal wells will be drilled. 

The state of a given well site at the beginning of time $t$ is represented by using a well-site-level state variable $s_{t}$ as follows:
\begin{equation}
    s_{t} \ = \ 
    \begin{cases}
        \ 0 \hspace{0.5cm} \text{when the well site has not been drilled yet} \\
        \ 1 \hspace{0.5cm} \text{when the well site is already drilled.}
    \end{cases}
\label{Equation:DCDP-Model_State-Variable}
\end{equation}

For a given well site that has not been developed, the firm makes a choice at time $t$ between two alternatives $a_{t} \in \{ 0, 1 \}$, which is a well-site-level control variable:
\begin{equation}
    a_{it} \ = \ 
    \begin{cases}
        \ 0 \hspace{0.5cm} \text{if the firm decides not to drill a well in the site} \\
        \ 1 \hspace{0.5cm} \text{if the firm decides to drill a well in the site.}
    \end{cases}
\label{Equation:DCDP-Model_Control-Variable}
\end{equation}
Intuitively, the two alternatives are constrained. To be specific, the constraint depends on the value of $s_{t}$:
\begin{equation}
    a_{t}(s_{t}) \ \in \ 
    \begin{cases}
        \ \{ 0, 1 \} \hspace{0.5cm} \text{when} \hspace{0.2cm} s_{t} = 0 \\ 
        \ \{ 0 \} \hspace{0.85cm} \text{when} \hspace{0.2cm} s_{t} = 1.
    \end{cases}
\label{Equation:DCDP-Model_Constraints-of-Control-Variable}
\end{equation}

The oil production from a horizontal well drilled at time $t$ is assumed to occur only during the same period:
\begin{equation}
\begin{split}
     q_{t} \ 
     & = \ \alpha a_{t},
\end{split}
\label{Equation:DCDP-Model_Oil-Production}
\end{equation}
where $\alpha$ is the amount of oil produced from a well. For simplicity, it is also assumed that $\alpha$ is a constant across well locations. This formulation can certainly be modified to allow for production decline over time. However, we are interested in focusing on firms' investment and drilling decisions, so we abstract away from production declines. In an empirical exercise, account for the fact that a well produces for multiple periods by assuming that firms sell their production forward and receive the present value of revenue at the time they drill and complete the well. 

We assume the linear cost of drilling a horizontal well. That is,
\begin{equation}
\begin{split}
     c_{t} \
     & = \ c_{0} \ + \ c_{1} a_{t}.
\end{split}
\label{Equation:DCDP-Model_Cost}
\end{equation}
For simplicity, we also assume that the drilling cost for the marginal well is uniform across well sites (i.e., $c_{0}$ and $c_{1}$ are the same for all well locations.).\footnote{As in Section \ref{C3-SubSection:A-Limitation-of-AKS-style-Model}, we take the assumption of the negligible extraction costs.} 

Let $R_{t}$ denote the remaining number of well sites at time $t$. In the AKS framework, $R_{t}$ is the remaining reserves. Therefore, the two can be equal but do not have to be. Without loss of generality, we fix the initial level of well sites to be 1 (i.e., $R_{0} = 1$).

Intuitively, the aggregate drilling at time $t$ can be defined as a function of the remaining level of well sites at time $t$ and the probability of drilling a given well location that has not been drilled yet at time $t$:
\begin{equation}
\begin{split}
     D_{t} \
     & \equiv \ \int_{\lambda_{a} R_{t}} a_{it} \ di.
\end{split}
\label{Equation:DCDP-Model_Aggregate-Drilling}
\end{equation}
From this definition, the evolution path of the remaining well sites is governed by the following relationship:
\begin{equation}
\begin{split}
     \dot{R}_{t} \
     \equiv \ -D_{t} \hspace{0.2cm} (= -R_{t} Pr_{t}).
\end{split}
\label{Equation:DCDP-Model_Reserve}
\end{equation}
In addition, due to the assumption that $\alpha$ is uniform across well sites, the aggregate oil production $Q_{t}$ is proportional to $D_{t}$:
\begin{equation}
\begin{split}
     Q_{t} \
     \equiv \ \alpha D_{t} \ = \ \alpha \lambda_{a} R_{t} Pr_{t}.
\end{split}
\label{Equation:DCDP-Model_Aggregate-Oil-Production}
\end{equation}

The utility obtained from an individual well site can be represented by exploiting an additively separable utility function\footnote{Intuitively, this definition can be understood to mean that the drilling of a well site and the production of oil from the site occur together at time $t$. See the description of $q_{t}$ for further details.}:
\begin{equation}
\begin{split}
     U(s_{t}, p_{t}, a_{t}, \epsilon_{t}) \ 
     & = \ \tilde{U}(s_{t}, p_{t}, a_{t}) \ + \ \epsilon(a_{t}) \\
     & = \ 
     \begin{cases}
          \ \epsilon_{0,t} \hspace{5.2cm} \text{if} \hspace{0.2cm} s_{t} = 0 \text{ and } a_{t} = 0 \\
          \ u(s_{t}, p_{t}, a_{t}) \ - \ c(s_{t}, p_{t}, a_{t}) \ + \ \epsilon_{1,t} \hspace{0.5cm} \text{if} \hspace{0.2cm} s_{t} = 0 \text{ and } a_{t} = 1 \\ 
          \ 0 \hspace{5.5cm} \text{if} \hspace{0.2cm} s_{t} = 1.
     \end{cases}
\end{split}
\label{Equation:DCDP-Model_Utility-Function}
\end{equation}
where $u(\cdot)$ represents the instantaneous utility obtained from the consumption of oils. Here, we normalize $u(0) = 0$.

In the utility function, $\epsilon(a_{t})$, a component of the utility of an alternative $a$ at time $t$, is an idiosyncratic cost shock at time $t$.\footnote{We can regard $\epsilon(a_{t})$ as an element of the unobservable state vector $\epsilon_{t}$. In our case, $\epsilon_{t} = ( \epsilon_{0, t}, \ \epsilon_{1, t} )$.} In our context, $\epsilon(a_{t})$, which relies on the firm's choice, can be perceived as a composite cost element that affects the firm's decision at time $t$ between drilling today and drilling in the future and that varies over time. For example, $\epsilon(a_{t})$ could include capacity-constraint-induced costs, whose value varies with $a_{t}$. With the interpretation of $\epsilon(a_{t})$, it is not required to specifically model a set of constraints at time $t$ in our framework. 

Definitely, $\epsilon(a_{t})$ drives the firm's drilling decision at time $t$. Because the choice-specific cost shock is observable only by the firm, without $\epsilon(a_{t})$, the (observable) state variable cannot perfectly explain the firm's drilling decision at time $t$ in our model. Intuitively, when the firm decides whether to drill a well into a given well site or not at time $t$, $a_{t} = 0$ will be the optimal choice if the value of the utility function conditional on $a_{t} = 0$ is greater than or equal to that conditional on $a_{t} = 1$. Mathematically, 
\begin{equation}
\begin{split}
     \epsilon_{0, t} \
     & \geq \ u(\alpha) \ - \ (c_{0} + c_{1}) \ + \ \epsilon_{1, t} \\
     \epsilon_{0, t} \ - \ \epsilon_{1, t} \ 
     & \geq \ u(\alpha) \ - \ (c_{0} + c_{1}).
\end{split}
\label{Equation:DCDP-Model_Decision-Rule}
\end{equation}
The decision rule implies that the magnitude of $\epsilon_{0, t} - \epsilon_{1, t}$ determines the optimal choice. 

Since $\epsilon(a_{t})$ is not observable, utilizing the decision rule directly is infeasible. However, according to \cite{Euler-Equations-for-the-Estimation-of-Dynamic-Discrete-Choice-Structural-Models_Aguirregabiria-and-Magesan_2013}, the expected value of $\epsilon(a_{t})$ conditional on alternative $a_{t}$ being chosen under the decision rule can be expressed with $Pr_{t}$. To be specific, when $a_{t} = 1$, the conditional expected value of $\epsilon_{1}$, denoted $e_{1}$, is given as follows\footnote{In other words, $e(a_{t})$ indicate the mean of the cost shock conditional on choice $a_{t}$.}:
\begin{equation}
\begin{split}
     e_{1} \
     & \equiv \ E[\epsilon_{1} \ | \ a_{t} = 1] \ = \ \sigma \left( \gamma \ - \ \ln (Pr_{t}) \right).
\end{split}
\label{Equation:DCDP-Model_Expected-Value-of-Epsilon}
\end{equation}
Here, $\gamma$ is Euler's constant. And it is assumed that $\epsilon(a_{t})$ follow the Type 1 Extreme Value (T1EV) distribution with the location parameter 0 and the scale parameter $\sigma$ and are independently and identically distributed. The expected value allows our analytical as well as empirical analysis of fracking firms' drilling decisions to be tractable without observing $\epsilon(a_{t})$. 


% Social Planner's Problem
\subsection{Social Planner's Problem and Necessary Conditions}
\label{C3-SubSection:Social-Planners-Problem-and-Necessary-Conditions}
In the continuous-time DCDP framework, the social planner's problem is given by
\begin{equation}
\begin{split}
     W^{sp^{*}} \ 
     & = \ \underset{\{Pr_{t}\}_{t = 0}^{\infty}}{\max} \hspace{0.1cm} \int_{0}^{\infty} e^{-rt} W_{t}^{sp} \ dt
\end{split}
\label{Equation:Social-Planners-Problem_Formulation}
\end{equation}
subject to
\begin{equation}
\begin{split}
    \dot{R}_{t} \ = \ -\lambda_{a} R_{t} Pr_{t} \ + \ E, \hspace{0.3cm} R_{0} \ = \ R(0) \ = \ 1 \hspace{0.2cm} \text{given,}
\end{split}
\label{Equation:Social-Planners-Problem_Law-of-Motion}
\end{equation}
\begin{equation}
\begin{split}
    R_{t} \ \geq \ 0, \hspace{0.3cm} 0 \ < \ Pr_{t} \ < \ 1.
\end{split}
\end{equation}
The third term in the square bracket represents the expected value of choice-specific utility shocks for the undeveloped well sites at time $t$. In this formulation, we further assume that fracking firms will have additional new well sites (denoted $E$) augmented at a constant rate due to their exploration efforts. 

The current-value Hamiltonian-Lagrangian of the social planner's problem is given by
\begin{equation}
\begin{split}
    \mathcal{H}_{sp} \ 
    & = \ u(\alpha R_{t} Pr_{t}) \ - \ c(R_{t} Pr_{t}) \ + \ R_{t} \big\{ Pr_{t} \sigma \big( \gamma - \ln(Pr_{t}) \big) \ + \ (1 - Pr_{t}) \sigma \big( \gamma - \ln( 1 - Pr_{t}) \big) \big\} \\
    & \hspace{1.0cm} + \ \pi_{t} (-R_{t}Pr_{t} + E) \ + \ \lambda_{1,t} (R_{t}) \ + \ \lambda_{2,t} (1 - Pr_{t}) \ + \ \lambda_{3,t} (Pr_{t}).
\end{split}
\label{Equation:Social-Planners-Problem_Hamiltonian-Lagrangian}
\end{equation}

The necessary conditions of the current-value Hamiltonian-Lagrangian are as follows:
\begin{equation}
\begin{split}
    & \lambda_{a} R_{t} \big\{ \alpha u'(\alpha \lambda_{a} R_{t} Pr_{t}) \ - \ c'(\lambda_{a} R_{t} Pr_{t}) \ - \ \sigma \ln(Pr_{t}) \ + \ \sigma \ln(1 - Pr_{t}) \ - \ \pi_{t} \big\} \ \leq \ 0,
%    \hspace{0.2cm} 0 \ < \ Pr_{t} \ < \ 1, \\
%    \ - \ \lambda_{2,t} \ + \ \lambda_{3,t} \ \leq \ 0, \\
%    & \hspace{0.5cm} Pr_{t} \ \geq \ 0,  \hspace{0.2cm} \text{C.S.},
\end{split}
\label{Equation:Social-Planners-Problem_Necessary-Conditions_Drilling-Probability}
\end{equation}
\begin{equation}
\begin{split}
    \dot{\pi}_{t} \ 
    & = \ r \pi_{t} \ - \ \left\{ f_{t} \ + \ \lambda_{a} \sigma \big( \gamma \ - \ \ln(1 - Pr_{t}) \big) \right\},
%     \ - \ \lambda_{1,t},
\end{split}
\label{Equation:Social-Planners-Problem_Necessary-Conditions_Costate-Variable}
\end{equation}
\begin{equation}
\begin{split}
    \lim_{t \rightarrow \infty} e^{-rt} (R_{t} \pi_{t}) \ = \ 0.
\end{split}
\label{Equation:Social-Planners-Problem_Transversality-Condition}
\end{equation}

The costate variable $\pi_{t}$ can be expressed as a function of $R_{t}$ and $Pr_{t}$ from condition (\ref{Equation:Social-Planners-Problem_Necessary-Conditions_Drilling-Probability}) if $Pr_{t} \in (0,1)$ (i.e., $\lambda_{2,t}, \lambda_{3,t} = 0$):
\begin{equation}
\begin{split}
    \pi_{t} \ 
    & = \ \alpha u'(\alpha \lambda_{a} R_{t} Pr_{t}) \ - \ c'(\lambda_{a} R_{t} Pr_{t}) \ - \ \sigma \ln(Pr_{t}) \ + \ \sigma \ln(1 - Pr_{t}) \\
    & = \ \big\{ \alpha u'(\alpha \lambda_{a} R_{t} Pr_{t}) \ - \ c'(\lambda_{a} R_{t} Pr_{t}) \ + \ \sigma \big( \gamma \ - \ \ln(Pr_{t}) \big) \big\} \ - \ \sigma \big( \gamma \ - \ \ln(1 - Pr_{t}) \big).
\end{split}
\label{Equation:Social-Planners-Problem_Meaning-of-Costate-Variable}
\end{equation}
And when $R_{t} > 0$, condition (\ref{Equation:Social-Planners-Problem_Necessary-Conditions_Costate-Variable}) is given by
\begin{equation}
\begin{split}
    \dot{\pi}_{t} \ 
    & = \ r \pi_{t} \ - \ \sigma \big( \gamma \ + \ \ln(1 - Pr_{t}) \big)
\end{split}
\label{Equation:Social-Planners-Problem_Simplified-Costate-Variable}
\end{equation}
We will discuss the implications of those two conditions later. 


The social planner's problem has a steady state $(R_{ss}, \pi_{ss})$. By definition, each of the two state variables of the dynamic optimization problem is on its nullcline at the steady state. In other words, the change in each of them is zero at the steady state (i.e., $\dot{R}_{ss}, \dot{\pi}_{ss} = 0$). From the necessary conditions (\ref{Equation:Social-Planners-Problem_Law-of-Motion}) and (\ref{Equation:Social-Planners-Problem_Necessary-Conditions_Costate-Variable}), it is obvious that at steady state $R_{t}$ and $\pi_{t}$ satisfy the following system of equations:
\begin{equation}
    \begin{cases}
        \begin{split}
        \ R_{ss} \
        & = \ \frac{E}{ \ \lambda_{a} Pr_{ss} \ } \\
        \ \pi_{ss} \
        & = \ \frac{ \ f_{t} \ + \ \lambda_{a} \sigma \left( \gamma \ - \ \ln(1 - Pr_{ss}) \right) \ }{r}.
        \end{split}
    \end{cases}
\label{Equation:Social-Planners-Problem_System-of-Equations-for-Steady-State}
\end{equation}
On top of the system of equations, the necessary condition (\ref{Equation:Social-Planners-Problem_Meaning-of-Costate-Variable}) has to hold at the steady state. Solving the three equations simultaneously, we can identify $(R_{ss}, \pi_{ss})$, including the value of the control variable at the steady state (i.e., $Pr_{ss}$). Of note, the first equation in (\ref{Equation:Social-Planners-Problem_System-of-Equations-for-Steady-State}) demonstrates that $R_{ss}$ can be positive if $E \neq 0$.

The steady state $(R_{ss}, \pi_{ss})$ has the saddle property.\footnote{Refer to \textit{9.5 Steady states in autonomous infinite-horizon problems} in \cite{Optimal-Control-Theory-and-Static-Optimization-in-Economics_Leonard-and-Long_1992}.} Using a Taylor series approximation, $\dot{R}_{t}$ and $\dot{\pi}_{t}$ can be linearized near the steady state\footnote{Details are provided in \ref{C3-Appendix_Derivations_Linearization-near-the-Steady-State-of-the-Social-Planners-Problem}.}:
\begin{align}
\begin{split}
    \begin{pmatrix}
        \dot{R}_{t} \\
        \dot{\pi}_{t}
    \end{pmatrix} \ 
    & = \ 
    \begin{pmatrix}
        -Pr_{t} & 0 \\
        0 & r
    \end{pmatrix}
    \begin{pmatrix}
        R_{t} \ - \ R_{ss} \\
        \pi_{t} \ - \ \pi_{ss}.
    \end{pmatrix}
\end{split}
\label{Equation:Social-Planners-Problem_Linearized-System-of-Equations-near-Steady-State}
\end{align}
Since the determinant of the coefficient matrix is negative, the steady state of the infinite-horizon maximization problem is a saddle point. The phase diagram in Figure \ref{Figure:Phase-Diagram_Saddle-Point} confirms it too.


\subsubsection{Implications of Necessary Conditions}
\label{C3-SubSubSection:Necessary-Conditions}
Necessary conditions (\ref{Equation:Social-Planners-Problem_Necessary-Conditions_Costate-Variable}) and (\ref{Equation:Social-Planners-Problem_Meaning-of-Costate-Variable}) have important economic implications of the optimal path of well sites' depletion. Firstly, necessary condition (\ref{Equation:Social-Planners-Problem_Meaning-of-Costate-Variable}) directly provides us with what $\pi_{t}$ means. In this condition, the three terms in the curly bracket collectively mean the net benefit from the output (i.e., oil or gas) produced from the marginally drilled well site at time $t$. Of note, the last term among them is the expected value of the marginal well location's cost shock at time $t$ when the firm decides to drill it (i.e., $e(1)$). The remaining term in this condition represents the opportunity cost of drilling the marginal site at time $t$.\footnote{If the firm decides not to drill a horizontal well into the marginal well site, then the expected value of $\epsilon_{0,t}$ conditional on $a_{t} = 0$ (i.e., $e(0)$) is the only gain the firm gets from the decision.} Hence, the necessary condition indicates that the costate variable $\pi_{t}$ implies the net shadow value of the marginally drilled well site in the current-value term at time $t$. 

Necessary condition (\ref{Equation:Social-Planners-Problem_Necessary-Conditions_Costate-Variable}) enables us to understand what $\pi_{t}$ means from a different perspective. We can re-write this necessary condition as follows\footnote{We ignore an arbitrary integration constant $\mathcal{C}$ in the derivation because $\mathcal{C} = 0$ from the fact that $Pr_{t}$ is a constant at the steady state as well as equation (\ref{Equation:Social-Planners-Problem_System-of-Equations-for-Steady-State}).}:
\begin{equation}
\begin{split}
    & \frac{ \ \dot{\pi}_{t} \ + \ \left\{ f_{t} \ + \ \lambda_{a} \sigma \big( \gamma \ - \ \ln(1 - Pr_{t}) \big) \right\} \ }{r} \\
    & = \ \alpha u'(\alpha \lambda_{a} R_{t} Pr_{t}) \ - \ c'(\lambda_{a} R_{t} Pr_{t}) \ + \ \sigma \big( \gamma \ - \ \ln(Pr_{t}) \big) \ - \ \sigma \big( \gamma \ - \ \ln(1 - Pr_{t}) \big).
\end{split}
\label{Equation:Social-Planners-Problem_Euler-Equation}
\end{equation}
This equation implies that $\pi_{t}$ is the marginally undrilled well site's aggregate expected utility (i.e., the sum of the expected value of $\epsilon_{0,t}$'s over time), as the current value at time $t$, if the well site will remain undeveloped. Therefore, it is clear that drilling well locations is economic depletion in our framework, as extracting an exhaustible resource is in Hotelling's model. 

Collectively, necessary conditions (\ref{Equation:Social-Planners-Problem_Necessary-Conditions_Costate-Variable}) and (\ref{Equation:Social-Planners-Problem_Meaning-of-Costate-Variable}) suggest that on the optimal path of drilling, the marginal undeveloped well site will be drilled at time $t$ if the net gains from drilling it at time $t$ equal the undrilled site' aggregate future gains from time $t$. In other words, at the margin, drilling a horizontal well today is an optimal choice for the firm if its value is indifferent to the value of simply holding it forever. Indeed, this implication holds under the Hotelling framework. 

Necessary condition (\ref{Equation:Social-Planners-Problem_Necessary-Conditions_Costate-Variable}) demonstrates significant implications of $\pi_{t}$'s growth over time. We can re-express this condition as follows\footnote{From equation (\ref{Equation:Social-Planners-Problem_Euler-Equation}), $\pi_{t} = e^{rt} \sigma \big( \gamma \ - \ \ln(1 - Pr_{t}) \big) \int_{t}^{\infty} e^{-r\tau} d\tau = \sigma \big( \gamma \ - \ \ln(1 - Pr_{t}) \big) / r$ at the steady state.}:
\begin{equation}
\begin{split}
%    \dot{\pi}_{t} \ 
%    & = \ r \pi_{t} \ - \ \sigma \big( \gamma \ + \ \ln(1 - Pr_{t}) \big) \\
    \dot{\pi}_{t} \
    & = \ r \left\{ \pi_{t} \ - \ \frac{ \ f_{t} \ + \ \sigma \big( \gamma \ - \ \ln(1 - Pr_{t}) \big) \ }{r} \right\}.
%    \frac{\dot{\pi_{t}}}{\pi_{t}} \ + \  \frac{\sigma \big( \gamma \ - \ \ln(1 - Pr_{t}) \big)}{\pi_{t}} \ 
%    & = \ r
\end{split}
\label{Equation:Social-Planners-Problem_Growth-Rate-of-Costate-Variable}
\end{equation}
This expression clearly indicates that $\pi_{t}$ grows slower than the rate of interest $r$. The necessary condition also suggests that $\pi_{t}$ increases concavely with $Pr_{t}$ and converges in the limit, unlike the exponential growth of the shadow price on the law of motion in Hotelling's framework.\footnote{From the beginning of drilling (i.e., $t = 0$), $R_{t}$ decreases. If $Pr_{t}$ is maintained at a lower level, the rate of drilling will quickly converge to zero. So, $Pr_{t}$ must continuously grow to keep drilling. In equation (\ref{Equation:Social-Planners-Problem_Growth-Rate-of-Costate-Variable}), $Pr_{t}$'s increase leads to the reduction in $\dot{\pi}_{t}$.} 

The value of $\sigma$, which is the dispersion parameter of the I.I.D. T1EV cost shocks, provides two interesting implications. First, the social planner's problem reverts to the Hotelling model of the optimal extraction of a nonrenewable resource when $\sigma$ goes to zero. Taking limits to zero for necessary conditions (\ref{Equation:Social-Planners-Problem_Necessary-Conditions_Costate-Variable}) and (\ref{Equation:Social-Planners-Problem_Meaning-of-Costate-Variable}) yields the followings, which are identical to the two necessary conditions in Hotelling's classic model of depletion\footnote{In the limiting case, we ignore the flow utility $f_{t}$, which is not introduced in Hotelling's theoretical model.}:
\begin{equation}
\begin{cases}
        \begin{split}
        \ \lim_{\sigma \to 0} \dot{\pi}_{t} \
        & = \ r\pi_{t} \\
        \ \lim_{\sigma \to 0} \pi_{t} \
        & = \ \alpha u'(\alpha \lambda_{a} R_{t} Pr_{t}) \ - \ c'(\lambda_{a} R_{t} Pr_{t}).
        \end{split}
    \end{cases}
\label{Equation:Social-Planners-Problem_Reverting-to-the-Hotelling-Model}
\end{equation}
Intuitively, the limiting case that $\sigma$ takes a value of zero means that the drilling decision for the marginal well location depends only on drilling costs and the interest rate $r$, which are not stochastic, unlike cost shocks $\epsilon(a_{t})$'s.

Second, the magnitude of $\sigma$ determines the rate of drilling, and also production. To be specific, an increase in the magnitude of $\sigma$ reduces drilling. When the value of $\sigma$ grows, the importance of the cost shocks increases relative to the observable components in the utility function (i.e., $\tilde{U}(\cdot)$). In other words, as more utility comes from the cost shocks, the option value of each well location increases. Therefore, a larger value of $\sigma$ makes the social planner wait for a better shock, which in turn, delays well drilling. 

Transversality condition (\ref{Equation:Social-Planners-Problem_Transversality-Condition}) rules out too aggressive depletion of well sites. Note that the transversality condition holds even when $R_{t} \neq 0$.


% Firm's Problem
\subsection{Firm's Problem}
\label{C3-SubSection:Firms-Problem}
In this section, we develop the firm's problem under the settings of our DCDP model in continuous time. We first introduce additionally required building blocks. In particular, we integrate heterogeneity in the quality of well sites into the problem. Following \cite{Estimation-of-Dynamic-Discrete-Choice-Models-in-Continuous-Time_ABBE_2016}, we formulate the value function for a given well site. And then, utilizing the value function, we show that the oil market clears. In addition, we formulate firm-level optimal paths by aggregating the firm's well-level drilling decisions. 

\subsubsection{Firm's Decisions on Drilling Well Sites}
\label{C3-SubSubSection:Firms-Decisions-on-Drilling-Well-Sites}
In this section, we develop the firm's problem under the settings of our DCDP model in continuous time. Following \cite{Estimation-of-Dynamic-Discrete-Choice-Models-in-Continuous-Time_ABBE_2016}, we formulate the value function for a particular potential well site owned by the firm $i$ that is forward-looking and discounts future payoffs at rate $\rho \in (0, \infty)$. Specifically, when the site is in state $k$, its value function is given as follows\footnote{Detailed derivation is presented in \ref{C3-Appendix_Derivations_Value-Function-in-Continuous-Time}.}:
\begin{equation}
\begin{split}
    % \left( \rho \ + \ \sum_{\ell \neq k} \lambda_{k\ell} \ + \ \lambda_{d} \right) V_{ik} \ 
    % & = \ f_{ik} \ + \ \sum_{\ell \neq k} \lambda_{k\ell} V_{i\ell} \ + \ \lambda_{d} E\bigg[ \underset{a}{\max} \left\{ V_{i,\ell(i, a, k)} \ + \ \psi_{iak} \ + \ \epsilon_{iak} \right\} \bigg].
%    V_{ik} \ 
%    & = 
%    \ \frac{
%        \ f_{ik} \ + \ \sum_{\ell \neq k} \lambda_{k\ell} V_{i\ell} \ + \ \lambda_{a} E\Big[ \underset{a \in \mathcal{A}}{\max} \left\{ V_{i,\ell(i, a, k)} \ + \ \psi_{iak} \ + \ \epsilon_{iak} \right\} \Big] \ 
%    }{
%        \rho \ + \ \sum_{\ell \neq k} \lambda_{k\ell} \ + \ \lambda_{a}
%    }.
    & -\dot{V}_{ik} (t) \ + \ \left( \rho \ + \ \lambda_{a} \ + \ \sum_{\ell \neq k} \lambda_{k\ell} \right) V_{ik}(t) \\
    & \hspace{1.0cm} = \ (f_{ik} \ + \ \lambda_{a} \tilde{f}_{ik}) \ + \ \lambda_{a} E\Big[ \underset{a \in \mathcal{A}}{\max} \left\{ V_{i,\ell(i, a, k)} (t) \ + \ \psi_{iak} \ + \ \tilde{\psi}_{iak} \ + \ \epsilon_{iak} \right\} \Big] \ + \ \sum_{\ell \neq k} \lambda_{k\ell} V_{i\ell}(t)
\end{split}
\label{Equation:Firms-Problem_Value-Function}
\end{equation}
For the site, the value function $V_{ik}$ represents the present discounted value of all payoffs obtained from starting at state $k$ and behaving optimally in all subsequent periods.\footnote{Although $V_{ik}$ varies over time, we omit the $t$ subscript for simplicity.} Here, it is assumed that the firm $i$'s drilling decisions have no impact on the price of oil in the market.\footnote{In other words, the assumption implies that the resulting state of taking action $a$ by the firm $i$, denoted $\ell(i, a, k)$, is $k$.} $\dot{V}_{ik}$ is the time derivative of $V_{ik}$. The three terms in the round bracket on the left-hand side are the sum of the discount factor and the rates at which the state can change. The right-hand side consists of the flow payoff, the expected value relying on the firm's decisions, and the rate-weighted value related to exogenous state transitions. The expectation is for the joint distribution of $\epsilon_{i0k}$ and $\epsilon_{i1k}$. While adding in the T1EV cost shocks is tedious, it also allows us to reconcile an empirical, firm-level model with aggregate time paths of resource extraction. 

For a given opportunity to choose an action $a \in \mathcal{A}$, the probability of drilling a horizontal well into the potential site conditional on state $k$, denoted $Pr_{k}$, can be defined as follows\footnote{For given values of parameters, we can compute the value of each $Pr_{k}$, $k = 1, 2, \cdots, K$, by using value function iterations.}:
\begin{equation}
\begin{split}
	Pr_{k} \
	& \equiv \ \Pr \big[ \ \psi_{i1k} + \epsilon_{i1k} \ \geq \ V_{i,\ell(i,0,k)} + \psi_{i0k} + \epsilon_{i0k} \ | \ k \ \big].
\end{split}
\label{Equation:Firms-Problem_CCP}
\end{equation}
As shown in \cite{Estimation-of-Dynamic-Discrete-Choice-Models-in-Continuous-Time_ABBE_2016}, for each action $a \in \mathcal{A}$, the second term on the right-hand side of equation (\ref{Equation:Firms-Problem_Value-Function}) is
\begin{equation}
%\small
\begin{split}
	\lambda_{a} E\Big[ \underset{a \in \mathcal{A}}{\max} \left\{ V_{ik} + \psi_{iak} + \epsilon_{iak} \right\} \Big] 
	& =  
	\underbrace{\lambda_{a} \left\{ V_{ik} + \sigma \big( \gamma - \ln(1 - Pr_{k}) \big) \right\} }_{\text{if \ $a = 0$}} = \underbrace{\lambda_{a} \left\{ \psi_{i1k} + \sigma \big( \gamma - \ln(Pr_{k}) \big) \right\} }_{\text{if \ $a = 1$}},
\end{split}
\label{Equation:Firms-Problem_Emax}
\end{equation}
where the choice-specific instantaneous payoff function for $a = 1$ (i.e., $\psi_{i1k}$) is the function (\ref{Equation:DCDP-Model_Payoff-Function_Firms-Problem}).

Under the assumption that there is no exogenous demand shock, some algebraic manipulation on the value function (\ref{Equation:Firms-Problem_Value-Function}), with the expressions in (\ref{Equation:Firms-Problem_Emax}), yields the Euler equation that drives the dynamics of the firm's optimal drilling decisions\footnote{See \ref{C3-Appendix_Derivations_Euler-Equation-for-the-Firms-Problem} for details.}:
\begin{equation}
\begin{split}
    & \frac{ \ \dot{V}_{ik} \ + \ f_{ik} \ + \ \lambda_{a} \sigma \big( \gamma \ - \ \ln(1 - Pr_{k}) \big) \ }{\rho} \\
    & = \ \alpha p_{k} \ - \ c \ + \ \sigma \big( \gamma \ - \ \ln(Pr_{k}) \big) \ - \ \sigma \big( \gamma \ - \ \ln(1 - Pr_{k}) \big).
\end{split}
\label{Equation:Firms-Problem_Euler-Equation}
\end{equation}
The left-hand side of this equation represents the sum of the instantaneous change in $V_{ik}$ and the firm's expected payoff when deciding not to drill a horizontal well into the site as the current value at the time point of decision. In the equation, the right-hand side, which is equal to $V_{ik}$ as shown in \ref{C3-Appendix_Derivations_Euler-Equation-for-the-Firms-Problem}, is the firm's payoff if it chooses to drill a horizontal well into the site, including the opportunity cost of the decision. In other words, $V_{ik}$ is the well location's net shadow value as the current value, which is represented as $\pi_{t}$ in the necessary conditions for the social planner's problem. Based on the relationship between $V_{ik}$ and $\pi_{t}$, it is evident that the Euler equation (\ref{Equation:Firms-Problem_Euler-Equation}) drawn from the firm $i$'s well-site-level decisions coincides with the Euler equation (\ref{Equation:Social-Planners-Problem_Euler-Equation}) of the social planner's welfare maximization problem.\footnote{There are two differences between the Euler equations. First, the rate of interest $r$ is used in the social planner's problem, whereas the discount rate $\rho$ is utilized in the firm's problem. Second, we exploit two different choice-specific instantaneous payoff functions, which are demonstrated in (\ref{Equation:DCDP-Model_Payoff-Function_Social-Planners-Problem}) and (\ref{Equation:DCDP-Model_Payoff-Function_Firms-Problem}), in the two dynamic optimization problems. Of note, both of the choice-specific instantaneous payoff functions indicate the net benefit obtained from the output of drilling activities.}
\afterpage{
    \begin{figure}[t!]
        \centering
        \includegraphics[scale = 0.15]{04_Chapter-3/00A_Figures/Figure_Equlibrium-Paths_Endogenous-Price.png}
        \caption{Equilibrium Paths under Unexpected Demand Shocks}
        \caption*{
            {\small
            \textit{Note}: 
            This figure shows the equilibrium paths of drilling probability, drilling, and the measure of well sites, which are obtained from a simulation for two unanticipated demand shocks. For these simulation results, we assume the identical parameter values and cost function utilized to draw the phase diagrams in Figure \ref{Figure:Phase-Diagram_Saddle-Point}.
        }}
        \label{Figure:Equilibrium-Paths-under-Unexpected-Demand-Shocks}
    \end{figure}
}




\section{Equilibrium Dynamics with Oil Prices}
\label{C3-Section:Equilibrium-Dynamics-with-Oil-Prices}
This section examines how the time paths for optimal drilling and production vary with oil prices. Using the Implicit Function Theorem, we first predict the impact of a sudden price variation on the drilling probability, which governs the optimal paths of drilling and production. We then demonstrate how the optimal drilling and production paths respond to unexpected demand shocks. We also investigate the heterogeneous impacts of unanticipated demand shocks on drilling well sites of different quality. Finally, we compute the equilibrium paths under two different scenarios for oil prices: exogenous and endogenous oil prices.


\subsection{Impacts of Unexpected Demand Shocks}
\label{C3-SubSection:Impacts-of-Unexpected-Demand-Shocks}
\afterpage{
    \begin{figure}[t!]
        \centering
        \includegraphics[scale = 0.15]{04_Chapter-3/00A_Figures/Figure_Equlibrium-Paths_Endogenous-Price.png}
        \caption{Equilibrium Paths under Unexpected Demand Shocks}
        \caption*{
            {\small
            \textit{Note}: 
            This figure shows the equilibrium paths of drilling probability, drilling, and the remaining well sites, which are obtained from a simulation for two unanticipated demand shocks. For these simulation results, we assume the identical parameter values and cost function utilized to draw the phase diagrams in Figure \ref{Figure:Phase-Diagram_Saddle-Point}.
        }}
        \label{Figure:Equilibrium-Paths-under-Unexpected-Demand-Shocks}
    \end{figure}
}
The Implicit Function Theorem (IFT) allows us to predict the impact of sudden demand shocks on the equilibrium paths for drilling and production. Applying the IFT to equation (\ref{Equation:Firms-Problem_Euler-Equation}) (i.e., the Euler equation of the firm's problem) yields
\begin{equation}
\begin{split}
    % \frac{ \ f_{ik} \ + \ \lambda_{a} \sigma \big( \gamma \ - \ \ln(1 - Pr_{k}) \big) \ }{\rho} \
    % & = \ \left\{ \psi_{i1k} \ + \ \sigma \big( \gamma \ - \ \ln(Pr_{k}) \big) \right\} \ - \ \sigma \big( \gamma \ - \ \ln(1 - Pr_{k}) \big) \\
    % \frac{1}{\rho} \frac{\partial f_{ik}}{\partial p_{k}} \ + \ \frac{\lambda_{a} \sigma}{\rho (1 - Pr_{k})} \frac{\partial Pr_{k}}{\partial p_{k}} \
    % & = \ \frac{\partial \psi_{i1k}}{\partial p_{k}} \ - \ \frac{\sigma}{Pr_{k}} \frac{\partial Pr_{k}}{\partial p_{k}} \ - \ \frac{\sigma}{1 - Pr_{k}} \frac{\partial Pr_{k}}{\partial p_{k}} \\
    % \left( \frac{\lambda_{a} \sigma}{\rho (1 - Pr_{k})} \ + \ \frac{\sigma}{Pr_{k}} \ + \ \frac{\sigma}{1 - Pr_{k}} \right) \frac{\partial Pr_{k}}{\partial p_{k}} \
    % & = \ -\frac{1}{\rho} \frac{\partial f_{ik}}{\partial p_{k}} \ + \ \frac{\partial \psi_{i1k}}{\partial p_{k}} \\
    \frac{\partial Pr_{k}}{\partial p_{k}} \
    & = \ \left\{ \frac{\rho Pr_{k} (1 - Pr_{k})}{\sigma (\lambda_{a} Pr_{k} + \rho)} \right\} \left( \frac{\partial \psi_{i1k}}{\partial p_{k}} \ - \ \frac{1}{\rho} \frac{\partial f_{ik}}{\partial p_{k}} \right).
\end{split}
\label{Equation:Equilibrium-Paths_Applying-IFT}
\end{equation}
This resulting equation suggests that an unexpected positive price shock will lead to a higher drilling rate if the cost-shock-induced incremental gains from drilling are larger than those from waiting (i.e., $\partial \psi_{i1k}/\partial p_{k} - (1/\rho)(\partial f_{ik}/\partial p_{k}) > 0$).\footnote{If $Pr_{k} \in (0, 1)$, every term in the curly bracket is always positive.}
\afterpage{
    \begin{figure}[t!]
        \centering
        \includegraphics[scale = 0.11]{04_Chapter-3/00A_Figures/Figure_Impact-of-Demand-Shocks-on-Drilling-of-Well-Sites-with-Heterogeneous-Quality}
        \caption{Heterogeneous Impacts of Unexpected Demand Shocks on Drilling and Production}
        \caption*{
            {\small
            \textit{Note}: 
            This figure demonstrates how the firm's drilling activity at well sites of heterogeneous quality responds to unexpected demand shocks. As shown in the first panel, the drilling probability of low-quality well sites shows a higher sensitivity to the first negative demand shock. The second panel demonstrates that although the drilling probability of low-quality well locations more sensitively responds to the second positive demand shock, the probability is still lower than that of high-quality well sites. The third panel illustrates the impacts of the demand shocks on oil extraction productivity. Clearly, the first negative shock discontinuously increases per-drilling oil production due to the high sensitivity of low-quality location drilling. To the second positive demand shock, the per-drilling production showed the opposite reaction. For this simulation, we assume that a dispersion parameter of $\sigma = 1$, an interest rate of $r = 0.05$, an initial number of well sites of $R_{0}^{g} = 1$, additional well sites of $E^{g} = 0$, where $g \in \{ L, H \}$. Also, it is assumed that a flow payoff function of $f(p_{k}) = 20 - 0.5p_{t}$ and that an instantaneous payoff function of $\psi_{iak} (p_{k}, a) = [\{ g_{i}^{L} + 1.5(1 - g_{i}^{L}) \} p_{k} - 2]a$. 
        }}
        \label{Figure:Heterogeneous-Impacts-of-Unexpected-Price-Shocks-on-Equilibrium-Paths}
    \end{figure}
}

Figure \ref{Figure:Equilibrium-Paths-under-Unexpected-Demand-Shocks} depicts how the paths of drilling probability, drilling, reserves, and oil price respond to two unanticipated demand shocks.\footnote{Given parameter values utilized for this simulation, the condition for the positive relationship between drilling probability and unexpected price shocks in (\ref{Equation:Equilibrium-Paths_Applying-IFT}) (i.e., $\partial \psi_{i1k} / \partial p_{k} - (1/\rho)(\partial f_{ik} / \partial p_{k}) > 0$) holds.} The first negative demand shock causes drilling probability discontinuously decreases. Due to the reduction in drilling probability, drilling demonstrates a discontinuous decrease too. Moreover, the negative demand shock also reduces the depletion rate of the remaining well locations. As shown in the last panel in the figure, the oil price jumps down on impact after the negative demand shock, then gradually rises.\footnote{Drilling, and thus production, rapidly diminishes for a while after $t = 0$. Then, its rate of change gradually decreases, and drilling eventually converges to a lower bound. In other words, the time path of drilling has a convex profile. Because equation (\ref{Equation:Firms-Problem_Oil-Prices}) determines the oil price at time $t$, the time path for the endogenous oil price is a concave curve.} The later positive demand shock induces the opposite reactions in the equilibrium paths. 


\subsection{Heterogeneous Impacts of Unanticipated Demand Shocks on Drilling Well Sites of Different Quality}
\label{C3-SubSection:Heterogeneous-Impacts-of-Unanticipated-Demand-Shocks-on-Drilling-Well-Sites-of-Different-Quality}
To examine how the firm's drilling activity at well sites of heterogeneous quality responds to unexpected demand shocks, we suppose that a particular well location $i$ has a quality type that falls in one from the quality set $\mathcal{Q} = \{ L(ow), H(igh) \}$. Furthermore, we also assume that when the well location is not drilled yet, the choice-specific instantaneous payoff $\psi_{i \alpha k}$ takes the following functional form:
\begin{align}
    \psi_{iak}(p_{k}, a; \boldsymbol{\theta}_{\psi}) \
    & = \ \big[ \big\{ g_{i}^{L} \ + \ (1 - g_{i}^{L}) \alpha^{H} \big\} p_{k} \ - \ c \big] a,
\label{Equation:Firms-Problem_Instantaneous-Payoff-for-Heterogeneous-Quality}
\end{align}
where $g_{i}^{L}$ is a binary indicator with the value of one when well location $i$ is a low-quality well site and $\alpha^{H}$, which is greater than 1, is the normalized oil production from the site $i$ whose quality type is $H$.\footnote{In the formulation, we implicitly normalize the oil production from a well location to 1.} 

Figure \ref{Figure:Heterogeneous-Impacts-of-Unexpected-Price-Shocks-on-Equilibrium-Paths} illustrates the results from a simulation. As discussed in Section \ref{C3-SubSubSection:The-Role-of-Geological-Quality-in-Horizontal-Drilling}, our empirical analysis reveals that fracking firms in North Dakota more significantly reduced drilling at low-quality well sites, more than at the high-quality ones, when experiencing sharp oil price declines. The time paths for drilling and production presented in the figure clearly show that the model-predicted elasticity of drilling s greater on low-quality well sites, just as illustrated in Figure \ref{Figure:High-Sensitivity-of-Firm-Level-Low-Quality-Well-Drilling}.\footnote{The term $Pr_{k}(1 - Pr_{k})$ in equation (\ref{Equation:Equilibrium-Paths_Applying-IFT}) is a parabolic curve that goes to zero as $Pr_{k}$ approaches to zero or one and that has its maximum value at $Pr_{k} = 1/2$. These properties of the term suggest the possibility that the drilling of high-quality well sites is less responsive to a demand shock.} As demonstrated in the third panel of the figure, per-drilling production, i.e., productivity, increases discontinuously due to the negative demand shock. Consequently, as described in the fourth panel, per-drilling oil production is also improved. 


\subsection{Exogenous vs. Endogenous Oil Prices}
\label{C3-SubSection:Exogenous-vs-Endogenous-Oil-Prices}
This section examines the equilibrium path of drilling for each of the endogenous and exogenous oil prices. In the endogenous-price scenario, the market clearing oil price is determined from equation (\ref{Equation:Firms-Problem_Oil-Prices}). Moreover, the exogenous price means the case of a constant oil price, in which the oil production industry is small relative to the world oil market.\footnote{In other words, $\widebar{p}_{1} = 0$ in equation (\ref{Equation:Firms-Problem_Oil-Prices}) in the exogenous-price scenario.}
 
\afterpage{
    \begin{figure}[t!]
        \centering
        \includegraphics[scale = 0.13]{04_Chapter-3/00A_Figures/Figure_Reserves-and-Drilling-Paths_Endogenous-and-Exogenous-Prices.png}
        \caption{Time Paths under Endogenous and Exogenous Oil Prices}
        \caption*{
            {\small
            \textit{Note}: 
            This figure depicts differences in the time paths for drilling and the measure of well locations under endogenous and exogenous oil prices. This figure takes the assumptions exploited in Figure \ref{Figure:Phase-Diagram_Saddle-Point}. See the text for details.
        }}
        \label{Figure:Time-Paths-for-Drilling-and-Reserves-under-Endogenous-and-Exogenous-Oil-Prices}
    \end{figure}
}
The Euler equation (\ref{Equation:Firms-Problem_Euler-Equation}) allows us to predict the differences in drilling and production time paths between the two price scenarios. The endogenous price is lower than the exogenous price when there is oil production (i.e., $Q > 0$). In the Euler equation, a lower price, thus a lower instantaneous payoff of drilling (i.e., $\psi_{i1k}$),  implies a lower probability of drilling. In other words, endogenizing the time path of oil prices makes the probability of drilling at time $t = 0$ decrease due to the initial production (i.e., $Q_{0} > 0$). Since increasing drilling, and thus production, causes oil prices endogenously determined to fall, there would be no incentive to rapidly raise the rate of drilling. For these reasons, drilling under endogenous oil prices (denoted $D_{t}^{en}$) would be small relative to that under exogenous oil prices (denoted $D_{t}^{ex}$) for some period after $t = 0$. But at some time point, $D_{t}^{en}$ would be larger than $D_{t}^{ex}$ because of the lower level of undrilled reserves in the exogenous-price case. Figure \ref{Figure:Time-Paths-for-Drilling-and-Reserves-under-Endogenous-and-Exogenous-Oil-Prices}, which shows the time paths for drilling and the remaining reserves for each type of oil price, supports the predictions.




\section{Conclusion}
\label{C3-Section:Conclusion}
In this paper, we first present two interesting empirical findings about the drilling decisions of hydraulic fracturing companies in North Dakota. First, the oil producers drilled horizontal wells into sites of heterogeneous quality, not in a strict order, but simultaneously. Second, the drilling of low-quality well sites was more responsive to the significant plunge in oil prices in the second half of 2014 than that of high-quality ones. 

We develop a continuous-time Discrete Choice Dynamic Programming (DCDP) to explain the empirical facts that are not captured in the economic models developed in other papers. Our theoretical framework is analytically tractable. Moreover, our economic model, in which choice-specific cost shocks at each decision opportunity allow us to avoid modeling specific constraints (e.g., capacity and transmission constraints), provides us with insightful implications for how the drilling activity of fracking firms operating in North Dakota evolves in response to oil price changes. Furthermore, in our model of optimal drilling, well-level drilling decisions naturally lead to the optimal firm-level path of drilling. 

In addition to the analytical advantages, our model is also empirically manageable. We can easily take our model to detailed well-level drilling and production data. Therefore, utilizing the empirically estimated values of structural parameters, we can perform counterfactual analysis on real-world issues in the oil and gas industry, such as a change in the severance tax rate in North Dakota. In other words, our theoretical framework can bridge Hotelling-style theoretical approaches with empirical research that relies heavily on microeconomic data. 



% ------- Appendix -------
\appendix
\chapter{Appendixes for Chapters}
\label{Chapter:Appendixes-for-Chapters}
%\counterwithin{figure}{chapter}
\section{For Chapter 1}
\subsection{Additional Figure(s) and Table(s)}
% 1. Figure(s)
% \afterpage{
    \begin{figure}[ht!]
        \centering
        \includegraphics[scale = 0.133]{02_Chapter-1/00A_Figures/Figure_SMUD-Residential-Rates_Fixed-Charge.png}
        \caption{Fixed Charge of SMUD Residential Rates}
        \caption*{
            {\small
            \textit{Note}: 
            The figure shows how SMUD changed the monthly fixed charge over time. The same fixed charge applies to households that choose one of the three major residential rate plans (i.e., RSCH, RSEH, and RSGH).
        }}
        \label{Figure:SMUD-Residential-Rates_Fixed-Charge}
    \end{figure}
% }

\afterpage{
    \begin{figure}[t!]
        \centering
        \includegraphics[scale = 0.105]{02_Chapter-1/00A_Figures/Figure_Household-Average-Daily-Consumption-by-Month.png}
        \caption{Household Average Daily Electricity Consumption by Month of Year}
        \caption*{
            {\small
            \textit{Note}: 
            This figure depicts, for each of SMUD residential rate plans, how households' average daily electricity consumption varied across months of the year. The red and blue dot-dash lines represent the lower and higher base usage quantities in each month of the year, respectively. The three rate plans show similar consumption and seasonal patterns. 
        }}
        \label{Figure:Household-Average-Daily-Electricity-Consumption-by-Month-of-Year}
    \end{figure}
}
\clearpage

% \afterpage{
    \begin{figure}[ht!]
        \centering
        \includegraphics[scale = 0.120]{02_Chapter-1/00A_Figures/Figure_Residuals-of-the-Average-Daily-Electricity-Consumption-in-Period-0-over-NC0_From-BW5-to-BW40.png}
        \caption{The Impact of the Change in the Marginal Price due to Surpassing the Lower Base Usage Quantity}
        \caption*{
            {\small
            \textit{Note}: 
            In this figure, scatter dots correspond to the mean of residuals, computed by bins with a bandwidth of 0.5\%, from a regression of households' average daily electricity consumption in Period 1 on $\widebar{NC}_{0}$, HDDs and CDDs. As described, a linear model fits those scatter points well, even for a wide bandwidth. 
        }}
        \label{Figure:The-Impact-of-the-Change-in-the-MP-due-to-Surpassing-the-Lower-BUQ}
    \end{figure}
% }
\clearpage



% 2. Table(s)
\afterpage{
    \begin{table}[t!]
        \centering
        \caption{Robustness Checks: For Different Functional Forms, 3rd- and 4th-Order Polynomial Models}
        \label{Table:Robustness-Checks_Functional-Forms_3rd-and-4th-Order-Polynomial-Models}
        \vspace{0.3cm}
        \footnotesize
        \begin{adjustbox}{scale = 0.83}
            \begin{threeparttable}
                \begin{tabular}{@{\extracolsep{1pt}}lcccccccc}
                    \\[-5.5ex]
                    \hline \hline
                    \\[-3.0ex]
                     & \multicolumn{8}{c}{Dependent Variable} \\
                    \\[-3.0ex]
                    \cline{2-9}
                    \\[-3.0ex]
                     & \multicolumn{8}{c}{Average Daily Electricity Consumption  (kWh/Day)} \\
                    \\[-3.0ex]
                     & (1) & (2) & (3) & (4) & (5) & (6) & (7) & (8) \\
                    \\[-3.0ex]
                    \hline
                    \\[-2.0ex]
                    $\mathbb{1}$[Treatment] & 0.049 & $-$0.022 & $-$0.056$^{***}$ & $-$0.101$^{***}$ & 0.129$^{**}$ & 0.019 & $-$0.049$^{**}$ & $-$0.068$^{**}$ \\ 
                    & (0.042) & (0.025) & (0.017) & (0.028) & (0.051) & (0.037) & (0.021) & (0.027) \\ 
                    & & & & & & & & \\ 
                    NC0 & 0.143$^{***}$ & 0.211$^{***}$ & 0.215$^{***}$ & 0.225$^{***}$ & 0.111$^{**}$ & 0.207$^{***}$ & 0.213$^{***}$ & 0.217$^{***}$ \\ 
                    & (0.022) & (0.010) & (0.006) & (0.010) & (0.043) & (0.016) & (0.008) & (0.011) \\ 
                    & & & & & & & & \\ 
                    $\mathbb{1}$[Treatment] $\times$ NC0 & 0.056 & 0.005 & $-$0.007 & 0.0003 & $-$0.035 & $-$0.027 & $-$0.008 & $-$0.001 \\ 
                    & (0.035) & (0.013) & (0.005) & (0.005) & (0.079) & (0.021) & (0.008) & (0.010) \\ 
                    & & & & & & & & \\ 
                    NC0$^{2}$ & $-$0.020$^{***}$ & $-$0.002 & $-$0.0003 & $-$0.0001 & $-$0.035$^{**}$ & $-$0.003 & $-$0.001 & $-$0.001$^{*}$ \\ 
                    & (0.005) & (0.001) & (0.0002) & (0.0002) & (0.017) & (0.003) & (0.001) & (0.001) \\ 
                    & & & & & & & & \\ 
                    $\mathbb{1}$[Treatment] $\times$ NC0$^{2}$ & 0.022$^{***}$ & 0.001 & 0.0004 & $-$0.0004 & 0.092$^{***}$ & 0.010$^{**}$ & 0.001 & 0.001$^{**}$ \\ 
                    & (0.008) & (0.001) & (0.0003) & (0.0003) & (0.024) & (0.005) & (0.001) & (0.001) \\ 
                    & & & & & & & & \\ 
                    NC0$^{3}$ & $-$0.001$^{***}$ & $-$0.00005 & $-$0.00000 & 0.00000 & $-$0.004 & $-$0.0001 & $-$0.00002 & $-$0.00003$^{*}$ \\ 
                    & (0.0003) & (0.00003) & (0.00001) & (0.00000) & (0.002) & (0.0002) & (0.00004) & (0.00002) \\ 
                    & & & & & & & & \\ 
                    $\mathbb{1}$[Treatment] $\times$ NC0$^{3}$ & 0.001$^{**}$ & 0.0001 & $-$0.00000 & 0.00000 & $-$0.005 & $-$0.0005$^{*}$ & $-$0.00001 & $-$0.00000 \\ 
                    & (0.001) & (0.00005) & (0.00001) & (0.00001) & (0.005) & (0.0003) & (0.0001) & (0.00004) \\ 
                    & & & & & & & & \\ 
                    NC0$^{4}$ &  &  &  &  & $-$0.0001 & $-$0.00000 & $-$0.00000 & $-$0.00000$^{*}$ \\ 
                    &  &  &  &  & (0.0001) & (0.00000) & (0.00000) & (0.00000) \\ 
                    & & & & & & & & \\ 
                    $\mathbb{1}$[Treatment] $\times$ NC0$^{4}$ &  &  &  &  & 0.001$^{***}$ & 0.00002$^{*}$ & 0.00000 & 0.00000$^{**}$ \\ 
                    &  &  &  &  & (0.0002) & (0.00001) & (0.00000) & (0.00000) \\ 
                    & & & & & & & & \\ 
                    Average Daily CDDs & 1.146$^{***}$ & 1.146$^{***}$ & 1.135$^{***}$ & 1.133$^{***}$ & 1.146$^{***}$ & 1.146$^{***}$ & 1.135$^{***}$ & 1.133$^{***}$ \\ 
                    & (0.106) & (0.105) & (0.109) & (0.129) & (0.106) & (0.105) & (0.109) & (0.129) \\ 
                    & & & & & & & & \\ 
                    Average Daily HDDs & 0.428$^{***}$ & 0.431$^{***}$ & 0.375$^{***}$ & 0.742$^{***}$ & 0.428$^{***}$ & 0.431$^{***}$ & 0.375$^{***}$ & 0.742$^{***}$ \\ 
                    & (0.106) & (0.104) & (0.128) & (0.202) & (0.106) & (0.104) & (0.128) & (0.202) \\ 
                    & & & & & & & & \\
                    \hline
                    \\[-2.0ex]
                    Bandwidth & 10\% & 20\% & 30\% & 40\% & 10\% & 20\% & 30\% & 40\% \\ 
                    FEs: Billing Year-by-Month & Yes & Yes & Yes & Yes & Yes & Yes & Yes & Yes \\ 
                    Observations & 2,378,864 & 4,702,081 & 6,276,579 & 3,904,120 & 2,378,864 & 4,702,081 & 6,276,579 & 3,904,120 \\ 
                    Adjusted R$^{2}$ & 0.293 & 0.334 & 0.536 & 0.592 & 0.293 & 0.334 & 0.536 & 0.592 \\
                    \\[-2.0ex]
                    \hline \hline
                    \\[-4.5ex]
                \end{tabular}
                \begin{tablenotes}[flushleft]
                    \footnotesize
                    \item \textit{Note}: This table reports the results of robustness checks for different functional forms, specifically the third- and fourth-order polynomial models. Standard errors in parentheses are clustered at the household and billing year-by-month levels; * $p < 0.1$, ** $p < 0.05$, and *** $p < 0.01$.
                \end{tablenotes}
            \end{threeparttable}
        \end{adjustbox}
        
    \end{table}
}

\clearpage

\afterpage{
    \begin{table}[t!]
        \centering
        \caption{Robustness Checks: For Different Bandwidths}
        \label{Table:Robustness-Checks_BWs_Without-FEs}
        \vspace{0.3cm}
        \footnotesize
        \begin{adjustbox}{scale = 0.85}
            \begin{threeparttable}
                \begin{tabular}{@{\extracolsep{5pt}}lcccccccc}
                    \\[-5.5ex]
                    \hline \hline
                    \\[-3.0ex]
                     & \multicolumn{8}{c}{Dependent Variable} \\
                    \\[-3.0ex]
                    \cline{2-9}
                    \\[-3.0ex]
                     & \multicolumn{8}{c}{Average Daily Electricity Consumption  (kWh/Day)} \\
                    \\[-3.0ex]
                     & (1) & (2) & (3) & (4) & (5) & (6) & (7) & (8) \\
                    \\[-3.0ex]
                    \hline
                    \\[-2.0ex]
                    $\mathbb{1}$[Treatment] & $-$0.014 & $-$0.054$^{*}$ & $-$0.053$^{*}$ & $-$0.084$^{***}$ & $-$0.076$^{***}$ & $-$0.072$^{**}$ & $-$0.097$^{*}$ & $-$0.118$^{*}$ \\ 
                    & (0.032) & (0.030) & (0.028) & (0.028) & (0.029) & (0.031) & (0.056) & (0.064) \\ 
                    & & & & & & & & \\ 
                    NC0 & 0.169$^{***}$ & 0.197$^{***}$ & 0.202$^{***}$ & 0.204$^{***}$ & 0.204$^{***}$ & 0.199$^{***}$ & 0.214$^{***}$ & 0.211$^{***}$ \\ 
                    & (0.010) & (0.007) & (0.006) & (0.006) & (0.006) & (0.006) & (0.009) & (0.009) \\ 
                    & & & & & & & & \\ 
                    $\mathbb{1}$[Treatment] $\times$ NC0 & 0.038$^{***}$ & 0.001 & $-$0.010$^{***}$ & $-$0.008$^{***}$ & $-$0.009$^{***}$ & $-$0.014$^{***}$ & $-$0.018$^{***}$ & $-$0.017$^{***}$ \\ 
                    & (0.012) & (0.004) & (0.003) & (0.003) & (0.003) & (0.003) & (0.004) & (0.005) \\ 
                    & & & & & & & & \\ 
                    Average Daily CDDs & 0.749$^{***}$ & 0.753$^{***}$ & 0.755$^{***}$ & 0.757$^{***}$ & 0.758$^{***}$ & 0.767$^{***}$ & 0.932$^{***}$ & 1.143$^{***}$ \\ 
                    & (0.122) & (0.121) & (0.120) & (0.119) & (0.118) & (0.114) & (0.124) & (0.124) \\ 
                    & & & & & & & & \\ 
                    Average Daily HDDs & 0.280$^{***}$ & 0.281$^{***}$ & 0.282$^{***}$ & 0.284$^{***}$ & 0.286$^{***}$ & 0.152$^{**}$ & 0.637$^{***}$ & 1.033$^{***}$ \\ 
                    & (0.079) & (0.078) & (0.078) & (0.077) & (0.077) & (0.066) & (0.101) & (0.131) \\ 
                    & & & & & & & & \\ 
                    (Constant) & 19.947$^{***}$ & 19.973$^{***}$ & 19.972$^{***}$ & 19.965$^{***}$ & 19.937$^{***}$ & 19.720$^{***}$ & 17.769$^{***}$ & 15.117$^{***}$ \\ 
                    & (0.948) & (0.941) & (0.937) & (0.932) & (0.926) & (0.829) & (1.082) & (1.159) \\ 
                    & & & & & & & & \\
                    \hline
                    \\[-2.0ex]
                    Bandwidth & 5\% & 10\% & 15\% & 20\% & 25\% & 30\% & 35\% & 40\% \\ 
                    FEs: Billing Year-by-Month & No & No & No & No & No & No & No & No \\ 
                    Observations & 1,186,630 & 2,378,864 & 3,566,318 & 4,702,081 & 5,816,854 & 6,276,579 & 4,093,259 & 3,904,120 \\ 
                    Adjusted R$^{2}$ & 0.105 & 0.120 & 0.144 & 0.175 & 0.210 & 0.349 & 0.394 & 0.468 \\
                    \\[-2.0ex]
                    \hline \hline
                    \\[-4.5ex]
                \end{tabular}
                \begin{tablenotes}[flushleft]
                    \footnotesize
                    \item \textit{Note}: * $p < 0.1$, ** $p < 0.05$, and *** $p < 0.01$.
                \end{tablenotes}
            \end{threeparttable}
        \end{adjustbox}
    \end{table}
}

\clearpage

\afterpage{
    \begin{table}[t!]
        \centering
        \caption{Robustness Checks: For Different Bandwidths, Only RSGH Rate Code}
        \label{Table:Robustness-Checks_BWs_RSGH}
        \vspace{0.3cm}
        \small
        \begin{adjustbox}{scale = 0.72}
            \begin{threeparttable}
                \begin{tabular}{@{\extracolsep{5pt}}lcccccccc}
                    \\[-5.5ex]
                    \hline \hline
                    \\[-3.0ex]
                    & \multicolumn{8}{c}{Dependent Variable} \\
                    \\[-3.0ex]
                    \cline{2-9}
                    \\[-3.0ex]
                    & \multicolumn{8}{c}{Average Daily Electricity Consumption  (kWh/Day)} \\
                    \\[-3.0ex]
                    & (1) & (2) & (3) & (4) & (5) & (6) & (7) & (8) \\
                    \\[-3.0ex]
                    \hline
                    \\[-2.0ex]
                    $\mathbb{1}$[Treatment] & $-$0.055$^{***}$ & $-$0.060$^{***}$ & $-$0.055$^{***}$ & $-$0.065$^{***}$ & $-$0.064$^{***}$ & $-$0.058$^{***}$ & $-$0.068$^{***}$ & $-$0.080$^{***}$ \\ 
                    & (0.020) & (0.016) & (0.015) & (0.014) & (0.013) & (0.013) & (0.021) & (0.024) \\ 
                    & & & & & & & & \\ 
                    NC0 & 0.211$^{***}$ & 0.215$^{***}$ & 0.215$^{***}$ & 0.216$^{***}$ & 0.216$^{***}$ & 0.217$^{***}$ & 0.234$^{***}$ & 0.229$^{***}$ \\ 
                    & (0.008) & (0.006) & (0.005) & (0.005) & (0.005) & (0.006) & (0.008) & (0.010) \\ 
                    & & & & & & & & \\ 
                    $\mathbb{1}$[Treatment] $\times$ NC0 & $-$0.005 & $-$0.010$^{***}$ & $-$0.013$^{***}$ & $-$0.012$^{***}$ & $-$0.014$^{***}$ & $-$0.016$^{***}$ & $-$0.021$^{***}$ & $-$0.020$^{***}$ \\ 
                    & (0.006) & (0.003) & (0.002) & (0.001) & (0.002) & (0.002) & (0.003) & (0.004) \\ 
                    & & & & & & & & \\ 
                    Average Daily CDDs & 1.170$^{***}$ & 1.172$^{***}$ & 1.174$^{***}$ & 1.174$^{***}$ & 1.172$^{***}$ & 1.171$^{***}$ & 1.162$^{***}$ & 1.190$^{***}$ \\ 
                    & (0.106) & (0.108) & (0.108) & (0.107) & (0.107) & (0.106) & (0.114) & (0.126) \\ 
                    & & & & & & & & \\ 
                    Average Daily HDDs & 0.224$^{**}$ & 0.227$^{**}$ & 0.227$^{**}$ & 0.228$^{**}$ & 0.228$^{**}$ & 0.229$^{**}$ & 0.547$^{***}$ & 0.708$^{***}$ \\ 
                    & (0.090) & (0.090) & (0.090) & (0.089) & (0.088) & (0.087) & (0.133) & (0.186) \\ 
                    & & & & & & & & \\
                    \hline
                    \\[-2.0ex]
                    Rate Code & RSGH & RSGH & RSGH & RSGH & RSGH & RSGH & RSGH & RSGH \\ 
                    Bandwidth & 5\% & 10\% & 15\% & 20\% & 25\% & 30\% & 35\% & 40\% \\ 
                    FEs: Billing Year-by-Month & Yes & Yes & Yes & Yes & Yes & Yes & Yes & Yes \\ 
                    Observations & 967,546 & 1,941,332 & 2,909,164 & 3,832,683 & 4,738,070 & 5,604,830 & 3,396,312 & 3,191,411 \\ 
                    Adjusted R$^{2}$ & 0.475 & 0.486 & 0.503 & 0.524 & 0.547 & 0.571 & 0.576 & 0.613 \\
                    \\[-2.0ex]
                    \hline \hline
                    \\[-4.5ex]
                \end{tabular}
                \begin{tablenotes}[flushleft]
                    \footnotesize
                    \item \textit{Note}: This table shows the results of regressions with observations only for households selecting the RSGH rate plan. Standard errors in parentheses are clustered at the household and billing year-by-month levels to allow correlations across households in a given month; * $p < 0.1$, ** $p < 0.05$, and *** $p < 0.01$.
                \end{tablenotes}
            \end{threeparttable}
        \end{adjustbox}
    \end{table}
}

\clearpage



\clearpage
\section{For Chapter 2}
\subsection{Additional Figure(s) and Table(s)}
% 1. Figure(s)
%\afterpage{
    \begin{figure}[!ht]
        \centering
        \includegraphics[scale = 0.13]{03_Chapter-2/00A_Figures/Figure_Location-of-Weather-Stations.png}
        \caption{Weather Stations from which Historical Weather Data have been collected}
        \caption*{
            {\small
            \textit{Note}: This figure demonstrates the location of each weather station listed in Table \ref{Table:Correlations-in-Average-Daily-Temperatures-among-Weather-Stations}. As is evident from the map, the weather stations are distributed throughout Ireland.
        }}
        \label{Figure:Weather-Stations}
    \end{figure}
%}
\clearpage



% 2. Table(s)
\afterpage{
    \begin{table}[t!]
        \centering
        \caption{Correlations in Average Daily Temperatures between Weather Stations}
        \label{Table:Correlations-in-Average-Daily-Temperatures-among-Weather-Stations}
        \vspace{0.2cm}
        \begin{adjustbox}{scale = 1.0}
            \begin{tabular}{
                >{\raggedright}m{4.8cm} |
                >{\centering}m{5.5cm} 
                >{\centering\arraybackslash}m{5.5cm}
            }
                \hline \hline
                \multicolumn{1}{c|}{Stations} & \multicolumn{2}{c}{Correlation Coefficients} \\
                \cline{2-3}
                \multicolumn{1}{c|}{}  & \multicolumn{1}{c}{For Sample Period} & \multicolumn{1}{c}{For Experiment Period} \\
                \hline
                Ballyhaise & 0.98291 & 0.98244 \\
                Belmullet & 0.96089 & 0.96361 \\
                Cork Airport & 0.97121 & 0.97130 \\
                Gurteen & 0.98389 & 0.98307 \\
                Johnstown & 0.98189 & 0.97958 \\
                Mace & 0.95870 & 0.95921 \\
                Malin & 0.95632 & 0.95705 \\
                Markree Castle & 0.97194 & 0.97179 \\
                Moore Park & 0.98057 & 0.97798 \\
                Mount Dillon & 0.97945 & 0.97782 \\
                Mullingar & 0.98876 & 0.98654 \\
                Newport Furnace & 0.97015 & 0.97211 \\
                Oak Park & 0.99074 & 0.98925 \\
                Shannon Airport & 0.97696 & 0.97582 \\
                Sherkin Island & 0.95342 & 0.95411 \\
                \hline \hline
            \end{tabular}
        \end{adjustbox}
        \begin{tablenotes}[flushleft]
            \small
            \textit{Note}: For each weather station, historical weather data from the weather station at Dublin airport is utilized to compute the two correlation coefficients. I do not provide the $p$-value of each correlation coefficient because it is arbitrarly small in magnitude. And the experiment period is the period between July 2009 to December 2010, while the sample period is the second half of 2009 and 2010. 
        \end{tablenotes}
    \end{table}
}


\afterpage{
    \begin{landscape} 
    \begin{table}[t!]
        \centering
        \caption{Hourly Average Treatment Effects in and near the Peak Rate Period}
        \label{Table:Hourly-Average-Treatment-Effects-in-and-near-the-Peak-Rate-Period_For-Appendix}
        \vspace{0.3cm}
        \small
        \begin{adjustbox}{scale = 0.75}
            \begin{threeparttable}
                \begin{tabular}{@{\extracolsep{1pt}}lcccccccccccc}
                    \\[-4.0ex]
                    \hline \hline
                    \\[-3.0ex]
%                    & \multicolumn{15}{c}{Dependent Variable} \\
%                    \\[-3.0ex]
%                    \cline{2-16}
%                    \\[-3.0ex]
                    & \multicolumn{12}{c}{Hourly Electricity Consumption  (kWh/Hour)} \\
                    \\[-3.0ex]
                    & (1) & (2) & (3) & (4) & (5) & (6) & (7) & (8) & (9) & (10) & (11) & (12) \\
                    \\[-3.0ex]
                    \hline
                    \\[-2.0ex]
                    $\mathbb{1}$[Treatment \& Post] & $-$0.048$^{***}$ & $-$0.053$^{*}$ & $-$0.002 & $-$0.049 & $-$0.125$^{***}$ & $-$0.161$^{***}$ & $-$0.119$^{***}$ & $-$0.249$^{***}$ & $-$0.082$^{***}$ & $-$0.055$^{*}$ & $-$0.015 & $-$0.113$^{**}$ \\
                    & (0.016) & (0.027) & (0.017) & (0.031) & (0.020) & (0.036) & (0.022) & (0.044) & (0.020) & (0.030) & (0.021) & (0.048) \\
                    & & & & & & & & & & & & \\
                    \hline
                    \\[-2.0ex]
                    Description of Period & Pre-Peak & Pre-Peak & Pre-Peak & Pre-Peak & Peak & Peak & Peak & Peak & Post-Peak & Post-Peak & Post-Peak & Post-Peak \\
                    Period of Hours & 15 to 16 & 15 to 16 & 15 to 16 & 15 to 16 & 17 to 18 & 17 to 18 & 17 to 18 & 17 to 18 & 19 to 20 & 19 to 20 & 19 to 20 & 19 to 20 \\
                    Tariff Group & A & B & C & D & A & B & C & D & A & B & C & D \\
                    Price Change in the Peak Rate Period & +6 & +12 & +18 & +24 & +6 & +12 & +18 & +24 & +6 & +12 & +18 & +24 \\
                    FEs: Household by Half-Hourly Time Window & Yes & Yes & Yes & Yes & Yes & Yes & Yes & Yes & Yes & Yes & Yes & Yes \\
                    FEs: Day of Week by Half-Hourly Time Window & Yes & Yes & Yes & Yes & Yes & Yes & Yes & Yes & Yes & Yes & Yes & Yes \\
                    FEs: Month of Year & Yes & Yes & Yes & Yes & Yes & Yes & Yes & Yes & Yes & Yes & Yes & Yes \\
                    Observations & 506,540 & 326,800 & 511,700 & 331,960 & 506,540 & 326,800 & 511,700 & 331,960 & 506,540 & 326,800 & 511,700 & 331,960 \\
                    Adjusted R$^{2}$ & 0.312 & 0.330 & 0.320 & 0.327 & 0.384 & 0.397 & 0.383 & 0.367 & 0.371 & 0.389 & 0.376 & 0.361 \\
                    \\[-2.0ex]
                    \hline \hline
                    \\[-4.5ex]
                \end{tabular}
                \begin{tablenotes}[flushleft]
                    \footnotesize
                    \item \textit{Note}: * $p < 0.1$, ** $p < 0.05$, and *** $p < 0.01$.
                \end{tablenotes}
            \end{threeparttable}
        \end{adjustbox}
    \end{table} 
    \end{landscape}
}

\clearpage

%\afterpage{
    \begin{table}[t!]
        \centering
        \vspace{0.5cm}
        \footnotesize
        \begin{ThreePartTable}
            \renewcommand\TPTminimum{\textwidth}
    
            \begin{TableNotes}[flushleft]
                \footnotesize
                \item \textit{Note}: * $p < 0.1$, ** $p < 0.05$, and *** $p < 0.01$.
            \end{TableNotes}
            
            \begin{landscape}
            \begin{longtable}{@{\extracolsep{1.5pt}}lcccccccc}
                \caption{Breakdown of Hourly Average Treatment Effects}
                \label{Table:Breakdown-of-Hourly-ATEs_For-Appendix} \\

                \\[-4.0ex]
                \hline \hline
                \\[-3.0ex]
                & \multicolumn{8}{c}{Hourly Electricity Consumption  (kWh/Hour)} \\
                \\[-3.0ex]
                & (1) & (2) & (3) & (4) & (5) & (6) & (7) & (8) \\
                \\[-3.0ex]
                \hline
                \\[-2.0ex]
                \endfirsthead

                \multicolumn{9}{c}{{\bfseries \tablename \ \thetable{} -- continued from previous page}} \\

                \hline \hline
                \\[-3.0ex]
                & \multicolumn{8}{c}{Hourly Electricity Consumption  (kWh/Hour)} \\
                \\[-3.0ex]
                & (1) & (2) & (3) & (4) & (5) & (6) & (7) & (8) \\
                \\[-3.0ex]
                \hline
                \\[-2.0ex]
                \endhead

                \multicolumn{9}{r}{{\footnotesize{\textit{(Continued on next page...)}}}} \\
                \endfoot
                \insertTableNotes
                \endlastfoot

                HDDs & 0.016$^{***}$ & 0.016$^{***}$ & 0.016$^{***}$ & 0.015$^{***}$ & 0.047$^{***}$ & 0.047$^{***}$ & 0.047$^{***}$ & 0.047$^{***}$ \\
                & (0.004) & (0.004) & (0.004) & (0.004) & (0.004) & (0.004) & (0.004) & (0.004) \\
                & & & & & & & & \\
                HDDs$^{*}$ & 0.010 & 0.009 & 0.009 & 0.010 & $-$0.018$^{***}$ & $-$0.018$^{***}$ & $-$0.018$^{***}$ & $-$0.018$^{***}$ \\
                & (0.007) & (0.007) & (0.007) & (0.007) & (0.007) & (0.007) & (0.007) & (0.007) \\
                & & & & & & & & \\
                $\mathbb{1}$[Treatment] & 0.009 & 0.106 & $-$0.014 & 0.174$^{**}$ & 0.057 & 0.189$^{**}$ & $-$0.017 & 0.150$^{**}$ \\
                & (0.050) & (0.071) & (0.049) & (0.072) & (0.055) & (0.083) & (0.052) & (0.072) \\
                & & & & & & & & \\
                $\mathbb{1}$[Treatment] $\times$ HDDs & 0.001 & 0.004 & 0.001 & 0.0003 & 0.009$^{**}$ & 0.008 & 0.008$^{**}$ & 0.011 \\
                & (0.003) & (0.005) & (0.003) & (0.004) & (0.004) & (0.007) & (0.004) & (0.007) \\
                & & & & & & & & \\
                $\mathbb{1}$[Treatment] $\times$ HDDs$^{*}$ & $-$0.0001 & $-$0.012$^{*}$ & 0.0004 & 0.002 & $-$0.012$^{***}$ & $-$0.016$^{**}$ & $-$0.013$^{***}$ & $-$0.008 \\
                & (0.004) & (0.006) & (0.003) & (0.005) & (0.004) & (0.008) & (0.004) & (0.007) \\
                & & & & & & & & \\
                $\mathbb{1}$[Post] & 0.014 & 0.015 & 0.014 & 0.011 & 0.048 & 0.049 & 0.047 & 0.047 \\
                & (0.022) & (0.022) & (0.022) & (0.023) & (0.040) & (0.040) & (0.040) & (0.040) \\
                & & & & & & & & \\
                $\mathbb{1}$[Post] $\times$ HDDs & $-$0.007 & $-$0.007 & $-$0.007 & $-$0.006 & $-$0.015$^{**}$ & $-$0.015$^{**}$ & $-$0.015$^{**}$ & $-$0.015$^{**}$ \\
                & (0.005) & (0.005) & (0.005) & (0.005) & (0.006) & (0.006) & (0.006) & (0.006) \\
                & & & & & & & & \\
                $\mathbb{1}$[Post] $\times$ HDDs$^{*}$ & 0.002 & 0.002 & 0.002 & 0.001 & 0.006 & 0.007 & 0.006 & 0.006 \\
                & (0.008) & (0.008) & (0.008) & (0.008) & (0.009) & (0.009) & (0.009) & (0.009) \\
                & & & & & & & & \\
                $\mathbb{1}$[Treatment \& Post] & $-$0.038 & $-$0.039 & 0.020 & $-$0.039 & $-$0.040 & $-$0.050 & 0.006 & $-$0.025 \\
                & (0.023) & (0.030) & (0.026) & (0.038) & (0.029) & (0.037) & (0.027) & (0.040) \\
                & & & & & & & & \\
                $\mathbb{1}$[Treatment \& Post] $\times$ HDDs & 0.001 & $-$0.003 & $-$0.0002 & 0.001 & $-$0.003 & 0.0005 & 0.0003 & $-$0.009 \\
                & (0.003) & (0.004) & (0.003) & (0.004) & (0.003) & (0.005) & (0.003) & (0.006) \\
                & & & & & & & & \\
                $\mathbb{1}$[Treatment \& Post] $\times$ HDDs$^{*}$ & $-$0.002 & 0.008 & $-$0.004 & $-$0.005 & 0.005 & 0.010 & 0.004 & 0.008 \\
                & (0.004) & (0.006) & (0.004) & (0.006) & (0.004) & (0.007) & (0.003) & (0.006) \\
                & & & \\
                \hline
                \\[-2.0ex]
                Description of Period & Pre-Peak & Pre-Peak & Pre-Peak & Pre-Peak & Post-Peak & Post-Peak & Post-Peak & Post-Peak \\
                Period of Hours & 15 to 16 & 15 to 16 & 15 to 16 & 15 to 16 & 19 to 20 & 19 to 20 & 19 to 20 & 19 to 20 \\
                Tariff Group & A & B & C & D & A & B & C & D \\
                Price Change in the Peak Rate Period & +6 & +12 & +18 & +24 & +6 & +12 & +18 & +24 \\
                Knot & 10 & 10 & 10 & 10 & 10 & 10 & 10 & 10 \\
                FEs: Day of Week by Half-Hourly Time Window & Yes & Yes & Yes & Yes & Yes & Yes & Yes & Yes \\
                Observations & 506,540 & 326,800 & 511,700 & 331,960 & 506,540 & 326,800 & 511,700 & 331,960 \\
                Adjusted R$^{2}$ & 0.024 & 0.024 & 0.023 & 0.025 & 0.041 & 0.040 & 0.039 & 0.043 \\
                \\[-2.0ex]
                \hline \hline
                \\[-4.5ex]

            \end{longtable}
            \end{landscape}
            
        \end{ThreePartTable}
    \end{table}
%}

\clearpage

% \afterpage{
    \begin{ThreePartTable}
        \centering
        \scriptsize
        \vspace{0.3cm}
        \renewcommand\TPTminimum{\textwidth}

        \begin{TableNotes}[flushleft]
            \footnotesize
%            \item
            \item \textit{Note}: * $p < 0.1$, ** $p < 0.05$, and *** $p < 0.01$.
        \end{TableNotes}

        \begin{longtable}{@{\extracolsep{0pt}}lcccccc}
            \caption{Hourly Treatment Effects as a Linear Function of Peak-Rate-Period Price Changes}
            \label{Table:Hourly-ATEs-as-a-Linear-Function-of-Peak-Rate-Period-Price-Changes_For-Appendix} \\

            \\[-4.0ex]
            \hline \hline
            \\[-3.0ex]
            & \multicolumn{6}{c}{Hourly Electricity Consumption  (kWh/Hour)} \\
            \\[-3.0ex] 
            & (1) & (2) & (3) & (4) & (5) & (6) \\ 
            \\[-3.0ex]
            \hline
            \\[-2.0ex]
            \endfirsthead

            \multicolumn{7}{c}{{\bfseries \tablename \ \thetable{} -- continued from previous page}} \\

            \hline \hline
            \\[-3.0ex]
            & \multicolumn{6}{c}{Hourly Electricity Consumption  (kWh/Hour)} \\
            \\[-3.0ex]
            & (1) & (2) & (3) & (4) & (5) & (6) \\ 
            \\[-3.0ex]
            \hline
            \\[-2.0ex]
            \endhead

            \multicolumn{7}{r}{{\footnotesize{\textit{(Continued on next page...)}}}} \\
            \endfoot
            \insertTableNotes
            \endlastfoot

            HDDs & 0.016$^{***}$ & 0.042$^{***}$ & 0.047$^{***}$ & 0.016$^{***}$ & 0.042$^{***}$ & 0.047$^{***}$ \\ 
            & (0.004) & (0.006) & (0.004) & (0.004) & (0.006) & (0.005) \\ 
            & & & & & & \\ 
            HDDs$^{*}$ & 0.010 & 0.001 & $-$0.018$^{***}$ & 0.010 & 0.001 & $-$0.018$^{**}$ \\ 
            & (0.007) & (0.010) & (0.007) & (0.007) & (0.010) & (0.007) \\ 
            & & & & & & \\ 
            $\mathbb{1}$[Treatment] & $-$0.020 & $-$0.018 & 0.064 &  &  &  \\ 
            & (0.059) & (0.073) & (0.065) &  &  &  \\ 
            & & & & & & \\ 
            $\mathbb{1}$[Treatment] $\times$ $\Delta$PC & 0.004 & 0.005 & $-$0.0003 &  &  &  \\ 
            & (0.003) & (0.004) & (0.003) &  &  &  \\ 
            & & & & & & \\ 
            $\mathbb{1}$[Treatment] $\times$ HDDs & 0.001 & 0.013$^{**}$ & 0.009 & 0.001 & 0.013$^{**}$ & 0.009 \\ 
            & (0.004) & (0.005) & (0.005) & (0.005) & (0.006) & (0.006) \\ 
            & & & & & & \\ 
            $\mathbb{1}$[Treatment] $\times$ HDDs$^{*}$ & $-$0.003 & $-$0.011$^{*}$ & $-$0.014$^{***}$ & $-$0.003 & $-$0.011 & $-$0.014$^{**}$ \\ 
            & (0.005) & (0.006) & (0.005) & (0.006) & (0.007) & (0.007) \\ 
            & & & & & & \\ 
            $\mathbb{1}$[Treatment] $\times$ HDDs $\times$ $\Delta$PC & $-$0.00001 & $-$0.0004 & 0.00003 & $-$0.00001 & $-$0.0004 & 0.00003 \\ 
            & (0.0002) & (0.0003) & (0.0003) & (0.0002) & (0.0003) & (0.0003) \\ 
            & & & & & & \\ 
            $\mathbb{1}$[Treatment] $\times$ HDDs$^{*}$ $\times$ $\Delta$PC & 0.0001 & 0.0003 & 0.0001 & 0.0001 & 0.0003 & 0.0001 \\ 
            & (0.0003) & (0.0003) & (0.0003) & (0.0003) & (0.0004) & (0.0003) \\ 
            & & & & & & \\ 
            $\mathbb{1}$[Post] & 0.013 & 0.045 & 0.047 & 0.013 & 0.045 & 0.047 \\ 
            & (0.022) & (0.036) & (0.040) & (0.024) & (0.038) & (0.042) \\ 
            & & & & & & \\ 
            $\mathbb{1}$[Post] $\times$ HDDs & $-$0.007 & $-$0.015$^{*}$ & $-$0.015$^{**}$ & $-$0.007 & $-$0.015$^{*}$ & $-$0.015$^{**}$ \\ 
            & (0.005) & (0.008) & (0.006) & (0.005) & (0.008) & (0.006) \\ 
            & & & & & & \\ 
            $\mathbb{1}$[Post] $\times$ HDDs$^{*}$ & 0.002 & 0.007 & 0.006 & 0.002 & 0.007 & 0.006 \\ 
            & (0.008) & (0.013) & (0.009) & (0.008) & (0.014) & (0.010) \\ 
            & & & & & & \\ 
            $\mathbb{1}$[Treatment \& Post] & $-$0.045 & $-$0.028 & $-$0.053 & $-$0.045 & $-$0.028 & $-$0.053 \\ 
            & (0.029) & (0.035) & (0.035) & (0.032) & (0.039) & (0.038) \\ 
            & & & & & & \\ 
            & & & & & & \\ 
            & & & & & & \\ 
            $\mathbb{1}$[Treatment \& Post] $\times$ $\Delta$PC & 0.002 & $-$0.005$^{**}$ & 0.002 & 0.002 & $-$0.005$^{**}$ & 0.002 \\ 
            & (0.002) & (0.002) & (0.002) & (0.002) & (0.002) & (0.002) \\ 
            & & & & & & \\ 
            $\mathbb{1}$[Treatment \& Post] $\times$ HDDs & $-$0.0001 & $-$0.010$^{**}$ & $-$0.001 & $-$0.0001 & $-$0.010$^{*}$ & $-$0.001 \\ 
            & (0.004) & (0.004) & (0.004) & (0.005) & (0.006) & (0.005) \\ 
            & & & & & & \\ 
            $\mathbb{1}$[Treatment \& Post] $\times$ HDDs$^{*}$ & 0.001 & 0.012$^{**}$ & 0.005 & 0.001 & 0.012 & 0.005 \\ 
            & (0.005) & (0.006) & (0.005) & (0.007) & (0.008) & (0.007) \\ 
            & & & & & & \\ 
            $\mathbb{1}$[Treatment \& Post] $\times$ HDDs $\times$ $\Delta$PC & 0.00001 & 0.0002 & $-$0.0001 & 0.00001 & 0.0002 & $-$0.0001 \\ 
            & (0.0002) & (0.0002) & (0.0003) & (0.0002) & (0.0003) & (0.0003) \\ 
            & & & & & & \\ 
            $\mathbb{1}$[Treatment \& Post] $\times$ HDDs$^{*}$ $\times$ $\Delta$PC & $-$0.0002 & $-$0.0003 & 0.00004 & $-$0.0002 & $-$0.0003 & 0.00004 \\ 
            & (0.0003) & (0.0003) & (0.0003) & (0.0004) & (0.0004) & (0.0004) \\ 
            & & & & & & \\
            \hline
            \\[-2.0ex]
            Description of Period & Pre-Peak & Peak & Post-Peak & Pre-Peak & Peak & Post-Peak \\ 
            Period of Hours & 15 to 16 & 17 to 18 & 19 to 20 & 15 to 16 & 17 to 18 & 19 to 20 \\ 
            Knot & 10 & 10 & 10 & 10 & 10 & 10 \\ 
            FEs: Household & No & No & No & Yes & Yes & Yes \\ 
            FEs: Day of Week by Half-Hourly Time Window & Yes & Yes & Yes & Yes & Yes & Yes \\ 
            Observations & 1,006,200 & 1,006,200 & 1,006,200 & 1,006,200 & 1,006,200 & 1,006,200 \\ 
            Adjusted R$^{2}$ & 0.024 & 0.047 & 0.040 & 0.288 & 0.343 & 0.356 \\
            \\[-2.0ex]
            \hline \hline
            \\[-4.5ex]

        \end{longtable}
    \end{ThreePartTable}
% }

\clearpage



\section{For Chapter 3}
(To be added ...)



% ------- Bibliography -------
\bibliographystyle{05_Bibliography/aea}
\bibliography{05_Bibliography/Bibliography}


\end{document}
