From SMUD, I obtain household-level monthly billing history of residential consumers in the Sacramento area from 2004 to 2013. For each monthly record, account ID, premise ID, rate code, billing start and end dates, monthly consumption with its breakdown into each tier, monthly fixed charge, monthly variable charge only for kWh usage, total monthly bill, and an indicator related to solar adoption are included in the data. Of note, in my empirical analysis, I assume that a pair of account and premise IDs corresponds to an individual household. And because the monthly billing data contain no price and base usage quantity information, I append historical price schedules and base usage quantities presented by SMUD. Unfortunately, my monthly billing data also lack any socioeconomic and demographic information. 

First, I focus on households that consistently used one of the three major residential rates (i.e., RSGH, RSCH, and RSEH) in their billing history. Second, I only utilize billing records before 2010, from which SMUD got down to install smart meters. Third, I focus on households whose number of billing periods is greater than or equal to 24. Fourth, I focus on SMUD residential customers with reliable billing records only.\footnote{To be specific, I exclude, from the sample used for my empirical analysis, households that have billing records being applied to any of the following conditions: 1) observations whose length of the billing period is either less than 27 or greater than 34; 2) observations with negative values for either quantities or charges; 3) observations having overlapping billing periods within a pair of account and premise IDs; and 4) observations whose number of days from the previous billing period is greater than 14.} Fifth, my sample only includes households that crossed the lower threshold at least once in their billing history.\footnote{In other words, I drop always-light- and always-heavy-users from my sample. } The procedure results in 16,322,353 billing records for 365,975 households. Table \ref{Table:Summary-Statistics} provides summary statistics for my sample. Furthermore, Figure \ref{Figure:Household-Average-Daily-Electricity-Consumption-by-Month-of-Year} shows, for each rate code, households' average daily electricity consumption by month of the year. 

To take account of the impact of weather conditions, especially outdoor temperatures, on household electricity consumption, I utilize the Local Climatological Data (LCD) for the Sacramento International Airport, published by the National Oceanic and Atmospheric Administration (NOAA). Using daily heating degree days (HDDs) and daily cooling degree days (CDDs) in the LCD between 2004 and 2009, I calculate each monthly billing period's accumulated HDDs and CDDs, which are used to compute the average daily HDDs and CDDs. 
