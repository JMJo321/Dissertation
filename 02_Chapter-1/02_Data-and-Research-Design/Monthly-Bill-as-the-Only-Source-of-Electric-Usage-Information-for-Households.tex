Before 2009, there was no feasible way for SMUD residential customers to access real-time information related to their electricity use. SMUD initiated installations of smart meters, allowing its residential and business customers to view their electricity usage online when they want, in late 2009. The electric service completed it in the first quarter of 2012. Also, the three types of bill alert SMUD are offering were introduced in 2017.\footnote{SMUD provides its customers with three types of bill alerts, via text or e-mail, as a billing service: 1) Mid-Bill Alerts send an alert on the 16th day of a customer's billing period and advise what his usage has been and what the cost is as of that day, 2) High Bill Alters compare a customer's current billing cycle to the same billing cycle in the previous year and alerts the customer if their current usage is running higher than before, and 3) Bill Threshold allows a customer to know when his bill has reached a certain amount set in advance by himself.} Therefore, for households using SMUD-delivering electricity, the only practical source of information about their electricity consumption had been their monthly bill statements, which send out (either e-mail or U.S. mail) after 3 or 4 business days from the last day of each billing cycle. 

The issue of households' welfare losses due to their response to discontinuous changes in the lagged marginal price suggests the importance of providing seemly price information in an appropriate manner. Many studies about various time-varying electricity pricing show that households changed their consumption behavior in response to the information about consumption and prices \citep{Dynamic-Pricing-of-Electricity-in-the-Mid-Atlantic-Region_Econometric-Results-from-the-Baltimore-Gas-and-Electric-Company-Experiment_Faruqui-et-al_2011, Knowledge-is-Less-Power_Jessoe-and-Rapson_2014, The-Effect-of-Information-on-TOU-Electricity-Use:An-Irish-Residential-Study_Pon_2017, Information-vs-Automation-and-Implications-for-Dynamic-Pricing_Bollinger-and-Hartmann_2020}. My empirical finding demonstrates that even under IBP, such information, though lagged, still plays a role in household electricity consumption. In this respect, providing household-specific as well as current price information for residential consumers, via text messages or app notifications regularly, could encourage them to respond to \textit{true} price signals rather than lagged ones, which in turn avoid the negative impact on household welfare. Based on the dissipating effect of intermittently salient information discussed in \cite{Dynamic-Salience-with-Intermittent-Billing_Gilbert-and-Zivin_2014}, a high frequency of informing the latest tailored price information might maximize households' behavior change in electricity consumption. In addition, because sending such information-bearing notifications is available at a very low cost these days, this type of information provision would be a practical policy instrument for utilities, especially in developing countries where the transition toward dynamic electricity pricing is difficult due to substantial investments in installing smart metering systems. 
