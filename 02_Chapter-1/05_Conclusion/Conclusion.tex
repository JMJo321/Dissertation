In this paper, I examine how the electricity consumption of residential customers of SMUD responded to the marginal price informed through monthly bill statements under Increasing-Block Pricing (IBP). In a setting with a valid regression discontinuity design, my empirical analysis shows that households reduced their electricity consumption in response to the discontinuous price change in the marginal price in the immediately preceding billing cycle. In other words, the empirical results of my analysis reveal an inefficiency of IBP. At the same time, the interesting response demonstrates the potential to induce desired behavioral changes in household electricity consumption by providing appropriate price information regularly. On top of that, my RD results also indicate that the unit price of the highest block should be high enough to achieve the intended energy conservation by heavy electricity users. In conclusion, my research suggests implications for how IBP can be used more effectively without any welfare losses, especially in the case that transitioning toward time-varying electricity pricing, which is more efficient, is not feasible for some reason. 