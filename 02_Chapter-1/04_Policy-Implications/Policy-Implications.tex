Utilizing the Regression Discontinuity (RD) design described in Section \ref{C1-Sub-Sub-Section:Regression-Discontinuity-Design}, I show that under Increasing-Block Pricing (IBP), household electricity consumption responded to the lagged marginal price informed through their monthly bill statements. Between 2004 and 2010, SMUD residential customers had no feasible way to know, in real time, how much electricity they had consumed since the beginning of a billing cycle, how much they paid for the marginal unit, and so on. In such a situation, their behavior to exploit available information---, for example, the information contained in their monthly bills delivered after several days from the last day of each billing cycle---as much as possible in their decisions for electricity consumption seems reasonable. Nevertheless, it is undeniable that their electricity consumption responding to the lagged marginal price is suboptimal. As discussed, the marginal price household responded to is not for the marginal unit in the current billing period but for the last unit in the previous one. Responding to \textit{wrong} price signals, SMUD residential customers reduced their electricity consumption. In other words, the \textit{informed} consumption decisions made by households, based on the lagged marginal price, caused welfare losses to them. 

The issue of households' welfare losses due to their response to discontinuous changes in the lagged marginal price suggests the importance of providing seemly price information in an appropriate manner. Many studies about various time-varying electricity pricing show that households changed their consumption behavior in response to the information about consumption and prices \citep{Dynamic-Pricing-of-Electricity-in-the-Mid-Atlantic-Region_Econometric-Results-from-the-Baltimore-Gas-and-Electric-Company-Experiment_Faruqui-et-al_2011, Knowledge-is-Less-Power_Jessoe-and-Rapson_2014, The-Effect-of-Information-on-TOU-Electricity-Use:An-Irish-Residential-Study_Pon_2017, Information-vs-Automation-and-Implications-for-Dynamic-Pricing_Bollinger-and-Hartmann_2020}. My empirical finding demonstrates that even under IBP, such information, though lagged, still plays a role in household electricity consumption. In this respect, providing household-specific as well as current price information for residential consumers, via text messages or app notifications, could encourage them to respond to \textit{true} price signals rather than lagged ones, which in turn avoid the negative impact on household welfare. Based on the dissipating effect of intermittently salient information discussed in \cite{Dynamic-Salience-with-Intermittent-Billing_Gilbert-and-Zivin_2014}, a high frequency of informing the latest tailored price information might maximize households' behavior change in electricity consumption. In addition, because sending such information-bearing notifications is available at a very low cost these days, this type of information provision would be a practical policy instrument for utilities, especially in developing countries where the transition toward dynamic electricity pricing is difficult due to substantial investments in installing smart metering systems. 

No response to the lagged marginal price at the higher base usage quantity presented in Section \ref{C1-SubSubSection:Heterogeneity-in-Household-Response-to-the-Lagged-Marginal-Prices} suggests an implication for setting a price for each block in IBP. As discussed, the result might be attributable to the relatively small price increase at the cutoff point. This explanation is likely to parallel with the discussion, in \cite{Does-Marginal-Price-Matter?-A-Regression-Discontinuity-Approach-To-Estimating-Water-Demand_Nataraj-and-Hanemann_2011}, that regarding water demand management, imposing a higher marginal price seems required to deter consumption by heavy users. In light of this paper and my RD estimates at the higher threshold, setting a high enough price for the highest block in IBP seems necessary to curb demand from heavy electricity consumers, which appears to contribute to a higher spot price in the wholesale electricity market. 
