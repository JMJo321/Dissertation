Utilizing the Regression Discontinuity (RD) design described in Section \ref{C1-Sub-Sub-Section:Regression-Discontinuity-Design}, I show that under Increasing-Block Pricing (IBP), residential electricity consumption responded to the lagged marginal prices, informed by their monthly bill statements. Before 2010, SMUD residential customers had no feasible way to know, in real time, how much electricity they had consumed since the beginning of a billing cycle, how much they paid for the marginal unit, and so on. In such a situation, it seems reasonable for them to use the available information---for example, the information contained in their monthly bills, which are delivered after several days after the last day of each billing cycle---as much as possible to make decisions about their electricity use. 

Nevertheless, it is undeniable that their electricity consumption responding to the lagged marginal price is suboptimal. As discussed, the marginal price to which households responded is not for the marginal unit in the current billing period but for the last unit in the previous one. In response to \textit{wrong} price signals, SMUD residential customers reduced their electricity consumption. Moreover, they applied the lagged price signals to all units of electricity consumption in a billing month, not just the last unit in a billing month. In other words, the \textit{informed} consumption decisions made by households, based on the lagged marginal prices, caused them welfare losses. 

The issue of household welfare losses due to their response to the discontinuous change in the lagged marginal prices suggests the importance of providing seemly price information in an appropriate manner. Many studies about various time-varying electricity pricing show that households changed their consumption behavior in response to the information about consumption and prices \citep{Dynamic-Pricing-of-Electricity-in-the-Mid-Atlantic-Region_Econometric-Results-from-the-Baltimore-Gas-and-Electric-Company-Experiment_Faruqui-et-al_2011, Knowledge-is-Less-Power_Jessoe-and-Rapson_2014, The-Effect-of-Information-on-TOU-Electricity-Use:An-Irish-Residential-Study_Pon_2017, Information-vs-Automation-and-Implications-for-Dynamic-Pricing_Bollinger-and-Hartmann_2020}. My empirical finding demonstrates that even under IBP, such information, though lagged, still plays a role in household electricity consumption. In this respect, providing household-specific as well as current price information to residential consumers, via text messages or app notifications, could encourage them to respond to \textit{true} price signals rather than lagged ones, which in turn avoids the negative impact on household welfare. Based on the dissipating effect of intermittently salient information discussed in \cite{Dynamic-Salience-with-Intermittent-Billing_Gilbert-and-Zivin_2014}, a high frequency of advising the latest tailored price information might maximize the behavioral change of households in electricity consumption.

My empirical results also suggest the significance of ensuring people understand the information provided correctly. Using an up-to-date machine learning technique, \cite{Peaking-Interest:How-Awareness-Drives-the-Effectiveness-of-Time-of-Use-Electricity-Pricing_Prest_2020} shows that the most critical driver of households' response to Time-Of-Use electricity pricing is their awareness of intraday price changes under the dynamic pricing. Given the empirical evidence suggested in \cite{Misunderstanding-Nonlinear-Prices_2020_(Shaffer)}, a subset of households in my sample may be driving the identified response to the lagged marginal prices.\footnote{For the households showing no response to the lagged price signals, there are two possible explanations: 1) they interpreted the signals correctly; or 2) they did not pay attention to it.} If this is the case, the household group perceived the variation in the lagged marginal prices informed by their monthly billing statements. However, their identified response to the lagged price signals clearly indicates that they misunderstood it as price information for right now. Herefore, in my empirical context, the result of my empirical analysis could suggest that it is crucial not only to get residential consumers to be aware of price changes, but also to get them to interpret the price signals correctly. 
