Sacramento Municipal Utility District (SMUD), which is the nation's sixth-largest community-owned electric utility, provides electricity to most of Sacramento County and small portions of adjoining Placer and Yolo Counties.\footnote{According to the company information presented on \href{https://www.smud.org/en/Corporate/About-us/Company-Information}{SMUD's website}, the size of this utility's service area is 900 square miles.} As of December 31, 2020, the total number of accounts served by SMUD is 644,723.

Residential rates chosen by most of SMUD residential customers have an increasing nonlinear block-tier structure.\footnote{Although SMUD offers the Time-of-Day (TOD) rate, only 5\% of residential customers adopt that rate.} The three most popular rates for SMUD residential customers are Standard General Service (RSGH), Standard Closed Electric-Heated Service (RCEH), and Standard Open Electric-Heated Service (RSEH).\footnote{Specifically, more than 75\% of SMUD residential customers in my dataset choose the RSGH rate, whereas 2\% and 20\% of households in my dataset adopt the RSCH and RSEH rate, respectively.} For those residential rates, the marginal price of the energy charge is a step function of monthly consumption relative to a base usage quantity per month, which varies seasonally. Figure \ref{Figure:SMUD-Residential-Rates_Variable-Charge-and-Base-Usage} illustrates variations in price and base usage quantity over time. Two points are noteworthy from this figure: first, both tier rates and base usage quantities of the energy charges show substantial seasonality; second, the structure of the residential rates has changed from three-tier to two-tier since September 2009.

In addition to the variable charge (i.e., the energy charge), households choosing one of the three rates should pay a per-month fixed charge, called the System Infrastructure Fixed Charge. As shown in Figure \ref{Figure:SMUD-Residential-Rates_Fixed-Charge}, the unit price of the fixed charge significantly increased between 2009 and 2014. 

