Before 2009, there was no feasible way for SMUD residential customers to access real-time information related to their electricity use. SMUD initiated installations of smart meters, allowing its residential and business customers to view their electricity usage online when they want, in late 2009. The electric service completed it in the first quarter of 2012. Also, the three types of bill alert SMUD are offering were introduced in 2017.\footnote{SMUD provides its customers with three types of bill alert, via text or e-mail, as a billing service: 1) Mid Bill Alerts send an alert on the 16th day of a customer's billing period and advise what his usage has been and what the cost is as of that day, 2) High Bill Alters compare a customer's current billing cycle to the same billing cycle in the previous year and alerts the customer if the usage is running higher than before, and 3) Bill Threshold allows a customer to know when his bill has reached a certain amount set in advance by himself.} Therefore, for households using SMUD-delivering electricity, the only practical source of information about their electricity consumption had been their monthly bill statements, which send out (either e-mail or U.S. mail) after 3 or 4 business days from the last day of each billing cycle. 

Based on the discussion above, it is obvious that the pieces of information directly observable or inferable with some computation from the monthly bill for the previous billing cycle were the most recent ones relevant to electricity consumption. Naturally, it is likely that before late 2009, households' decisions about electricity consumption in a billing period were made, at least partly, based on their consumption history in the preceding billing period. \cite{Dynamic-Salience-with-Intermittent-Billing_Gilbert-and-Zivin_2014} corroborates the evidence of this reasoning. To be specific, the paper demonstrates that households reduce their electricity consumption following receipt of a bill, which includes lagged, rather than real-time, information associated with their electricity consumption. Moreover, the paper's authors frame the response as residential consumers' dynamic behavior under intermittent expenditure signals. Even though the aim of the study is not to reveal the underlying mechanisms of information-signal-induced behavioral changes but rather to better understand them, such behavioral changes seem to support that when making consumption decisions, households utilize certain information from their electricity bills, which could allow them to gauge their consumption status in an ongoing billing period roughly.