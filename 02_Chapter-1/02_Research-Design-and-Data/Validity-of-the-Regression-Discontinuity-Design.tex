Two pieces of evidence support the assumption that base usage quantities do not correspond to jumps in household characteristics. First, as illustrated in Figure \ref{Figure:SMUD-Billing-Data_Histogram_By-Season-and-Rate-Code}, each density plot of the running variable is very smooth, without any bump (i.e., excess mass), around base usage quantities at which marginal prices jump. The set of density plots that show apparent continuity at the thresholds suggests households' inability to precisely adjust their electricity consumption in order not to be subject to a higher marginal price. 

Second, Figure \ref{Figure:Average-Daily-Electricity-Consumption-over-Normalized-Electricity-Consumption} demonstrates that households' average daily electricity consumption during Period 0 evolved smoothly around the cutoff points. This figure allows me, at a minimum, not to reject the assumption of local randomization around the base usage quantities, even though examining an observed covariate around the thresholds is not also a direct test for the validity of the assumption. 
