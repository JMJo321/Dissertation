Table \ref{Table:RD-Results} summarizes the regression results of several alternate specifications for the bandwidth of 10\%. Column (1) reports estimates from the most straightforward RD specification without any control and FEs. Column (2) adds controls for households' cooling and heating needs, significantly driving household electricity consumption. In addition to those two controls, column (3) uses billing month-by-year FEs. Adding the FEs attenuates the estimate of interest. Moreover, the standard errors of the estimated treatment effect are substantially smaller, suggesting that controlling for time-varying factors is important.\footnote{Table \ref{Table:Robustness-Checks_BWs_Without-FEs} shows the RD estimates of the treatment effect from specifications without the billing month-by-year FEs. From this table, it is convincing that including the FEs is necessary to reduce sampling variance.} In this specification, the estimated treatment effect indicates a discontinuous reduction in households' electricity demand by 0.040 kWh, which amounts to 0.2\% of their average daily electricity consumption. This estimate is statistically different from 0 at the 5\% level. Columns from (4) to (6) additionally include the interaction term between the binary indicator and the running variable. Adding the interaction term to the specifications has only minimal impact on estimates. 

The identified reduction in household electricity consumption convincingly demonstrates that households respond to lagged marginal prices. As discussed in Section \ref{C1-Sub-Sub-Section:Regression-Discontinuity-Design}, the discontinuous increase in the marginal price at the lower base usage quantity was not followed by any change in the average price. Moreover, the households in my sample were able to notice the price jump only through their monthly bill statements, which were delivered a few days after the first day of the new billing month. Collectively, my estimates reveal an inefficiency stemming from households' responses to nonlinear electricity pricing because the lagged marginal price reflects their consumption history, not their contemporaneous consumption. In other words, under IBP, the untimely price signal drives, at least partly, households' electricity consumption. 

Importantly, the estimated discontinuous decrease in residential electricity consumption also suggests that SMUD residential customers overreacted to the lagged marginal price under nonlinear electricity pricing. The discontinuous change in the marginal price at the lower base usage quantity occurred in a billing cycle (i.e., in Period 0). And my estimates show that in the following billing cycle (i.e., in Period 1), the customers reduced their electricity consumption as a response to the price variation. Consequently, the sharp increase in the marginal price at the cutoff point in Period 0 affected all consumption, not the marginal one, in Period 1. That is, households excessively applied the lagged marginal price to every unit of electricity consumption during a billing month. 

Inspired by \cite{Misunderstanding-Nonlinear-Prices_2020_(Shaffer)}, the estimates could be interpreted differently. The paper finds that a subgroup of less than 10\% of households, which applies the marginal price to all consumption, was driving the seemingly overall response, in which the primary response to the average price is masked. If this is also true in my setting, then the measured decrease in household electricity consumption would be attributed to a subset of my sample. Suppose that there are two distinct types of SMUD residential customers: households over-responding to the lagged marginal price and those not responding to it.\footnote{Here, I do not consider the type of households that respond to the average price because the change in the marginal price at the threshold does not accompany any change in the average price.} My back-of-the-envelope calculation suggests that about 4\% of over-responders produce the estimated treatment effect.\footnote{In my rough computations, I exploit the price elasticities provided in \cite{Do-Consumers-Respond-to-Marginal-or-Average-Price?-Evidence-from-Nonlinear-Electricity-Pricing_2014_(Ito)} as well as \cite{SmartPricing-Options-Final-Evaluation_SMUD_2014}.} Interestingly, this calculation parallels the finding in the paper. Figure \ref{} visualizes how the share of over-responding households varies with a range of price elasticities of residential electricity consumption for a given RD estimate. 
