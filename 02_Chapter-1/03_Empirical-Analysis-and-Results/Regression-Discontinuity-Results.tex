Table \ref{Table:RD-Results} summarizes the regression results of several alternate specifications for the bandwidth of 20\%. Column (1) reports estimates from the simplest RD specification without any control and FEs. Column (2) adds controls for households' cooling and heating needs, significantly driving household electricity consumption. In addition to those two controls, columns (3) and (4) use household and billing-year-by-billing-month FEs, respectively. Here, the estimated treatment effects are substantially smaller, suggesting that controlling for both household-specific and time-varying factors is important. Column (5) shows results from the specification with the controls and both FEs. The estimated treatment effect is a discontinuous reduction in households' average daily electricity consumption by 0.024 kWh (i.e., about 0.1\% of the average daily consumption). This estimate is statistically different from 0 at the 5\% level. Columns from (6) to (10) additionally include the interaction term between the binary indicator and running variables. Adding the interaction terms to the specifications has only minimal impact on the estimates.  

The identified reduction in household electricity consumption clearly demonstrates that households respond to lagged marginal prices. As discussed in Section \ref{Sub-Sub-Section:Regression-Discontinuity-Design}, the discontinuous increase in the marginal price at the lower base usage quantity was not followed by any change in the average price. Moreover, the households in my sample were able to notice the price jump only through their monthly bill statements, which were delivered a few days after the first day of the new billing month. Collectively, my estimates reveal an inefficiency stemming from households' responses to nonlinear electricity pricing because the lagged marginal price reflects their consumption history, not their contemporaneous consumption. In other words, under IBP, the untimely price signal drives households' electricity consumption. 

Importantly, the estimated treatment effect also indicates that SMUD residential customers overreacted to the lagged marginal price under nonlinear electricity pricing. The discontinuous change in the marginal price at the lower base usage quantity occurred in a billing cycle (i.e., in Period 0). And my estimates show that in the following billing cycle (i.e., in Period 1), the customers reduced their electricity consumption as a response to the price variation. Consequently, the sharp increase in the marginal price at the cutoff point in Period 0 affected all consumption, not the marginal one, in Period 1. That is, households excessively applied the lagged marginal price to every unit of electricity consumption. 

Inspired by \cite{Misunderstanding-Nonlinear-Prices_2020_(Shaffer)}, the estimates could be interpreted differently. The paper finds that a subgroup of less than 10\% of households was driving the seemingly overall response. If this is also true in my setting, then the measured decrease in household electricity consumption would be attributed to a subset of my sample. Let us suppose that there are two distinct types of SMUD residential customers: households over-responding to the lagged marginal price and those not responding to it.\footnote{Here, I do not consider the type of households that respond to the average price because the change in the marginal price at the threshold does not accompany any change in the average price.} My back-of-the-envelope calculation suggests that about 4\% of over-responders produce the estimated treatment effect.\footnote{In my rough computation, I exploit the price elasticity for the lagged marginal price provided in \cite{Do-Consumers-Respond-to-Marginal-or-Average-Price?-Evidence-from-Nonlinear-Electricity-Pricing_2014_(Ito)}.} Interestingly, this calculation parallels the finding in the paper. 
