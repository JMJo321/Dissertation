\noindent
\textit{Regression Discontinuity Results for Different Bandwidths and Functional Forms} 
--- 
Table \ref{Table:Robustness-Checks_BWs} summarizes the regression results for a set of different bandwidths. For the three cases where the bandwidth utilized is greater than 20\%, the treatment estimates range from -0.021 to -0.032 and are significant at least at the 10\% level. Interestingly, this table clearly shows that the wider the bandwidth employed, the larger the estimated treatment effect. In other words, the treatment estimates approach zero as I move even closer to the lower base usage quantity.

There are several possible explanations for this monotonic trend in the treatment effect. First, it may be more difficult or demanding for SMUD residential customers near the threshold to notice, from their monthly bill statements, whether their electricity consumption in the previous billing month barely exceeded the lower base usage quantity, which in turn made them experience a discontinuous increase in the marginal price. Second, households whose electricity consumption just surpassed the cutoff point in a billing cycle could intentionally ignore the lagged price signal in the subsequent billing cycle. Some of them likely understood that their immediate electricity consumption was utterly irrelevant to the signal. It is also possible that adjusting their electricity consumption pattern against the lagged marginal price during a whole billing month led to too much cost for some treated households very near the threshold compared to its benefit. Third, households near the lower base usage quantity may respond differently to the lagged marginal price compared to those farther from the threshold. Specifically, conditional on a given magnitude of the increase in the lagged marginal price, heavy electricity consumers could be more responsive to the price signal. 

Table \ref{Table:Robustness-Checks_Functional-Forms} presents the regression results from other specifications having different functional forms. As illustrated in Figure \ref{Figure:Average-Daily-Electricity-Consumption-over-Normalized-Electricity-Consumption}, a linear regression function seems highly reasonable on both sides of the threshold. And the high dependency of the sign of the treatment estimates on adding higher-order polynomials of the running variables confirms it strongly. 
\\

\noindent
\textit{Falsification Test} 
---
Table \ref{Table:Robustness-Checks_Falsification-Test} summarizes the results from the falsification test that examines treatment effects at four distinct cutoff points (i.e., -15\%, -10\%, 10\%, and 15\% of the normalized electricity consumption in Period 0). In the falsification test, I only use bandwidths less than the distance between a false threshold and the (actual) lower base usage quantity to avoid capturing some of the treatment effect. As clearly demonstrated, no estimate is statistically significant at the 10\% level, suggesting that the RD approach is valid. 
