\noindent
\textit{\textbf{Treatment Effects by Season}} --- 
Table \ref{Table:Heterogeneity-in-Household-Responses:Treatment-Effect-by-Season} summarizes how households responded differently to the lagged marginal price in different seasons. In this table, based on the billing month of Period 0, the summer season is from June to September, while the winter season is from December to March. Of note, there was no change in the lower base usage quantity during each season. For each of the three rate codes (i.e., RSCH, RSEH, and RSGH), the first column (i.e., columns (1), (4), and (7) in the table) presents the treatment effect obtained by exploiting meter readings from all months. 

The two rate groups for households with electric heating (i.e., RSCH and RSEH) shared identical base usage quantities in my sample. The only difference between them was that households choosing the RSCH rate plan paid a much lower price in the winter season. Generally, households with electric heating show very similar consumption changes, though the large standard errors of the coefficient of interest make it difficult to say anything conclusive about any difference between the two rate groups. 

Residential customers adopting the RSGH rate plan, which experienced relatively small seasonal changes in base usage quantities, show similar reductions in daily electricity consumption in both seasons, except that the RD estimate for the summer season demonstrates much larger standard errors. 


\par \vspace{0.5cm}
\noindent
\textit{\textbf{Treatment Effects at the Higher Base Usage Quantity}} ---
Table \ref{Table:Heterogeneity-in-Household-Responses:At-the-Higher-BUQ} presents the results of applying the RD design to the higher base usage quantity. In other words, the RD estimates shown in this table demonstrate how SMUD residential customers responded to the lagged marginal price at the higher base usage quantity. Interestingly, no estimated treatment effect is statistically significant at the 5\% level. In other words, the table highlights that households did not respond to the lagged marginal price at the higher base usage quantity. 

There is a number of possible reasons for the empirical finding. First, considering the consumption reductions around the lower base usage quantity, this finding may indicate that households' responsiveness to the lagged marginal price varies with the level of electricity consumption. If heavy electricity consumers, presumably high-income households, tend to pay less attention to how much they are paying for their marginal electricity consumption, no response at the higher cutoff point seems reasonable. Second, no change in households' consumption behavior near the higher base usage quantity could be attributed to the relatively small magnitude of the increases in the marginal price at the threshold. Specifically, the average marginal price increase at the lower base usage quantity was about 74\%, whereas it was only about 18\% at the higher threshold. 
