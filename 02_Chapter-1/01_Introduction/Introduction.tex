Increasing-Block Pricing (IBP) is one of the most common electricity rate plans. Under the pricing scheme, the marginal price increases, in a nonlinear way, with the customer's usage. Specifically, the marginal price customers pay for their marginal electricity consumption is a step function of their aggregate consumption in a billing cycle. Naturally, the relationship between price and quantity consumed under IBP would imply that for customers, knowing how much electricity they have used from the first day of the current billing month is inevitable to properly adjust their consumption behavior according to the price variations determined by their past consumption decisions in that billing month. 

Electricity bills, usually issued every month, have been a primary, perhaps for some residential customers the only, source of information about their electricity consumption in a billing cycle, including the total amount of consumption and the marginal price. Unlike a decade ago, many households can access (near-)real-time data about their electricity consumption these days because of the diffusion of smart metering. But as illustrated in \cite{Dynamic-Salience-with-Intermittent-Billing_Gilbert-and-Zivin_2014}, households change their consumption behavior after receiving a monthly bill, even though the pieces of information, which are available directly or after some computations from it, are not up-to-date. In other words, the paper shows that electricity bills, which are intermittent tough, determine household electricity consumption at least partly. 

Economists have studied how households respond to monthly bill-providing information, especially to different prices. \cite{Do-Consumers-Respond-to-Marginal-or-Average-Price?-Evidence-from-Nonlinear-Electricity-Pricing_2014_(Ito)}, demonstrating that when faced with IBP, residential consumers respond to the average price rather than the marginal price, presents convincing evidence that households respond more to lagged average prices than contemporaneous ones. More recently, \cite{Misunderstanding-Nonlinear-Prices_2020_(Shaffer)} shows that not all households uniformly respond to the average price. To be specific, this paper exhibits that most residential consumers respond to the average price, but their response can be veiled by the response of a small number of households mistakenly applying the marginal price to all consumption units, not the last unit. And in both papers, it is stressed that the two distinct types of misunderstanding nonlinear prices will lead to households' inefficient electricity consumption, like over- or under-consumption. 

This paper examines another piece of information available from monthly electricity bills that could alter a household's consumption behavior: whether the total electricity consumption in a billing cycle surpasses a base usage quantity at which the marginal price sharply jumps. Specifically, I investigate how households adjust their electricity consumption in a billing month when they know that their total electricity consumption in the previous billing month just exceeded a threshold at which the marginal price discontinuously increases. In other words, focusing on two consecutive billing cycles, I study how households respond to a delayed price signal, indicating that they were subjected to a higher margin price in the previous billing month. 

To measure the impact of surpassing a base usage quantity in a billing cycle (denoted Period 0) on household electricity consumption in the following billing cycle (denoted Period 1), I utilize the sharp discontinuity in the marginal price at the threshold. Under IBP, a household's aggregate electricity consumption at some point completely determines the price the household pays for the marginal unit of consumption at that point. In other words, in my setting, the known-to-consumer price determination mechanism is what determines a household's treatment status. On top of that, households had no practical way to know how much electricity they had consumed from the first day of the very billing cycle at the point of consumption, especially before installing smart meters. They were able to find out whether their total consumption in a billing cycle surpassed the threshold only after receiving their monthly bills at the beginning of the subsequent billing cycle. Therefore, it is compelling that residential consumers were not able to control their electricity consumption precisely to avoid crossing the cutoff point. Resultingly, the two points demonstrate that my RD design clearly satisfies the fundamental assumption required for identification: around the threshold, the treatment assignment is random. 

Implementing the RD design to monthly billing records from hundreds of thousands of residential consumers in Sacramento, CA, I find households' interesting behavioral changes with respect to their electricity consumption: the treated households (i.e., households whose aggregate electricity consumption in a billing month exceeded the first base usage quantity) reduced their electricity consumption by 0.1\% in the following billing month. Importantly, this result suggests another inefficiency stemming from households' responses to nonlinear electricity pricing---households also respond to the delayed marginal price, reflecting not their contemporaneous consumption level but their past one. In addition, my empirical analysis demonstrates that among the treated households, the larger the electricity consumption during Period 0, the greater the magnitude of the reduction in electricity consumption during Period 1. I also examine heterogeneity in responsiveness to receiving the delayed price signal across households based on different rate codes and seasons. Furthermore, my investigation into the impacts of the discontinuous increase in the marginal price shows that reductions in household electricity consumption persist over multiple billing cycles but tend to dissipate gradually. 

The rest of this paper proceeds as follows. In Section 2, I discuss my empirical approach and data. Section 3 presents the results from my empirical analysis. Section 4 describes the policy implications of my key findings, and Section 5 concludes. 