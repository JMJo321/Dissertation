Increasing-Block Pricing (IBP) is one of the most common electricity rate plans. Under the pricing scheme, the marginal price increases, in a nonlinear way, with the customer's usage. Specifically, the price consumers pay for their marginal electricity consumption is a step function of their aggregate consumption in a billing cycle. Naturally, the relationship between price and quantity consumed under IBP would imply that for consumers, knowing how much electricity they have used from the first day of the current billing month is inevitable to properly adjust their consumption behavior according to the price variations determined by their past consumption decisions in that billing month. 

Economists have studied how residential consumers respond to nonlinear electricity pricing. \cite{Do-Consumers-Respond-to-Marginal-or-Average-Price?-Evidence-from-Nonlinear-Electricity-Pricing_2014_(Ito)} demonstrates that when faced with IBP, households respond to average prices rather than marginal prices. More recently, \cite{Misunderstanding-Nonlinear-Prices_2020_(Shaffer)} shows that not all households uniformly respond to the average price. To be specific, this paper exhibits that most residential consumers respond to the average price, but their response can be veiled by the response of a small number of households mistakenly applying the marginal price to all inframarginal units of consumption, not the last unit. And in both papers, it is stressed that households' misunderstanding of nonlinear electricity pricing leads to their inefficient electricity consumption, such as over- or under-consumption.

Electricity bills, usually issued every month, have been a primary, perhaps for some households the only, source of information about their electricity consumption (e.g., the total charges, the amount of consumption, and the marginal price). Inherently, the consumption-relevant information on a monthly electricity statement is not up to date---a statement principally includes \textit{delayed} information about the previous billing cycle. Household electricity consumption, however, is affected, at least partly, by the intermittent bills. As illustrated in \cite{Dynamic-Salience-with-Intermittent-Billing_Gilbert-and-Zivin_2014}, households change their consumption behavior after receiving a monthly bill. And in addition to the key finding discussed above, \cite{Do-Consumers-Respond-to-Marginal-or-Average-Price?-Evidence-from-Nonlinear-Electricity-Pricing_2014_(Ito)} presents convincing evidence that households respond more to lagged average prices, which are available in monthly electricity statements, than contemporaneous ones. 

This paper examines households' responses to the marginal price again but from a different point of view. Specifically, I investigate how residential consumers under IBP adjust their electricity consumption in a billing month according to the marginal price's discontinuous increase at a base usage quantity in the previous billing month, not the current one. In other words, focusing on two consecutive billing cycles, I study how households respond to the delayed price signal, transferred probably through their monthly electricity bills. 

To estimate the impact of the change in the marginal price due to surpassing a base usage quantity in a billing cycle on household electricity consumption in the following billing cycle, I adopt a Regression Discontinuity (RD) approach that exploits the sharp discontinuity in the marginal price at the threshold. Under IBP, a household's (within-billing-cycle) aggregate electricity consumption at some point completely determines the price the household pays for the marginal unit of consumption at that point. In other words, in my empirical setting, the known-to-consumer price determination mechanism is what determines a household's treatment status in a billing month, absolutely depending on whether the total electricity consumption of a treated household during a billing month exceeds the base usage quantity. On top of that, the households in my sample had no practical way to know how much electricity they had consumed from the first day of a billing cycle to the point of the marginal electricity consumption, especially before installing smart meters. They were able to find out whether their total consumption in a billing cycle surpassed the threshold only after receiving their monthly bills at the beginning of the subsequent billing cycle. Therefore, it is compelling that residential consumers were not able to control their electricity consumption precisely to avoid crossing the cutoff point. Thus, the two points demonstrate that my RD design clearly satisfies the fundamental assumption required for identification: around the threshold, the treatment assignment is random.

Implementing the RD design to monthly billing records from hundreds of thousands of residential consumers in Sacramento, CA, I find households' interesting behavioral changes with respect to their electricity consumption: around the lower base usage quantity, the treated households, which were subjected to a higher marginal price because their aggregate electricity consumption in a billing month exceeded the threshold, reduced their daily electricity consumption by 0.1\% in the following billing month, compared to the households assigned to the control group. Because only the marginal price discontinuously increases, not the average prices, at the threshold, this result is convincing evidence that households respond to marginal prices. More importantly, the result suggests another inefficiency stemming from households' responses to nonlinear electricity pricing: 1) households also respond to the \textit{delayed} marginal price, reflecting not their contemporaneous consumption level but their past one; and 2) the delayed marginal price is applied to all units of electricity consumption, not the last unit, during a billing cycle. I also examine heterogeneity in households' responsiveness to the delayed marginal price across rate codes and seasons. Furthermore, my investigation into the impacts of the discontinuous increase in the marginal price on household electricity consumption demonstrates that reductions in household electricity consumption persist over multiple billing cycles but tend to dissipate gradually. 

The rest of this paper proceeds as follows. In Section 2, I discuss my empirical approach and data. Section 3 presents the results from my empirical analysis. Section 4 describes the policy implications of my key findings, and Section 5 concludes. 
