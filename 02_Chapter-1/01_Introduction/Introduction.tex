Increasing-Block Pricing (IBP) is one of the most common electricity rate plans.\footnote{As of the end of December 2022, about 40\% of the residential consumers of Pacific Gas and Electric Company (PG\&E), which is one of the largest utilities in California, were on the IBP.} Under the pricing scheme, the marginal price increases, in a nonlinear way, with the customer's usage. Specifically, the price consumers pay for their marginal electricity consumption is a step function of their aggregate consumption in a billing cycle. 

Economists have studied how residential consumers respond to nonlinear electricity pricing. \cite{Do-Consumers-Respond-to-Marginal-or-Average-Price?-Evidence-from-Nonlinear-Electricity-Pricing_2014_(Ito)} demonstrates that when faced with IBP, households respond to average prices rather than marginal prices. More recently, \cite{Misunderstanding-Nonlinear-Prices_2020_(Shaffer)} shows that not all households uniformly respond to average prices. To be specific, this paper exhibits that most residential consumers respond to average prices, but their response can be veiled by the response of a small number of households mistakenly applying the marginal price to all inframarginal units of consumption, not the last unit. And in both papers, it is stressed that households' misunderstanding of nonlinear electricity pricing leads to their inefficient electricity consumption, such as over- or under-consumption.

Electricity bills, usually issued every month, have been a primary, perhaps for some households the only, source of information about their electricity consumption (e.g., the total charges, the amount of consumption, and the marginal price). Inherently, the consumption-relevant information on a monthly electricity statement is not up to date---a statement principally includes information about the previous billing cycle. Household electricity consumption, however, is affected, at least partly, by the intermittent bills. As illustrated in \cite{Dynamic-Salience-with-Intermittent-Billing_Gilbert-and-Zivin_2014}, households change their consumption behavior after receiving a monthly bill. And in addition to the key finding discussed above, \cite{Do-Consumers-Respond-to-Marginal-or-Average-Price?-Evidence-from-Nonlinear-Electricity-Pricing_2014_(Ito)} presents convincing evidence that households respond more to lagged average prices, which are available in monthly electricity statements, than contemporaneous ones. 

There are two issues when residential consumers change their electricity consumption according to price signals delivered through monthly electricity bills. The first issue is that they respond to the incorrect prices. Whatever households' perceived price of nonlinear price schedules is, the price signals are false because these signals reflect not their contemporaneous consumption level but their past one. And households' exaggerated responses to the price signals are the second issue. Suppose that residential consumers adjust their electricity consumption according to the lagged prices. In that case, the behavioral change implies that all consumption in a billing cycle is subject to the previous billing cycle's prices. Importantly, such an implication is analogous with households' over-responses to the (current) marginal price discussed in \cite{Misunderstanding-Nonlinear-Prices_2020_(Shaffer)}. 

On both those fronts, this paper examines households' responses to the marginal price again. For residential consumers under IBP, I measure how much electricity consumption they reduced in a billing month in response to the discontinuous increase in the marginal price at the lower base usage quantity in the immediately preceding billing month, which was not accompanied by any discontinuous change in average prices. In other words, focusing on two consecutive billing cycles, I investigate how households responded to the sharp increase in the \textit{lagged} marginal price signaled through their monthly electricity bills.

To estimate the impact of the change in the marginal price due to surpassing a base usage quantity in a billing cycle on household electricity consumption in the following billing cycle, I adopt a Regression Discontinuity (RD) approach that exploits the sharp discontinuity in the marginal price at the threshold. Under IBP, a household's (within-billing-cycle) aggregate electricity consumption at some point completely determines the price the household pays for the marginal unit of consumption at that point. Under the known-to-consumer price determination mechanism, my identification relies on the assumption that households cannot manipulate whether they fall on one side of the cutoff or the other conditional on their monthly consumption being close to the cutoff point. In my empirical setting, the assumption is convincing for two reasons. First of all, the households in my sample had no practical way to know how much electricity they had consumed from the first day of a billing cycle to the point of marginal electricity consumption, especially before installing smart meters. They were able to find out whether their total consumption in a billing cycle surpassed the threshold only after receiving their monthly bills at the beginning of the subsequent billing cycle. On top of that, it is not feasible for them to adjust their electricity consumption precisely to avoid crossing the cutoff point due to their indispensable consumption (e.g., consumption for refrigerators and lighting) and too high information cost imposed when micromanaging electricity demand according to the change of outdoor conditions. Thus, the two points demonstrate that my RD design clearly satisfies the fundamental assumption required for identification: around the threshold, the treatment assignment is random.

Implementing the RD design to monthly billing records from hundreds of thousands of residential consumers in Sacramento, CA, I find households' interesting behavioral change with respect to electricity consumption: around the lower base usage quantity, the households that were subject to a higher marginal price because their aggregate electricity consumption in a billing month exceeded the cutoff point reduced their average daily electricity consumption by about 0.16\% in the following billing month, compared to the households experiencing no change in the marginal price. Because only the marginal price discontinuously increased, not the average price, at the threshold, this result is convincing evidence that households, conservatively at least some of them, respond to marginal prices. In line with \cite{Misunderstanding-Nonlinear-Prices_2020_(Shaffer)}, if the identified reduction in household electricity consumption is entirely attributable to behavioral changes of a subset of residential consumers, my estimates imply that approximately 11\% of households in my sample reacted to the discontinuous jump in the marginal price. More importantly, the result suggests other inefficiencies stemming from households' responses to nonlinear electricity pricing: 1) households also respond to the lagged marginal price, which is irrelevant to their present consumption level; and 2) the lagged marginal price is applied to all units of electricity consumption, not the last unit, during a billing cycle. 

I also examine the heterogeneity in household responsiveness to the lagged marginal price across rate codes and seasons. In addition, my investigation of the multi-period impact of the discontinuous increase in the lagged marginal price on household electricity consumption shows whether households' use of electric heating drove the impact. 

The rest of this paper proceeds as follows. In Section 2, I discuss my empirical approach and data. Section 3 presents the results from my empirical analysis. Section 4 describes the policy implications of my key findings, and Section 5 concludes. 
